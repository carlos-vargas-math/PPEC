\documentclass{book}
\usepackage{fullpage}
\usepackage{ifthen}
\usepackage{xparse}
\usepackage{etoolbox}
\usepackage[utf8]{inputenc}
% \usepackage[spanish,es-nodecimaldot]{babel}

\usepackage[T1]{fontenc}
\usepackage{fourier}
\usepackage{paratype}


\usepackage{enumerate}
\usepackage{hyperref}
\usepackage{amsmath,amssymb}
\usepackage{epigraph}
\setlength\epigraphwidth{.8\textwidth}
\usepackage{fdsymbol}
\usepackage{mathrsfs}
\usepackage{pgfplots}
\usepackage{pstricks-add}
\usepackage{tikz}
\usetikzlibrary{arrows}
\usetikzlibrary{babel}
\usepackage{multicol}
\pgfplotsset{compat=1.15}
\usepackage{mathrsfs}
\usepackage{color}
\usepackage{graphicx}
\usepackage{caption}
\usepackage{subcaption}
\usepackage{pdfpages}
%\usepackage[1-17]{pagesel}
\usepackage{wrapfig}

\newcommand{\ZZ}{\mathbb{Z}}
\DeclareMathOperator{\mcd}{mcd}

\usepackage{amsthm}
\theoremstyle{definition}
\newcommand{\sen}{\mathrm{sen}}
\newtheorem*{comentario}{Comentario}
\newtheorem{definicion}{Definici\'on}[chapter]
\newtheorem{teorema}{Teorema}[chapter]
\newtheorem{corolario}{Corolario}[chapter]
\newtheorem{problema*}{Problema}
\newtheorem{ejemplo}{Ejemplo}[chapter]
\newtheorem{problema}{Problema}[chapter]
\newtheorem{proposicion}{Proposici\'on}[chapter]
\newtheorem{observacion}{Observaci\'on}[chapter]
% \newtheorem{ejercicio}{Ejercicio}[chapter]
\newtheorem{exercise}{Exercise}[chapter]
\newenvironment{solucion}{\begin{proof}[Solución]\small}{\end{proof}}

\newboolean{tutorialmode}
\setboolean{tutorialmode}{false}  % Set to false for print/final version

\newcommand{\tutpagebreak}{\ifthenelse{\boolean{tutorialmode}}{\newpage}{}}

\usepackage{ifthen}
\newboolean{showhints}
\setboolean{showhints}{false} % or false to hide all suggestions

\newcommand{\hint}[1]{%
  \ifthenelse{\boolean{showhints}}{%
    \par\smallskip\textit{Hint:} #1%
  }{}%
}

\begin{document}

%\tableofcontents
\pagenumbering{arabic}


% \chapter{Repaso de Aritmética, Álgebra y Notación}

En este capítulo introductorio resumiremos la mayoría de las ideas y conceptos básicos de matemáticas que se aprenden en la educación básica. Varios de ellos serán revisados con mucho mayor detalle y en un contexto más amplio, con miras a la matemática pre-universitaria, en los capítulos posteriores.

\section{Teorema fundamental de la aritmética, mcm y mcd}

El {\bf teorema fundamental de la aritmética} establece que todo número natural se \emph{factoriza como producto} de {\bf números primos}: $$n=p_1^{\alpha_1}\cdot p_2^{\alpha_2}\cdots p_k^{\alpha_k}.$$
{\bf Factorizar} es la acción de descomponer un número o una expresión como producto de números más pequeños o expresiones más sencillas, llamados {\bf factores}.

Por ejemplo: 
$$12=2^2\cdot 3,\quad 15=3\cdot 5,\quad 20= 2^2\cdot 5,\quad 36=2^2\cdot 3^2, \quad 49=7^2, \quad 50=2\cdot 5^2, \quad 100=2^2\cdot 5^2, \quad 101=101.$$

$$(x+y)^3=(x+y)(x+y)(x+y)=x^3+3x^2y+3xy^2+y^3, \quad (x-y)^2 =(x-y)(x-y)=x^2-2xy+y^2.$$

$$(x+y+z)^2=(x+y+z)(x+y+z)=x^2+y^2+z^2+3xy+3xz+3yz$$

Dos de las más inmediatas aplicaciones del teorema fundamental de la aritmética es para calcular el {\bf mínimo común múltiplo} $\mathrm{mcm}[n,m]$ y el {\bf máximo común divisor} $\mathrm{mcd}(n,m)$. Para calcularlos, se deben considerar simultáneamente las factorizaciones de ambos números $$n=p_1^{\alpha_1}\cdot p_2^{\alpha_2} \cdots p_k^{\alpha_k}, \quad m=q_1^{\beta_1}\cdot q_2^{\beta_2} \cdots q_k^{\beta_k},$$
donde se vale que algunas $\alpha_i$, $\beta_j, i\neq j$ sean ceros. 

El {\bf máximo común divisor} $$(m,n):= \mathrm{mcd}(n,m):= p_1^{\min\{\alpha_1,\beta_1\}}\cdot p_2^{\min\{\alpha_2,\beta_2\}} \cdots p_k^{\min\{\alpha_k,\beta_k\}}$$ se calcula multiplicando todos los factores primos que aparecen en \emph{ambas} factorizaciones, tomando el \emph{mínimo exponente para cada primo}.

En contraparte, el {\bf minimo comun multiplo} toma los maximos para cada par de exponentes: $$[m,n]:= \mathrm{mcm}(n,m):= p_1^{\max\{\alpha_1,\beta_1\}}\cdot p_2^{\max\{\alpha_2,\beta_2\}} \cdots p_k^{\max\{\alpha_k,\beta_k\}}$$. 

Por ejemplo, la siguiente tabla contiene los valores de $\mathrm{mcd}(n,m)$ para $n,m= 12,15,20,36,49,50,100,101$:
\begin{center}
\begin{tabular}{|c||c|c|c|c|c|c|c|c|c|} 
 \hline
  $m\setminus n$& $12$ & $15$ & $20$ & $36$ & $49$ & $50$& $99$ & $100$ & $101$ \\ 
  \hline
  \hline
  $12$ & $12$ & $3$ & $4$ & $12$ & $1$ & $2$ & $3$ & $4$ & $1$ \\
  \hline
  $15$ & $3$ & $15$ & $5$ & $3$ & $1$ & $5$ & $3$ & $5$ & $1$ \\
  \hline
  $20$ & $4$ & $5$ & $20$ & $4$ & $1$ & $10$ & $1$ & $20$ & $1$ \\
  \hline
  $36$ & $12$ & $3$ & $4$ & $36$ & $1$ & $2$ & $9$ & $4$ & $1$ \\ 
  \hline
  $49$ & $1$ & $1$ & $1$ & $1$ & $49$ & $1$ & $1$ & $1$ & $1$ \\
  \hline
    $50$ & $2$ & $5$ & $10$ & $2$ & $1$ & $50$ & $1$ & $50$ & $1$ \\
  \hline
    $99$ & $3$ & $3$ & $1$ & $9$ & $1$ & $1$ & $99$ & $1$ & $1$ \\
  \hline
    $100$ & $4$ & $5$ & $20$ & $4$ & $1$ & $50$ & $1$ & $100$ & $1$ \\
  \hline
    $101$ & $1$ & $1$ & $1$ & $1$ & $1$ & $1$ & $1$ & $1$ & $101$ \\
  \hline
  \end{tabular}    
\end{center}
Observa que dos números consecutivos no comparten primos, por lo tanto, el $\mathrm{mcd}(n,n+1)=1$. El máximo común divisor juega un rol muy importante en teoría de números.

Por otro lado, el mínimo común múltiplo $\mathrm{mcm}[n,m]$ se construye multiplicando todos los primos que aparecen en \emph{alguna} de las factorizaciones, tomando como exponente el \emph{máximo exponente para cada primo}.


Por ejemplo, la siguiente tabla contiene los valores de $\mathrm{mcm}[n,m]$ para $n,m= 12,15,20,36,49,50,100,101$:
\begin{center}
\begin{tabular}{|c||c|c|c|c|c|c|c|c|c|} 
 \hline
  $m\setminus n$& $12$ & $15$ & $20$ & $36$ & $49$ & $50$& $99$ & $100$ & $101$ \\ 
  \hline
  \hline
  $12$ & $12$ & $60$ & $60$ & $36$ & $588$ & $300$ & $1188$ & $300$ & $1212$ \\
  \hline
  $15$ & $60$ & $15$ & $60$ & $180$ & $735$ & $150$ & $495$ & $300$ & $1515$ \\
  \hline
  $20$ & $60$ & $60$ & $20$ & $180$ & $980$ & $100$ & $1980$ & $100$ & $2020$ \\
  \hline
  $36$ & $36$ & $180$ & $180$ & $36$ & $1740$ & $900$ & $396$ & $900$ & $3636$ \\ 
  \hline
  $49$ & $588$ & $735$ & $980$ & $1740$ & $49$ & $2450$ & $4851$ & $4900$ & $4949$ \\
  \hline
    $50$ & $300$ & $150$ & $100$ & $900$ & $2450$ & $50$ & $4950$ & $100$ & $5050$ \\
  \hline
    $99$ & $1188$ & $495$ & $1980$ & $396$ & $4851$ & $4950$ & $99$ & $9900$ & $9999$ \\
  \hline
    $100$ & $300$ & $300$ & $100$ & $900$ & $4900$ & $100$ & $9900$ & $100$ & $10100$ \\
  \hline
    $101$ & $1212$ & $1515$ & $2020$ & $3636$ & $4949$ & $5050$ & $9999$ & $10100$ & $101$ \\
  \hline
  \end{tabular}    
\end{center}

\begin{ejercicio}
Lista todos los números primos menores que $50$.
\end{ejercicio}

\begin{ejercicio}
Factoriza los siguientes números. 
$$60,\quad 240, \quad 150,  \quad 300, \quad  1000,\quad 1001.$$
\end{ejercicio}

\begin{ejercicio}
Completa las tablas con los $\mathrm{mcd}(n,m)$ (izq.) y $\mathrm{mcm}[n,m]$ (der.)
\end{ejercicio}

\begin{tabular}{|c||c|c|c|c|c|c|} 
 \hline
  $m\setminus n$& $60$ & $150$ & $240$ & $300$ & $1000$ & $1001$ \\ 
  \hline
  \hline
  $60$&  &  &  &  &  &  \\
  \hline
  $150$&  &  &  &  &  &  \\ 
  \hline
  $240$&  &  &  &  &  &  \\ 
  \hline
  $300$&  &  &  &  &  &  \\ 
  \hline
  $1000$&  &  &  &  &  &  \\
  \hline
  $1001$&  &  &  &  &  &  \\ 
  \hline
  \end{tabular}    
  \hspace{1cm}
  \begin{tabular}{|c||c|c|c|c|c|c|} 
 \hline
  $m\setminus n$& $60$ & $150$ & $240$ & $300$ & $1000$ & $1001$ \\ 
  \hline
  \hline
  $60$&  &  &  &  &  &  \\ 
  \hline
  $150$&  &  &  &  &  &  \\ 
  \hline
  $240$&  &  &  &  &  &  \\ 
  \hline
  $300$&  &  &  &  &  &  \\ 
  \hline
  $1000$&  &  &  &  &  &  \\
  \hline
  $1001$&  &  &  &  &  &  \\ 
  \hline
  \end{tabular}    

\begin{ejercicio}
Demuestra que para cualesquiera números enteros $m,n$, se cumple que $$mn=(m,n)[m,n].$$
\end{ejercicio}

\section{Simplificando fracciones y expresiones algebraicas}

Una {\bf fracción} $\frac{n}{d}$ representa una {\bf proporción}. 

Un ejemplo bastante útil para entender fracciones, por la simetría circular de los objetos involucrados y nuestra familiaridad con la acción de partirlos en pedazos iguales, es pensar en rebanadas de pizza o de pastel. 

El {\bf denominador} $d\neq 0$ nos indica en cuántos \emph{pedazos iguales} se cortan las pizzas. El {\bf numerador} nos indica cuántas rebanadas le tocan a una persona (puede que me toque más de una pizza completa $n>d$, o que deba pizzas que no he pagado $n<0$).

Una misma proporción se puede representar de \emph{muchas maneras}. Por ejemplo, obtengo la misma cantidad de pastel si éste se divide en $20$ rebanadas idénticas de las que tomo $4$, a que si el mismo pastel se divide en $10$ rebanadas y tomo $2$. 

    \begin{tikzpicture}
\draw[fill=gray!120] (0,
0) -- +(36:1) arc (36:-36:1);
\draw (0,0) circle [radius=1];
    \foreach \i in {0,...,9} \draw[dashed, black, line width=0.1pt] 
    (18*\i:1) -- (-180+18*\i:1);
\end{tikzpicture}         
\hspace{2cm}
    \begin{tikzpicture}
\draw[fill=gray!120] (0,0) -- +(36:1) arc (36:-36:1);
\draw (0,0) circle [radius=1];
    \foreach \i in {0,...,4} \draw[dashed, black, line width=0.1pt] 
    (36*\i:1) -- (-180+36*\i:1);
\end{tikzpicture}         
\hspace{2cm}
    \begin{tikzpicture}
\draw[fill=gray!120] (0,0) -- +(17.1428*3:1) arc (17.1428*3:-17.1428*6:1);
\draw (0,0) circle [radius=1];
    \foreach \i in {0,...,20} \draw[dashed, black, line width=0.1pt] 
    (17.1428*\i:1) -- (0,0);
\end{tikzpicture}
\hspace{2cm}
    \begin{tikzpicture}
\draw[fill=gray!120] (0,0) -- +(25.7142*2:1) arc (25.7142*2:-25.7142*4:1);
\draw (0,0) circle [radius=1];
    \foreach \i in {0,...,13} \draw[dashed, black, line width=0.1pt] 
    (25.7142*\i:1) -- (-180+25.7142*\i:1);
\end{tikzpicture}

Es decir, $\frac{4}{20}=\frac{2}{10}$ y de la misma manera $\frac{9}{21}=\frac{6}{14}$. Decimos que estas fracciones son {\bf equivalentes}. En general, dos fracciones $\frac{a}{b}, \frac{c}{d}$ son equivalentes si y solo si $a\cdot d=b\cdot c$. 

En el ejemplo anterior $4\cdot 10 = 2 \cdot 20$ y $9\cdot 14 = 6 \cdot 21$.

Hay una cantidad \emph{infinita de fracciones equivalentes} a una fracción dada $\frac{a}{b}$. Por ejemplo, si multiplico el numerador y el denominador por cualquier número $c$ que no sea cero, obtengo una fracción equivalente $\frac{c\cdot a}{c \cdot b}=\frac{a}{b}$. Entonces, por ejemplo $$\frac{2}{3}=\frac{4}{6}=\frac{6}{9}=\frac{8}{12}=\frac{10}{15}=\cdots, \quad \frac{7}{4}=\frac{14}{8}=\frac{21}{12}=\frac{-7}{-4}=\frac{-14}{-8}=\cdots$$

Normalmente, para facilitar cuentas, estamos interesados en la expresión que involucre los números más pequeños. 

Decimos que una fracción $\frac{a}{b}$ se encuentra {\bf simplificada} si el máximo común divisor $\mathrm{mcd}(a,b)=1$. Con esto se logran los numeradores y denominadores más pequeños. 

\begin{ejercicio}
Para cada una de las siguientes fracciones, escribe 5 fracciones equivalentes. Después encuentra la fracción equivalente simplificada.
\end{ejercicio}

\begin{ejercicio}
A continuación se presentan 20 fracciones. Agrúpalas por conjuntos de fracciones equivalentes.
\end{ejercicio}

Recordemos ahora las reglas para realizar operaciones con fracciones.

El caso más simple es el de la suma y resta con fracciones \emph{con el mismo denominador}. Aquí simplemente se tiene que $$\frac{a}{b}+\frac{c}{b}=\frac{a+c}{b},\quad \frac{a}{b}-\frac{c}{b}=\frac{a-c}{b}.$$
Por ejemplo:
$$\frac{3}{12}+\frac{10}{12}=\frac{13}{12},\quad \frac{4}{15}-\frac{6}{15}=\frac{-2}{15}, \quad \frac{8}{7}+\frac{5}{7}=\frac{13}{7},\quad \frac{12}{8}-\frac{5}{8}=\frac{7}{8}$$

Cuando las fracciones \emph{no tienen el mismo denominador}, se tienen que encontrar \emph{fracciones equivalentes con los mismos denominadores}: Por ejemplo

\begin{ejercicio}
Resuelve las siguientes sumas de fracciones con denominadores distintos.
\end{ejercicio}

Para multiplicar dos, tres o más fracciones la regla es sencilla. Simplemente se multiplican todos los numeradores arriba y se multiplican todos los denominadores abajo. $$\frac{a}{b}\cdot\frac{c}{d}=\frac{ac}{bd},\quad \frac{a}{b}\cdot\frac{c}{d}\cdot\frac{e}{f}=\frac{ace}{bdf}, \quad \frac{n_1}{d_1}\cdot\frac{n_2}{d_2}\cdots \frac{n_k}{d_k}=\frac{n_1n_2\cdots n_k}{d_1d_2\cdots d_k}.$$ 

En principio no es necesario buscar fracciones equivalentes, como en la suma, aunque en general es más eficiente trabajar con fracciones simplificadas.

\begin{ejercicio}
¿Cuánto valen la siguiente multiplicación de fracciones: $$\left(\frac{60}{21}\right)\times\left(\frac{1}{33}\right)\times\left(\frac{6}{5}\right)\times \left(\frac{35}{8}\right)?$$
\end{ejercicio}

\begin{ejercicio}
¿Cuánto vale $$\left(\frac{1}{2}\right)\times\left(\frac{2}{3}\right)\times\left(\frac{3}{4}\right)\times\cdots \times\left(\frac{998}{999}\right)\times\left( \frac{999}{1000}\right)?$$
\end{ejercicio}


La división es la operación inversa a la multiplicación, en el sentido de que:

Cuando el numerador y el denominador presentan algún {\bf factor común}, se puede \emph{simplificar la fracción cancelando el factor común}. Por ejemplo $$\frac{a\times c}{b\times c}=\frac{a}{b}, \quad \frac{a\times b\times c\times d \times e}{f \times c\times g \times a }=\frac{b\times d\times e}{f\times g},$$

%Utilizamos la conmutatividad de la muntuplicación para reordenar las fracciones y cancelarlas.

Es importante mencionar que las cancelaciones solo aplican para factores y no para sumas. Por ejemplo, \emph{no se vale} que en la fracción $\frac{a+c}{b+c}$ se cancelen las $c$'s para obtener $\frac{a+c}{b+c}=\frac{a}{b}$.

Cuando de multiplicaciones se trate, ¡se vale cancelar \emph{cualquier cosa que no sea cero}! (números fraccionarios, decimales, irracionales, complejos, etc).

Por ejemplo, la multiplicación:
$$\frac{\pi}{\sqrt{2}}\times \frac{4}{3 \pi}\times \frac{\sqrt{8}}{10}=\frac{\pi \times 4\times \sqrt{8}}{\sqrt{2} \times 3 \times \pi \times 10}.$$ 
me enfrentaría con una situación horrorosa, pues los números $$\pi=3.141592\dots,\quad  \sqrt{2}=1.4142135\dots, \quad \sqrt{8}=2.8284271\dots$$ son \emph{irracionales}. Esto significa que sus expansiones decimales nunca se acaban y nunca se repiten.

Pero, como $\sqrt{8}=\sqrt{4\times 2}=\sqrt{4}\times \sqrt{2}=2\times \sqrt{2}$, ¡puedo cancelar todos los factores irracionales!:
$$\frac{\pi}{\sqrt{2}}\times \frac{4}{3 \pi}\times \frac{\sqrt{8}}{10}=\frac{\pi}{\sqrt{2}}\times \frac{4}{3 \times \pi}\times \frac{2\times \sqrt{2}}{2\times 5}=\frac{\pi \times \sqrt{2}\times 2 \times 4}{\pi \times \sqrt{2}\times 2\times 3\times 5}=\frac{4}{15}.$$

Este truco, aunque sencillo, se utiliza para simplificar cuentas muy frecuentemente, en todas las áreas de las matemáticas.


\begin{ejercicio}
Muestra que si $\frac{a}{b}=\frac{c}{d}$, entonces $\frac{a}{b}=\frac{a+c}{b+d}$.
\end{ejercicio}

\begin{ejercicio}
Muestra que si $\frac{a}{b}=\frac{a+c}{b+d}$, entonces $\frac{a}{b}=\frac{c}{d}$.
\end{ejercicio}

Los dos ejercicios anteriores juntos muestran que la igualdad $\frac{a}{b}=\frac{c}{d}$ se verifica {\bf si y solamente si} $\frac{a}{b}=\frac{a+c}{b+d}$ se verifica.
%ToDO (ver sección ?).

%%%%%%%%%%%%%%%%%%%%%%%%%%%%%%%%%%%%%%%%%%%%%%%%%%%%%%%%%%%%

\newpage

\section{Póngale nombre}

Para organizar mejor un cálculo o una demostración matemática es conveniente \emph{ponerles nombres} a los objetos que deseamos estudiar. Así podemos referirnos a ellos con facilidad y manipularlos para resolverlos, compararlos o estudiarlos.

Se le puede poner nombre a números (naturales, enteros, racionales, etc.), objetos geométricos (ángulos, lados, áreas, circunferencias etc.), conjuntos, funciones y muchos objetos matemáticos más.

Entre otros caracteres, se utilizan frecuentemente en matemáticas: 
\begin{itemize}
    \item letras minúsculas del abecedario: $a,b,c,\dots d, \dots f,g,h, \dots k, \dots m,n, \dots p,q,r, \dots ,x,y,z,$ 
para nombrar números constantes, divisores, funciones, naturales, primos , incógnitas, coordenadas, etc).
    \item letras mayúsculas: $A,B,C,\dots, X,Y,Z$, (para nombrar conjuntos, puntos en el plano, etc).
    \item letras griegas minúsculas: $\alpha$, $\beta$, $\gamma$, $\delta$, $\theta$, etc. (para nombrar ángulos, permutaciones, la constante irracional $\pi=3.141592\dots $, etc). 
    \item símbolos especiales $\mathbb{N}, \mathbb{Z}, \mathbb{Q}, \mathbb{R}, \mathbb{C}$ (para nombrar, respectivamente, a los conjuntos de números naturales, enteros, racionales, reales y complejos). En estas notas, utilizaremos $\mathbb P$, y $\mathbb E$ para denotar a las funciones que calculan la probabilidad de un evento y su valor esperado (en probabilidad). 
\end{itemize}

Sin embargo, estas convenciones no son absolutamente estrictas y tenemos en principio la libertad de elegir el nombre que se nos antoje y que nos parezca más conveniente para un objeto matemático de nuestro interés. 

Supongamos que me piden calcular la suma de los primeros cien números naturales $1+2+3+4+\cdots +98+99+100$. Como al principio no sé cuánto vale, podría llamarle  <<$x$>> a esa suma: $$x:=1+2+3+4\cdots +99+100,$$ para después intentar calcular el valor de $x$. Como dijimos, tenemos absoluta libertad de elegir el nombre y <<$x$>> es solamente una elección de una infinidad de posibilidades.

Al mismo objeto matemático bien le podríamos poner el nombre <<$s$>>, $$s:=1+2+3+4+\cdots+99+100,$$ porque así la letra $s$ me recuerda que se trata de una suma, y le puedo preguntar: <<¿Qué tranza, ése?>>.

Los dos puntos al lado izquierdo del signo igual  $:=$ se utilizan para \emph{nombrar por primera vez a un objeto matemático}. Solamente se utilizan al momento del bautizo. Un vez que mi objeto tiene nombre, los puntos se omiten. Por ejemplo, después de calcular la suma, obtengo que $s=5050$ (ver sig. capítulo).

Los nombres pueden incluir {\bf índices}. Por ejemplo, si ahora me piden calcular el valor de tres sumas, la primera $1+2+3+4+\cdots+9+10$, la segunda $1+2+3+4+\cdots+99+100$ y la tercera $1+2+3+4+\cdots +999+1000$, podría llamarles $$s_1:=1+2+3+4+\cdots+9+10,\quad s_{2}:=1+2+3+4+\cdots+99+100, \quad s_{3}:=1+2+3+4+\cdots +999+1000,$$ para así distinguir entre las tres sumas, manipularlas y calcularlas.

Aunque no sepa el valor específico de $s_1$, $s_2$ y $s_3$ puedo asegurar sin muchos problemas que el primero es menor que el segundo, que a su vez es menor que el tercero y comunicar todo esto de forma práctica y compacta: $$s_1<s_2<s_3.$$

Si así lo quisiera, el nombre puede ser aún más informativo. Por ejemplo, podría llamarles $$s_{10}:=1+2+3+4+\cdots+9+10,\quad s_{100}:=1+2+3+4+\cdots+99+100, \quad s_{1000}:=1+2+3+4+\cdots +999+1000,$$
para que la variable (el <<nombre>>), me indique directamente cuántos números se están sumando.

A veces una notación sofisticada con índices nos puede ser muy útil. En otras ocasiones, por el contrario, la misma notación sofisticada podría resultar tediosa o poco práctica.

De la misma forma que asignamos $$s_{100}:=1+2+3+\dots+99+100,$$ podríamos nombrar al producto de los primeros números <<$\mathrm{perro}_{100}$>>, o: $$p_{100}:=1\times 2\times 3\times \cdots \times 99\times 100.$$

Esta última definición, aunque es válida, sería un despropósito. Los productos de los primeros $n$ enteros positivos son números bastante utilizados en matemáticas, pues aparecen de manera elemental en problemas de combinatoria, álgebra, probabilidad y análisis. Entonces es natural que los matemáticos ya se hayan puesto de acuerdo en cómo llamarles:
$$100!:=1\times 2\times 3\times \cdots \times 99\times 100.$$
El $100!$ se lee <<{\bf cien factorial}>>. Por convención se define $0!:=1$ (ver Sección 2.4).

Por su parte, aunque la suma de los primeros naturales no tiene un nombre por si solo, está determinada por un símbolo bastante estándar en matemáticas, llamado el {\bf coeficiente binomial} $${n\choose k}=\frac{n!}{k!(n-k)!}=\frac{n\times (n-1)\times \cdots \times 3\times2\times1}{k\times(k-1)\times \cdots \times 2\times 1\times (n-k)\times(n-k-1)\times \cdots \times 2\times 1},$$
(casualmente definido en términos de factoriales).

A partir de aquí tenemos que jubilar al símbolo $\times$ como signo de multiplicación. Es mucho mas claro y recomendable escribir $2\cdot 3 = 6$, $(2)(3)=6$, $2(3)=6$, $(2)3=6$, que escribir $2\times 3 = 6$. Sobre todo cuando se escribe a mano, el $\times$ se puede confundir con una letra $x$ 

Como en particular (ver primer capítulo sobre conteo): $$1+2+3+\cdots+n={n+1\choose 2}=\frac{n(n+1)}{2},$$
no hay necesidad de establecer un nuevo símbolo universal para denotar la suma de los primeros $n$ números naturales.

El encontrar buenos nombres para objetos matemáticos es un arte que se perfecciona con el tiempo. La regla dorada es \emph{no repetir el mismo nombre para dos objetos matemáticos distintos}. Si no respetamos esta regla confundiremos de forma terrible quien nos esté leyendo, o peor, \emph{calificando} (y probablemente a nosotros mismos).

\section{Productos notables}

\section{Simplificación de ecuaciones algebraicas (Igualdad)}

\section{Desigualdades}

\section{Notación de conjuntos y funciones}

\section{Conjuntos famosos}

\section{Los números complejos}


%%%%%%%%%%%%%%%%%%%%%%%%%%%%%%%%%%%%%%%%%%%%%%%%%%%%%%%%%%%%
\newpage



%\section{Manipulación de expresiones algebraicas}

%\subsection{Si y solamente si}

%Una expresión súper utilizada en la jerga matemática es el mentado <<si y solamente si>>. Es tan importante y frecuente que se abrevia <<si y solo si>>, <<ssi>> o con el símbolo de relación <<$\Longleftrightarrow$>>. 

%En esta sección explicaremos qué significa eso de <<P si y solamente si Q>>. Al principio este concepto podrá parecer un poco extraño, pero después de varios ejercicios se irá clarificando.

%Para empezar, <<$\Longleftrightarrow$>> no nos resulta completamente ajeno: en los ejercicios \ref{} y \ref{} nos pidieron demostrar exactamente que, para $a,b,c,d\in \mathbb Z$ con $b,d\neq 0$,  \[\frac{a}{b}=\frac{c}{d} \Longleftrightarrow  \frac{a}{b}=\frac{a+c}{b+d}\] 

%ToDo: Breve explicación de ssi.

%Ejemplo 2: Otro?.

%Ejemplo 3: Mediatriz.

%\subsection{Igualdades}

%\subsection{Desigualdades}




% \input{Capitulos/2Conteo}
% \chapter{Geometría elemental I}

\section{Configuraciones de líneas}

Dos rectas $l,l'$ en el plano se llaman {\bf paralelas} si no se intersectan en ningún punto. En el caso contrario, dos rectas $l,m$ no paralelas tienen un único punto de intersección.

Los dos pares de ángulos $(\alpha,\gamma)$ y $(\beta,\delta)$ se llaman {\bf opuestos por el vértice} y miden lo mismo.

\begin{ejercicio}
Figura (a): ¿Cuánto vale $\gamma+\beta$ en grados y en radianes?
\end{ejercicio}
%Obs: Introducir medida de ángulos en radianes vs grados.

Cuando tenemos dos paralelas $l, l'$ y una tercera línea transversal $m$, se obtienen dos colecciones de ángulos correspondientes.
%Obs: Ángulos correspondientes entre paralelas.

\definecolor{ffqqtt}{rgb}{1,0,0.2}
\definecolor{qqwwzz}{rgb}{0,0.4,0.6}
\begin{figure}[h]
\begin{subfigure}{.32\textwidth}
\centering
\begin{tikzpicture}[line cap=round,line join=round,>=triangle 45,x=1.5cm,y=1.5cm]
(8.55629629629625,4.036296296296293);
\draw [shift={(6.260681625951539,0.7951480173142947)},line width=1pt,color=qqwwzz,fill=qqwwzz,fill opacity=0.1] (0,0) -- (-33.436267283393505:0.22222222222222107) arc (-33.436267283393505:29.626460977420926:0.22222222222222107) -- cycle;
\draw [shift={(6.260681625951539,0.7951480173142947)},line width=1pt,color=ffqqtt,fill=ffqqtt,fill opacity=0.1] (0,0) -- (29.62646097742092:0.22222222222222107) arc (29.62646097742092:146.56373271660652:0.22222222222222107) -- cycle;
\draw [shift={(6.260681625951539,0.7951480173142947)},line width=1pt,color=qqwwzz,fill=qqwwzz,fill opacity=0.1] (0,0) -- (146.5637327166065:0.22222222222222107) arc (146.5637327166065:209.62646097742092:0.22222222222222107) -- cycle;
\draw [shift={(6.260681625951539,0.7951480173142947)},line width=1pt,color=ffqqtt,fill=ffqqtt,fill opacity=0.1] (0,0) -- (-150.37353902257908:0.22222222222222107) arc (-150.37353902257908:-33.436267283393505:0.22222222222222107) -- cycle;
\draw [line width=1pt] (6.260681625951539,0.7951480173142947) -- (8.52,2.08);
\draw [line width=1pt] (6.260681625951539,0.7951480173142947) -- (4.86,1.72);
\draw [line width=1pt] (6.260681625951539,0.7951480173142947) -- (4.691743294280375,-0.09709167290125631);
\draw [line width=1pt] (6.260681625951539,0.7951480173142947) -- (7.896641069926044,-0.2850548691377711);
\draw [fill=black] (6.260681625951539,0.7951480173142947) circle (1pt);
\draw[color=qqwwzz] (6.787407407407369,0.7351851851851834) node {$\alpha$};
\draw[color=ffqqtt] (6.250740740740704,1.1385185185185167) node {$\beta$};
\draw[color=qqwwzz] (5.802962962962928,0.7785185185185167) node {$\gamma$};
\draw[color=ffqqtt] (6.256296296296259,0.4029629629629612) node {$\delta$};
\draw[color=black] (7.507407407407365,1.6318518518518499) node {$l$};
\draw[color=black] (7.347407407407366,0.2096296296296279) node {$m$};
\end{tikzpicture}
  \caption{Ángulos opuestos por el vértice}
  \label{fig:sfig1}
\end{subfigure}%
\begin{subfigure}{.32\textwidth}
  \centering
   \begin{tikzpicture}[line cap=round,line join=round,>=triangle 45,x=.6cm,y=.6cm]
 \clip(5,-3) rectangle (14,4);
\draw [shift={(10.178618247765447,2.0141992314003194)},line width=1pt,fill=black,fill opacity=0.10000000149011612] (0,0) -- (1.577662875497712:0.6) arc (1.577662875497712:49.4456883534957:0.6) -- cycle;
\draw [shift={(10.178618247765447,2.0141992314003194)},line width=1pt,fill=black,fill opacity=0.10000000149011612] (0,0) -- (49.44568835349569:0.6) arc (49.44568835349569:181.5776628754977:0.6) -- cycle;
\draw [shift={(10.178618247765447,2.0141992314003194)},line width=1pt,fill=black,fill opacity=0.10000000149011612] (0,0) -- (-178.4223371245023:0.6) arc (-178.4223371245023:-130.55431164650432:0.6) -- cycle;
\draw [shift={(10.178618247765447,2.0141992314003194)},line width=1pt,fill=black,fill opacity=0.10000000149011612] (0,0) -- (-130.55431164650432:0.6) arc (-130.55431164650432:1.5776628754976951:0.6) -- cycle;
\draw [shift={(7.554985102983721,-1.0517906645364654)},line width=1pt,fill=black,fill opacity=0.10000000149011612] (0,0) -- (1.5776628754977127:0.6) arc (1.5776628754977127:49.44568835349569:0.6) -- cycle;
\draw [shift={(7.554985102983721,-1.0517906645364654)},line width=1pt,fill=black,fill opacity=0.10000000149011612] (0,0) -- (49.44568835349569:0.6) arc (49.44568835349569:181.5776628754977:0.6) -- cycle;
\draw [shift={(7.554985102983721,-1.0517906645364654)},line width=1pt,fill=black,fill opacity=0.10000000149011612] (0,0) -- (-178.4223371245023:0.6) arc (-178.4223371245023:-130.55431164650432:0.6) -- cycle;
\draw [shift={(7.554985102983721,-1.0517906645364654)},line width=1pt,fill=black,fill opacity=0.10000000149011612] (0,0) -- (-130.55431164650432:0.6) arc (-130.55431164650432:1.5776628754977207:0.6) -- cycle;
\draw [->,line width=1pt] (4.58,1.86) -- (13.966279351681408,2.118520405872581);
\draw [->,line width=1pt] (2.9,-1.18) -- (12.34,-0.92);
\draw [->,line width=1pt] (5.46,-3.5) -- (12.34,4.54);
\draw[color=black] (11.434074074074011,2.348518518518512) node {$\alpha_1$};
\draw[color=black] (9.914074074074018,2.848518518518512) node {$\alpha_2$};
\draw[color=black] (9.234074074074021,1.568518518518513) node {$\alpha_3$};
\draw[color=black] (11.134074074074013,1.2485185185185135) node {$\alpha_4$};
\draw[color=black] (8.854074074074026,-0.6114814814814842) node {$\beta_1$};
\draw[color=black] (7.314074074074032,-0.07148148148148487) node {$\beta_2$};
\draw[color=black] (6.554074074074035,-1.4314814814814834) node {$\beta_3$};
\draw[color=black] (8.434074074074028,-1.8714814814814829) node {$\beta_4$};
\draw[color=black] (5.89407407407403,2.328518518518512) node {$l$};
\draw[color=black] (11.534074074074022,-0.4114814814814842) node {$l'$};
\draw[color=black] (8.554074074074027,0.7885185185185142) node {$m$};
\end{tikzpicture}
  \caption{Dos paralelas y una transversal.}
  \label{fig:sfig2}
\end{subfigure}
\begin{subfigure}{.32\textwidth}
  \centering
\begin{tikzpicture}[line cap=round,line join=round,>=triangle 45,x=1.0cm,y=1.0cm]
\clip(-1.9068768493010926,-1.3161412100806018) rectangle (5.005683093561883,1.4
2223344556677895);
\draw [line width=1.pt,domain=-1.9068768493010926:5.005683093561883] plot(\x,{(-0.--0.41223344556678004*\x)/0.7918273645546372});
\draw [line width=1.pt,domain=-1.9068768493010926:5.005683093561883] plot(\x,{(-0.--0.9785021936537096*\x)/0.4347209468421588});
\draw [line width=1.pt,domain=-1.9068768493010926:5.005683093561883] plot(\x,{(-0.--0.9274869911233556*\x)/-0.42233445566778943});
\draw[color=black] (.5,.5) node {$\alpha_1$};
\draw[color=black] (0,.5) node {$\alpha_2$};
\draw[color=black] (-.5,.2) node {$\alpha_3$};
\draw[color=black] (-.5,-.5) node {$\alpha_4$};
\draw[color=black] (0,-.5) node {$\alpha_5$};
\draw[color=black] (.5,-.2) node {$\alpha_6$};
\end{tikzpicture}
  \caption{Tres líneas concurrentes}
  \label{fig:sfig2}
\end{subfigure}
  \caption{}
\end{figure}


\begin{ejercicio}
Figura (b): ¿Qué relaciones hay entre los ocho ángulos de la figura? 
\end{ejercicio}

\begin{ejercicio}
Figura (c). ¿Qué relaciones guardan los ángulos $\alpha_1,\alpha_2,\alpha_3, \alpha_4, \alpha_5, \alpha_6$?
\end{ejercicio}
\newpage

Decimos que tres líneas  $(l, m, n)$ se encuentran {\bf en posición general} si no son  concurrentes y no hay dos paralelas. Éstas delimitan un único triángulo.

\begin{ejercicio}
Demuestra que la suma de los tres ángulos interiores de un triángulo es igual a media vuelta al círculo (es decir: $180^{\circ}$ o $\pi$ radianes).

Sug: Trace una paralela.
\end{ejercicio}

A las magnitudes de los lados y ángulos de un triángulo con vértices $A,B,C$ se les suele llamar $|AB|=c$, $|AC|=b$, $|BC|=a$, y a los ángulos $\angle CAB=\alpha, \angle ABC= \beta, \angle BCA = \gamma$, como en la figura.

\begin{figure}[h]
    \centering
    \begin{subfigure}{.4\textwidth}
    \begin{tikzpicture}[line cap=round,line join=round,>=triangle 45,x=1.0cm,y=1.0cm]
\draw [shift={(1.15,-1.125)},line width=1.pt,fill=black,fill opacity=0.10000000149011612] (0,0) -- (-1.1233027140754284:0.45) arc (-1.1233027140754284:57.68038349181983:0.45) -- cycle;
\draw [shift={(4.975,-1.2)},line width=1.pt,fill=black,fill opacity=0.10000000149011612] (0,0) -- (136.8240898323761:0.45) arc (136.8240898323761:178.87669728592456:0.45) -- cycle;
\draw [shift={(2.545,1.08)},line width=1.pt,fill=black,fill opacity=0.10000000149011612] (0,0) -- (-122.3196165081802:0.45) arc (-122.3196165081802:-43.17591016762391:0.45) -- cycle;
\draw [line width=1.pt] (2.545,1.08)-- (4.975,-1.2);
\draw [line width=1.pt] (2.545,1.08)-- (1.15,-1.125);
\draw [line width=1.pt] (1.15,-1.125)-- (4.975,-1.2);
\draw[color=black] (0.865,-1.0875) node {$B$};
\draw[color=black] (5.23,-1.1625) node {$A$};
\draw[color=black] (2.485,1.4325) node {$C$};
\draw[color=black] (3.94,0.2575) node {$b$};
\draw[color=black] (1.675,0.2475) node {$a$};
\draw[color=black] (3.1,-1.4675) node {$c$};
\draw[color=black] (1.8,-0.8375) node {$\beta$};
\draw[color=black] (4.3,-1.0425) node {$\alpha$};
\draw[color=black] (2.55,0.4075) node {$\gamma$};
\end{tikzpicture}
    \caption{}
\end{subfigure}
\begin{subfigure}{.55\textwidth}
 \definecolor{rvwvcq}{rgb}{0.36235294117647059,0.546078431372549,0.6629411764705882}
\begin{tikzpicture}[line cap=round,line join=round,>=triangle 45,x=.8cm,y=.8cm]
\clip(3.4074074074073883,-1) rectangle (13.51407407407399,3.025185185185175);
\fill[line width=1pt,color=rvwvcq,fill=rvwvcq,fill opacity=0.40000000149011612] (5.618177814240684,-0.2985247635908286) -- (6.234795894642444,0.42205799315773773) -- (7.656691608979234,-0.24237925653235157) -- cycle;
\fill[line width=1pt,color=rvwvcq,fill=rvwvcq,fill opacity=0.40000000149011612] (5.344159897563122,1.8810467768396621) -- (10.906520143356797,2.034247376829742) -- (9.223994419902349,0.06803999069983671) -- cycle;
\draw [line width=1pt,domain=3.4074074074073883:15.51407407407399] plot(\x,{(-4.2788--0.26*\x)/9.44});
\draw [line width=1pt,domain=3.4074074074073883:15.51407407407399] plot(\x,{(--16.3676--0.26*\x)/9.44});
\draw [line width=1pt,domain=3.4074074074073883:15.51407407407399] plot(\x,{(--73.6928-8.04*\x)/-6.88});
\draw [line width=1pt,domain=3.4074074074073883:15.51407407407399] plot(\x,{(-7.138--1*\x)/-2.14});
\draw [line width=1pt,domain=3.4074074074073883:15.51407407407399] plot(\x,{(-9.3696--1*\x)/-2.14});
\draw [line width=1pt,domain=3.4074074074073883:15.51407407407399] plot(\x,{(--47.224-8.04*\x)/-6.88});
\draw[color=black] (12.1,-0.501481481481486) node {$m'$};
\draw[color=black] (12.1,1.5918518518518452) node {$m$};
\draw[color=black] (7.6,2.8) node {$l'$};
\draw[color=black] (4.100740740740719,1) node {$n'$};
\draw[color=black] (4.354074074074051,2.5518518518518447) node {$n$};
\draw[color=black] (11.34074074074069,2.8) node {$l$};
\end{tikzpicture}
  %dfnemrdm
      \caption{}
\end{subfigure}
      \caption{}
\end{figure}

\begin{ejercicio}
Si se cumplen las desigualdades $a<b<c$, ¿qué puedes decir de $\alpha, \beta, \gamma$?
\end{ejercicio}

\begin{teorema}[Desigualdad del triángulo]
Si un triángulo tiene lados $a,b,c$, entonces se cumplen las siguientes desigualdades: $$a\leq b+c,\quad b\leq a+c ,\quad c\leq a+b.$$
\end{teorema}

\begin{ejercicio}
¿Qué pasa si pones algún <<$=$>> en los dos anteriores ejercicios?
\end{ejercicio}

\begin{ejercicio}
¿Cuánto vale la suma de los ángulos interiores de un $n$-ágono?
\end{ejercicio}
%\vspace{2cm}

Considera tres líneas  $(l, m, n)$ en posición general.
Si considero otras tres líneas $(l', m', n')$ no concurrentes, tales que $l$ es paralela a $l'$, $m$ es paralela a $m'$ y $n$ es paralela a $n'$, se obtiene un {\bf triángulo semejante}.

Recordemos que dos triángulos $ABC$ y $A'B'C'$ se llaman {\bf semejantes} si sus ángulos correspondientes son iguales: es decir $$\measuredangle ABC=\measuredangle A'B'C',\quad \measuredangle BCA=\measuredangle B'C'A', \quad \measuredangle CAB=\measuredangle C'A'B'$$

\begin{ejercicio}
Demuestra que los triángulos delimitados por las rectas $(l,m,n)$ y $(l',m',n')$ de la figura son semejantes.
\end{ejercicio}

\begin{ejercicio}
¿Cuántos triángulos semejantes puedes encontrar en la figura?
\end{ejercicio}

\begin{ejercicio}
Si se dibujan $k_1$ rectas paralelas a $l$, $k_2$ rectas paralelas a $m$ y $k_3$ rectas paralelas a $n$ y no hay tres rectas concurrentes, ¿cuántos triángulos semejantes se forman?
\end{ejercicio}
\newpage

\section{Semejanza de triángulos}

Existen otras maneras de verificar que dos triángulos $ABC$ y $A'B'C'$ son semejantes.

\begin{teorema}[Criterios de semejanza de triángulos]
Los siguientes enunciados son equivalentes (cada uno significa que el triángulo $ABC$ es semejante al triángulo $A'B'C'$):
\begin{enumerate}
\item Sus ángulos correspondientes son iguales: es decir $$\measuredangle ABC=\measuredangle A'B'C',\quad \measuredangle BCA=\measuredangle B'C'A', \quad \measuredangle CAB=\measuredangle C'A'B'$$
\item Sus lados correspondientes son proporcionales, es decir: $$\frac{|AB|}{|A'B'|}=\frac{|BC|}{|B'C'|}=\frac{|CA|}{|C'A'|}.$$
\item Un ángulo es igual y los lados incidentes son proporcionales, es decir: $$\frac{|AB|}{|AC|}=\frac{|A'B'|}{|A'C'|},\quad \measuredangle CAB=\measuredangle C'A'B'.$$
\end{enumerate}
\end{teorema}

\begin{teorema}
Teoremas de Tales:
\end{teorema}

\begin{tikzpicture}[line cap=round,line join=round,>=triangle 45,x=1.0cm,y=1.0cm]
\clip(-2,-2) rectangle (5,3.5);
\draw [line width=1.pt,domain=-5.18:17.82] plot(\x,{(-0.--2.36*\x)/3.78});
\draw [line width=1.pt,domain=-5.18:17.82] plot(\x,{(-0.-1.7*\x)/3.84});
\draw [line width=1.pt,domain=-2.18:7] plot(\x,{(-15.4884--4.06*\x)/-0.06});
\draw [line width=1.pt,domain=-2.18:5.82] plot(\x,{(-9.265146898319307--4.06*\x)/-0.06});
\draw[color=black] (0.14,0.33) node {$A$};
\draw[color=black] (3.92,2.73) node {$B$};
\draw[color=black] (3.98,-1.33) node {$C$};
\draw[color=black] (2.44,-0.65) node {$E$};
\draw[color=black] (2.4,1.75) node {$D$};
\draw[color=black] (0,3) node {{\Large $\frac{|AB|}{|AD|}=\frac{|AC|}{|AE|}=\frac{|BC|}{|DE|}$}};
\draw[color=black] (0,1.5) node {{\Large $\frac{|AD|}{|DB|}=\frac{|AE|}{|EC|}$}};
\end{tikzpicture}
%-------------------------------------------------
\begin{tikzpicture}[line cap=round,line join=round,>=triangle 45,x=1.0cm,y=1.0cm]
\clip(-5.18,-2) rectangle (5,3.5);
\draw [line width=1.pt,domain=-5.18:7.82] plot(\x,{(-0.--2.36*\x)/3.78});
\draw [line width=1.pt,domain=-3.18:7.82] plot(\x,{(-0.-1.7*\x)/3.84});
\draw [line width=1.pt,domain=-3.18:5.82] plot(\x,{(-15.4884--4.06*\x)/-0.06});
\draw [line width=1.pt,domain=-2:1] plot(\x,{(--7.663576991993467--4.06*\x)/-0.06});
\draw[color=black] (0.14,0.33) node {$A$};
\draw[color=black] (3.92,2.73) node {$B$};
\draw[color=black] (3.98,-1.33) node {$C$};
\draw[color=black] (-1.76,1.21) node {$E$};
\draw[color=black] (-1.74,-0.83) node {$D$};
\draw[color=black] (.2,3) node {{\Large $\frac{|AB|}{|AD|}=\frac{|AC|}{|AE|}=\frac{|BC|}{|DE|}$}};
\draw[color=black] (0.2,1.5) node {{\Large $\frac{|AD|}{|DB|}=\frac{|AE|}{|EC|}$}};
\end{tikzpicture}
\vspace{1cm}

\begin{tikzpicture}[line cap=round,line join=round,>=triangle 45,x=1.0cm,y=1.0cm]
\clip(-3.18,-2) rectangle (5,4);
\draw [line width=1.pt,domain=-5.18:17.82] plot(\x,{(-0.--2.16*\x)/3.38});
\draw [line width=1.pt,domain=-5.18:17.82] plot(\x,{(-0.-1.7*\x)/3.84});
\draw [line width=1.pt,domain=-5.18:17.82] plot(\x,{(-14.0404--3.86*\x)/-0.46});
\draw [line width=1.pt,domain=-5.18:17.82] plot(\x,{(--6.947114382272223--3.86*\x)/-0.46});
\draw [line width=1.pt,domain=-5.18:17.82] plot(\x,{(-0.--0.18*\x)/5.88});
\draw[color=black] (0.14,0.33) node {$A$};
\draw[color=black] (3.52,2.53) node {$B$};
\draw[color=black] (3.98,-1.33) node {$C$};
\draw[color=black] (-1.76,1.21) node {$D$};
\draw[color=black] (-1.54,-0.73) node {$E$};
\draw[color=black] (3.76,0.45) node {$G$};
\draw[color=black] (-1.66,0.27) node {$H$};
\draw[color=black] (0.2,1.5) node {{\Large $\frac{|DH|}{|HE|}=\frac{|CG|}{|GB|}$}};
\draw[color=black] (0.2,3) node {{\Large $\frac{|DH|}{|CG|}=\frac{|HE|}{|GB|}$}};
\end{tikzpicture}
\hspace{1cm}
%---------------------------------------------------------------
\begin{tikzpicture}[line cap=round,line join=round,>=triangle 45,x=1.0cm,y=1.0cm]
\clip(-2,-2) rectangle (5,3.5);
\draw [line width=1.pt,domain=-5.18:17.82] plot(\x,{(-0.--2.16*\x)/3.38});
\draw [line width=1.pt,domain=-5.18:17.82] plot(\x,{(-0.-1.7*\x)/3.84});
\draw [line width=1.pt,domain=-5.18:17.82] plot(\x,{(-14.0404--3.86*\x)/-0.46});
\draw [line width=1.pt,domain=-5.18:17.82] plot(\x,{(-8.616141495611153--3.86*\x)/-0.46});
\draw [line width=1.pt,domain=-5.18:17.82] plot(\x,{(-0.--0.18*\x)/5.88});
\draw[color=black] (0.14,0.33) node {$A$};
\draw[color=black] (3.52,2.53) node {$B$};
\draw[color=black] (3.98,-1.33) node {$C$};
\draw[color=black] (2.5,-0.67) node {$D$};
\draw[color=black] (2.22,1.65) node {$E$};
\draw[color=black] (3.76,0.45) node {$G$};
\draw[color=black] (2.36,0.39) node {$H$};
\draw[color=black] (0.2,1.5) node {{\Large $\frac{|DH|}{|HE|}=\frac{|CG|}{|GB|}$}};
\draw[color=black] (0.2,3) node {{\Large $\frac{|DH|}{|CG|}=\frac{|HE|}{|GB|}$}};
\end{tikzpicture}

Demostración: Ejercicio.
\newpage

\begin{ejercicio}
[Caracterización de un paralelogramo] Sea ABCD un cuadrilátero. Demuestra que los siguientes enunciados son equivalentes (todos signfican que el cuadrilátero es un paralelgramo) 
\begin{enumerate}
    \item $|AB|\parallel |CD|$ y $|BC|\parallel |DA|$ (dos pares de lados opuestos son paralelos).
    \item $|AB|= |CD|$ y $|BC|= |DA|$ (dos pares de lados opuestos miden lo mismo).
    \item $\angle ABC= \angle BCD$ y $\angle BCD= \angle DAB$ (dos pares de ángulos opuestos miden lo mismo).
    \item Las diagonales $AC$ y $BD$ se intersectan en el punto medio.
    \item $|AB|\parallel |CD|$ y $|AB|= |CD|$ (dos pares de lados opuestos son paralelos y del mismo tamaño).
\end{enumerate}
\end{ejercicio}
%Sug:

\begin{problema}
En los lados $AB, BC, CD, DA$ de un cuadrilátero convexo $ABCD$ se toman los puntos medios $P,Q,R,S$.
\begin{enumerate}
    \item Demuestra que $PQRS$ es un paralelogramo.
    \item Demuestra que el área del paralelogramo $PQRS$ es la mitad del área del cuadrilátero $ABCD$.
\end{enumerate}
\end{problema}
%\vspace{5cm}
%Sug: Los lados son paralelos a las diagonales AC y BD. $$\mathrm a[ABCD]=\frac{1}{2}(\mathrm a[ABC]+\mathrm a[BCD]+\mathrm a[CDA]+\mathrm a[DAB]).$$ Por otro lado $\mathrm a[PQRS]=\mathrm a[ABCD]-\text{¿qué?}$

\begin{tikzpicture}[line cap=round,line join=round,>=triangle 45,x=4.0cm,y=4.0cm]
\draw [line width=1.pt] (0.,0.)-- (0.5037037037037022,0.44);
\draw [line width=1.pt] (0.5037037037037022,0.44)-- (1.,0.);
\draw [line width=1.pt] (1.,0.)-- (0.32037037037036886,-0.56);
\draw [line width=1.pt] (0.32037037037036886,-0.56)-- (0.,0.);
\draw [line width=1.pt] (0.2518518518518511,0.22)-- (0.7518518518518511,0.22);
\draw [line width=1.pt] (0.7518518518518511,0.22)-- (0.6601851851851844,-0.28);
\draw [line width=1.pt] (0.6601851851851844,-0.28)-- (0.16018518518518443,-0.28);
\draw [line width=1.pt] (0.16018518518518443,-0.28)-- (0.2518518518518511,0.22);
\draw[color=black] (-0.07962962962963109,0.0205555555555558) node {$A$};
\draw[color=black] (1.053703703703702,0.026111111111111356) node {$C$};
\draw[color=black] (0.4925925925925911,0.5083333333333333) node {$B$};
\draw[color=black] (0.3148148148148133,-0.618333333333333) node {$D$};
\draw [fill=black] (0.2518518518518511,0.22) circle (1.5pt);
\draw[color=black] (0.2037037037037022,0.2983333333333335) node {$P$};
\draw [fill=black] (0.7518518518518511,0.22) circle (1.5pt);
\draw[color=black] (0.803703703703702,0.2872222222222224) node {$Q$};
\draw [fill=black] (0.6601851851851844,-0.28) circle (1.5pt);
\draw[color=black] (0.7037037037037022,-0.3072222222222219) node {$R$};
\draw [fill=black] (0.16018518518518443,-0.28) circle (1.5pt);
\draw[color=black] (0.0925925925925911,-0.29055555555555523) node {$S$};
\end{tikzpicture}

\newpage

\section{Pitágoras y el cálculo de distancias}

\begin{ejercicio}
Sea $ABC$ un triángulo rectángulo. Demuestra que la altura desde el ángulo recto divide al triángulo en dos triángulos semejantes al original.
\end{ejercicio}
%\vspace{3cm}

\begin{ejercicio}
Enuncia y demuestra el Teorema de Pitágoras (puedes ayudarte de la figura)
\end{ejercicio}
%Sug: Puedes apoyarte de este dibujo %bt4naud4

\begin{figure}
    \begin{subfigure}{.4\textwidth}
\begin{tikzpicture}[line cap=round,line join=round,>=triangle 45,x=4.0cm,y=4.0cm]
\draw [line width=1.pt] (0.,0.)-- (1.,0.);
\draw [line width=1.pt] (0.,0.)-- (0.31697534608537803,0.46529772840562295);
\draw [line width=1.pt] (0.31697534608537803,0.46529772840562295)-- (1.,0.);
\draw [dashed, line width=1.pt] (0.31697534608537803,0.46529772840562295)-- (0.31697534608537803,0.);
\draw[color=black] (-0.046296296296297765,-0.012777777777777527) node {$B$};
\draw[color=black] (1.053703703703702,-0.00722222222222197) node {$C$};
\draw[color=black] (0.28703703703703554,0.5483333333333333) node {$A$};
\end{tikzpicture}

\vspace{1cm}

\begin{tikzpicture}[line cap=round,line join=round,>=triangle 45,x=2cm,y=2cm]
\draw [line width=1pt] (-1,-1)-- (0.22666666666666835,-1);
\draw [line width=1pt] (1,-1)-- (1,0.2266666666666683);
\draw [line width=1pt] (1,1)-- (-0.22666666666666824,1);
\draw [line width=1pt] (-1,1)-- (-1,-0.22666666666666815);
\draw [line width=1pt] (-1,-0.22666666666666815)-- (-1,-1);
\draw [line width=1pt] (0.22666666666666835,-1)-- (1,-1);
\draw [line width=1pt] (1,0.2266666666666683)-- (1,1);
\draw [line width=1pt] (-0.22666666666666824,1)-- (-1,1);
\draw [line width=1pt] (-0.22666666666666824,1)-- (-1,-0.22666666666666815);
\draw [line width=1pt] (-1,-0.22666666666666815)-- (0.22666666666666835,-1);
\draw [line width=1pt] (0.22666666666666835,-1)-- (1,0.2266666666666683);
\draw [line width=1pt] (1,0.2266666666666683)-- (-0.22666666666666824,1);
\draw[color=black] (-0.36,-1.1811111111111103) node {$a$};
\draw[color=black] (1.1244444444444466,-0.4788888888888901) node {$a$};
\draw[color=black] (0.5111111111111126,1.076666666666661) node {$a$};
\draw[color=black] (-1.097777777777778,0.4277777777777741) node {$a$};
\draw[color=black] (-1.08,-0.6744444444444451) node {$b$};
\draw[color=black] (0.68,-1.1633333333333327) node {$b$};
\draw[color=black] (1.1688888888888909,0.6233333333333291) node {$b$};
\draw[color=black] (-0.6355555555555551,1.076666666666661) node {$b$};
\draw[color=black] (-0.4577777777777772,0.28555555555555223) node {$c$};
\draw[color=black] (-0.28,-0.5144444444444456) node {$c$};
\draw[color=black] (0.4666666666666681,-0.39) node {$c$};
\draw[color=black] (0.34222222222222354,0.4722222222222184) node {$c$};
\end{tikzpicture}
    \caption{~}
\end{subfigure}
    \begin{subfigure}{.5\textwidth}
\begin{tikzpicture}[line cap=round,line join=round,>=triangle 45,x=1.0cm,y=1.0cm]
\begin{axis}[
x=1.0cm,y=1.0cm,
axis lines=middle,
ymajorgrids=true,
xmajorgrids=true,
xmin=-1.260000000000003,
xmax=5.740000000000006,
ymin=-1.5399999999999914,
ymax=8.100000000000003,
xtick={-8.0,-7.0,...,14.0},
ytick={-1.0,0.0,...,10.0},]
\draw [fill=black] (1.,3.) circle (1pt);
\draw[color=black] (0.74,2.67) node {$P$};
\draw [fill=black] (3.,7.) circle (1pt);
\draw[color=black] (3.14,7.37) node {$Q$};
\draw [fill=black] (0.,6.) circle (1pt);
\draw[color=black] (0.14,6.33) node {$R$};

\draw [line width=1pt] (1,3)-- (3,7);

%\draw [dashed, line width=1pt] (-1,6.5)-- (5,3.5);
\end{axis}
\end{tikzpicture}    
    \caption{~}
    \end{subfigure}
    \end{figure}

El teorema de Pitágoras nos permite calcular {\bf distancias} en el plano. La distancia entre dos puntos $P=(x_1,y_1), Q=(x_2,y_2)$ está dada por 
$$\mathrm d(P,Q):=\sqrt{(x_2-x_1)^2+(y_2-y_1)^2}.$$

\begin{ejercicio}
Sean $P=(1,3)$, $Q=(3,7)$, $R=(0,6)$ tres puntos en el plano. Calcula las distancia entre $P$, $Q$ y $R$.
\end{ejercicio}

\begin{ejercicio}\label{P:LugGeoMed}
El punto $R$ del ejercicio anterior está a la \emph{misma distancia} del punto $P$ y el punto $Q$. Encuentra {\bf todos} los puntos en el plano que están a la misma distancia de $P$ y de $Q$.
\end{ejercicio}

\begin{ejercicio}
Construye un triángulo equilátero con regla (sin graduar) y un compás.
\end{ejercicio}

\begin{ejercicio}
Sea $l$ una línea y $P$ que no se encuentra en $l$. Construye usando regla (sin graduar) y un compás la recta perpendicular a $l$ que pasa por $P$
\end{ejercicio}
%\vspace{3cm}
%Sug:

\newpage 
La distancia entre una recta $l$ y un punto $P$ se define como la mínima de las distancias de $P$ a otro punto $Q$ en la recta $l$: $$\mathrm d(P,l):=\min_{Q\in l} |PQ|.$$

\begin{tikzpicture}[line cap=round,line join=round,>=triangle 45,x=1.0cm,y=1.0cm]
\draw [line width=1.pt] (0.,0.)-- (5.46,2.74);
\draw[color=black] (5.6,3.11) node {$l$};
\draw [fill=black] (3.04,-0.32) circle (1.0pt);
\draw[color=black] (3.18,0.05) node {$P$};

\draw [dashed, line width=1.pt] (3.04,-0.32)-- (2.300154772878304,1.1542901241184165);
\end{tikzpicture}
\hspace{1.5cm}
\begin{tikzpicture}[line cap=round,line join=round,>=triangle 45,x=3.0cm,y=3.0cm]
\draw [line width=1.pt] (-0.8518518518518546,0.14962962962966514)-- (0.7007407407407388,-0.52);
\draw [line width=1.pt] (-0.5792592592592619,-0.5911111111110763)-- (0.7896296296296277,0.4992592592592951);
%\draw [dashed, line width=1.pt] (-0.9420181123163018,-0.3030243517345032)--(1.031738133035407,-0.03952553703119757);
%\draw [dashed, line width=1.pt] (-0.18183600689444354,0.6379804703164009)-- (0.016319659468979272,-0.8463183660067091);
\end{tikzpicture}

\begin{ejercicio}
Sea $P$ un punto y $l$ una recta que no pasa por $P$. Muestra que el punto de $l$ más cercano a $P$ es es el punto $Q$ en $l$ tal que el segmento $PQ$ es perpendicular a $l$.
\end{ejercicio}


\begin{ejercicio}
Considera dos lineas no paralelas. ¿Cuáles son los puntos que equidistan de las dos rectas?

¿Qué pasa si las rectas son paralelas?
\end{ejercicio}

\begin{problema} %Shariguin
Considera un triángulo equilátero y un punto $P$ en su interior. Demuestra que la suma de las distancias desde $P$ a los lados del triángulo es igual a la altura del triángulo.
\end{problema}
%Sug: Primero verificar punto en la base, trazar una paralela a un lado. Para un punto en el interior trazar una paralela a la base.

\begin{tikzpicture}[line cap=round,line join=round,>=triangle 45,x=4.0cm,y=4.0cm]
\draw [line width=1.pt] (0.,0.)-- (1.,0.);
\draw [line width=1.pt] (1.,0.)-- (0.5,0.8660254037844388);
\draw [line width=1.pt] (0.5,0.8660254037844388)-- (0.,0.);
\draw [dashed, line width=1.pt] (0.802731149212046,0.3416796723154605)-- (0.5592592592592577,0.2011111111111113);
\draw [dashed, line width=1.pt] (0.22689848041758312,0.39299969624342584)-- (0.5592592592592577,0.2011111111111113);
\draw [dashed, line width=1.pt] (0.5592592592592577,0.)-- (0.5592592592592577,0.2011111111111113);
\draw[color=black] (-0.046296296296297765,-0.012777777777777527) node {$B$};
\draw[color=black] (1.053703703703702,-0.00722222222222197) node {$C$};
\draw[color=black] (0.4981481481481466,0.9594444444444444) node {$A$};
\draw [fill=black] (0.5592592592592577,0.2011111111111113) circle (1.5pt);
\draw[color=black] (0.5592592592592578,0.30944444444444463) node {$P$};
\end{tikzpicture}
\newpage

\section{Triángulos, sus círculos y cevianas}

Sea ABC un triágulo. Para cada ángulo del triángulo se consideran las siguientes rectas notables (en este caso desde el vércice $A$):

\begin{definicion}  
\begin{enumerate}
    \item La {\bf altura} es la perpendicular a $BC$ desde $A$. A su intersección $D$ con la base se le llama el {\bf pie de altura}. Nótese que a veces el pie de altura se encuentra fuera del segmento.
    \item La {\bf bisectriz} (interior) es la recta que divide al ángulo $\angle BAC$ en dos ángulos iguales. 
    \item La {\bf medatriz} es la recta perpendicular al $BC$ por el punto medio $M$. 
    \item La {\bf mediana} es el segmento que une al vértice $A$ con $M$.
\end{enumerate}
\end{definicion}

\begin{tikzpicture}[line cap=round,line join=round,>=triangle 45,x=5.0cm,y=5.0cm]
\draw [line width=1.pt] (0.,0.)-- (1.,0.);
\draw [line width=1.pt] (0.,0.)-- (0.1785733882030177,0.666172839506168);
\draw [line width=1.pt] (0.1785733882030177,0.666172839506168)-- (1.,0.);
\draw [line width=1.pt] (0.1785733882030177,0.666172839506168)-- (0.5,0.);
\draw [line width=1.pt] (0.5,0.7767901234567852)-- (0.5,-0.18320987654321455);
\draw [line width=1.pt] (0.1785733882030177,0.820246913580242)-- (0.1785733882030177,-0.17530864197531332);
\draw [line width=1.pt] (0.1785733882030177,0.666172839506168)-- (0.5,0.);
\draw [line width=1.pt] (0.12845239890423074,0.8206482811827577)-- (0.4538779828195871,-0.18232994234172117);
\draw[color=black] (-0.05846364883401933,-0.007407407407412123) node {$B$};
\draw[color=black] (1.0358573388203016,-0.019259259259263972) node {$C$};
\draw[color=black] (0.12721536351165968,0.7274074074074025) node {$A$};
\draw [fill=black] (0.5,0.) circle (1.0pt);
\draw[color=black] (0.5499314128943757,-0.027160493827165205) node {$M$};
\draw [fill=black] (0.1785733882030177,0.) circle (1.0pt);
\draw[color=black] (0.21412894375857328,-0.031111111111115815) node {$D$};
\draw [fill=black] (0.39471934252198965,0.) circle (1.0pt);
\draw[color=black] (0.37610425240054857,-0.031111111111115815) node {$E$};
\end{tikzpicture}
\hspace{1cm}
\begin{tikzpicture}[line cap=round,line join=round,>=triangle 45,x=5.0cm,y=5.0cm]
\draw [line width=1.pt] (0.,0.)-- (1.,0.);
\draw [line width=1.pt] (0.,0.)-- (-0.2955006858710564,0.4449382716049335);
\draw [line width=1.pt] (-0.2955006858710564,0.4449382716049335)-- (1.,0.);
\draw [line width=1.pt] (-0.2955006858710564,0.4449382716049335)-- (0.5,0.);
\draw [line width=1.pt] (0.5,0.4567901234567853)-- (0.5,-0.11604938271605407);
\draw [line width=1.pt] (-0.2955006858710564,0.5990123456790075)-- (-0.2955006858710564,-0.18320987654321455);
\draw [line width=1.pt] (-0.2955006858710564,0.4449382716049335)-- (0.5,0.);
\draw [line width=1.pt] (-0.4531606560454268,0.5667155129547207)-- (0.4278000217141357,-0.11374231636331512);
\draw [line width=1.pt] (1.,0.)-- (-0.5167352537722909,0.);
\draw[color=black] (-0.03475994513031562,-0.03506172839506643) node {$B$};
\draw[color=black] (1.0358573388203016,-0.019259259259263972) node {$C$};
\draw[color=black] (-0.2401920438957477,0.506172839506168) node {$A$};
\draw [fill=black] (0.5,0.) circle (1.0pt);
\draw[color=black] (0.5499314128943757,-0.027160493827165205) node {$M$};
\draw [fill=black] (0.2805425374203728,0.) circle (1.0pt);
\draw[color=black] (0.2615363511659807,-0.031111111111115815) node {$E$};
\draw [fill=black] (-0.2955006858710564,0.) circle (1.0pt);
\draw[color=black] (-0.3468587105624144,-0.019259259259263972) node {$D$};
\end{tikzpicture}

\begin{tikzpicture}[line cap=round,line join=round,>=triangle 45,x=5.0cm,y=5.0cm]
\draw [line width=1.pt] (0.,0.)-- (1.,0.);
\draw [line width=1.pt] (0.,0.)-- (0.5,0.4988);
\draw [line width=1.pt] (0.5,0.4988)-- (1.,0.);
\draw [line width=1.pt] (0.5,0.4988)-- (0.5,0.);
\draw [line width=1.pt] (0.5,0.4988)-- (0.5,0.);
\draw[color=black] (-0.03475994513031562,-0.03506172839506643) node {$B$};
\draw[color=black] (1.0358573388203016,-0.019259259259263972) node {$C$};
\draw[color=black] (0.5538820301783264,0.5614814814814766) node {$A$};
\draw [fill=black] (0.5,0.) circle (1.0pt);
\draw[color=black] (0.5499314128943757,-0.047160493827165205) node {$M=D=E$};
\end{tikzpicture}
\hspace{1cm}
\begin{tikzpicture}[line cap=round,line join=round,>=triangle 45,x=5.0cm,y=5.0cm]
\draw [line width=1.pt] (0.,0.)-- (1.,0.);
\draw [line width=1.pt] (0.,0.)-- (0.3326474622770918,0.46864197530863716);
\draw [line width=1.pt] (0.3326474622770918,0.46864197530863716)-- (1.,0.);
\draw [line width=1.pt] (0.3326474622770918,0.46864197530863716)-- (0.5,0.);
\draw [line width=1.pt] (0.5,0.4567901234567853)-- (0.5,-0.11604938271605407);
\draw [line width=1.pt] (0.3326474622770918,0.6227160493827112)-- (0.3326474622770918,-0.15950617283951085);
\draw [line width=1.pt] (0.3326474622770918,0.46864197530863716)-- (0.5,0.);
\draw [line width=1.pt] (0.3069321817729737,0.6178720267636074)-- (0.44016077436864154,-0.15527566358108486);
\draw[color=black] (-0.03475994513031562,-0.03506172839506643) node {$B$};
\draw[color=black] (1.0358573388203016,-0.019259259259263972) node {$C$};
\draw[color=black] (0.3879561042524004,0.5298765432098717) node {$A$};
\draw [fill=black] (0.5,0.) circle (1.0pt);
\draw[color=black] (0.538079561042524,-0.02320987654321459) node {$M$};
\draw [fill=black] (0.3326474622770918,0.) circle (1.0pt);
\draw[color=black] (0.2931412894375856,-0.02320987654321459) node {$D$};
\draw [fill=black] (0.4134037156829415,0.) circle (1.0pt);
\draw[color=black] (0.38400548696844977,-0.027160493827165205) node {$E$};
\end{tikzpicture}

\begin{ejercicio}
Demuestra que dos medianas se intersectan en proporción $2:1$

Concluye que las tres medianas de un triángulo se intersectan en un mismo punto.
\end{ejercicio}

\begin{tikzpicture}[line cap=round,line join=round,>=triangle 45,x=4.0cm,y=4.0cm]
\draw [line width=1.pt] (0.,0.)-- (1.,0.);
\draw [line width=1.pt] (0.,0.)-- (0.32037037037036886,0.6566666666666667);
\draw [line width=1.pt] (0.32037037037036886,0.6566666666666667)-- (1.,0.);
\draw [line width=1.pt] (0.32037037037036886,0.6566666666666667)-- (0.5,0.);
\draw [line width=1.pt] (1.,0.)-- (0.16018518518518443,0.32833333333333337);
\draw[color=black] (-0.046296296296297765,-0.012777777777777527) node {$A$};
\draw[color=black] (1.053703703703702,-0.00722222222222197) node {$B$};
\draw[color=black] (0.3592592592592577,0.7594444444444445) node {$C$};
\draw[color=black] (0.4870370370370355,-0.06277777777777752) node {$D$};
\draw[color=black] (0.09814814814814667,0.3761111111111113) node {$E$};
\draw[color=black] (0.48148148148147996,0.30944444444444463) node {$G$};
\end{tikzpicture}
\newpage 

\subsection*{Mediatrices}

\begin{ejercicio}
Construye la mediatriz de un segmento $AB$, utilizando regla (sin graduación) y un compás.
\end{ejercicio}
%Sug: Se abre el compás a una distancia $r>\frac{1}{2}|AB|$. Se dibujan dos círculos de radio r, con centros en A y en B. Las intersecciones de los círculos están en la mediatriz y por tanto la determinan.


\begin{ejercicio}\label{P:LugGeoMed3}
Sean $P$, $Q$ y $R$ tres puntos en el plano. Encuentra todos los puntos que equidistan de $P$, $Q$ y  $R$.
\end{ejercicio}

\begin{ejercicio}
Demuestra que las tres mediatrices de un triángulo $ABC$ concurren en un mismo punto $O$. 

Muestra que $O$ es el centro del círculo que pasa por los puntos $A,B,C$. 

¿Para qué tipo de triángulos $ABC$ se encuentra $O$ en el interior del triángulo?
\end{ejercicio}

\begin{tikzpicture}[line cap=round,line join=round,>=triangle 45,x=2.5cm,y=2.5cm]
\draw [line width=1.pt] (0.,0.) circle (2.5cm);
\draw [line width=1pt] (-0.8944271909999159,-0.4472135954999581)-- (1.,0.);
\draw [line width=1pt] (1.,0.)-- (-0.42256246490594884,0.9063338034370166);
\draw [line width=1pt] (-0.42256246490594884,0.9063338034370166)-- (-0.8944271909999159,-0.4472135954999581);
\draw [line width=1pt] (0.05278640450004207,-0.22360679774997905)-- (0.,0.);
\draw [line width=1pt] (0.,0.)-- (-0.8944271909999159,-0.4472135954999581);
\draw [line width=1pt] (0.,0.)-- (1.,0.);
\draw [line width=1pt] (0.2887187675470256,0.4531669017185083)-- (0.,0.);
\draw [line width=1pt] (0.,0.)-- (-0.42256246490594884,0.9063338034370166);
\draw [line width=1pt] (0.,0.)-- (-0.6584948279529323,0.22956010396852927);
\draw[color=black] (-0.05,-.1) node {$O$};
\end{tikzpicture}
\hspace{1cm}
\begin{tikzpicture}[line cap=round,line join=round,>=triangle 45,x=2.5cm,y=2.5cm]
\draw [line width=1.pt] (0.,0.) circle (2.5cm);
\draw [line width=1pt] (-0.9258815429358703,0.37781393363756743)-- (1.,0.);
\draw [line width=1pt] (1.,0.)-- (-0.222804702522627,0.9748631004063102);
\draw [line width=1pt] (-0.222804702522627,0.9748631004063102)-- (-0.9258815429358703,0.37781393363756743);
\draw [line width=1pt] (0.037059228532064836,0.18890696681878372)-- (0.,0.);
\draw [line width=1pt] (0.,0.)-- (-0.9258815429358703,0.37781393363756743);
\draw [line width=1pt] (0.,0.)-- (1.,0.);
\draw [line width=1pt] (0.3885976487386865,0.4874315502031551)-- (0.,0.);
\draw [line width=1pt] (0.,0.)-- (-0.222804702522627,0.9748631004063102);
\draw [line width=1pt] (0.,0.)-- (-0.5743431227292487,0.6763385170219388);
\draw[color=black] (0,-.1) node {$O$};
\end{tikzpicture}

Al círculo y su centro $O$ del ejercicio anterior se les llama, respectivamente, el {\bf circuncírculo} y {\bf circuncentro} de $ABC$.

\begin{ejercicio}
Sea $ABC$ un triángulo y $O$ su circuncentro. Muestre que $\measuredangle AOC = 2\measuredangle ABC$.

Deja fijo al circuncírculo de $ABC$ y al lado $BC$. Si colocas otro punto $A'$ distinto en la circunferencia. ¿Cuál es la relación entre el ángulo $\measuredangle BA'C$ y el $\measuredangle BAC$?
\end{ejercicio}
Obs: Hay dos respuestas diferentes, dependiendo de \emph{donde en el círculo} se ubique $A'$

\begin{tikzpicture}[line cap=round,line join=round,>=triangle 45,x=3.0cm,y=3.0cm]
\draw [line width=1.pt] (0.,0.) circle (3.cm);
\draw [line width=1.pt] (-0.7730733704825902,-0.634316611678023)-- (-0.6043864813742317,0.7966912709023964);
\draw [line width=1.pt] (-0.6043864813742317,0.7966912709023964)-- (1.,0.);
\draw [line width=1.pt] (-0.7730733704825902,-0.634316611678023)-- (1.,0.);
\draw [line width=1.pt] (-0.7730733704825902,-0.634316611678023)-- (0.,0.);
\draw [line width=1.pt] (0.,0.)-- (1.,0.);

\draw[color=black] (1.0671604938271592,0.03259259259259451) node {$A$};
\draw[color=black] (-0.6267901234567913,0.870617283950618) node {$B$};
\draw[color=black] (-0.8166666666666678,-0.695061728395033) node {$C$};
\draw[color=black] (0,0.1) node {$O$};

%\draw[color=black] (0.2804938271604927,1.0286419753086424) node {$E$};
%\draw[color=black] (0.71111111111111,-0.654320987654318) node {$F$};
%\draw[color=black] (-0.031604938271606056,-0.583209876543207) node {$n$};
%\draw[color=black] (0.9086419753086409,-0.35407407407407154) node {$p$};

%\draw [line width=1.pt] (0.6846193950183727,-0.7289007367019719)-- (1.,0.);
%\draw [line width=1.pt] (-0.7730733704825902,-0.634316611678023)-- (0.2510831802060881,0.967965514167523);
%\draw [line width=1.pt] (0.2510831802060881,0.967965514167523)-- (1.,0.);
%\draw [line width=1.pt] (-0.6043864813742317,0.7966912709023964)-- (0.,0.);
%\draw [line width=1.pt] (0.6846193950183727,-0.7289007367019719)-- (-0.7730733704825902,-0.634316611678023);
\end{tikzpicture}

\begin{ejercicio}
¿Para cuáles triángulos se encuentra el circuncentro en el interior del triángulo?
\end{ejercicio}

\newpage

\subsection*{Cuadriláteros cíclicos}

Caracterización de cuadriláteros cíclicos.

1. Un par de ángulos opuestos suman $180^{\circ}$.

\begin{tikzpicture}[line cap=round,line join=round,>=triangle 45,x=3.0cm,y=3.0cm]
\draw [line width=1.pt] (0.,0.) circle (3.cm);
\draw [line width=1.pt] (-0.8024623409461336,-0.5967027663445607)-- (1.,0.);
\draw [line width=1.pt] (1.,0.)-- (0.,1.);
\draw [line width=1.pt] (0.,1.)-- (-1.,0.);
\draw [line width=1.pt] (-1.,0.)-- (-0.8024623409461336,-0.5967027663445607);
\draw[color=black] (1.1111111111111118,0.07333333333333297) node {$B$};
\draw[color=black] (0.002962962962962966,1.163703703703703) node {$C$};
\draw[color=black] (-1.1466666666666672,0.0318518518518515) node {$D$};
\draw[color=black] (-0.9392592592592598,-0.7088888888888889) node {$E$};
\end{tikzpicture}
\hspace{1cm}
\begin{tikzpicture}[line cap=round,line join=round,>=triangle 45,x=3.0cm,y=3.0cm]
\draw [line width=1.pt] (0.,0.) circle (3.cm);
\draw [line width=1.pt] (-0.7240976963409272,0.6896974163745739)-- (1.,0.);
\draw [line width=1.pt] (1.,0.)-- (0.5014412639610989,0.8651916890476327);
\draw [line width=1.pt] (0.5014412639610989,0.8651916890476327)-- (-0.17755861123271632,0.9841102273511383);
\draw [line width=1.pt] (-0.17755861123271632,0.9841102273511383)-- (-0.7240976963409272,0.6896974163745739);
\draw[color=black] (1.1111111111111118,-0.009629629629629967) node {$B$};
\draw[color=black] (0.5777777777777781,1.009629629629629) node {$C$};
\draw[color=black] (-0.21037037037037049,1.1340740740740733) node {$D$};
\draw[color=black] (-0.8977777777777782,0.7666666666666662) node {$E$};
\end{tikzpicture}

2. Un par de ángulos inscritos son iguales.

\begin{tikzpicture}[line cap=round,line join=round,>=triangle 45,x=3.0cm,y=3.0cm]
\clip(-1.92888888888889,-1.4585185185185183) rectangle (2.5451851851851863,1.8955555555555545);
\draw [line width=1.pt] (0.,0.) circle (3.cm);
\draw [line width=1.pt] (-0.3549961771582605,-0.9348677522532377)-- (1.,0.);
\draw [line width=1.pt] (1.,0.)-- (0.5014412639610989,0.8651916890476327);
\draw [line width=1.pt] (0.5014412639610989,0.8651916890476327)-- (-0.17755861123271655,0.9841102273511383);
\draw [line width=1.pt] (-0.17755861123271655,0.9841102273511383)-- (-0.3549961771582605,-0.9348677522532377);
\draw [line width=1.pt] (-0.17755861123271655,0.9841102273511383)-- (1.,0.);
\draw [line width=1.pt] (0.5014412639610989,0.8651916890476327)-- (-0.3549961771582605,-0.9348677522532377);
\draw[color=black] (1.093333333333334,0.025925925925925578) node {$B$};
\draw[color=black] (0.58962962962963,0.9622222222222216) node {$C$};
\draw[color=black] (-0.24,1.1459259259259253) node {$D$};
\draw[color=black] (-0.4533333333333336,-0.945925925925926) node {$E$};
\end{tikzpicture}


\newpage
Cuando hacemos que el punto $D$ se acerce a $C$ el segmento $CD$ se vuelve tangente a la circunferencia y obtenemos los {\bf ángulos semi-inscritos}:

\begin{tikzpicture}[line cap=round,line join=round,>=triangle 45,x=3.0cm,y=3.0cm]
\draw [line width=1.pt] (0.,0.) circle (3.cm);
\draw [line width=1.pt] (-0.3549961771582605,-0.9348677522532377)-- (1.,0.);
\draw [line width=1.pt] (1.,0.)-- (0.5014412639610989,0.8651916890476327);
\draw [line width=1.pt] (0.5014412639610989,0.8651916890476327)-- (0.3741393361546969,0.9273725018253036);
\draw [line width=1.pt] (0.3741393361546969,0.9273725018253036)-- (-0.3549961771582605,-0.9348677522532377);
\draw [line width=1.pt] (0.3741393361546969,0.9273725018253036)-- (1.,0.);
\draw [line width=1.pt] (0.5014412639610989,0.8651916890476327)-- (-0.3549961771582605,-0.9348677522532377);
\draw[color=black] (1.093333333333334,0.025925925925925578) node {$B$};
\draw[color=black] (0.6014814814814818,0.9385185185185179) node {$C$};
\draw[color=black] (0.31703703703703723,1.0688888888888883) node {$D$};
\draw[color=black] (-0.42962962962962986,-0.9992592592592593) node {$A$};
\end{tikzpicture}
\hspace{1cm}
\begin{tikzpicture}[line cap=round,line join=round,>=triangle 45,x=3.0cm,y=3.0cm]
\draw [line width=1.pt] (0.,0.) circle (3.cm);
\draw [line width=1.pt] (-0.3549961771582605,-0.9348677522532377)-- (1.,0.);
\draw [line width=1.pt] (1.,0.)-- (0.5014412639610989,0.8651916890476327);
\draw [line width=1.pt] (0.5014412639610989,0.8651916890476327)-- (0.500545798299617,0.8657100575854476);
\draw [line width=1.pt] (0.500545798299617,0.8657100575854476)-- (-0.3549961771582605,-0.9348677522532377);
\draw [line width=1.pt] (0.500545798299617,0.8657100575854476)-- (1.,0.);
\draw [line width=1.pt] (0.5014412639610989,0.8651916890476327)-- (-0.3549961771582605,-0.9348677522532377);
\draw [line width=1.pt] (1.4160229051759798,0.33575726351714175)-- (-0.376632896313735,1.3734926079254157);
\draw[color=black] (1.093333333333334,0.025925925925925578) node {$B$};
\draw[color=black] (0.6014814814814818,0.9385185185185179) node {$D=C$};
\draw[color=black] (-0.42962962962962986,-0.9992592592592593) node {$A$};
\end{tikzpicture}

\begin{ejercicio}[Potencia de un punto a una circunferencia]
En la siguiente figura, demuestra que $|PQ|\cdot|PR|=|PS|\cdot|PT|$
\end{ejercicio}

\begin{tikzpicture}[line cap=round,line join=round,>=triangle 45,x=3.0cm,y=3.0cm]
\draw [line width=1.pt] (0.,0.) circle (3.cm);
\draw [line width=1.pt] (1.7807407407407416,1.1844444444444437)-- (-1.4548148148148157,-0.3740740740740743);
\draw [line width=1.pt] (1.7807407407407416,1.1844444444444437)-- (-0.16296296296296306,-1.174074074074074);
\draw[color=black] (1.8696296296296306,1.246666666666666) node {$P$};
\draw[color=black] (-1.075555555555556,-0.05111111111111143) node {$R$};
\draw[color=black] (0.008888888888888894,-1.0762962962962963) node {$T$};
\draw[color=black] (0.7259259259259263,0.7962962962962957) node {$Q$};
\draw[color=black] (1.0577777777777784,0.20962962962962922) node {$S$};
\end{tikzpicture}

\newpage
\begin{ejercicio}
En la siguiente figura, demuestra que $|PQ|\cdot|PR|=|PS|^2$
\end{ejercicio}

\begin{tikzpicture}[line cap=round,line join=round,>=triangle 45,x=3.0cm,y=3.0cm]
\draw [line width=1.pt] (0.,0.) circle (3.cm);
\draw [line width=1.pt] (1.7807407407407416,1.1844444444444437)-- (-1.4548148148148157,-0.3740740740740743);
\draw [line width=1.pt] (1.7807407407407416,1.1844444444444437)-- (0.45883752636980857,-1.2508821931279661);
\draw[color=black] (1.8696296296296306,1.246666666666666) node {$P$};
\draw[color=black] (-1.075555555555556,-0.05111111111111143) node {$Q$};
\draw[color=black] (0.7259259259259263,0.7962962962962957) node {$R$};
\draw[color=black] (0.9629629629629635,-0.5548148148148149) node {$S$};
\end{tikzpicture}

\begin{ejercicio}[Potencia de un punto interior]
En la siguiente figura, demuestra que $|PQ|\cdot|PR|=|PS|\cdot|PT|$
\end{ejercicio}

\begin{tikzpicture}[line cap=round,line join=round,>=triangle 45,x=3.0cm,y=3.0cm]
\draw [line width=1.pt] (0.,0.) circle (3.cm);
\draw [line width=1.pt] (-0.8743319740778694,0.4853283415432235)-- (0.6869537824455602,-0.7267011082857505);
\draw [line width=1.pt] (1.,0.)-- (-0.8232492334325737,-0.56768010327356);
\draw[color=black] (1.093333333333334,0.025925925925925578) node {$S$};
\draw[color=black] (-0.9985185185185191,0.6007407407407402) node {$Q$};
\draw[color=black] (-0.921481481481482,-0.6140740740740741) node {$T$};
\draw[color=black] (0.7911111111111115,-0.7681481481481481) node {$R$};
\draw[color=black] (0.0918518518518519,-0.4125925925925928) node {$P$};
\end{tikzpicture}

\begin{ejercicio}
Demuestra que las tres alturas de un triángulo concurren en un punto.
\end{ejercicio}
%Sug: Cíclicos.

\begin{ejercicio}
Sean $A$ y $B$ dos puntos en el plano y $P$ y $Q$ otros dos puntos en el plano. Demuestra que $AP^2-BP^2 = AQ^2-BQ^2$ si y solo si $AB$ es perpendicular a $PQ$. 
\end{ejercicio}

\subsection*{Bisectrices}

Además de las mediatrices, hay otras ternas de líneas asociadas a triángulos que concurren en un solo punto.

La bisectriz de un ángulo $\angle ABC$ es la semi-recta que parte al ángulo en dos ángulos iguales.

Obs: La continuación de la bisectriz de $\angle ABC$ es la bisectriz del ángulo complementario $\angle CBA$.

\begin{ejercicio}
Construye la bisectriz de un ángulo $ABC$, utilizando regla (sin graduación) y un compás.
\end{ejercicio}
%Sug. La bisectriz es la mediatriz dede un triángulo isósceles.
\vspace{3cm}

Todos los puntos en la bisectriz de un ángulo se encuentran a la misma distancia de las dos rectas. 

\begin{ejercicio}
Demuestra que las tres bisectrices de un triángulo $ABC$ concurren en un punto $I$.

\begin{tikzpicture}[line cap=round,line join=round,>=triangle 45,x=3.0cm,y=3.0cm]
\draw [line width=1.pt] (-0.6666666666666671,0.005185185185184847)-- (1.1881481481481488,0.05259259259259224);
\draw [line width=1.pt] (-0.6666666666666671,0.005185185185184847)-- (-0.062222222222222255,1.2733333333333325);
\draw [line width=1.pt] (-0.062222222222222255,1.2733333333333325)-- (1.1881481481481488,0.05259259259259224);
\draw [line width=1.pt] (-0.6666666666666671,0.005185185185184847)-- (0.47655820677936916,0.7473202130758552);
%\draw [line width=1.pt] (1.1881481481481488,0.05259259259259224)-- (-0.35538876925859014,0.6582584209236986);
\draw [line width=1.pt] (-0.062222222222222255,1.2733333333333325)-- (0.15993791218778133,0.026312458766129195);
\draw[color=black] (-0.7733333333333338,-0.0688888888888892) node {$A$};
\draw[color=black] (-0.08592592592592596,1.4125925925925917) node {$B$};
\draw[color=black] (1.2829629629629637,0.02) node {$C$};
\draw[color=black] (0.1570370370370371,0.41111111111111065) node {$I$};
\end{tikzpicture}

Demuestra que $I$ es el centro del círculo inscrito (tangente a los tres segmentos $AB$, $BC$ y $CA$).
\end{ejercicio}

\begin{ejercicio}
Muestra que el área del triángulo $ABC$ es $sr$, donde $s=\frac{1}{2}(a+b+c)$ es el {\bf semiperímetro} y $r$ es el inradio.
\end{ejercicio}

\begin{ejercicio}
Mostrar que en el siguiente dibujo el incentro de ABC se encuentra en el arco de la circunferencia tangente por $B$ y $C$. 
\end{ejercicio}

\begin{tikzpicture}[line cap=round,line join=round,>=triangle 45,x=2.5cm,y=2.5cm]
\draw [line width=1.pt] (0.,0.) circle (2.5cm);
\draw [line width=1.pt] (1.9414814814814823,1.1395061728395062)-- (-0.0703992984840427,0.9975188914366258);
\draw [line width=1.pt] (-0.0703992984840427,0.9975188914366258)-- (0.8365984515330039,-0.5478166033378141);
\draw [line width=1.pt] (0.8365984515330039,-0.5478166033378141)-- (1.9414814814814823,1.1395061728395062);
\draw [line width=1.pt] (1.9414814814814823,1.1395061728395062)-- (-1.1801528483527446,0.9191986991358906);
\draw [line width=1.pt] (1.9414814814814823,1.1395061728395062)-- (0.32084571431010445,-1.3354487026674366);

\draw[color=black] (2.036296296296297,1.231358024691358) node {$A$};
\draw[color=black] (-0.132592592592593,1.1483950617283951) node {$B$};
\draw[color=black] (0.8985185185185186,-0.6412345679012339) node {$C$};
\draw [fill=black] (0.8624272336757908,0.5061810610980253) circle (1.5pt);
\draw[color=black] (0.9044444444444446,0.6032098765432101) node {$I$};

%Sug
%\draw [line width=1.pt] (1.9414814814814823,1.1395061728395062)-- (0.8624272336757908,0.5061810610980253);
%\draw [line width=1.pt] (0.8624272336757908,0.5061810610980253)-- (-0.0703992984840427,0.9975188914366258);
\end{tikzpicture}
\newpage


\begin{ejercicio}
Demuestra que las bisectrices exteriores a $B$ y a $C$ de un triángulo $ABC$ y la prolongación de la bisectriz interior en $A$ se concurren en un punto. 

\begin{tikzpicture}[line cap=round,line join=round,>=triangle 45,x=2.0cm,y=2.0cm]
\draw [line width=1.pt] (-0.6666666666666671,0.005185185185184847)-- (1.1881481481481488,0.05259259259259224);
\draw [line width=1.pt] (-0.6666666666666671,0.005185185185184847)-- (-0.062222222222222255,1.2733333333333325);
\draw [line width=1.pt] (-0.062222222222222255,1.2733333333333325)-- (1.1881481481481488,0.05259259259259224);
\draw [line width=1.pt] (1.797130661258658,1.6045820677645852) circle (2*1.5359228243041165cm);
\draw [line width=1.pt] (-0.6666666666666671,0.005185185185184847)-- (0.9998845393436135,3.5016749703440055);
\draw [line width=1.pt] (3.2607120882575087,0.10556547285097838)-- (-0.6666666666666671,0.005185185185184847);
\draw [line width=1.pt] (-0.062222222222222255,1.2733333333333325)-- (1.797130661258658,1.6045820677645852);
\draw [line width=1.pt] (1.1881481481481488,0.05259259259259224)-- (1.797130661258658,1.6045820677645852);
\draw [line width=1.pt] (-0.6666666666666671,0.005185185185184847)-- (1.797130661258658,1.6045820677645852);
\draw[color=black] (-0.8859259259259267,-0.08888888888888981) node {$A$};
\draw[color=black] (-0.13925925925925953,1.5111111111111086) node {$B$};
\draw[color=black] (1.2474074074074082,-0.0755555555555565) node {$C$};
\end{tikzpicture}

Ese punto es el centro del círculo (ubicado el exterior de $ABC$) que es tangente al lado $BC$ y a las prolongaciones de $AB$ y $AC$.
\end{ejercicio}

\begin{tikzpicture}[line cap=round,line join=round,>=triangle 45,x=.3*1.0cm,y=.3*1.0cm]
\clip(-18.21083333333336,-21.64583333333331) rectangle (32.41416666666669,17.639166666666657);
\draw [line width=1.pt] (3.24470263161417,1.0941423224029345) circle (.3*2.2986308406263434cm);
\draw [line width=1.pt] (-3.9980188870669684,4.827456150909049) circle (.3*5.77648087898075cm);
\draw [line width=1.pt] (12.571517704455092,7.434090396204657) circle (.3*8.960739345557396cm);
\draw [line width=1.pt] (4.661446593065524,-7.911643952542564) circle (.3*6.6521265489400045cm);
\draw [line width=1.pt] (-12.626391107896858,-0.6507529239794501)-- (20.171535947279477,-1.7979633643853818);
\draw [line width=1.pt] (-4.612509334841702,15.12577437595197)-- (15.941712305724394,-11.694978252591575);
\draw [line width=1.pt] (-3.52837533235122,-11.085592069439267)-- (6.049052534494206,15.833959738485555);
\draw[color=black] (-0.9983333333333422,-1.5645833333333266) node {$B$};
\draw[color=black] (2.309166666666661,7.277916666666667) node {$A$};
\draw[color=black] (9.396666666666668,-1.902083333333326) node {$C$};
\draw [fill=black] (-3.9980188870669684,4.827456150909049) circle (1.pt);
\draw [fill=black] (4.661446593065524,-7.911643952542564) circle (1.pt);
\draw [fill=black] (12.571517704455092,7.434090396204657) circle (1.pt);
\draw [fill=black] (3.24470263161417,1.0941423224029345) circle (1.pt);
\draw [fill=black] (3.1643499428687836,-1.20308365404996) circle (1.pt);
\draw [fill=black] (0.5869226729370871,8.341149682874779) circle (1.pt);
\draw [fill=black] (12.25827928146155,-1.5211723754498359) circle (1.pt);
\draw [fill=black] (-1.605841381044761,-5.681870629380531) circle (1.pt);
\draw [fill=black] (9.94141066039625,-3.8653163495414415) circle (1.pt);
\draw [fill=black] (-4.199945948128214,-0.9454942912168329) circle (1.pt);
\draw [fill=black] (4.129174803881926,10.43770421428485) circle (1.pt);
\draw [fill=black] (1.4442812374372613,2.891196689976321) circle (1.pt);
\draw [fill=black] (5.459148236859981,1.9834894958046612) circle (1.pt);
\draw [fill=black] (5.069185096473359,2.4923438375286686) circle (1.pt);
\draw [fill=black] (1.0790520684912361,1.864636566329447) circle (1.pt);
\draw [fill=black] (4.893983236790102,-1.2635830072414624) circle (1.pt);
\end{tikzpicture}

\begin{tikzpicture}[line cap=round,line join=round,>=triangle 45,x=.5*1.0cm,y=.5*1.0cm]
\clip(-5.508333333333349,-17.205833333333306) rectangle (12.541666666666678,7.18416666666666);
\draw [line width=1.pt] (3.24470263161417,1.0941423224029345) circle (.5*2.2986308406263434cm);
\draw [line width=1.pt] (-3.9980188870669684,4.827456150909049) circle (.5*5.77648087898075cm);
\draw [line width=1.pt] (12.571517704455092,7.434090396204657) circle (.5*8.960739345557396cm);
\draw [line width=1.pt] (4.661446593065524,-7.911643952542564) circle (.5*6.6521265489400045cm);
\draw [line width=1.pt] (-12.626391107896858,-0.6507529239794501)-- (20.171535947279477,-1.7979633643853818);
\draw [line width=1.pt] (-4.612509334841702,15.12577437595197)-- (15.941712305724394,-11.694978252591575);
\draw [line width=1.pt] (-3.52837533235122,-11.085592069439267)-- (6.049052534494206,15.833959738485555);
\draw[color=black] (-0.653333333333341,-1.3883333333333274) node {$A$};
\draw[color=black] (2.361666666666662,6.801666666666662) node {$B$};
\draw[color=black] (8.931666666666667,-1.7033333333333274) node {$C$};
\draw [fill=black] (-3.9980188870669684,4.827456150909049) circle (1.pt);
\draw [fill=black] (4.661446593065524,-7.911643952542564) circle (1.pt);
\draw [fill=black] (12.571517704455092,7.434090396204657) circle (1.pt);
\draw [fill=black] (3.24470263161417,1.0941423224029345) circle (1.pt);
\draw[color=black] (3.666666666666663,1.7616666666666685) node {$I$};
\draw [fill=black] (3.1643499428687836,-1.20308365404996) circle (1.pt);
\draw [fill=black] (0.5869226729370871,8.341149682874779) circle (1.pt);
\draw [fill=black] (12.25827928146155,-1.5211723754498359) circle (1.pt);
\draw [fill=black] (-1.605841381044761,-5.681870629380531) circle (1.pt);
\draw [fill=black] (9.94141066039625,-3.8653163495414415) circle (1.pt);
\draw [fill=black] (-4.199945948128214,-0.9454942912168329) circle (1.pt);
\draw [fill=black] (4.129174803881926,10.43770421428485) circle (1.pt);
\draw [fill=black] (1.4442812374372613,2.891196689976321) circle (1.pt);
\draw [fill=black] (5.459148236859981,1.9834894958046612) circle (1.pt);
\draw [fill=black] (5.069185096473359,2.4923438375286686) circle (1.pt);
\draw [fill=black] (1.0790520684912361,1.864636566329447) circle (1.pt);
\draw [fill=black] (4.893983236790102,-1.2635830072414624) circle (1.pt);
\end{tikzpicture}

\begin{ejercicio}
Si $P$ y $Q$ son los puntos de tangencia del ex-círculo con las prolongaciones de $AB$ y $AC$, muestra que $AP=AQ=s$, donde $s$ es el semi-perímetro de $ABC$.
\end{ejercicio}
\newpage

\section{Problemas}

\begin{problema}
% [OMM '00]
Mostrar que $PQRS$ es un cuadrilátero cíclico.

\begin{tikzpicture}[line cap=round,line join=round,>=triangle 45,x=1.0cm,y=1.0cm]
\draw [line width=1.pt] (-0.8725925925925934,1.3844444444444421) circle (0.9261629326299876cm);
\draw [line width=1.pt] (-0.8725925925925934,1.3844444444444421)-- (1.3140740740740748,3.6111111111111067);
\draw [line width=1.pt] (1.3140740740740748,3.6111111111111067)-- (4.1807407407407435,1.1977777777777756);
\draw [line width=1.pt] (4.1807407407407435,1.1977777777777756)-- (1.1674074074074081,-0.8155555555555558);
\draw [line width=1.pt] (1.1674074074074081,-0.8155555555555558)-- (-0.8725925925925934,1.3844444444444421);
\draw [line width=1.pt] (1.3140740740740748,3.6111111111111067) circle (2.194663168830257cm);
\draw [line width=1.pt] (4.1807407407407435,1.1977777777777756) circle (1.5525965707068303cm);
\draw [line width=1.pt] (1.1674074074074081,-0.8155555555555558) circle (2.071446598032874cm);
\draw [line width=1.pt] (-0.24285777234504735,0.7053186579029721)-- (-0.22365876967415543,2.0452490080260213);
\draw [line width=1.pt] (-0.22365876967415543,2.0452490080260213)-- (2.9929989919992828,2.1976905988112825);
\draw [line width=1.pt] (2.9929989919992828,2.1976905988112825)-- (2.889781780845985,0.3352344019188893);
\draw [line width=1.pt] (2.889781780845985,0.3352344019188893)-- (-0.24285777234504735,0.7053186579029721);
\draw[color=black] (-1.2259259259259272,1.4744444444444427) node {$A$};
\draw[color=black] (1.254074074074075,4.0944444444444414) node {$B$};
\draw[color=black] (4.474074074074078,1.1944444444444429) node {$C$};
\draw[color=black] (1.1140740740740749,-1.065555555555556) node {$D$};
\draw[color=black] (-0.18592592592592627,2.454444444444442) node {$P$};
\draw[color=black] (2.8940740740740765,2.674444444444442) node {$Q$};
\draw[color=black] (3.0340740740740766,0.6744444444444431) node {$R$};
\draw[color=black] (-0.24592592592592633,0.37444444444444325) node {$S$};
\end{tikzpicture}
\end{problema}
%Sug: Angulitos.


\begin{problema}
%[Shariguin]
Sea $ABC$ isósceles con base $BC$. Se considera el círculo que es tangente interiormente al circuncírculo de $ABC$ y al lado $BC$. Calcular el radio del círculo pequeño en términos de $\angle BAC=\alpha$ y $BC$.

\begin{tikzpicture}[line cap=round,line join=round,>=triangle 45,x=2.5cm,y=2.5cm]
\draw [line width=1.pt] (2.,0.34788560533031926) circle (2.5*1.0587843946696809cm);
\draw [line width=1.pt] (2.,-0.35544939466968073) circle (2.5*0.35544939466968073cm);
\draw [line width=1.pt] (1.,0.)-- (3.,0.);
\draw [line width=1.pt] (2.,1.40667)-- (3.,0.);
\draw [line width=1.pt] (2.,1.40667)-- (1.,0.);
\draw[color=black] (1.9933333333333334,1.53) node {$A$};
\draw[color=black] (0.78,-0.02666666666666516) node {$B$};
\draw[color=black] (3.18,-0.02666666666666516) node {$C$};
\end{tikzpicture}
\end{problema}

\newpage


\begin{problema}
%ref% 
%[OMM '99]
Sea ABCD un trapecio con $AB$ paralela a $CD$. Las bisectrices exteriores a los ángulos $B$ y $C$ se intersectan en $P$. Las bisectrices exteriores a los ángulos $A$ y $D$ se intersectan en $Q$. Mostrar que $|PQ|$ es igual a la mitad del perímetro de $ABCD$.  

\begin{tikzpicture}[line cap=round,line join=round,>=triangle 45,x=2.0cm,y=2.0cm]
\draw [line width=1.pt] (0.3,0.)-- (2.4,0.);
\draw [line width=1.pt] (2.4,0.)-- (2.22,1.12);
\draw [line width=1.pt] (2.22,1.12)-- (0.86,1.12);
\draw [line width=1.pt] (0.86,1.12)-- (0.3,0.);
\draw [line width=1.pt] (-0.5133333333333334,0.)-- (0.3,0.);
\draw [line width=1.pt] (0.3,0.)-- (-0.046099033699941165,0.56);
\draw [line width=1.pt] (-0.046099033699941165,0.56)-- (0.86,1.12);
\draw [line width=1.pt] (0.06,1.12)-- (0.86,1.12);
\draw [line width=1.pt] (2.22,1.12)-- (3.1533333333333333,1.12);
\draw [line width=1.pt] (2.22,1.12)-- (2.8771860364994892,0.56);
\draw [line width=1.pt] (2.8771860364994892,0.56)-- (2.4,0.);
\draw [line width=1.pt] (2.4,0.)-- (3.18,0.);
\draw [line width=1.pt] (-0.046099033699941165,0.56)-- (2.8771860364994892,0.56);
\draw[color=black] (0.2066666666666667,-0.14666666666666517) node {$A$};
\draw[color=black] (2.4733333333333336,-0.16) node {$B$};
\draw[color=black] (2.2866666666666666,1.36) node {$C$};
\draw[color=black] (0.8066666666666668,1.3866666666666685) node {$D$};
\draw[color=black] (-0.24666666666666667,0.653333333333335) node {$P$};
\draw[color=black] (3.086666666666667,0.6666666666666684) node {$Q$};
\end{tikzpicture}
%Sug: excírculos.
\end{problema}


\begin{problema}
%[OMM '05]
Sea $O$ el circuncentro del triángulo acutángulo $ABC$, y sea $P$ cualquier punto en el segmento $BC$. Supón que el circuncírculo de $BPO$ intersecta al segmento $AB$ en el punto $R$ y que el circuncírculo de $COP$ intersecta a $CA$ en el punto $Q$.

(i) Considera el triángulo $PQR$, muestra que es semejante al triángulo $ABC$ y que $O$ es su ortocentro.

(ii) Muestra que los circuncírculos de los triángulos $BPO$, $COP$, $PQR$ tienen el mismo radio.

\begin{tikzpicture}[line cap=round,line join=round,>=triangle 45,x=.5*3.0cm,y=.5*3.0cm]
\draw [line width=1.pt] (1.9133333333333329,0.6512599681020734) circle (.5*6.392355364859109cm);
\draw [line width=1.pt] (3.066415326319379,0.7527761833316287) circle (.5*3.4726261849684987cm);
\draw [line width=1.pt] (1.507839302308568,1.7354546993687834) circle (.5*3.472626184968496cm);
\draw [line width=1.pt] (-0.11333333333333334,-0.006666666666665152)-- (3.94,-0.006666666666665152);
\draw [line width=1.pt] (3.94,-0.006666666666665152)-- (2.006666666666667,2.78);
\draw [line width=1.pt] (2.006666666666667,2.78)-- (-0.11333333333333334,-0.006666666666665152);
\draw [line width=1.pt] (0.634254628627948,0.9760118493704902)-- (2.660921295294614,1.8369709145983386);
\draw [line width=1.pt] (0.634254628627948,0.9760118493704902)-- (2.1928306526387566,-0.006666666666665151);
\draw [line width=1.pt] (2.1928306526387566,-0.006666666666665151)-- (2.660921295294614,1.8369709145983386);
\draw[color=black] (-0.31333333333333335,-0.04) node {$A$};
\draw[color=black] (4.153333333333333,-0.08) node {$B$};
\draw[color=black] (2.033333333333333,3.0533333333333355) node {$C$};
\draw [fill=black] (1.9133333333333329,0.6512599681020734) circle (1.0pt);
\draw[color=black] (2.06,0.96) node {$O$};
\draw[color=black] (2.7533333333333334,2.08) node {$P$};
\draw[color=black] (0.4066666666666667,1.0266666666666684) node {$Q$};
\draw[color=black] (2.14,-0.1866666666666652) node {$R$};
\end{tikzpicture}
\hspace{1cm}
\begin{tikzpicture}[line cap=round,line join=round,>=triangle 45,x=.5*3.0cm,y=.5*3.0cm]
\draw [line width=1.pt] (-0.11333333333333334,-0.006666666666665152)-- (3.94,-0.006666666666665152);
\draw [line width=1.pt] (3.94,-0.006666666666665152)-- (2.006666666666667,2.78);
\draw [line width=1.pt] (2.006666666666667,2.78)-- (-0.11333333333333334,-0.006666666666665152);
\draw [line width=1.pt] (0.634254628627948,0.9760118493704902)-- (2.660921295294614,1.8369709145983386);
\draw [line width=1.pt] (0.634254628627948,0.9760118493704902)-- (2.1928306526387566,-0.006666666666665151);
\draw [line width=1.pt] (2.1928306526387566,-0.006666666666665151)-- (2.660921295294614,1.8369709145983386);
\draw[color=black] (-0.31333333333333335,-0.04) node {$A$};
\draw[color=black] (4.153333333333333,-0.08) node {$B$};
\draw[color=black] (2.033333333333333,3.0533333333333355) node {$C$};
\draw [fill=black] (1.9133333333333329,0.6512599681020734) circle (1.0pt);
\draw[color=black] (2.06,0.96) node {$O$};
\draw[color=black] (2.7533333333333334,2.08) node {$P$};
\draw[color=black] (0.4066666666666667,1.0266666666666684) node {$Q$};
\draw[color=black] (2.14,-0.1866666666666652) node {$R$};
\end{tikzpicture}
%Sug: cíclicos, isóceles.
\end{problema}
\newpage

\begin{problema}
En la figura se muestra un cuadril\'atero $ABCD$. Si $BC=AD$, ?`cu\'anto mide el \'angulo $ADC$?

\begin{tikzpicture}[line cap=round,line join=round,>=triangle 45,x=1.0cm,y=1.0cm]
\draw [line width=1.pt] (1.76,-3.14)-- (-2.68,-0.4);
\draw [line width=1.pt] (1.76,-3.14)-- (1.78,0.66);
\draw [line width=1.pt] (1.78,0.66)-- (-2.58,1.48);
\draw [line width=1.pt] (-2.58,1.48)-- (-2.68,-0.4);
\draw [line width=1.pt] (-2.68,-0.4)-- (1.78,0.66);
\draw[color=black] (-3.18,-0.59) node {$A$};
\draw[color=black] (-1.95,-0.50) node {$50^{\circ}$};
\draw[color=black] (-2.84,1.85) node {$B$};
\draw[color=black] (-2.14,1.15) node {$75^{\circ}$};
\draw[color=black] (1.92,1.03) node {$C$};
\draw[color=black] (0.72,0.67) node {$30^{\circ}$};
\draw[color=black] (1.9,-3.27) node {$D$};
\end{tikzpicture}

% Demostraci\'on. Por suma de \'angulos en $ABC$ se tiene $\angle BAC=180-(75+30)=75$, y entonces $ABC$ es is\'osceles y $AC=BC=AD$, por lo que $ACD$ tambi\'en es is\'osceles y $\angle ADC=\angle ACD=(180-50)/2=65$.

%(P23 Nivel Ol\'impico MatPreolimpicasMLPS)
\end{problema}

\begin{problema}
En la figura $ABCD$ es un cuadrado y $OBC$ es un tri\'angulo equil\'atero. ?`Cu\'anto
mide el \'angulo $\angle OAC$?

\begin{tikzpicture}[line cap=round,line join=round,>=triangle 45,x=4.0cm,y=4.0cm]
\draw [line width=1.pt] (0.,1.)-- (0.,0.);
\draw [line width=1.pt] (0.,0.)-- (1.,0.);
\draw [line width=1.pt] (1.,0.)-- (1.,1.);
\draw [line width=1.pt] (1.,1.)-- (0.,1.);
\draw [line width=1.pt] (0.49753086419753084,0.8155555555555556)-- (0.,0.);
\draw [line width=1.pt] (0.49753086419753084,0.8155555555555556)-- (1.,0.);
\draw [line width=1.pt] (0.49753086419753084,0.8155555555555556)-- (0.,1.);
\draw [line width=1.pt] (0.,1.)-- (1.,0.);
\draw[color=black] (-0.027407407407407193,1.074320987654321) node {$A$};
\draw[color=black] (-0.027407407407407193,-0.0548148148148148) node {$B$};
\draw[color=black] (1.0269135802469138,-0.0548148148148148) node {$C$};
\draw[color=black] (1.0269135802469138,1.074320987654321) node {$D$};
\draw[color=black] (0.5251851851851852,0.8886419753086421) node {$O$};
\end{tikzpicture}

%(P53 Nivel Ol\'impico MatPreolimpicasMLPS)
\end{problema}



\begin{problema}
En la figura, $ABCD$ es un cuadrado y $CED$ es un tri\'angulo equil\'atero. ?`Cu\'anto
mide el \'angulo $\alpha$?

\begin{tikzpicture}[line cap=round,line join=round,>=triangle 45,x=3.0cm,y=3.0cm]
\draw [line width=1.pt] (0.,1.)-- (0.,0.);
\draw [line width=1.pt] (0.,0.)-- (1.,0.);
\draw [line width=1.pt] (1.,0.)-- (1.,1.);
\draw [line width=1.pt] (1.,1.)-- (0.,1.);
\draw [line width=1.pt] (0.,1.)-- (0.5014814814814815,1.7735802469135802);
\draw [line width=1.pt] (1.,1.)-- (0.5014814814814815,1.7735802469135802);
\draw [line width=1.pt] (0.,0.)-- (0.5014814814814815,1.7735802469135802);
\draw [line width=1.pt] (1.,0.)-- (0.5014814814814815,1.7735802469135802);
\draw[color=black] (-0.08518518518518559,1.0713580246913583) node {$D$};
\draw[color=black] (-0.08518518518518559,-0.06049382716049337) node {$A$};
\draw[color=black] (1.052592592592593,-0.05456790123456745) node {$B$};
\draw[color=black] (1.0703703703703709,1.0713580246913583) node {$C$};
\draw[color=black] (0.5014814814814815,1.9365432098765434) node {$E$};
\draw[color=black] (0.5014814814814815,1.5365432098765434) node {$\alpha$};
\end{tikzpicture}

%(P11 Nivel Estudiante MatPreolimpicasMLPS)
\end{problema}

\begin{problema}
Los \'angulos en las esquinas de la estrella son los marcados. ?`Cu\'anto
vale $x?$

\begin{tikzpicture}[line cap=round,line join=round,>=triangle 45,x=3.5cm,y=3.5cm]
\draw [line width=1.pt] (0.9044444444444447,0.03135802469135847)-- (-0.008148148148148491,0.8491358024691361);
\draw [line width=1.pt] (0.9044444444444447,0.03135802469135847)-- (0.5962962962962963,1.1869135802469137);
\draw [line width=1.pt] (0.07481481481481453,0.1558024691358029)-- (0.5962962962962963,1.1869135802469137);
\draw [line width=1.pt] (0.07481481481481453,0.1558024691358029)-- (1.2837037037037042,0.8728395061728398);
\draw [line width=1.pt] (1.2837037037037042,0.8728395061728398)-- (-0.008148148148148491,0.8491358024691361);
\draw[color=black] (0.751851851851852,0.30172839506172877) node {$25^{\circ}$};
\draw[color=black] (0.2018518518518516,0.7987654320987657) node {$60^{\circ}$};
\draw[color=black] (1.0040740740740746,0.8020987654320991) node {$45^{\circ}$};
\draw[color=black] (0.5725925925925927,1.0013580246913583) node {$x$};
\draw[color=black] (0.22925925925925905,0.32469135802469173) node {$15^{\circ}$};
\end{tikzpicture}

(P41 Nivel Estudiante MatPreolimpicasMLPS)
\end{problema}


\begin{problema}
En la figura, los puntos $A,P,Q$ y $R$ est\'an sobre la circunferencia con centro
$C$; $ABCD$ es un cuadrado; la recta $PR$ pasa por $B$ y $D$; la recta $QR$ pasa por $C$.
Determinar el \'angulo $\angle PQR$.

\begin{tikzpicture}[line cap=round,line join=round,>=triangle 45,x=3.0cm,y=3.0cm]
\draw [line width=1.pt] (0.,0.) circle (3.cm);
\draw [line width=1.pt] (0.,0.)-- (-0.13542605358503557,0.6940171352426272);
\draw [line width=1.pt] (-0.13542605358503557,0.6940171352426272)-- (-0.8294431888276628,0.5585910816575916);
\draw [line width=1.pt] (-0.8294431888276628,0.5585910816575916)-- (-0.6940171352426272,-0.13542605358503557);
\draw [line width=1.pt] (0.,0.)-- (-0.6940171352426272,-0.13542605358503557);
\draw [line width=1.pt] (-0.8984756614567335,-0.43902333169193314)-- (0.8984756614567335,0.43902333169193314);
\draw [line width=1.pt] (0.8984756614567335,0.43902333169193314)-- (0.06903247262907086,0.9976144133495249);
\draw [line width=1.pt] (-0.8984756614567335,-0.43902333169193314)-- (0.06903247262907086,0.9976144133495249);
\draw[color=black] (0.057037037037036734,-0.07827160493827115) node {$C$};
\draw[color=black] (-0.9148148148148156,0.6683950617283954) node {$A$};
\draw[color=black] (-0.808148148148149,-0.07827160493827115) node {$D$};
\draw[color=black] (-0.18,0.8165432098765434) node {$B$};
\draw[color=black] (0.08666666666666636,1.1483950617283951) node {$P$};
\draw[color=black] (0.9992592592592594,0.4609876543209879) node {$Q$};
\draw[color=black] (-0.974074074074075,-0.46938271604938214) node {$R$};
\end{tikzpicture}

%(P29 Nivel Semifinal MatPreolimpicasMLPS)
\end{problema}

\begin{problema}
En el hex\'agono regular de la figura, cada lado mide $\sqrt{3}$ y se dibujaron
dos cuadrados sobre los lados, como se muestra.


Probar que el tri\'angulo $ABC$ es equil\'atero.

\begin{tikzpicture}[line cap=round,line join=round,>=triangle 45,x=3.0cm,y=3.0cm]
\draw [line width=1.pt] (0.2886751345948127,0.6547005383792519)-- (0.5773502691896257,1.1547005383792521);
\draw [line width=1.pt] (0.,0.)-- (1.,0.);
\draw [line width=1.pt] (1.,0.)-- (1.5,0.8660254037844387);
\draw [line width=1.pt] (1.5,0.8660254037844387)-- (1.,1.7320508075688776);
\draw [line width=1.pt] (0.,1.7320508075688779)-- (-0.5,0.8660254037844395);
\draw [line width=1.pt] (-0.5,0.8660254037844395)-- (0.,0.);
\draw [line width=1.pt] (1.,1.7320508075688776)-- (0.,1.7320508075688779);
\draw [line width=1.pt] (0.,1.7320508075688779)-- (0.,1.1547005383792521);
\draw [line width=1.pt] (-0.5,0.8660254037844395)-- (0.,1.1547005383792521);
\draw [line width=1.pt] (0.,1.1547005383792521)-- (0.5773502691896257,1.1547005383792521);
\draw [line width=1.pt] (0.5773502691896257,1.1547005383792521)-- (0.577350269189626,1.7320508075688776);
\draw [line width=1.pt] (-0.21132486540518738,0.36602540378443954)-- (0.2886751345948127,0.6547005383792519);
\draw [line width=1.pt] (0.2886751345948127,0.6547005383792519)-- (0.,1.1547005383792521);
\draw[color=black] (-0.132592592592593,1.24320987654321) node {$A$};
\draw[color=black] (0.35333333333333317,0.5854320987654323) node {$B$};
\draw[color=black] (0.6792592592592592,1.2254320987654321) node {$C$};
\end{tikzpicture}

%(P31 Nivel Semifinal MatPreolimpicasMLPS)
\end{problema}

\begin{problema}
Sobre cada lado de un paralelogramo se dibuja un cuadrado (hacia el exterior del
paralelogramo y de manera que el lado del cuadrado sea el lado respectivo del paralelogramo).
Probar que los centros de los cuatro cuadrados son los v\'ertices de otro cuadrado.

\begin{tikzpicture}[line cap=round,line join=round,>=triangle 45,x=2.0cm,y=2.0cm]
\draw [line width=1.pt] (0.18740740740740716,-0.06938271604938225)-- (1.0051851851851854,-0.07530864197530816);
\draw [line width=1.pt] (0.18740740740740716,-0.06938271604938225)-- (0.6081481481481481,0.9380246913580248);
\draw [line width=1.pt] (-0.10592592592592581,0.6446913580246917)-- (1.02,1.343950617283951);
\draw [line width=1.pt] (1.02,1.343950617283951)-- (1.7192592592592595,0.21802469135802485);
\draw [line width=1.pt] (1.7192592592592595,0.21802469135802485)-- (0.5933333333333334,-0.4812345679012343);
\draw [line width=1.pt] (0.5933333333333334,-0.4812345679012343)-- (-0.10592592592592581,0.6446913580246917);
\draw [line width=1.pt] (0.6081481481481481,0.9380246913580248)-- (1.4259259259259265,0.9320987654320989);
\draw [line width=1.pt] (1.4259259259259265,0.9320987654320989)-- (1.0051851851851854,-0.07530864197530816);
\draw [line width=1.pt] (1.0051851851851854,-0.07530864197530816)-- (2.012592592592592,-0.49604938271604926);
\draw [line width=1.pt] (2.012592592592592,-0.49604938271604926)-- (2.433333333333333,0.5113580246913575);
\draw [line width=1.pt] (2.433333333333333,0.5113580246913575)-- (1.4259259259259265,0.9320987654320989);
\draw [line width=1.pt] (1.4259259259259265,0.9320987654320989)-- (1.431851851851852,1.7498765432098773);
\draw [line width=1.pt] (1.431851851851852,1.7498765432098773)-- (0.6140740740740738,1.755802469135803);
\draw [line width=1.pt] (0.6140740740740738,1.755802469135803)-- (0.6081481481481481,0.9380246913580248);
\draw [line width=1.pt] (-0.3992592592592587,1.3587654320987657)-- (0.6081481481481481,0.9380246913580248);
\draw [line width=1.pt] (-0.3992592592592587,1.3587654320987657)-- (-0.82,0.3513580246913588);
\draw [line width=1.pt] (-0.82,0.3513580246913588)-- (0.18740740740740716,-0.06938271604938225);
\draw [line width=1.pt] (0.18740740740740716,-0.06938271604938225)-- (0.1814814814814813,-0.8871604938271604);
\draw [line width=1.pt] (0.1814814814814813,-0.8871604938271604)-- (0.9992592592592595,-0.8930864197530863);
\draw [line width=1.pt] (0.9992592592592595,-0.8930864197530863)-- (1.0051851851851854,-0.07530864197530816);
\end{tikzpicture}

%(P8 Nivel Final MatPreolimpicasMLPS)
\end{problema}


\begin{problema}
Demostrar que la bisectriz del \'angulo recto de un tri\'angulo rect\'angulo divide
por la mitad el \'angulo entre la mediana y la altura bajadas sobre la hipotenusa.

\begin{tikzpicture}[line cap=round,line join=round,>=triangle 45,x=4.0cm,y=4.0cm]
\draw [line width=1.pt] (-1.,0.)-- (1.,0.);
\draw [line width=1.pt] (-1.,0.)-- (-0.7289686274214113,0.6845471059286888);
\draw [line width=1.pt] (-0.7289686274214113,0.6845471059286888)-- (1.,0.);
\draw [line width=1.pt] (-0.7289686274214113,0.6845471059286888)-- (0.,0.);
\draw [line width=1.pt] (-0.7289686274214113,0.6845471059286888)-- (-0.7289686274214113,0.);
\draw [line width=1.pt] (-0.7289686274214113,0.6845471059286888)-- (-0.4327386422474257,0.);
\end{tikzpicture}

%(SHARIGUIN I.35)
\end{problema}

\begin{problema}
Se da una circunferencia y un punto $A$ fuera de \'esta. $AB$ y $AC$ son
tangentes a la circunferencia ($B$ y $C$ son los puntos de tangencia). Demostrar
que el centro de la circunferencia inscrita en el tri\'angulo $ABC$ se halla
en la circunferencia dada.

\begin{tikzpicture}[line cap=round,line join=round,>=triangle 45,x=2.5cm,y=2.5cm]
\draw [line width=1.pt] (0.,0.) circle (2.5cm);
\draw [line width=1.pt] (1.9414814814814823,1.1395061728395062)-- (-0.0703992984840427,0.9975188914366258);
\draw [line width=1.pt] (-0.0703992984840427,0.9975188914366258)-- (0.8365984515330039,-0.5478166033378141);
\draw [line width=1.pt] (0.8365984515330039,-0.5478166033378141)-- (1.9414814814814823,1.1395061728395062);
\draw [line width=1.pt] (1.9414814814814823,1.1395061728395062)-- (-1.1801528483527446,0.9191986991358906);
\draw [line width=1.pt] (1.9414814814814823,1.1395061728395062)-- (0.32084571431010445,-1.3354487026674366);

\draw[color=black] (2.036296296296297,1.231358024691358) node {$A$};
\draw[color=black] (-0.132592592592593,1.1483950617283951) node {$B$};
\draw[color=black] (0.8985185185185186,-0.6412345679012339) node {$C$};
\draw [fill=black] (0.8624272336757908,0.5061810610980253) circle (1.5pt);
\draw[color=black] (0.9044444444444446,0.6032098765432101) node {$I$};

%Sug
%\draw [line width=1.pt] (1.9414814814814823,1.1395061728395062)-- (0.8624272336757908,0.5061810610980253);
%\draw [line width=1.pt] (0.8624272336757908,0.5061810610980253)-- (-0.0703992984840427,0.9975188914366258);
\end{tikzpicture}

%(SHARIGUIN I.56)
\end{problema}

\begin{problema}
Los \'angulos del cuadril\'atero inscrito $ABCD$ son $\angle DAB=\alpha$, 
$\angle ABC=\beta$, $\angle BKC=\gamma$, donde $K$ es el punto de intersecci\'on
de las diagonales. Hallar $\angle ACD$.

\begin{tikzpicture}[line cap=round,line join=round,>=triangle 45,x=3.0cm,y=3.0cm]
\draw [line width=1.pt] (0.,0.) circle (3.cm);
\draw [line width=1.pt] (0.6210731195021287,0.783752626937795)-- (-0.7276875317562231,0.6859087811994651);
\draw [line width=1.pt] (-0.7276875317562231,0.6859087811994651)-- (-0.984429099196579,-0.17578210561661872);
\draw [line width=1.pt] (-0.984429099196579,-0.17578210561661872)-- (0.5327825335555444,-0.8462521916888223);
\draw [line width=1.pt] (0.5327825335555444,-0.8462521916888223)-- (0.6210731195021287,0.783752626937795);
\draw [line width=1.pt] (-0.7276875317562231,0.6859087811994651)-- (0.5327825335555444,-0.8462521916888223);
\draw [line width=1.pt] (-0.984429099196579,-0.17578210561661872)-- (0.6210731195021287,0.783752626937795);
\draw[color=black] (0.7207407407407408,0.8639506172839508) node {$A$};
\draw[color=black] (-0.7844444444444452,0.8165432098765434) node {$B$};
\draw[color=black] (-1.1162962962962972,-0.18493827160493778) node {$C$};
\draw[color=black] (0.6377777777777778,-0.943456790123456) node {$D$};
\draw[color=black] (-0.298518518518519,0.3069135802469139) node {$K$};
\end{tikzpicture}

%(SHARIGUIN I.79)
\end{problema}

\begin{problema}
Supongamos que $M$ y $N$ son los puntos de tangencia de una circunferencia inscrita
con los lados $BC$ y $BA$ del tri\'angulo $ABC$; $K$, el punto de intersecci\'on de la
bisectriz del \'angulo $A$ con la recta $MN$. Demostrar que $\angle AKC=90^\circ$.


\begin{tikzpicture}[line cap=round,line join=round,>=triangle 45,x=5.0cm,y=5.0cm]
\draw [line width=1.pt] (-0.3992592592592598,-0.36567901234567846)-- (-0.10888888888888928,0.529135802469136);
\draw [line width=1.pt] (-0.10888888888888928,0.529135802469136)-- (1.1829629629629632,-0.3597530864197525);
\draw [line width=1.pt] (1.1829629629629632,-0.3597530864197525)-- (-0.3992592592592598,-0.36567901234567846);
\draw [line width=1.pt] (0.07687326819735947,-0.018246935369252804) circle (5*0.34564638483012994cm);
\draw [line width=1.pt] (-0.3992592592592598,-0.36567901234567846)-- (0.6360369217846424,0.3897726355122104);
\draw [line width=1.pt] (-0.25189618744067716,0.08843984162587176)-- (0.6360369217846424,0.3897726355122104);
\draw [line width=1.pt] (0.6360369217846424,0.3897726355122104)-- (1.1829629629629632,-0.3597530864197525);
\draw[color=black] (-0.49407407407407467,-0.398271604938271) node {$A$};
\draw[color=black] (1.2837037037037042,-0.3923456790123451) node {$C$};
\draw[color=black] (-0.1562962962962967,0.6683950617283954) node {$B$};
\draw[color=black] (-0.34,0.16469135802469176) node {$M$};
\draw[color=black] (0.26444444444444426,0.3839506172839509) node {$N$};
\draw[color=black] (0.6555555555555556,0.4965432098765435) node {$K$};
\end{tikzpicture}

%(SHARIGUIN I.255)
\end{problema}

\begin{problema}
En el tri\'angulo is\'osceles $ABC$ se tiene que $\angle C>90$. Sean $O$ el circuncentro del tri\'angulo,
$I$ el incentro y $D$ el punto sobre $BC$ de tal manera que las l\'ineas $OD$ y $BI$ son perpendiculares.
Prueba que $ID$ y $AC$ son paralelas.


%(PRE 1999)
\end{problema}


\begin{problema}
Sea $ABC$ un tri\'angulo rect\'angulo con \'angulo recto en $A$, tal que $AB<AC$.
Sea $M$ el punto medio de $BC$ y $D$ la intersecci\'on de $AC$ con la perpendicular a $BC$ que pasa
por $M$. Sea $E$ la intersecci\'on de la paralela a $AC$ que pasa por $M$ con la perpendicular a
$BD$ que pasa por $B$. Demuestra que los tri\'angulos $AEM$ y $MCA$ son semejantes si y s\'olo si
$\angle ABC=60^\circ$.

%(20a. OMM P2) 
\end{problema}


\begin{problema}
En el tri\'angulo $ABC$, cuyo $\angle B=60^\circ$, la bisectriz del \'angulo $A$
corta $BC$ en el punto $M$. En el lado $AC$ se toma un punto $K$ de modo que
$\angle AMK=30^\circ$. Hallar $\angle OKC$, donde $O$ es el centro de la circunferencia
circunscrita alrededor del tri\'angulo $AMC$.


%(SHARIGUIN I.250) 
\end{problema}

\begin{problema}
Demostrar que si en el tri\'angulo un \'angulo es igual a $120^\circ$, el
tri\'angulo formado por los pies de sus bisectrices es rect\'angulo.


%(SHARIGUIN I.258)
\end{problema}


%Herón, Brahmagupta, inradio.
\newpage 

\begin{problema}
%[IMO '04]
Let $ABC$ be an acute-angled triangle with $AB\neq AC$. The circle with diameter $BC$ intersects the sides $AB$ and $AC$ at $M$ and $N$ respectively. Denote by $O$ the midpoint of the side $BC$. The bisectors of the angles $\angle BAC$ and $\angle MON$ intersect at $R$. Prove that the circumcircles of the triangles $BMR$ and $CNR$ have a common point lying on the side $BC$.
\end{problema}


Ejes radicales:

\begin{tikzpicture}[line cap=round,line join=round,>=triangle 45,x=6*1.0cm,y=6*1.0cm]
\clip(-0.8209327846364884,-1.1866666666666712) rectangle (2.1420301783264746,1.1125925925925877);
\draw [line width=1.pt] (0.,0.) circle (6*0.14828532235939648cm);
\draw [line width=1.pt] (1.,0.) circle (6*0.5103703703703697cm);
\draw [line width=1.pt] (-0.4095317472344304,0.)-- (0.8152025199410664,-0.4757392209139965);
\draw [line width=1.pt] (-0.4095317472344304,0.)-- (0.8152025199410664,0.4757392209139965);
\draw [line width=1.pt] (0.38075531093762083,0.3069813260147082)-- (0.38075531093762083,-0.3069813260147082);
\draw [line width=1.pt] (-0.4095317472344304,0.)-- (1.6354732510288061,0.);
\draw [line width=1.pt] (-0.4095317472344304,0.)-- (1.559947488368141,0.7650300495900147);
\draw [line width=1.pt] (-0.4095317472344304,0.)-- (1.5592174193449682,-0.7647464595267607);
\draw [line width=1.pt] (0.09766897172028498,0.11157647059453128)-- (0.6638416501549568,-0.38402536215912336);
\draw [line width=1.pt] (0.09766897172028498,-0.11157647059453128)-- (0.6638416501549568,0.38402536215912336);
\draw [fill=black] (0.,0.) circle (1pt);
\draw[color=black] (-0.022908093278463774,-0.031111111111115815) node {$O_1$};
\draw [fill=black] (1.,0.) circle (1pt);
\draw[color=black] (1.020054869684499,-0.03506172839506643) node {$O_2$};
\draw[color=black] (-0.46537722908093293,0.01234567901234096) node {$P_2$};
\draw [fill=black] (0.8152025199410664,0.4757392209139965) circle (1.5pt);
\draw [fill=black] (-0.053691898065824716,0.13822343111541988) circle (1.5pt);
\draw [fill=black] (-0.053691898065824716,-0.13822343111541985) circle (1.5pt);
\draw [fill=black] (0.8152025199410664,-0.4757392209139965) circle (1.5pt);
\draw[color=black] (0.2220301783264745,-0.04691358024691828) node {$P_1$};
\draw [fill=black] (0.09766897172028498,0.11157647059453128) circle (1.5pt);
\draw [fill=black] (0.09766897172028498,-0.11157647059453128) circle (1.5pt);
\draw [fill=black] (0.6638416501549568,0.38402536215912336) circle (1.5pt);
\draw [fill=black] (0.6638416501549568,-0.38402536215912336) circle (1.5pt);
\end{tikzpicture}

% \chapter{Divisibilidad y aritmética de residuos}
\label{cap:divisibilidad}

Un teorema fundamental que se debe tener en mente para muchos ejercicios y problemas en olimpiadas es el teorema fundamental de la aritmética. Lo presentamos al principio porque es el que más se usa:

\begin{teorema}[Teorema Fundamental de la Aritmética]
Todo número natural $n\geq 2$ se descompone de manera única (salvo por el orden de los factores) como producto de potencias de primos \[n=p_1^{\alpha_1}p_2^{\alpha_2}p_3^{\alpha_3}\dots p_k^{\alpha_k}\]
\end{teorema}
Por ejemplo: $$60=2^2\cdot 3\cdot 5,\quad  1000=2^3\cdot 5^3,\quad 1001=7\cdot 11 \cdot  13$$

En este capítulo revisaremos problemas relacionados con divisibilidad y aritmética de residuos.

A continuación presentamos ejercicios bastante representativos que aprenderemos a resolver a lo largo del capítulo.

Un concepto clave en teoría de números es la \textbf{relación de divisibilidad}. Por ejemplo, el número
$20$ es divisible entre $1, 2, 4, 5, 10, 20$.

\begin{ejercicio}
Muestra que si $n$ es impar entonces $8\mid n^2-1$, para todo número entero $n$.
\end{ejercicio}

\begin{ejercicio}
Muestra que $5\mid n^5+4n$ para todo número natural $n$.
\end{ejercicio}

Algo un poco más general es el concepto de \textbf{división con residuo}. Por ejemplo: los residuos de $130$ y $86$ al dividirlos entre $7$ son $4$ y $2$, respectivamente pues \[130=18\cdot 7+4,\quad 86=12\cdot 7+2.\]

\begin{ejercicio}
Calcula los residuos al dividir entre siete de los siguientes números: 
 \begin{enumerate}
     \item $n=130, 260, 390, 520$.
     \item $n=130, 130^2, 130^3, 130^4$.
     \item $n=86^k$, $k=0,1,2,3,4,5,6,7,8,9,10$.
     \item $n=2\cdot130+5\cdot86$.
     \item $n=2\cdot130^5+5\cdot86^{15}$.
 \end{enumerate}
\end{ejercicio}

Después veremos problemas del siguiente tipo, cuyas soluciones se conocían en China desde el siglo III.

\begin{ejercicio}
Considera un entero positivo $n$ tal que:  
\begin{itemize}
\item al dividir a $n$ entre $7$, sobra $6$,
\item al dividir a $n$ entre $11$, sobra $10$,
\item al dividir a $n$ entre $13$, sobra $12$.
\end{itemize}

Encuentra el mínimo valor posible de $n$.
\end{ejercicio}

¿Qué sucede si cambiamos el problema anterior con residuos más complicados? 

\begin{ejercicio}
Considera un entero positivo $n$ tal que:  
\begin{itemize}
\item al dividir a $n$ entre $25$, sobra $16$,
\item al dividir a $n$ entre $16$, sobra $8$,
\item al dividir a $n$ entre $9$, sobra $4$.
\item al dividir a $n$ entre $11$, sobra $2$.
\end{itemize}
¿Tiene solución?
Encuentra el mínimo valor posible de $n$.
\end{ejercicio}

Terminamos el capítulo demostrando el teorema de Wilson, el pequeño teorema de Fermat y el teorema (de la función $\varphi$) de Euler:

Para varios de los problemas anteriores resulta muy útil usar el lenguaje de congruencias o residuos módulo $n$. 

\section{Divisibilidad}
\begin{definicion}
$~$
\begin{itemize}
    \item   Para dos números enteros $a$ y $b$ se dice que
  \textbf{$a$ es divisible entre $b$} si existe $c \in \ZZ$ tal que $a = bc$.
    \item  En
  este caso también se escribe «$b \mid a$» y se dice que
  \textbf{$b$ divide a $a$}, o que $b$ es un \textbf{divisor} de $a$.
    \item   Cuando $a$ no es divisible entre $b$ (es decir, $b$ no divide al número $a$),
  vamos a escribir «$b \nmid a$».
\end{itemize}
\end{definicion}

\begin{ejercicio}
Un número natural $n$ cumple que $4\mid n$ y $3\mid n$. ¿Es cierto que $4\cdot 3=12\mid n$?
\end{ejercicio}

\begin{ejercicio}
Un número natural $n$ cumple que $4\mid n$ y $6\mid n$. ¿Es cierto que $4\cdot 6=24\mid n$?
\end{ejercicio}

Dos números naturales $m,n$ se llaman {\bf primos relativos} o {\bf coprimos} si no tienen ningún divisor en común mayor que $1$. Es decir, el {\bf máximo común divisor} $(n,m)=1$. 

\begin{ejercicio}
Muestra que si $p\mid n$, $q\mid n$ y $(p,q)=1$, entonces $pq\mid n$.
\end{ejercicio}

\begin{ejercicio}
¿Cuáles son los divisores positivos de $a=100$, $a=240$, $a=150$?
\end{ejercicio}

Para $a=60$, los diferentes divisores positivos son $$b = 1, 2, 3, 4, 5, 6, 10, 12, 15, 20, 30, 60.$$ Los mismos con el signo «$-$» son los divisores negativos de $a$. Esta propiedad del número $60$ de tener muchos divisores fue notada por los babilonios. Por esta razón la hora todavía se divide en $60$ minutos, el círculo en $360$ grados, etcétera. Por la misma razón algunos productos, como los huevos o las tortillas, se venden a veces por docena: los divisores positivos no triviales de $10$ son solamente $2$ y $5$, mientras que $12$ es divisible por $2$, $3$, $4$, $6$.

\begin{ejercicio}
  Sean $a$ y $b$ dos números enteros. Muestra que:

  \begin{enumerate}
  \item[1)] Si $b \mid a$, entonces $|b| \le |a|$.

  \item[2)] Si $b \mid a$ y $a \mid b$, entonces $a = \pm b$.
  \end{enumerate}
\end{ejercicio}
%Sug: Si $a = bc$, entonces $|a| = |b| \cdot |c|$.


\begin{ejercicio}
  Sean $a,b,c$ números enteros. Demuestra las siguientes propiedades de la relación de divisibilidad.

  \begin{enumerate}
  \item[1)] $1\mid a$, $a \mid a$, $a \mid 0$ para cualquier $a$,

  \item[2)] $a\mid 1$ si y solamente si $a = \pm 1$,

  \item[3)] $0\mid a$ si y solamente si $a = 0$,

  \item[4)] si $c \mid a$ y $c \mid b$, entonces $c \mid (a + b)$,

  \item[5)] si $c \mid b$ y $b \mid a$, entonces $c \mid a$,

  \item[6)] si $c \ne 0$, entonces $ac \mid bc$ implica $a\mid b$,

  \item[7)] $b \mid a$ si y solamente si $-b \mid a$.
  \end{enumerate}
\end{ejercicio}

Para $a = 31$, los únicos divisores son $b = \pm 1, \pm 31$. En cierto
sentido, el número $31$ no tiene divisores no triviales.  Es un {\bf número
primo}. El estudio de los números primos es uno de los principales objetivos de la teoría de números. Es importante mencionar que el $1$ no se considera primo.

\begin{ejercicio}
Escribe todos los números primos menores o iguales que $n=50$.
\end{ejercicio}

\begin{ejercicio}
Muestra que la cantidad de divisores positivos de un número $n=p_1^{\alpha_1}p_2^{\alpha_2}p_3^{\alpha_3}\dots p_k^{\alpha_k}$ esta dada por $(\alpha_1+1)(\alpha_2+1)(\alpha_3+1)\cdots (\alpha_k+1)$.
\end{ejercicio}


\begin{ejercicio}
Muestra que un número $n$ tiene una cantidad impar de divisores si y solo si $n$ es un cuadrado perfecto.
\end{ejercicio}

\begin{ejercicio}
Demuestra que \emph{no es posible} que haya una \emph{cantidad finita} de números primos $\{p_1,p_2,\dots,p_k\}$.
\end{ejercicio}
%Sug: ¿Qué pasaría con $n=p_1\cdot p_2\cdot \dots \cdot p_k+1$?

La función $\varphi(n)$ de Euler cuenta el número de enteros positivos $k\leq n$ que son primos relativos. Por ejemplo $\varphi(6)=2$ porque solamente $k=1,5$ son primos relativos con $6$.

\begin{ejercicio}
Calcula $\varphi (40), \varphi (120), \varphi (330)$.
\end{ejercicio}

\begin{ejercicio}
Calcula $\varphi (p^\alpha)$ para $p$ primo.
\end{ejercicio}

\begin{ejercicio}
Demuestra que $\varphi (mn)=\varphi (m)\varphi (n)$ siempre que $(m,n)=1$.

Deduce una fórmula general para $\varphi (n)$.
\end{ejercicio}

\section{División con residuo}

Un concepto poco más general que la divisibilidad es la noción de \textbf{división con residuo}: aunque $20$
no es divisible entre $7$, podemos escribir $\frac{20}{7} = 2+\frac{6}{7}$; es decir, $20 = 2\cdot 7 + 6$. Aquí el número $6$ es el \textbf{residuo}
de la división de $20$ entre $7$.

\begin{proposicion}[División con residuo]
  Para dos números enteros $a$ y $b \ne 0$, existen $q$ (cociente) y $r$
  (residuo) tales que
  \[ a = qb + r,
    \quad
    0 \le r < |b|. \]
  Además, estas propiedades definen a $q$ y $r$ de manera única.
\end{proposicion}

Algo que no es tan obvio a simple vista es que podemos calcular residuos de {\bf sumas, restas y productos} de maneras muy eficientes.  Por ejemplo: los residuos de $130$ y $86$ al dividirlos entre $7$ son $4$ y $2$, respectivamente pues \[130=18\cdot 7+4,\quad 86=12\cdot 7+2.\] 

\begin{ejercicio}
Calcula los residuos al dividir entre siete de los siguientes números: 
 \begin{enumerate}
     \item $n=130, 260, 390, 520$.
     \item $n=130, 130^2, 130^3, 130^4$.
     \item $n=86^k$, $k=0,1,2,3,4,5,6,7,8,9,10$
     \item $n=2\cdot130+5\cdot86$.
     \item $n=2\cdot130^5+5\cdot86^{15}$.
 \end{enumerate}
\end{ejercicio}

En todas las operaciones anteriores, bastaba con ignorar siempre las partes que fueran múltiplos de $7$ (tanto de los sumandos como de los factores), para que todos los ejercicios se trataran sobre multiplicaciones o sumas de números pequeños, entre $0,1,2,3,4,5,6$ y no en tediosos productos o sumas de números más grandes.

\begin{ejercicio}
Muestra que $5\mid n^5+4n$ para todo número natural $n$.
\end{ejercicio}

\begin{ejercicio}
Muestra que $3$ no divide a $n^2+1$ para ningún número natural $n$.
\end{ejercicio}

\begin{ejercicio}
Muestra que $n^3+2$ no es divisible entre $9$ para ningún número natural $n$.
\end{ejercicio}

En los ejercicios anteriores, puede resultar conveniente partir el problema en distintos casos, de acuerdo a su residuo. Para hablar de estas operaciones de forma abreviada se utiliza la notación de {\bf congruencias o reducciones módulo} $n$.

%%%%%%%%%%%%%%%%%%%%%%%%%%%%%%%%%%%%%%%%%%%%%%%%%%%%%%%%%%%%%%%%%%%%%%%%%%%%%%%%

\section{Reducción módulo $n$}

Decimos que un número entero $a$ es {\bf congruente a} $r$ {\bf módulo} $n$, y escribimos $$a\equiv r~(\mathrm {mod} n),$$ si y solo si $n\mid a-r$.

\begin{ejercicio}
Demuestra que $a\equiv b~(\mathrm {mod} n)$ si y solo si $a$ y $b$ tienen el mismo residuo al dividirse entre $n$.
\end{ejercicio}

\begin{ejercicio} Demuestra que la relación <<$\equiv$>> es una {\bf relación de equivalencia}. Es decir, la relación es:
\begin{itemize}
\item Reflexiva: $x\equiv x~(\mathrm {mod} n)$ para todo $x\in \mathbb Z$.
\item Simétrica: Si $x\equiv y ~(\mathrm {mod} n)$, entonces $y \equiv x ~(\mathrm {mod} n)$.
\item Transitiva: Si $x\equiv y~(\mathrm {mod} n)$ y $y\equiv z~(\mathrm {mod} n)$, entonces $x \equiv z~(\mathrm {mod} n)$.
\end{itemize}
\end{ejercicio}

\begin{ejercicio}
Muestra que si $a\equiv b ~(\mathrm {mod} n)$  y $c\equiv d (\mathrm {mod} n)$, entonces
\begin{itemize}
    \item $a+ c\equiv b+ d ~(\mathrm {mod} n)$
    \item $a- c\equiv b- d ~(\mathrm {mod} n)$
    \item $ac\equiv bd ~(\mathrm {mod} n)$
    \item $a^n\equiv b^n ~(\mathrm {mod} n)$
\end{itemize}
\end{ejercicio}

\begin{ejercicio}
Encuentra un contraejemplo donde $ac\equiv bc ~(\mathrm {mod} n)$ pero $a\nequiv b ~(\mathrm {mod} n)$ de la para el caso cuando $n$ y $c$ no son coprimos.
\end{ejercicio}

\begin{ejercicio}
  Encuentra el residuo de $6^{100}$ al dividirlo entre $7$.
\end{ejercicio}

\begin{ejercicio}
  Muestra que $1999^{1999}+2001^{2001}$ es divisible entre $125$.
\end{ejercicio}

\section{La identidad de Bezout} 

\begin{ejercicio}
En las siguientes expresiones se pueden substituir $x,y$ por cualquier entero (incluyendo enteros negativos). 
  
\begin{itemize}
      \item $15x +120y$
      \item $15x -33y$
      \item $187x -34y$
      \item $101x -34y$
\end{itemize}
  
  ¿Cuál es el mínimo valor positivo para cada expresión? 
  
  ¿Es única la solución $(x,y)$ para la cuál se obtiene el mínimo?  
\end{ejercicio}

\begin{proposicion}[Identidad de Bezout]
Sean $m,n$ enteros positivos y sea $d=(m,n)$ su máximo común divisor. Entonces existen enteros $x,y$ tales que $d=mx+ny$.

Además, $d$ es el mínimo valor positivo posible de la expresión $mx+ny$.
\end{proposicion}

Una demostración constructiva se sigue de la versión extendida del algoritmo Euclidiano de la división, pero no la revisaremos por el momento. A menudo es sencillo encontrar los coeficientes de forma directa y por lo pronto lo que utilizaremos en varias ocasiones es que la solución de $(m,n)=xm+yn$ existe.

La identidad de Bézout es de suma importancia. Por ejemplo, se puede utilizar para resolver el siguiente par de ejercicios. 

\begin{ejercicio}[Lema de Euclides]
Demuestra que si $n\mid ab$ y $n$ es coprimo con $a$, entonces $n\mid b$. 

Nota: Aquí en principio no se vale usar el Teorema Fundamental de la Aritmética porque justamente se utiliza este lema en la demostración del TFA.  
\end{ejercicio}
Sug: Por la identidad de Bézout, existen $x,y$ tales que $ax+ny=1$.

\begin{ejercicio}
Sea $n$ un entero positivo. Muestra que un número $k$ tiene inverso multiplicativo módulo $n$ si y solo si $(n,k)=1$. 

Es decir, muestra que existe un entero $j$ tal que $kj\equiv 1 ~(\mathrm {mod}~n)$ si y solo si $(n,k)=1$
\end{ejercicio}
Sug: Por la identidad de Bézout, existen $x,y$ tales que $kx+ny=1$.

\begin{ejercicio}
Demuestra que si $c$ y $n$ son primos relativos y $ac\equiv bc ~(\mathrm {mod} n)$, entonces se pueden <<cancelar>> las $c$'s para obtener $a\equiv b ~(\mathrm {mod} n)$
\end{ejercicio}


\section{Teorema chino del residuo}

Desde el siglo III se conocen soluciones a sistemas de congruencias. Por ejemplo, en los manuscritos del matemático chino Sun-Tzu se encuentra la solución  del sistema de congruencias:

$$n\equiv 2 ~(\mathrm {mod}~3), \quad n\equiv 3 ~(\mathrm {mod}~5), \quad n\equiv 2 ~(\mathrm {mod}~7),$$
cuya solución está dada por $n=23+105k$.

El teorema chino del residuo establece que cada una de las $105=3\cdot 5\cdot 7$ ternas de congruencias
$$n\equiv x ~(\mathrm {mod}~3), \quad n\equiv y ~(\mathrm {mod}~5), \quad n\equiv z ~(\mathrm {mod}~7),$$
tiene exactamente una solución para $0\leq n < 105=3\cdot 5\cdot 7$.

En otras palabras, si consideramos todos los posibles  $0\leq n\leq 104$, recorreremos todas las posibles ternas de congruencias.

A continuación se presenta una tabla con $n=0,1,2,\dots,14$, donde se muestra que se recorren todas las posibles parejas de congruencias, en modulo $3$ y $5$. 

\begin{tabular}{|c||c|c|c|c|c|c|c|c|c|c|c|c|c|c|c|} 
 \hline
  $n=$ & $0$ & $1$ & $2$ & $3$ & $4$ & $5$ & $6$ & $7$ & $8$ & $9$ & $10$ & $11$ & $12$ & $13$ & $14$ \\ 
  \hline
  \hline
  $\mathrm {mod}~3$ & $0$ & $1$ & $2$ & $0$ & $1$ & $2$ & $0$ & $1$ & $2$ & $0$ & $1$ & $2$ & $0$ & $1$ & $2$ \\ 
  \hline
  $\mathrm {mod}~5$ & $0$ & $1$ & $2$ & $3$ & $4$ & $0$ & $1$ & $2$ & $3$ & $4$ & $0$ & $1$ & $2$ & $3$ & $4$ \\
  \hline
  \end{tabular}

Para tres o más congruencias (de módulos coprimos) ocurre lo mismo. ¿Cómo se encuentra una solución en particular?

La estrategia es muy sencilla: En lugar de resolver el sistema:
$$n\equiv 2 ~(\mathrm {mod}~3), \quad n\equiv 3 ~(\mathrm {mod}~5), \quad n\equiv 2 ~(\mathrm {mod}~7),$$
resolvemos los tres sistemas
$$n_1\equiv 2 ~(\mathrm {mod}~3), \quad n_1\equiv 0 ~(\mathrm {mod}~5), \quad n_1\equiv 0 ~(\mathrm {mod}~7),$$
$$n_2\equiv 0 ~(\mathrm {mod}~3), \quad n_2\equiv 3 ~(\mathrm {mod}~5), \quad n_2\equiv 0 ~(\mathrm {mod}~7),$$
$$n_3\equiv 0 ~(\mathrm {mod}~3), \quad n_3\equiv 0 ~(\mathrm {mod}~5), \quad n_3\equiv 2 ~(\mathrm {mod}~7).$$

La suma $n=n_1+n_2+n_3$ será una solución del sistema de congruencias, si esta suma es mayor a $105$ restamos un múltiplo de $105$ hasta que encontremos una solución dentro del intervalo $0,1,2,3 \dots  103,104$.

Cada sistema simplificado es más sencillo porque sabemos que la solución tiene que ser un múltiplo del producto de los módulos con congruencia $0$. Para resolver estas congruencias de manera directa se puede usar la identidad de Bezout, que nos sirve para encontrar inversos multiplicativos.

Por ejemplo, para resolver la primera congruencia $$n_1\equiv 2 ~(\mathrm {mod}~3), \quad n_1\equiv 0 ~(\mathrm {mod}~5), \quad n_1\equiv 0 ~(\mathrm {mod}~7),$$ el número $n_1$ debe ser múltiplo de $35$ y queremos que además sea congruente a $2 ~(\mathrm {mod}~3)$. Directamente el primer múltiplo de $35$ resuelve esta congruencia, por lo que $n_1=35$.

Para la segunda congruencia $$n_2\equiv 0 ~(\mathrm {mod}~3), \quad n_2\equiv 3 ~(\mathrm {mod}~5), \quad n_2\equiv 0 ~(\mathrm {mod}~7),$$ necesitamos un múltiplo de $21$ congruente a $3 ~(\mathrm {mod}~5)$. El $21\equiv 1~(\mathrm {mod}~5)$ no funciona directamente, por lo que debemos buscar un múltiplo más grande. Claramente $n_2=3\cdot 21=63$ funciona.

Finalmente, para la tercera congruencia $$n_3\equiv 0 ~(\mathrm {mod}~3), \quad n_3\equiv 0 ~(\mathrm {mod}~5), \quad n_3\equiv 2 ~(\mathrm {mod}~7),$$ necesitamos un múltiplo de $15$ que sea congruente a $2 ~(\mathrm {mod}~7)$. Como $15\equiv 1~(\mathrm {mod}~7)$, el primer múltiplo que funciona es $n_2=30$.

Entonces $n=35+63-30=128$ es una solución al sistema de congruencias. Como es mayor o igual a $105$ le restamos $105$ (de esta forma no afectamos ninguna de las congruencias) y obtenemos $23$.

\begin{teorema}[Teorema Chino del Residuo]
Si $n_1,n_2,\dots, n_k$ son primos relativos entre sí (cada dos), y sea $N=n_1\cdot n_2 \cdot \cdots \cdot n_k$. Entonces, para cualesquiera enteros $x_1,x_2, \dots x_k$, el sistema de congruencias
$$n\equiv x_1 ~(\mathrm {mod}~n_1), \quad n\equiv x_2 ~(\mathrm {mod}~n_2), \quad \dots \quad  ,n\equiv x_k ~(\mathrm {mod}~n_k),$$
tiene exactamente una solución $0\leq m< N$. Todas las posibles soluciones son de la forma $m+Nk$.
\end{teorema}


\begin{ejercicio}
Considera un entero positivo $n$ tal que:  
\begin{itemize}
\item al dividir a $n$ entre $7$, sobra $6$,
\item al dividir a $n$ entre $11$, sobra $10$,
\item al dividir a $n$ entre $13$, sobra $12$.
\end{itemize}

Encuentra el mínimo valor posible de $n$.
\end{ejercicio}

\newpage
¿Qué sucede si cambiamos el problema anterior con residuos más complicados? 

\begin{ejercicio}
Supón que $n$ es un entero positivo tal que:  
\begin{itemize}
\item al dividir a $n$ entre $12$, sobra $10$,
\item al dividir a $n$ entre $16$, sobra $8$,
\item al dividir a $n$ entre $9$, sobra $4$.
\end{itemize}
¿Tiene solución el sistema de congruencias?
En caso afirmativo, encuentra el mínimo valor posible de $n$.
\end{ejercicio}

\begin{ejercicio}
Supón que $n$ es un entero positivo tal que:  
\begin{itemize}
\item al dividir a $n$ entre $25$, sobra $16$,
\item al dividir a $n$ entre $16$, sobra $8$,
\item al dividir a $n$ entre $9$, sobra $4$.
\end{itemize}
¿Tiene solución el sistema de congruencias?
En caso afirmativo, encuentra el mínimo valor posible de $n$.
\end{ejercicio}

\begin{ejercicio}
Encuentra el mínimo entero positivo $n$ tal que:  
\begin{itemize}
\item al dividir a $n$ entre $25$, sobra $16$,
\item al dividir a $n$ entre $16$, sobra $8$,
\item al dividir a $n$ entre $9$, sobra $4$.
\item al dividir a $n$ entre $11$, sobra $2$.
\end{itemize}
¿Tiene solución?
Encuentra el mínimo valor posible de $n$.
\end{ejercicio}


\newpage 
\section{Anillos de residuos}

Como toda relación de equivalencia, la reducción módulo $n$ parte al conjunto de números enteros $\mathbb Z$ en $n$ {\bf clases de equivalencia}. Cada clase de equivalencia contiene todos los enteros con el mismo residuo $r$ al dividir entre $n$, $r=0,1,2,3,\dots, n-1$.

Lo que se hace típicamente es fijar un representativo para cada clase de equivalencia. Lo más normal es usar los representantes $r\in \{0,1,2,3,\dots, n-1\}$. Otra posible lista de representantes que a veces conviene considerar es una lista más simétrica (para así sumar/multiplicar números más pequeños): por ejemplo $\{-3,-2,-1,0,+1+2,-3\}$, para los residuos módulo $7$ o $\{-1,0,+1+2\}$, para los residuos módulo $4$.

Como hemos visto, cuando queremos calcular el residuo módulo $n$ de una multiplicación, suma o potencia, nos basta conocer el residuo modulo $n$ de los sumandos o factores. Entonces nos interesa ver en concreto cómo se ven esas operaciones de forma resumida.

Tomando un conjunto de representantes, se pueden representar las operaciones de manera sucinta 

Por ejemplo, la multiplicación y suma de congruencias módulo $5$ está dada por:

\begin{center}
\begin{tabular}{|c||c|c|c|c|c|} 
 \hline
  $\plus$ & 0 & 1 & 2 & 3 & 4 \\ 
  \hline
  \hline
  0 & 0 & 1 & 2 & 3 & 4 \\ 
  \hline
  1 & 1 & 2 & 3 & 4 & 0 \\ 
  \hline
  2 & 2 & 3 & 4 & 0 & 1 \\ 
  \hline
  3 & 3 & 4 & 0 & 1 & 2 \\ 
  \hline
  4 & 4 & 0 & 1 & 2 & 3 \\ 
  \hline  
  \end{tabular}
  \hspace{2cm}
  \begin{tabular}{|c||c|c|c|c|c|} 
 \hline
  $\times$ & 0 & 1 & 2 & 3 & 4 \\ 
  \hline
  \hline
  0 & 0 & 0 & 0 & 0 & 0 \\ 
  \hline
  1 & 0 & 1 & 2 & 3 & 4 \\ 
  \hline
  2 & 0 & 2 & 4 & 1 & 3 \\ 
  \hline
  3 & 0 & 3 & 1 & 4 & 2 \\ 
  \hline
  4 & 0 & 4 & 3 & 2 & 1 \\ 
  \hline  
  \end{tabular}
  \end{center}
  

La multiplicación y suma de congruencias módulo $6$, por su parte, tiene las siguientes tablas:  
  \begin{center}
\begin{tabular}{|c||c|c|c|c|c|c|c|c|c|} 
 \hline
  $\plus$ & 0 & 1 & 2 & 3 & 4 & 5 \\ 
  \hline
  \hline
  0 & 0 & 1 & 2 & 3 & 4 & 5\\ 
  \hline
  1 & 1 & 2 & 3 & 4 & 5 & 0\\ 
  \hline
  2 & 2 & 3 & 4 & 5 & 0 & 1\\ 
  \hline
  3 & 3 & 4 & 5 & 0 & 1 & 2\\ 
  \hline
  4 & 4 & 5 & 0 & 1 & 2 & 3\\ 
  \hline   
  5 & 5 & 0 & 1 & 2 & 3 & 4\\ 
  \hline   
  \end{tabular}
  \hspace{2cm}
  \begin{tabular}{|c||c|c|c|c|c|c|c|} 
 \hline
  $\times$ & 0 & 1 & 2 & 3 & 4 & 5 \\ 
  \hline
  \hline
  0 & 0 & 0 & 0 & 0 & 0 & 0 \\ 
  \hline
  1 & 0 & 1 & 2 & 3 & 4 & 5 \\ 
  \hline
  2 & 0 & 2 & 4 & 0 & 2 & 4 \\ 
  \hline
  3 & 0 & 3 & 0 & 3 & 0 & 3 \\ 
  \hline
  4 & 0 & 4 & 2 & 0 & 4 & 2 \\ 
  \hline  
  5 & 0 & 5 & 4 & 3 & 2 & 1 \\ 
  \hline  
  \end{tabular}
  \end{center}

  \begin{center}
  \begin{tabular}{|c||c|c|c|c|c|c|c|c|c|} 
 \hline
  $+$ & 0 & 1 & 2 & 3 & 4 & 5 & 6 & 7 & 8 \\ 
  \hline
  \hline
  0 & 0 & 1 & 2 & 3 & 4 & 5 & 6 & 7 & 8 \\ 
  \hline
  1 & 1 & 2 & 3 & 4 & 5 & 6 & 7 & 8 & 0 \\
  \hline
  2 & 2 & 3 & 4 & 5 & 6 & 7 & 8 & 0 & 1\\ 
  \hline
  3 & 3 & 4 & 5 & 6 & 7 & 8 & 0 & 1 & 2\\ 
  \hline
  4 & 4 & 5 & 6 & 7 & 8 & 0 & 1 & 2 & 3\\ 
  \hline  
  5 & 5 & 6 & 7 & 8 & 0 & 1 & 2 & 3 & 4\\
    \hline
  6 & 6 & 7 & 8 & 0 & 1 & 2 & 3 & 4 & 5\\ 
    \hline
  7 & 7 & 8 & 0 & 1 & 2 & 3 & 4 & 5 & 6 \\ 
    \hline
  8 & 8 & 0 & 1 & 2 & 3 & 4 & 5 & 6 & 7 \\ 
  \hline  
  \end{tabular}
  \hspace{2cm}
  \begin{tabular}{|c||c|c|c|c|c|c|c|c|c|} 
 \hline
  $\times$ & 0 & 1 & 2 & 3 & 4 & 5 & 6 & 7 & 8 \\ 
  \hline
  \hline
  0 & 0 & 0 & 0 & 0 & 0 & 0 & 0 & 0 & 0 \\ 
  \hline
  1 & 0 & 1 & 2 & 3 & 4 & 5 & 6 & 7 & 8\\ 
  \hline
  2 & 0 & 2 & 4 & 6 & 8 & 1 & 3 & 5 & 7\\ 
  \hline
  3 & 0 & 3 & 6 & 0 & 3 & 6 & 0 & 3 & 6\\ 
  \hline
  4 & 0 & 4 & 8 & 3 & 7 & 2 & 6 & 1 & 5\\ 
  \hline  
  5 & 0 & 5 & 1 & 6 & 2 & 7 & 3 & 8 & 4\\
    \hline
  6 & 0 & 6 & 3 & 0 & 6 & 3 & 0 & 6 & 3 \\ 
    \hline
  7 & 0 & 7 & 5 & 3 & 1 & 8 & 6 & 4 & 2 \\ 
    \hline
  8 & 0 & 8 & 7 & 6 & 5 & 4 & 3 & 2 & 1 \\ 
  \hline  
  \end{tabular}
  \end{center}

\begin{observacion}
\begin{enumerate}
    \item En el caso de la suma aparecen todas las congruencias en cada columna y en cada fila exactamente una vez, parecido a un Sudoku. Además, las congruencias son constantes en las anti-diagonales.
    \item En el caso de la multiplicación la estructura es más complicada.
\end{enumerate} 
\end{observacion}   

Observa que si se ignoran todas las filas y columnas correspondientes a los residuos no invertibles, quedándonos solo con los residuos invertibles módulo $n$, se vuelve a formar un arreglo tipo <<sudoku>> en la tabla multiplicativa.

Los siguientes ejercicios pueden parecer laboriosos y repetitivos, como cuando el Sr. Miyagi le puso a lavar todos sus autos a Daniel Larusso. El objetivo es familiarizarnos con las simetrías que esconden las tablas de multiplicar. Detrás de esas simetrías están las ideas que permiten entender varias nociones importantes: raiz primitiva, órden de un elemento, así como las ideas detrás de la demostración del Teorema de Lagrange (que a su vez se utiliza para demostrar el Teorema de Euler).


La suma y multiplicación  modulo 1 son triviales. Todo entero es congruente a cero módulo 1. Entonces $0+0=0$ y $0\cdot 0=0$. 

\begin{tabular}{|c||c|} 
 \hline
  $\plus$ & 0  \\ 
  \hline
  \hline
  0 & 0  \\ 
  \hline
\end{tabular}
\hspace{2cm}
\begin{tabular}{|c||c|} 
 \hline
  $\times$ & 0  \\ 
  \hline
  \hline
  0 & 0  \\ 
  \hline
\end{tabular}    

Completa las tablas de suma y multiplicación que se solicitan en los siguientes ejercicios. Como en el caso multiplicativo nos interesan principalmente los residuos invertibles (es decir los que son primos relativos con el módulo), pondremos primero a éstos en las tablas multiplicativas.

\begin{ejercicio}
 Modulo 2

\begin{tabular}{|c||c|c|} 
 \hline
  $\plus$ & 0 & 1 \\ 
  \hline
  \hline
  0 &  &  \\ 
  \hline
  1 &  &  \\ 
  \hline
\end{tabular}
\hspace{2cm}
\begin{tabular}{|c||c|c|} 
 \hline
  $\times$ & 1 & 0 \\ 
  \hline
  \hline
  1 &  &  \\ 
  \hline
  0 &  &  \\ 
  \hline
\end{tabular}
\end{ejercicio}

\begin{ejercicio}
 Modulo 3

\begin{tabular}{|c||c|c|c|} 
 \hline
  $\plus$ & 0 & 1 & 2\\ 
  \hline
  \hline
  0 &  & & \\ 
  \hline
  1 &  & & \\
    \hline
  2 &  & & \\ 
  \hline
\end{tabular}
\hspace{2cm}
\begin{tabular}{|c||c|c|c|} 
 \hline
  $\times$ & 1 & 2 & 0\\ 
  \hline
  \hline
  1 &  & & \\ 
  \hline
  2 &  & & \\
    \hline
  0 &  & & \\ 
  \hline
\end{tabular}

Observa que el cuadrante de $2\times 2$ de la esquina superior izquierda tiene las mismas simetrías que la tabla de adición módulo $2$.
\end{ejercicio}


\begin{ejercicio}
Modulo 4. Completa las siguientes tablas aditivas modulo $4$. Notarás en sus simetrías que dos de ellas se parecen entre sí y otra es un poco distinta: ¿Cuales?

\begin{tabular}{|c||c|c|c|c|} 
 \hline
  $\plus$ & 0 & 1 & 2 & 3\\ 
  \hline
  \hline
  0 &  & & & \\ 
  \hline
  1 &  & & &\\
    \hline
  2 &  & & & \\
      \hline
  3 &  & & & \\ 
  \hline
\end{tabular}
\hspace{.5cm}
\begin{tabular}{|c||c|c|c|c|} 
 \hline
  $\plus$ & 0 & 2 & 1 & 3\\ 
  \hline
  \hline
  0 &  & & & \\ 
  \hline
  2 &  & & &\\
    \hline
  1 &  & & & \\
      \hline
  3 &  & & & \\ 
  \hline
\end{tabular}
\hspace{.5cm}
\begin{tabular}{|c||c|c|c|c|} 
 \hline
  $\plus$ & 0 & 3 & 2 & 1\\ 
  \hline
  \hline
  0 &  & & & \\ 
  \hline
  3 &  & & &\\
    \hline
  2 &  & & & \\
      \hline
  1 &  & & & \\ 
  \hline
\end{tabular}

Ahora completa las tablas multiplicativas.

\begin{tabular}{|c||c|c|c|c|} 
 \hline
  $\times$ & 0 & 1 & 2 & 3\\ 
  \hline
  \hline
  0 &  & & & \\ 
  \hline
  1 &  & & &\\
    \hline
  2 &  & & & \\
      \hline
  3 &  & & & \\ 
  \hline
\end{tabular}
\hspace{.5cm}
\begin{tabular}{|c||c|c|c|c|} 
 \hline
  $\times$ & 1 & 3 & 0 & 2\\ 
  \hline
  \hline
  1 &  & & & \\ 
  \hline
  3 &  & & &\\
    \hline
  0 &  & & & \\
      \hline
  2 &  & & & \\ 
  \hline
\end{tabular}

Observa que, en la última tabla, los subcuadrantes de $2\times 2$ se encuentran mucho mejor organizados. En particular, el cuadrante superior izquierdo tiene el mismo patrón de simetrías que la tabla aditiva módulo $2$.
\end{ejercicio}

\begin{ejercicio}
Pasemos a módulo $5$. Completa las siguientes tablas de sumas. Observa que ambas tienen el mismo patrón de simetrías.
 
  \begin{tabular}{|c||c|c|c|c|c|} 
 \hline
  $+$ & 0 & 1 & 2 & 3 & 4\\ 
  \hline
  \hline
  0 &  & & & & \\ 
  \hline
  1 &  & & & &\\
    \hline
  2 &  & & & & \\
      \hline
  3 &  & & & & \\ 
        \hline
  4 &  & & & & \\ 
  \hline
\end{tabular}
\hspace{.5cm}
\begin{tabular}{|c||c|c|c|c|c|} 
 \hline
  $+$ & 0 & 2 & 4 & 1 & 3\\ 
  \hline
  \hline
  0 &  & & & & \\ 
  \hline
  2 &  & & & &\\
    \hline
  4 &  & & & & \\
      \hline
  1 &  & & & & \\ 
        \hline
  3 &  & & & & \\ 
  \hline
\end{tabular}

Ahora pasemos al caso multiplicativo. Completa las tablas y compara sus simetrías con las tres tablas del ejercicio modulo $4$ (ignorando la fila y columna del cero).

 \begin{tabular}{|c||c|c|c|c|c|} 
 \hline
  $\times$ & 1 & 2 & 3 & 4 & 0\\ 
  \hline
  \hline
  1 &  & & & & \\ 
  \hline
  2 &  & & & &\\
    \hline
  3 &  & & & & \\
      \hline
  4 &  & & & & \\ 
        \hline
  0 &  & & & & \\ 
  \hline
\end{tabular}
\hspace{.5cm}
\begin{tabular}{|c||c|c|c|c|c|} 
 \hline
  $\times$ & 1 & 4 & 2 & 3 & 0\\ 
  \hline
  \hline
  1 &  & & & & \\ 
  \hline
  4 &  & & & &\\
    \hline
  2 &  & & & & \\
      \hline
  3 &  & & & & \\ 
        \hline
  0 &  & & & & \\ 
  \hline
\end{tabular}
\hspace{.5cm}
\begin{tabular}{|c||c|c|c|c|c|} 
 \hline
  $\times$ & 1 & 2 & 4 & 3 & 0\\ 
  \hline
  \hline
  1 &  & & & & \\ 
  \hline
  2 &  & & & &\\
    \hline
  4 &  & & & & \\
      \hline
  3 &  & & & & \\ 
        \hline
  0 &  & & & & \\ 
  \hline
\end{tabular}
\hspace{.5cm}
\begin{tabular}{|c||c|c|c|c|c|} 
 \hline
  $\times$ & 1 & 3 & 4 & 2 & 0\\ 
  \hline
  \hline
  1 &  & & & & \\ 
  \hline
  3 &  & & & &\\
    \hline
  4 &  & & & & \\
      \hline
  2 &  & & & & \\ 
        \hline
  0 &  & & & & \\ 
  \hline
\end{tabular}    
\end{ejercicio}

\begin{ejercicio}
Módulo $6$. Revisemos las congruencias módulo $6$, primero el caso aditivo. Como $6$ tiene varios divisores, se pueden formar varios patrones cambiando el orden en que aparecen los residuos en la tabla:

 \begin{tabular}{|c||c|c|c|c|c|c|} 
 \hline
  $+$ & 0 & 1 & 2 & 3 & 4 & 5\\ 
  \hline
  \hline
  0 &  & & & & & \\ 
  \hline
  1 &  & & & & &\\
    \hline
  2 &  & & & & & \\
      \hline
  3 &  & & & & &\\ 
        \hline
  4 &  & & & & &\\ 
          \hline
  5 &  & & & & &\\ 
  \hline
\end{tabular}
\hspace{.5cm}
 \begin{tabular}{|c||c|c|c|c|c|c|} 
 \hline
  $+$ & 0 & 2 & 4 & 1 & 3 & 5\\ 
  \hline
  \hline
  0 &  & & & & & \\ 
  \hline
  2 &  & & & & &\\
    \hline
  4 &  & & & & & \\
      \hline
  1 &  & & & & &\\ 
        \hline
  3 &  & & & & &\\ 
          \hline
  5 &  & & & & &\\ 
  \hline
\end{tabular}
\hspace{.5cm}
 \begin{tabular}{|c||c|c|c|c|c|c|} 
 \hline
  $+$ & 0 & 3 & 1 & 4 & 2 & 5\\ 
  \hline
  \hline
  0 &  & & & & & \\ 
  \hline
  3 &  & & & & &\\
    \hline
  1 &  & & & & & \\
      \hline
  4 &  & & & & &\\ 
        \hline
  2 &  & & & & &\\ 
          \hline
  5 &  & & & & &\\ 
  \hline
\end{tabular}

Observa que la segunda tabla se acomoda en subcuadrantes de $3\times 3$, mientras que la tercera tabla se acomoda en subcuadrantes de $2\times 2$.

El caso multiplicativo modulo $6$ no es demasiado interesante porque solo hay dos residuos invertibles ($1,5$) y por tanto no hay muchas maneras de ordenarlos:

 \begin{tabular}{|c||c|c|c|c|c|c|} 
 \hline
  $\times$ & 1 & 5 & 0 & 2 & 4 & 3\\ 
  \hline
  \hline
  1 &  & & & & & \\ 
  \hline
  5 &  & & & & &\\
    \hline
  0 &  & & & & & \\
      \hline
  2 &  & & & & &\\ 
        \hline
  4 &  & & & & &\\ 
          \hline
  3 &  & & & & &\\ 
  \hline
\end{tabular}

¿Cómo son las simetrías del cuadradito superior izquierdo $2\times 2$ con respecto a las tablas de módulos más pequeños?
\end{ejercicio}

\begin{definicion}
El orden aditivo del residuo $k$ módulo $n$ es el mínimo entero positivo $o_+(k)=m$ tal que $$\underbrace{k+k+k+\dots+k}_{o_+(k)=m~\text{veces}}=m\cdot k\equiv 0 (\mathrm{mod} n)$$ 
\end{definicion}
Por ejemplo, los órdenes aditivos modulo $12$ son:

\begin{tabular}{c||c|c|c|c|c|c|c|c|c|c|c|c|}
    $k=$ & 0 & 1 & 2 & 3 & 4 & 5 & 6 & 7 & 8 & 9 & 10 & 11 \\
    $o_+(k)$ & 1 & 12 & 6 & 4 & 3 & 12 & 2 & 12 & 3 & 4 & 6 & 12 
\end{tabular}

\begin{ejercicio}
Muestra que el orden aditivo de $k$ en módulo $n$ es simplemente $m=\frac{n}{d}$, donde $d=(k,n)$.
\end{ejercicio}
En particular, el órden aditivo de cada elemento siempre es un divisor del módulo. 

La razón por la que observamos subpatrones de $6\times 6$, $3\times 3$ y $2\times 2$, en el ejercicio aditivo módulo $6$, se debe a que los órdenes aditivos de los residuos $k=1,2,3$ son $o_+(1)=6$, $o_+(2)=3$ y $o_+(3)=2$, respectivamente.

En el siguiente ejercicio veremos que ocurre algo similar con el pedazo de las tablas multiplicativas correspondiente a los residuos invertibles. Sin embargo, en el caso multiplicativo el orden es un poco más misterioso. Calcularemos algunos ejemplos para obtener un poco de intuición.

\begin{definicion}
El orden multiplicativo de un residuo $k$ módulo $n$ es el mínimo entero positivo $o_{\times}(k)=m$ tal que $$\underbrace{k\cdot k\cdot k\dots k}_{o_{\times}(k)=m~\text{veces}}=k^m\equiv 1 (\mathrm{mod} n)$$ 
\end{definicion}

Por ejemplo, los órdenes multiplicativos modulo $3, 4, 5$ son:

\begin{tabular}{c||c|c|c|c|c|}
    $k=$ & 0 & 1 & 2 \\
    $o_{\times}(k)$ & - & 1 & 2  
\end{tabular}
\hspace{1.5cm}
\begin{tabular}{c||c|c|c|c|c|}
    $k=$ & 0 & 1 & 2 & 3\\
    $o_{\times}(k)$ & - & 1 & - & 2 
\end{tabular}
\hspace{1.5cm}
\begin{tabular}{c||c|c|c|c|c|}
    $k=$ & 0 & 1 & 2 & 3 & 4\\
    $o_{\times}(k)$ & - & 1 & 4 & 4 & 2 
\end{tabular}

\begin{ejercicio}
Calcula el orden multiplicativo módulo $6$ y $7$
\end{ejercicio}

\begin{tabular}{c||c|c|c|c|c|c|}
    $k=$ & 0 & 1 & 2 & 3 & 4 & 5 \\
    $o_{\times}(k)$ & &&&&& 
\end{tabular}
\hspace{1cm}
\begin{tabular}{c||c|c|c|c|c|c|c|}
    $k=$ & 0 & 1 & 2 & 3 & 4 & 5 & 6 \\
    $o_{\times}(k)$ & &&&&&& 
\end{tabular}

\begin{ejercicio}
Módulo $7$. Completa las tablas. 

i) Calcula la tabla de sumas de residuos:

 \begin{tabular}{|c||c|c|c|c|c|c|c|} 
 \hline
  $+$ & 0 & 4 & 1 & 5 & 2 & 6 & 3\\ 
  \hline
  \hline
  0 &  & & & & & &\\ 
  \hline
  4 &  & & & & & &\\
    \hline
  1 &  & & & & & &\\
      \hline
  5 &  & & & & & &\\ 
        \hline
  2 &  & & & & & &\\ 
          \hline
  6 &  & & & & & &\\ 
            \hline
  3 &  & & & & & &\\ 
  \hline
\end{tabular}
\hspace{.5cm}
 \begin{tabular}{|c||c|c|c|c|c|c|c|} 
 \hline
  $+$ & 0 & 1 & 2 & 3 & 4 & 5 & 6\\ 
  \hline
  \hline
  0 &  & & & & & &\\ 
  \hline
  1 &  & & & & & &\\
    \hline
  2 &  & & & & & &\\
      \hline
  3 &  & & & & & &\\ 
        \hline
  4 &  & & & & & &\\ 
          \hline
  5 &  & & & & & &\\ 
            \hline
  6 &  & & & & & &\\ 
  \hline
\end{tabular}

ii) Completa la tabla de ordenes aditivos módulo $7$.

\begin{tabular}{c||c|c|c|c|c|c|c|c|c|c|c|c|}
    $k=$ & 0 & 1 & 2 & 3 & 4 & 5 & 6 \\
    $o_{+}(k)$ & &&&&&& 
\end{tabular}

Como el orden aditivo de todos los residuos distintos de cero son iguales, se obtendrá la misma simetría en el caso aditivo, siempre que los residuos se ordenen de acuerdo a una progresión aritmética. Ahora pasemos al caso multiplicativo.

 \begin{tabular}{|c||c|c|c|c|c|c|c|} 
 \hline
  $\times$ & 1 & 2 & 4 & 3 & 6 & 5 & 0\\ 
  \hline
  \hline
  1 &  & & & & & &\\ 
  \hline
  2 &  & & & & & &\\
    \hline
  4 &  & & & & & &\\
      \hline
  3 &  & & & & & &\\ 
        \hline
  6 &  & & & & & &\\ 
          \hline
  5 &  & & & & & &\\ 
            \hline
  0 &  & & & & & &\\ 
  \hline
\end{tabular}
\hspace{.5cm}
 \begin{tabular}{|c||c|c|c|c|c|c|c|} 
 \hline
  $\times$ & 1 & 6 & 2 & 5 & 3 & 4 & 0\\ 
  \hline
  \hline
  1 &  & & & & & &\\ 
  \hline
  6 &  & & & & & &\\
    \hline
  2 &  & & & & & &\\
      \hline
  5 &  & & & & & &\\ 
        \hline
  3 &  & & & & & &\\ 
          \hline
  4 &  & & & & & &\\ 
            \hline
  0 &  & & & & & &\\ 
  \hline
\end{tabular}
\hspace{.5cm}
 \begin{tabular}{|c||c|c|c|c|c|c|c|} 
 \hline
  $\times$ & 1 & 3 & 2 & 6 & 4 & 5 & 0\\ 
  \hline
  \hline
  1 &  & & & & & &\\ 
  \hline
  3 &  & & & & & &\\
    \hline
  2 &  & & & & & &\\
      \hline
  6 &  & & & & & &\\ 
        \hline
  4 &  & & & & & &\\ 
          \hline
  5 &  & & & & & &\\ 
            \hline
  0 &  & & & & & &\\ 
  \hline
\end{tabular}

¿Como se comparan las simetrías de las tablas multiplicativas de residuos invertibles módulo $7$ con las tablas aditivas de residuos módulo $6$?
\end{ejercicio}

En el ejercicio anterior la última tabla multiplicativa módulo $7$ tiene las mismas simetrías que la tabla de sumas modulo $6$, porque el orden se basa en las potencias del residuo $3$: $1,3,2,6,4,5,1,\dots $, que se ciclan después de recorrer todos los residuos invertibles módulo $7$. Esto ocurre si y solo si $o_{\times}(k)=\varphi(n)$.

A este tipo de residuos de orden máximo, cuyas potencias recorren todos los residuos invertibles, se les llama raíces primitivas modulo $n$.

\begin{ejercicio}
Hasta ahora, para cada módulo $n=2,3,4,5,6,7$, hemos encontrado al menos una raíz primitiva ¿Cuáles son todas las raíces primitivas en cada caso $n=2,3,4,5,6,7$?
\end{ejercicio}

\begin{ejercicio}
Módulo $8$. Calcula los órdenes multiplicativos módulo $8$. Concluye que no hay raíces primitivas módulo $8$.

\begin{tabular}{c||c|c|c|c|c|c|c|c|c|c|c|c|c|}
    $k=$ & 0 & 1 & 2 & 3 & 4 & 5 & 6 & 7 \\
    $o_{\times}(k)$ & &&&&&& & 
\end{tabular}

\end{ejercicio}

\begin{ejercicio}
Encuentra las raíces primitivas módulo $n$ para $n=9, 10, 11$.

\begin{tabular}{c||c|c|c|c|c|c|c|c|c|c|c|c|c|c|}
    $k=$ & 0 & 1 & 2 & 3 & 4 & 5 & 6 & 7 & 8 \\
    $o_{\times}(k)$ & &&&&&& & &
\end{tabular}

\begin{tabular}{c||c|c|c|c|c|c|c|c|c|c|c|c|c|c|c|}
    $k=$ & 0 & 1 & 2 & 3 & 4 & 5 & 6 & 7 & 8 & 9 \\
    $o_{\times}(k)$ & &&&&&& & & &
\end{tabular}

\begin{tabular}{c||c|c|c|c|c|c|c|c|c|c|c|c|c|c|c|c|}
    $k=$ & 0 & 1 & 2 & 3 & 4 & 5 & 6 & 7 & 8 & 9 & 10 \\
    $o_{\times}(k)$ & &&&&&& & & & &
\end{tabular}

Usando esas raíces primitivas, elabora una tabla multiplicativa con patrones similares a los de la suma de residuos (puedes hacer la tabla solo con los $\varphi(9)=6$, $\varphi(10)=4$ y $\varphi(11)=10$ residuos invertibles).
\end{ejercicio}

\begin{ejercicio}
Módulo $12$. Calcula los órdenes multiplicativos módulo $12$. Concluye que no hay raíces primitivas módulo $12$.

\begin{tabular}{c||c|c|c|c|c|c|c|c|c|c|c|c|c|c|c|c|c|}
    $k=$ & 0 & 1 & 2 & 3 & 4 & 5 & 6 & 7 & 8 & 9 & 10 & 11 \\
    $o_{\times}(k)$ & &&&&&& & &&&&
\end{tabular}
\end{ejercicio}

\begin{ejercicio}
Demuestra que si existe una raíz primitiva módulo $n$, entonces de hecho existen exactamente $\varphi(\varphi(n))$ raíces primitivas en total.  
\end{ejercicio}

La siguiente proposición no es tan fácil de demostrar por lo que solo incluiremos el enunciado. Este nos indica exactamente para qué módulos existen raíces primitivas.

\begin{proposicion}
Existe una raíz primitiva en módulo $n$ si y solo si $n$ es de alguna de las siguientes formas:

$n=2$

$n=4$

$n=p^{\alpha}$, para algún primo impar $p$

$n=2p^{\alpha}$, para algún primo impar $p$.
\end{proposicion}

Observa que los órdenes multiplicativos de todos los residuos invertibles módulo $n$ siempre son divisores de $\varphi (n)$. De esta observación se sigue el Teorema de Euler.

Al hecho de que el orden multiplicativo divida $\varphi(n)$ se conoce como el Teorema de Lagrange. Este resultado se cumple y se demuestra en un contexto más general, en el marco de teoría de grupos elemental.

\newpage

\section{Elementos de teoría de grupos}

A continuación introducimos la definición de un grupo.

\begin{definicion}
Un {\bf grupo} es un conjunto $G$ dotado de una operación <<$\circledast$>> que manda a cualesquiera dos elementos $g,h\in G$ a otro elemento $(g\circledast h)\in G$, con las siguientes propiedades:
\begin{itemize}
    \item Asociatividad: Para cualesquiera  $a,b,c\in G$ se tiene que $(a\circledast b)\circledast c =a\circledast (b\circledast c)$.
    \item Neutro: Existe un elemento neutro $e\in G$, que cumple que $a\circledast e=a=e\circledast a$ para todo $a\in G$.
    \item Inversos: Para todo $a\in G$ existe un inverso $a^{-1}$, tal que $a\circledast a^{-1}=e= a^{-1}\circledast a$.
\end{itemize}
Si $a\circledast b=b\circledast a$ para todos $a,b\in G$ decimos que el grupo es {\bf conmutativo} o {\bf abeliano}.
\end{definicion}

Un aspecto importante de trabajar con grupos es que se vale cancelar operaciones repetidas gracias a la existencia de los inversos: Si $a,b,c\in G$ entonces
$$a\circledast c=b\circledast c\quad  \Longleftrightarrow \quad a\circledast c\circledast c^{-1}=b\circledast c\circledast c^{-1} \quad  \Longleftrightarrow \quad a=b $$

Aunque parezcan un poco abstractos, los grupos abundan en la naturaleza y los hemos estudiado desde la primaria sin darnos cuenta. Por ejemplo, considerando $G$ como cualquiera de los conjuntos de números $G=\mathbb Z, \mathbb Q, \mathbb R, \mathbb C$, y $\circledast =+$.

Si en lugar de la suma consideramos la multiplicación $\circledast :=\cdot$, entonces los conjuntos de números $\mathbb Q, \mathbb R, \mathbb C$ son grupos si les extirpamos el cero, que es el único que causa problemas por no tener inverso.  

Entonces ya hemos trabajado con grupos, solo que no lo sabíamos. De hecho, usando las reglas de distributividad para la suma y la multiplicación, sabemos manipular combinaciones de ambas operaciones.

Al trabajar con grupos nos enfocamos en una sola operación. Sorprendentemente, hay teoremas muy importantes, como el de Lagrange, que solo se basan en estas estructuras relativamente simples.

\begin{ejercicio}
El conjunto de números naturales $\mathbb N$ con $\circledast=+$ no es un grupo. ¿Por qué?
\end{ejercicio}

\begin{ejercicio}
Con respecto $\circledast=+$, los conjuntos de números $\mathbb Z, \mathbb Q, \mathbb R, \mathbb C$ sí son grupos.

¿Quién es el elemento neutro en cada caso?

¿Quién es el inverso de $n\in \mathbb Z$?

¿Quién es el inverso de $\frac{a}{b}\in \mathbb Q$?

¿Quién es el inverso de $\frac{2+8\sqrt{\pi}}{7}\in \mathbb R$?

¿Quién es el inverso de $z=a+b\mathrm{i}\in \mathbb C$?
\end{ejercicio}

Con respecto a la multiplicación $\circledast=\cdot$ es claro que los conjuntos de números $\mathbb Z, \mathbb Q, \mathbb R, \mathbb C$ no pueden ser grupos: aunque sí se cuenta con el elemento neutro  $e=1$, el cero nos causa problemas. 

El cero no puede tener inverso porque la multiplicación por cero siempre da cero, por tanto $0\circledast x=1$ no tiene solución.

Para remediar esto se consideran los conjuntos de números sin el cero:
$$\mathbb Z^{\times}:=\mathbb Z\setminus \{0\}, \quad \mathbb Q^{\times}:=\mathbb Q\setminus \{0\},  \quad \mathbb R^{\times}:=\mathbb R\setminus \{0\},  \quad \mathbb C^{\times}:=\mathbb C\setminus \{0\}.$$

\begin{ejercicio}
Con respecto a la multiplicación $\circledast=\cdot$, los conjuntos $\mathbb Q^{\times}, \mathbb R^{\times}, \mathbb C^{\times}$ sí son grupos 

¿Por qué $\mathbb Z^{\times}$ no es un grupo?

¿Quién es el elemento neutro en cada caso?

¿Quién es el inverso de $0\neq \frac{a}{b}\in \mathbb Q$?

¿Quién es el inverso de $\frac{2+8\sqrt{\pi}}{7}\in \mathbb R$?

¿Quién es el inverso de $z=2+\mathrm{i}$?

¿Quién es el inverso de $z=a+b\mathrm{i}\neq 0$?
\end{ejercicio}

Ambas operaciones $+,\cdot$ son conmutativas en $\mathbb Q, \mathbb R$ $\mathbb C$ y los tres conjuntos cumplen las reglas distributivas: $$a\cdot(b+c)=a\cdot b+a\cdot c, \quad (a+b)\cdot c= a\cdot c+b\cdot c.$$
A este tipo de estructuras algebraicas les conoce como {\bf campos}. Los campos son estructuras muy importantes en física y matemáticas (en particular en la teoría de números).

\begin{ejercicio}
El conjunto de matrices de $n\times n$ con entradas en $\mathbb Z, \mathbb Q, \mathbb R, \mathbb C$ es un grupo con respecto a la operación de suma de matrices.

¿Quién es el elemento neutro?

Escribe es el inverso de la matriz 

$\left(\begin{array}{ccc}
     a_{11} & a_{12} & a_{13}\\
     a_{21} & a_{22} & a_{23}\\
     a_{31} & a_{32} & a_{33}
\end{array}\right)
$
\end{ejercicio}

Nuevamente, como ocurrió con los conjuntos de números, el conjunto de matrices de $n\times n$ no es un grupo con respecto a la operación $\circledast$ de multiplicación de matrices. En concreto, hay matrices que fallan en tener inversos, como se muestra en el siguiente ejercicio.

\begin{ejercicio}
Para las siguientes matrices, decide si tienen o no inverso (multiplicativo), en caso afirmativo calcula el inverso.

$$
\left(\begin{array}{ccc}
     0 & 1 & 0\\
     0 & 0 & 1\\
     1 & 0 & 0
\end{array}\right),
\quad
\left(\begin{array}{ccc}
     0 & 1 & 0\\
     1 & 0 & 0\\
     0 & 0 & 1
\end{array}\right),
\quad
\left(\begin{array}{ccc}
     0 & 1 & 0\\
     1 & 0 & 0\\
     0 & 0 & 0
\end{array}\right),
\quad
\left(\begin{array}{cc}
     a & 0\\
     0 & b
\end{array}\right),
\quad
\left(\begin{array}{ccc}
     1 & 1\\
     1 & 1
\end{array}\right).
$$
\end{ejercicio}

\begin{proposicion}
Una matriz de $n\times n$ con entradas reales o complejas tiene inverso multiplicativo si y solo si su determinante es distinto de cero.  
\end{proposicion}

Todos los ejemplos anteriores de grupos consisten de conjuntos infinitos. Sin embargo, en esta ocasión estamos más interesados en {\bf grupos finitos}. Ahora presentamos algunos grupos finitos importantes. Nuestros principales ejemplos son residuos modulo $n$, con los que ya hemos trabajado.

\begin{ejercicio}
Muestra que el conjunto de residuos módulo $n$, $$\frac{\mathbb Z}{n \mathbb Z}=\{[k]:0\leq  k< n\}$$ es un grupo finito con respecto a la operación $\circledast$ de adición de residuos. 

¿Quién es el elemento neutro $e$?

¿Quién es el inverso del elemento $[k]$?

\end{ejercicio}
Obs: La asociatividad se hereda de la asociatividad de la suma usual en $\mathbb Z$, por lo que se cumple automáticamente.  

\begin{ejercicio}
Muestra que el conjunto de residuos invertibles módulo $n$, $$\left(\frac{\mathbb Z}{n \mathbb Z}\right)^{\times}=\{[k]: 0\leq k< n, (k,n)=1\}$$ es un grupo finito con respecto a la operación $\circledast$ de multiplicación de residuos. 

¿Quién es el elemento neutro $e$?

¿Cómo sabemos que existe el inverso del elemento $[k]$?

¿Es verdad que si se multiplica un residuo invertible módulo $n$ con otro, el producto vuelve a ser un residuo invertible?

\end{ejercicio}
Obs: La asociatividad se hereda de la asociatividad de la multiplicación usual en $\mathbb Z$, por lo que se cumple automáticamente.

Como observamos en el caso de operaciones con congruencias, estas quedan completamente determinadas por la tabla de multiplicación. Podemos hacer lo mismo con grupos finitos, listando sus elementos y calculando las tablas de multiplicar.

Si $G$ tiene un solo elemento, solo existe un único grupo, con tabla de multiplicar:

\begin{tabular}{|c||c|} 
 \hline
  $\circledast$ & e  \\ 
  \hline
  \hline
  e &  e \\ 
  \hline
\end{tabular}
\hspace{1cm}
\begin{tabular}{|c||c|} 
 \hline
  $+$ & 0  \\ 
  \hline
  \hline
  0 &  0 \\ 
  \hline
\end{tabular}
\hspace{1cm}
\begin{tabular}{|c||c|} 
 \hline
  $\times$ & 1  \\ 
  \hline
  \hline
  1 &  1 \\ 
  \hline
\end{tabular}

Observa que este grupo tiene la misma estructura que el grupo aditivo de residuos módulo $1$, y que el grupo multiplicativo de residuos invertibles módulo $2$.

Si $G$ tiene dos elementos $e,h$, solo existe un único grupo. Observa que para que $h$ tenga inverso necesitamos que $h\circledast h=e$. Entonces la tabla queda:

\begin{tabular}{|c||c|c|} 
 \hline
  $\circledast$ & $e$ & $h$ \\ 
  \hline
  \hline
  $e$ & $e$ & $h$  \\ 
  \hline
  $h$ & $h$ & $e$ \\ 
  \hline
\end{tabular}
\hspace{1cm}
\begin{tabular}{|c||c|c|} 
 \hline
  $+$ & 0 & 1 \\ 
  \hline
  \hline
  0 & 0 & 1 \\ 
  \hline
  1 & 1 & 0 \\ 
  \hline
\end{tabular}
\hspace{1cm}
\begin{tabular}{|c||c|c|} 
 \hline
  $\times$ & 1 & 2 \\ 
  \hline
  \hline
  1 & 1 & 2 \\ 
  \hline
  2 & 2 & 1 \\ 
  \hline
\end{tabular}

Observa que esta tabla es la misma que la del grupo aditivo de residuos módulo $2$, que es la misma que la tabla multiplicativa de residuos invertibles módulo $3$:

Hasta el momento únicamente hemos visto grupos finitos para los que la operación es conmutativa. 

Las permutaciones de $n$ elementos forman un grupo con respecto a la operación $\circledast$ de {\bf composición} de permutaciones, a partir de $n=3$ el grupo no es conmutativo.

La composición de permutaciones se define como la composición usual de funciones. 

Por ejemplo, sea $n=5$. Si tenemos dos permutaciones $\tau$ y $\sigma$ determinadas por las quintuplas $$(\sigma(1),\sigma(2),\sigma(3),\sigma(4), \sigma(5)), \quad (\tau(1),\tau(2),\tau(3),\tau(4),\tau(5)),$$
Las permutación $\sigma\circ \tau$ está determinada por la fórmula $$(\sigma\circ \tau(1),\sigma\circ \tau(2),\sigma\circ \tau(3),\sigma\circ \tau(4),\sigma\circ \tau(5))=(\sigma(\tau(1)),\sigma(\tau(2)),\sigma( \tau(3)),\sigma(\tau(4)),\sigma(\tau(5)))$$ 

En adelante vamos a omitir el símbolo $\circ$ como se hace cuando se multiplican variables y escribiremos $\sigma \tau$ en lugar de $\sigma \circ \tau$. 

Por ejemplo, supongamos que $$(\sigma(1),\sigma(2),\sigma(3),\sigma(4), \sigma(5))=(2,3,4,5,1), \quad (\tau(1),\tau(2),\tau(3),\tau(4),\tau(5))=(2,1,3,4,5),$$

Tenemos que $\sigma \tau$ está dado por $$(\sigma \tau(1),\sigma \tau(2),\sigma \tau(3),\sigma \tau(4), \sigma \tau (5))=(\sigma(\tau(1)),\sigma(\tau(2)),\sigma( \tau(3)),\sigma(\tau(4)),\sigma(\tau(5)))$$ $$=(\sigma(2),\sigma(1),\sigma(3),\sigma(4), \sigma(5))=(3,2,4,5,1).$$ 

De manera análoga $\tau \sigma$ está dado por $$(\tau\sigma(1),\tau\sigma(2), \tau(3)\sigma, \tau\sigma(4),  \tau \sigma (5))=(\tau(\sigma(1)),\tau(\sigma(2)),\tau(\sigma( 3)),\tau(\sigma(4)),\tau(\sigma(5)))$$ $$=(\tau(2),\tau(3),\tau(4),\tau(5),\tau(1))=(1,3,4,5,2).$$ 

Para cualquier grupo finito $G$ se puede definir el orden de un elemento $g\in G$.

\begin{definicion}
Sea $(G,\circledast)$ un grupo finito con neutro $e$. El orden de un elemento $g\in G$ se define como el mínimo entero positivo $o_{\circledast}(g)=m$ tal que $$\underbrace{g\circledast g\circledast g \circledast \dots \circledast g}_{o_{\circledast}(g)=m~\text{veces}}=g^{\circledast m}= e$$ 
\end{definicion}

\begin{ejercicio}
Demuestra que en cualquier grupo finito $(G, \circledast)$ y para cualquier $h\in G$, existe el orden $o(h)=m$.

Muestra que en ese caso los elementos de la lista $h, h^{\circledast 2}, h^{\circledast 3}, h^{\circledast 4}, \dots, h^{\circledast m}$ son todos distintos.
\end{ejercicio}

\begin{ejercicio}
Calcula los órdenes de las permutaciones $\sigma$, $\tau$ del ejemplo anterior. 
\end{ejercicio}

Para $n=1$ la situación no es muy interesante. Se obtiene el grupo trivial con un solo elemento $e$.

Como hemos visto, este grupo es idéntico (isomorfo) al grupo aditivo mod 1 (todo es congruente a 0 mod 1), o al grupo multiplicativo generado por el número 1. En todas estas situaciones solo hay un $e$ en el grupo y $e\circledast e=e$.

Para $n=2$ solo hay dos permutaciones: la identidad $(e(1),e(2))=(1,2)$ y la transposición $(\tau(1),\tau(2))=(2,1)$. La tabla para la composición se ve igual que la de los otros grupos de dos elementos que habíamos encontrado antes:

\begin{tabular}{|c||c|c|} 
 \hline
  $\circ$ & $e$ & $\tau$ \\ 
  \hline
  \hline
  $e$ & $e$ & $\tau$  \\ 
  \hline
  $\tau$ & $\tau$ & $e$ \\ 
  \hline
\end{tabular}

A partir de $3$ elementos el grupo de permutaciones no es conmutativo.

\begin{ejercicio}
Consideremos las permutaciones de tres elementos $$(\sigma(1),\sigma(2),\sigma(3))=(2,3,1),\quad (\tau(1), \tau(2), \tau(3))=(2,1,3).$$

Muestra que $\sigma \circ \tau$ no es la misma permutación que $\tau \circ \sigma$.

Muestra que la lista $\{e,\sigma, \sigma^2,\tau, \tau \sigma, \tau \sigma^2 \}$ contiene  cada una de las seis permutaciones de tres elementos.

Muestra que la lista $\{e,\tau, \sigma, \sigma\tau, \sigma^2, \sigma^2\tau \}$ contiene cada una de las seis permutaciones de tres elementos.

Completa las tablas de composiones de permutaciones (recuerda que ahora la operación no es conmutativa).

\end{ejercicio}

\begin{center}
\begin{tabular}{|c||c|c|c|c|c|c|} 
 \hline
  $\circ$ & $e$ & $\sigma$ & $\sigma^2$ & $\tau$ & $\tau\sigma$ & $\tau\sigma^2$ \\ 
  \hline
  \hline
$e$ & $e$ & $\sigma$ & $\sigma^2$ & $\tau$ & $\tau\sigma$ & $\tau\sigma^2$ \\
  \hline
$\sigma$ & $\sigma$ &  &  &  &  & \\
  \hline
$\sigma^2$ & $\sigma^2$ &  &  &  &  & \\
  \hline
$\tau$ & $\tau$ &  &  &  &  & \\
  \hline
$\tau\sigma$ & $\tau\sigma$ &  &  &  &  & \\
  \hline  
$\tau\sigma^2$ & $\tau\sigma^2$ &  &  &  &  & \\
    \hline
  \end{tabular}
\hspace{2cm}
\begin{tabular}{|c||c|c|c|c|c|c|} 
 \hline
  $\circ$ & $e$ & $\tau$ & $\sigma$ & $\sigma\tau$ & $\sigma^2$ & $\sigma^2\tau$ \\ 
  \hline
  \hline
$e$ & $e$ & $\tau$ & $\sigma$ & $\sigma\tau$ & $\sigma^2$ & $\sigma^2\tau$ \\
  \hline
$\tau$ & $\tau$ &  &  &  &  & \\
  \hline
$\sigma$ & $\sigma$ &  &  &  &  & \\
  \hline
$\sigma\tau$ & $\sigma\tau$ &  &  &  &  & \\
  \hline
$\sigma^2$ & $\sigma^2$ &  &  &  &  & \\
  \hline  
$\sigma^2\tau$ & $\sigma^2\tau$ &  &  &  &  & \\
    \hline
  \end{tabular}
\end{center}

Ahora estamos listos para demostrar el teorema de Lagrange.

\begin{teorema}[Teorema de Lagrange]
El orden de cualquier elemento de un grupo finito divide al número de elementos en el grupo. Es decir $$h^{|G|}=e$$  
\end{teorema}
Sug: llamemos $g_1=e$ y consideremos la lista de potencias del elemento $h$: $h, h^{\circledast 2}, h^{\circledast 3}, h^{\circledast 4}, \dots, h^{\circledast m}$ hasta que aparezca $e=h^{\circledast m}$. Por el ejercicio de arriba, todos estos elementos de $G$ son distintos.

Si la lista contiene a todos los elementos, terminamos pues el orden $m=|G|$, que en efecto divide a $|G|$.

Si la lista no contiene todos los elementos de $G$, consideremos alguno de los que faltan, digamos $g_2$, y consideremos la lista de elementos en la clase lateral $g_2h, g_2h^{\circledast 2}, g_2h^{\circledast 3}, g_2h^{\circledast 4}, \dots, g_2h^{\circledast m}$. Nuevamente se puede mostrar que todos los elementos de esta lista son distintos entre sí y son distintos a los elementos de la primera lista. 

Si después de esto hemos listado todos los elementos de $G$, entonces $|G|=2m$. 

En caso contrario tomamos otro elemento $g_3$ que no esté en ninguna de las listas y procedemos de la misma manera. El proceso debe terminar en algún momento porque hay un número finito de elementos. Como en cada paso se añaden exactamente $m$ elementos nuevos, $m\mid G$.

Para el caso del grupo aditivo de residuos, el Teorema de Lagrange no es muy sorprendente, pues nos dice que el orden aditivo es divisor de $n$, lo cual ya habíamos visto por nuestra cuenta antes.

Para el caso de permutaciones, el Teorema de Lagrange nos dice que el orden multiplicativo de una permutación de $n$ elementos es un divisor de $n!$. Aunque apreciamos esta información, no es suficientemente descriptiva. 

Pensándole mejor a las permutaciones podemos obtener una fórmula bonita para calcular exactamente el orden de una permutación:

(Añadir ejemplo)

\begin{ejercicio}
Calcular ordenes multiplicativos de las siguiente permutaciones:
\end{ejercicio}

\begin{ejercicio}
¿Se te ocurre una formula para calcular el orden multiplicativo de una permutación en general?
\end{ejercicio}

\section{Teoremas de Wilson, Fermat y Euler}

Para el caso del grupo multiplicativo de residuos invertibles módulo $n$, el Teorema de Lagrange justamente demuestra el Teorema de Euler.

\begin{ejercicio}
Sea $p$ un primo. Demuestra que $x^2\equiv 1~(\mathrm {mod} p)$ si y solo si $x\equiv \pm 1~(\mathrm {mod} p
)$
\end{ejercicio}

\begin{ejercicio}[Teorema de Wilson]
Sea $p$ un número primo. Entonces $$(p-1)!\equiv -1~(\mathrm {mod} p)$$
\end{ejercicio}

\begin{ejercicio}[Pequeño Teorema de Fermat]
Sean $a$ un número entero y $p$ un número primo. Entonces $$p|a^p-a$$
\end{ejercicio}

\begin{ejercicio}[Teorema de Euler]
Sea $n$ un número entero positivo y $a$ un entero, coprimo con $n$. Entonces $$a^{\varphi (n)}\equiv 1~(\mathrm {mod} n)$$
\end{ejercicio}

\begin{ejercicio}
Encuentra el entero positivo capicúa $n$ más pequeño tal que $2022|1001^{2018}-n$.
\end{ejercicio}
%$2022=2\cdot 3\cdot 337$. Como $1001=7\cdot 11\cdot 13$, $(2022,1001)=1$. Entonces $\varphi(2022)=1\cdot 2\cdot 336=672$ y como $2018=672\cdot 3+2$, tenemos que $1001^{2018}\equiv 1001^2~(\mathrm {mod} 2022)$. Como $1001^2\equiv 1111~(\mathrm {mod} 2022)$, el mínimo natural tal que $2022\mid 1001^{2018}-n$ es $n=1111$. Como además $1111$ es capicúa, es la respuesta al problema.
\newpage

\section{Ejercicios y Problemas}

%% Popurrí de Teoremas sencillos.

\begin{ejercicio}
  Demuestra que para todo $n > 0$ se cumple $n^2 \mid (n+1)^n - 1$.
\end{ejercicio}

\begin{ejercicio}
  Sea $a$ un número entero.

  \begin{enumerate}
  \item[a)] Demuestra que $6 \mid a\,(a+1)\,(a+2)$.

  \item[b)] Demuestra que si $a$ es impar, entonces $8 \mid (a^2 - 1)$.

  \item[c)] Demuestra que si $3\nmid a$, entonces $6 \mid (a^2 - 1)$.
  \end{enumerate}
\end{ejercicio}

\begin{ejercicio}
  ¿Para cuáles $a$ se cumple $a+1 \mid a^2 + 1$?
\end{ejercicio}

\begin{ejercicio}
  Demuestra que $(n+1) \mid {2n \choose n}$ para todo $n = 0,1,2,\ldots$
\end{ejercicio}

\begin{ejercicio}
  Demuestra que para todo $a > 0$ el número $3\,(1^5 + 2^5 + \cdots + a^5)$ es
  divisible por $1^3 + 2^3 + \cdots + a^3$.
\end{ejercicio}

\begin{ejercicio}
  Demuestra que para cualesquiera $a,b,c \in \ZZ$ se cumple
  $$9 \mid (a^3 + b^3 + c^3) \Longrightarrow 3 \mid abc.$$
\end{ejercicio}

\begin{ejercicio}
  Sea $a$ un entero.

  \begin{enumerate}
  \item[a)] Demuestra que para $n = 1,2,3,\ldots$ los números $a$ y $a^n$ tienen
    la misma paridad.

  \item[b)] Demuestra que si $a$ es impar, entonces el residuo de división de
    $a^2$ por $8$ es igual a $1$.
  \end{enumerate}
\end{ejercicio}

\begin{ejercicio}
  \label{ejerc:criterios-de-divisibilidad}
  Expresemos un entero $a \ge 0$ en la base $10$:
  $$a = a_0 + a_1\cdot 10 + a_2\cdot 10^2 + \cdots + a_k\cdot 10^k.$$

  Demuestre los siguientes criterios de divisibilidad.

  \begin{itemize}
  \item $2\mid a$ si y solamente si el último dígito de $a$ es par
    ($0, 2, 4, 6, 8$).

  \item $4\mid a$ si y solamente si los últimos dos dígitos forman un número
    $a_1 a_0$ que es divisible por $4$.

  \item $5\mid a$ si y solamente si el último dígito $a_0$ es $0$ o $5$.

  \item $10\mid a$ si y solamente si el último dígito $a_0$ es $0$.

  \item $11\mid a$ si y solamente si la suma alternante de los dígitos
    $\sum_i (-1)^i\,a_i$ es divisible por $11$, por ejemplo
    \[ 11\mid 87109, \quad 8 - 7 + 1 - 0 + 9 = 11. \]
  \end{itemize}
\end{ejercicio}

También existen criterios de divisibilidad por $7$, $13$, etc. pero son más
complicados y al mismo tiempo bastante inútiles.


\begin{problema}
%[OMM '87]
¿Cuántos enteros positivos dividen al número $20!$ (veinte factorial)?
\end{problema}

\begin{problema}
%[OMM '87]
Muestra que para todo $n$ entero positivo se tiene que $(n^3-n)(5^{8n+1}-3^{4n+2})$ es divisible entre $3804$.
\end{problema}
%\vspace{3cm}

\begin{problema}
%[OMM '87]
Muestra que para todo $n$ entero positivo se tiene que $$\frac{n^2+n-1}{n^2+2n}$$ es una fracción irreducible.
\end{problema}
%\vspace{3cm}

\begin{problema}
%[OMM '88]
Si $a$ y $b$ son enteros positivos, muestre que $19\mid 11a+2b$ si y solo si $19\mid 18a+5b$
\end{problema}
%\vspace{3cm}

\begin{problema}
%[OMM '89]
Encuentra enteros positivos $a$ y $b$ tales que:

$b^2$ sea múltiplo de $a$,

$a^3$ sea múltiplo de $b^2$,

$b^4$ sea múltiplo de $a^3$,

$a^5$ sea múltiplo de $b^4$,

pero $b^6$ no sea múltiplo de $a^5$.
\end{problema}
%Sug: intentar $a=p^m$, $b=p^n$. Condiciones se transforman en $$m\leq 2n\leq 3m\leq 4n \leq 5m >6n.$$ 
%\vspace{3cm}

\newpage
\begin{problema}
%[OMM '90]
  Demuestra que para todo $n > 0$ se cumple $n^2 \mid (n+1)^n - 1$.
\end{problema}
%Sug: Teorema del binomio.
%\vspace{3cm}


\begin{problema}
%[OMM '90]
  Se tiene una colección de diecinueve puntos con coordenadas enteras $\{P_1,P_2,\dots, P_19\}$, no tres colineales. Demuestra que hay un triángulo formado por tres de esos puntos, cuyo baricentro también tiene coordenadas enteras.
\end{problema}
%Sug: Congruencias módulo 3 en las coordenadas y casillas.
%\vspace{3cm}


\begin{problema}
%[OMM '04]
Encuentra todas las ternas de primos $p<q<r$, tales que $pqr+1$ es cuadrado perfecto y $25pq+r=2004$
\end{problema}
%Sug: Congruencias en la segunda propiedad para descartar casi todos los casos.
\vspace{3cm}


\begin{problema}
%[OMM '01]
Encuentra todos los números de siete dígitos con solo $3$'s y $7$'s que sean múltiplos de $3$ y de $7$.
\end{problema}

\section{Algoritmo Euclidiano de la división con residuo}


Cuando $b \nmid a$, la división $\frac{a}{b}$ no es posible en números enteros,
pero se puede usar la \textbf{división con residuo}. Esta también se conoce como
la \textbf{división euclidiana}, ya que aparece en los «Elementos» de Euclides.

\begin{proposicion}[División con residuo]
  Para dos números enteros $a$ y $b \ne 0$, existen $q$ (cociente) y $r$
  (residuo) tales que
  \[ a = qb + r,
    \quad
    0 \le r < |b|. \]
  Además, estas propiedades definen a $q$ y $r$ de manera única.
\end{proposicion}

La demostración de este teorema se encuentra en el apéndice.

La división con residuo se usa muy a menudo en la vida cotidiana. Por ejemplo,
en lugar de «$\frac{5}{4}$» a veces se escribe «$1\frac{1}{4}$».

\begin{ejemplo}
  Tenemos $b \mid a$ si y solamente si $r = 0$ y $q = a/b$.
\end{ejemplo}

\begin{ejemplo}
  Para $a = 15$ y $b = 7$ se tiene $15 = 2\cdot 7 + 1$, así que $(q,r) = (2,1)$.
\end{ejemplo}

\begin{ejemplo}
  El residuo de división por $b = 2$ es $r = 0$, cuando $a = 2q$ es un
  \textbf{número par} y el residuo es $r = 1$ cuando $q = 2q+1$ es un
  \textbf{número impar}.
\end{ejemplo}

He aquí una aplicación de la división con residuo.

\begin{proposicion}
  Para un entero $a > 1$ y números naturales $m, n$, se tiene
  $$(a^m - 1) \mid (a^n-1) \iff m \mid n.$$

  \begin{proof}
    Dividiendo con residuo
    $n = qm + r$,
    \begin{align*}
      \frac{a^n-1}{a^m-1} & = \frac{(a^{qm + r} - a^r) + (a^r - 1)}{a^m - 1} \\
                          & = \frac{a^{qm} - 1}{a^m - 1}\,a^r + \frac{a^r - 1}{a^m - 1} \\
                          & = \underbrace{a^r \, \sum_{0 \le i < q} a^{im}}_{\text{entero}} + \frac{a^r - 1}{a^m - 1}.
    \end{align*}
    Entonces,
    $$(a^m - 1) \mid (a^n-1) \iff (a^m - 1) \mid (a^r - 1).$$
    Pero ojo: siendo el residuo de división por $m$, sabemos que $r < |m|$, así
    que la única opción es $r = 0$. Entonces, $m \mid n$.
  \end{proof}
\end{proposicion}

%%%%%%%%%%%%%%%%%%%%%%%%%%%%%%%%%%%%%%%%%%%%%%%%%%%%%%%%%%%%%%%%%%%%%%%%%%%%%%%%

\section{Descomposición en base $b$}

El siguiente resultado, seguramente conocido al lector para el caso de $b = 10$,
también se demuestra usando la división con residuo.

\begin{teorema}
  Fijemos un entero $b \ge 2$. Todo entero $a \ge 0$ puede ser escrito como
  \begin{equation}
    \label{eqn:expresion-en-base-b}
    a = a_0 + a_1\,b + a_2\,b^2 + \cdots + a_k\,b^k,
  \end{equation}
  donde $0 \le a_i \le b-1$ y $a_k \ne 0$. Además, esta expresión es única.

  \begin{proof}
    Usando división con residuo, podemos escribir sucesivamente, hasta obtener
    $q_{k+1} = 0$,
    \begin{align*}
      a & = b q_0 + a_0 \\
        & = b \, (b q_1 + a_1) + a_0 \\
        & = b \, (b \, (b q_2 + a_2) + a_1) + a_0 \\
        & = \cdots \\
        & = a_0 + a_1\,b + a_2\,b^2 + \cdots + a_k\,b^k.
    \end{align*}

    Para la unicidad, supongamos que
    $$a = a_0' + a_1'\,b + a_2'\,b^2 + \cdots + a_k'\,b^k.$$
    Sin pérdida de generalidad, $a_0 \ge a_0'$. En este caso $a_0 - a_0'$ es un
    múltiplo de $b$, y además $0 \le a_0 - a_0' \le b-1$, así que $a_0 = a_0'$.
    Podemos pasar a los números $\frac{a - a_0}{b}$ y $\frac{a' - a_0'}{b}$ para
    concluir que $a_1 = a_1'$, etcétera.
  \end{proof}
\end{teorema}

\begin{comentario}
  Normalmente la expresión \eqref{eqn:expresion-en-base-b} se escribe como
  $$a_k a_{k-1} \cdots a_1 a_0.$$
  Por ejemplo,
  $$12345 = 10^4 + 2\cdot 10^3 + 3\cdot 10^2 + 4\cdot 10 + 5.$$
\end{comentario}

\begin{comentario}
  En la vida cotidiana se usa la base \textbf{decimal} ($b = 10$).

  En la informática son comunes la base \textbf{binaria} ($b = 2$),
  \textbf{octal} ($b = 8$), y \textbf{hexadecimal} ($b = 16$). Los dígitos
  hexadecimales más allá de $9$ normalmente se denotan por $A,B,C,D,E,F$.

  En el fondo, todos los datos en la computadora se representan como una
  sucesión de unos y ceros, es decir en la base binaria. De allí vienen las
  unidades tradicionales de información:


  \begin{center}
    \renewcommand{\arraystretch}{1.5}
    \begin{tabular}{lll}
      \hline
      \textbf{bit} & & dígito $0$ ó $1$ \\
      \hline
      \textbf{byte} & (\textbf{B}; \textbf{octeto}) & $8$ bits \\
      \hline
      \textbf{kilobyte} & (\textbf{KB}) & $2^{10}$ bytes \\
      \hline
      \textbf{megabyte} & (\textbf{MB}) & $2^{20}$ bytes \\
      \hline
      \textbf{gigabyte} & (\textbf{GB}) & $2^{30}$ bytes \\
      \hline
      \textbf{terabyte} & (\textbf{TB}) & $2^{40}$ bytes \\
      \hline
      \dots & \dots
    \end{tabular}
  \end{center}
    En el sistéma métrico los prefijos \emph{kilo-},
    \emph{mega-}, \emph{giga-}, \emph{tera-} significan $10^3$, $10^6$, $10^9$,
    $10^{12}$. Los mercadotécnicos sacaron provecho de esta confusión, y por
    esto un disco duro marcado «$1$~TB» normalmente contiene $10^{12}$ bytes,
    mucho menos de $2^{40}$. Así un disco «de $1$ terabyte» contiene un poco más
    de $931$ verdaderos gigabytes.
\end{comentario}

\begin{ejemplo}
  Para expresar $2021$ en la base $3$, podemos ecribir
  \begin{align*}
    2021 & = 3\cdot 673 + 2 \\
         & = 3\cdot (3\cdot 224 + 1) + 2 \\
         & = 3\cdot (3\cdot (3\cdot 74 + 2) + 1) + 2 \\
         & = 3\cdot (3\cdot (3\cdot (3\cdot 24 + 2) + 2) + 1) + 2 \\
         & = 3\cdot (3\cdot (3\cdot (3\cdot (3\cdot 8 + 0) + 2) + 2) + 1) + 2 \\
         & = 3\cdot (3\cdot (3\cdot (3\cdot (3\cdot (3\cdot \boxed{2} + \boxed{2}) + \boxed{0}) + \boxed{2}) + \boxed{2}) + \boxed{1}) + \boxed{2} \\
         & = 2 + 3 + 2\cdot 3^2 + 2\cdot 3^3 + 2\cdot 3^5 + 2\cdot 3^6.
  \end{align*}
\end{ejemplo}

\begin{comentario}
  Los números reales también admiten una expresión en la base $b$
  \[
    x = a_k a_{k-1} \cdots a_1 a_0, a_{-1} a_{-2} a_{-3} \cdots
    \longleftrightarrow
    x = \sum_i a_i\cdot b^i.
  \]
  Los dígitos no son exactamente únicos, como por ejemplo en el caso de
  $$1,000000\ldots = 0,999999\ldots$$

  El $x$ de arriba es un número racional si y solamente si los dígitos son
  «eventualmente periódicos»: es decir existe $n$ tal que $a_{-i} = a_{-(i+n)}$
  para todo $i$ suficientemente grande. 
\end{comentario}

\begin{proposicion}
  El número de los dígitos de $a \ge 0$ en la base $b \ge 2$ es igual a
  $$\lfloor\log_b (a)\rfloor + 1.$$

  \begin{proof}
    Notamos que $a$ tiene $n$ dígitos en la base $b$ si y solamente si
    \[
      b^{n-1} \le a < b^n
      \iff
      n-1 \le \log_b (a) < n
      \iff
      n = \lfloor\log_b (a)\rfloor + 1.
      \qedhere
    \]
  \end{proof}
\end{proposicion}

Como consecuencia, si pasamos de base $b_1$ a otra base $b_2 > b_1$, el número
de dígitos necesarios se disminuye proporcionalmente, con factor
$\log (b_2) / \log (b_1)$. En este sentido, la elección de base no es muy
importante, lo importante es no usar la base unaria
\[ 1 = |, ~ 2 = ||, ~ 3 = |||, ~ 4 = ||||, ~ \ldots \]

\begin{ejemplo}
  Hay evidencia de que hace cerca de 360 millones
    de años todavía había animales vertebrados con seis, siete, u ocho dedos en
    sus extremidades.
  Si la especie humana tuviera ocho dedos en cada mano en lugar de cinco,
  seguramente usaríamos la base hexadecimal. Esto nos daría una economía en
  dígitos de $\log(16)/\log(10) = 1.204119\ldots$ alrededor de $20\%$, que no es
  mucho.
\end{ejemplo}

\begin{ejemplo}
  Tenemos $3^6 < 2021 < 3^7$, así que $6 < \log_3 (2021) < 7$. Entonces,
  $a = 2021$ tiene $7$ dígitos en la base $3$.
\end{ejemplo}

Se conocen varios «criterios de divisibilidad» que se formulan en términos de la
expresión en la base $b$, normalmente $b = 10$. Vamos a probar el criterio de
divisibilidad por $3$, y dejaremos algunos otros criterios en el
ejercicio~\ref{ejerc:criterios-de-divisibilidad}.

\begin{proposicion}
  Expresemos un entero $a \ge 0$ en la base $10$:
  $$a = a_0 + a_1\cdot 10 + a_2\cdot 10^2 + \cdots + a_k\cdot 10^k.$$

  Ahora $3\mid a$ si y solamente si $3 \mid \sum_i a_i$.

  \begin{proof}
    Notamos que para cualquier $i \ge 1$, la división de $10^i$ por $3$ da
    residuo $1$. De esta forma se obtiene la expresión
    \[ a = a_0 + a_1 + a_2 + \cdots + a_k + (\text{algo divisible por }3). \qedhere \]
  \end{proof}
\end{proposicion}

La última proposición es un típico ejemplo de resultados acerca de los dígitos
de un número en cierta base. Estos normalmente no son muy profundos y pertenecen
al terreno de las «matemáticas recreativas». No hay que olvidar que la expresión
en base $b$ es nada más una manera cómoda de escribir los números.

En el último argumento, la idea de ignorar el resto de términos que son
divisibles por $3$ es algo que se llama la «reducción módulo $3$».

\begin{ejercicio}
  Consideremos los números que son sumas de diferentes potencias de $3$:
  \begin{align*}
    a_1 & = 3^0 = 1, \\
    a_2 & = 3^1 = 3, \\
    a_3 & = 3^0 + 3^1 = 4, \\
    a_4 & = 3^2 = 9, \\
    a_5 & = 3^0 + 3^2 = 10, \\
    a_6 & = 3^1 + 3^2 = 12, \\
    a_7 & = 3^0 + 3^1 + 3^2 = 13, \\
        & \cdots
  \end{align*}
  Encuentre el número $a_{100}$ en esta sucesión.
\end{ejercicio}

\begin{ejercicio}
  ¿Cuántos dígitos binarios tiene el número $10^n$?
\end{ejercicio}

\begin{ejercicio}
  ¿Cuántos dígitos decimales tiene $n!$ para $n = 2021$?
\end{ejercicio}

\begin{ejercicio}
  En la base hexadecimal los dígitos normalmente se denotan por
  $$0,1,2,3,4,5,6,7,8,9,A,B,C,D,E,F.$$
  Con ayuda de calculadora, exprese en la base decimal los números hexadecimales
  $BADCAFE$ y $DEADBEEF$.
\end{ejercicio}

A parte de la expresión en la base $b$ de la forma $\sum_i a_i\,b^i$ con
$0 \le a_i < b$, existen otras representaciones un poco más exóticas. Vamos a
explorar un par de estas en el siguiente ejercicio.

\begin{ejercicio}
  Sea $a \ge 0$ un número entero.

  \begin{enumerate}
  \item[a)] Demuestre que $a$ puede ser escrito de manera única como

    $$a = a_0 + a_1\,3 + a_2\,3^2 + \cdots + a_k\,3^k,$$
    donde $a_i \in \{ -1, 0, +1 \}$.

  \item[b)] Demuestre lo mismo con $b = 2n+1$ en lugar de $3$ los dígitos
    $$a_i \in \{ -n, \ldots, -1, 0, +1, \ldots, +n \}.$$

  \item[c)] Demuestre que $a$ puede ser escrito de manera única como
    $$a = a_1\cdot 1! + a_2\cdot 2! + a_3\,3! + \cdots + a_k\cdot k!,$$
    donde $0 \le a_i \le i$.
  \end{enumerate}

  ¿Cómo se expresa $a = 100$ en cada una de estas bases?
\end{ejercicio}

\begin{ejercicio}[N.\,Anning]
  Demuestre que la fracción
  $$\frac{101010101}{110010011}$$
  tiene el mismo valor si «$1$» en el medio del numerador y denominador se
  remplaza por un número impar de $1$'s:
  \[
    \frac{101010101}{110010011} =
    \frac{10101110101}{11001110011} =
    \frac{1010111110101}{1100111110011} = \cdots
  \]

  Esto es válido en cualquier base $b$.
\end{ejercicio}


 %%%%%%%%%%%%%%%%%%%%%%%%%%%%%%%%%%%%%%%%%%%%%%%%%%%%%%%%%%%%%%%%%%%%%%%%%%%%%%%%

% \chapter{Funciones, polinomios y series}

\section{Funciones}

\section{Polinomios}

\section{La pendiente de una función}

\section{La serie exponencial, el seno y el coseno}

\section{La identidad de Euler}

% 
\chapter{Elementos de probabilidad y estadística}

La 

\section{Introducción}

\section{Problemas básicos de combinatoria y probabilidad}

\begin{ejercicio}
Considera una baraja inglesa de $52$, cartas, trece de cada figura $\{\clubsuit
,\vardiamondsuit,\spadesuit,\varheartsuit\}$. 
Te dan cinco cartas al azar. 
\vspace{.3cm}


¿Cuál es la probabilidad de obtener:

\begin{itemize}
\item dos pares? (es decir: dos pares distintos de números y otro número distinto. )
\item un full? (dos pares y una tercia)
\item una escalera? (cinco cartas con números consecutivos)
\item una flor? (cinco cartas con la misma figura)
\item un pokar? (cuatro números iguales)
\item una flor escalera?
\end{itemize}
\end{ejercicio}


\begin{ejercicio}
Una caja grande contiene diez pares distintos de zapatos. Se extraen de la caja, al azar, $r\leq 20$ zapatos. 

¿Cuál es la probabilidad de que no se haya extraído ningún par de zapatos?

¿Cuál es la probabilidad de que se extraiga exactamente un par de zapatos?

¿Cuál es la probabilidad de que se extraigan exactamente dos pares de zapatos?
\end{ejercicio}


\begin{ejercicio}
A una fiesta asisten siete personas, cada una con su sombrero. De repente sonó una alarma, se apagó la luz y todos se salieron de la fiesta tomando un sombrero al azar. ¿Cuál es la probabilidad de que nadie se haya llevado su propio sombrero?
\end{ejercicio}


\section{Urnas}

\begin{ejercicio}
ejercicio de urnas sin reemplazo.
\end{ejercicio}

\begin{ejercicio}
ejercicio de urnas con reemplazo.
\end{ejercicio}
%(se obtienen distribuciones binomiales o multinomiales)

\begin{ejercicio}
ejercicio de urnas con $c$-extra-reemplazo...
\end{ejercicio}
%(se obtienen distribuciones binomiales o multinomiales)

\begin{ejercicio}
Hay tres urnas A, B y C con canicas blancas y negras.

La urna A tiene $10$ canicas negras y $20$ blancas.

La urna B tiene $15$ canicas negras y $15$ blancas.

La urna C tiene $40$ canicas negras y $10$ blancas.

Si se elige una urna al azar y luego se elige una canica al azar, cuál es la probabilidad de que se obtenga una canica negra?.
\end{ejercicio}

\begin{ejercicio}
% [IWYMICI '19]
Hay tres urnas A, B y C, que contienen 100, 80 y 50 canicas (algunas negras y algunas blancas), del mismo tamaño.

La urna A tiene 15 canicas negras.

Seleccionamos una urna al azar, y después se extrae una canica al azar.

Si la probabilidad de obtener una canica negra es exactamente $\frac{101}{600}$, ¿Cuál es el máximo número posible de canicas negras que puede tener la caja C?
\end{ejercicio}

¿Qué pasa cuando la probabilidad de elegir cada urna no es uniforme?

\begin{ejercicio}

Hay tres urnas A, B y C con canicas blancas y negras, como en el ejercicio anterior.

Esta vez, para elegir una urna, primero arrojo un dado. 

Si obtengo 1,2,o 3, elijo la urna 1. 

Si obtengo 4 o 5 elijo la urna B, y 

Si obtengo 6 elijo la urna C. 

Ya que elegí la urna con el dado, tomo canica al azar de la urna ganadora.

¿Cuál es la probabilidad de que obtenga una canica negra?.
\end{ejercicio}

\section{Variables aleatorias discretas y contínuas}

Una variable aleatoria es una función especial que se define formalmente en el contexto de teoría de la medida. En esta introducción no vamos a entrar en este tipo de detalles técnicos.

Por medio de variables aleatorias podemos expresar enunciados sobre probabilidades de fenómenos aleatorios de manera precisa y compacta, empleando los conceptos fundamentales asociados a dichas variables aleatorias, como lo son sus distribuciones, medias, varianzas, momentos, cumulantes, funciones características, que iremos introduciendo a lo largo de esta sección.

Un primer aspecto sobre una variable aleatoria es que esta puede ser abstracta, puede tener valores en conjuntos bastante generales, o bien puede tener valores en conjuntos muy bien estructurados como los números naturales, los reales, los complejos, etc.

Ejemplos básicos típicos de variables aleatorias son los resultados de lanzar dados o monedas. Un ejemplo que es mucho más útil para introducir de manera general el concepto de variable aleatoria discreta es el de una ruleta. 

\subsection{Variables aleatorias discretas: ruletas}

Pensemos en un disco cortado como pizza desde el centro en rebanadas de tamaños $p_1, p_2, p_3, \dots p_k$, donde $1=p_1+p_2+\cdots +p_k$. 

Si giramos el disco con fuerza, como si fuera una ruleta con una aguja (sin que hayamos entrenado como girarla), la probabilidad de que la aguja caiga en la rebanada número $(1, 2, 3, \dots, k)$ serían exactamente $(p_1, p_2, p_3, \dots, p_k)$.

¿Nos conviene o no girar la ruleta de la fortuna?

La respuesta depende de los premios o castigos que se asignen a cada rebanada de la ruleta. Una variable aleatoria es un concepto muy genérico, y los premios o castigos, como en los concursos de la tele o las series de ficción, pueden tener valores en conjuntos completamente distintos (e.g. ganar o perder cierta cantidad de dinero, grande o pequeña, ganar un auto, un viaje a la playa, una cabra). 

En ese sentido la conveniencia de un juego de azar es subjetiva puesto que depende del valor que cada jugador le asigne a los premios o castigos. Para que la conveniencia de un juego resulte un poco menos subjetiva, concentrémonos por el momento en una ruleta para la cual lo que se gana o se pierde siempre es dinero, en diversas cantidades, con distintas probabilidades.

Hay ruletas especiales que representan a variables aleatorias especiales. Es importante irse familiarizando con los nombres de las variables aleatorias básicas y los fenómenos aleatorios que describen.

\subsubsection*{Variables aleatorias constantes}

Antes de discutir las variables \emph{verdaderamente aleatorias}, es importante mencionar que las variables contantes también entran como un caso especial, dentro del universo de las variables aleatorias, como aquellas que solo tienen un único posible resultado y este ocurre con probabilidad $1$.

La forma en que uno modela una variable aleatoria constante $X = c$ es haciendo que esta tenga solo un único valor, completamente determinístico. Por ejemplo, si en los dos lados de una moneda ponemos cruz, la variable aleatoria $X$ asociada a lanzar esa moneda será la variable constante cruz: $\mathbb P(X=\text{cruz})=1$. Si en todas las caras de un dado pongo el número $4$, entonces el dado es la variable constante igual a 4, es decir $\mathbb P(X=4)=1$. 

De la misma manera, si en todas las rebanadas de una ruleta ponemos auto, entonces el ejercicio de girar la ruleta es predecible, sé que siempre obtendré un auto. $\mathbb P(X=\text{auto})=1$

Ya que dejamos en claro a lo que nos referimos con una variable aleatoria constante, pasemos a las variables aleatorias (verdaderamente aleatorias).

\subsubsection*{Variables aleatorias con solo dos resultados}

Las variables aleatorias no-constantes más sencillas son las que tienen dos posibles resultados: $a$ y $b$ con probabilidades respectivas $p$ y $1-p$, con $0 < p < 1$. En notación abreviada, $$\mathbb P (X=a)=p,\quad \mathbb P (X=b)=1-p.$$

Existen algunos casos particulares notables, cada uno tiene sus ventajas.

El juego aleatorio más sencillo es el lanzamiento de una moneda justa en donde los resultados son cara o cruz. Lo podemos representar con una variable aleatoria $X$ tal que
$$\mathbb P (X=\text{cara})=\frac{1}{2},\quad \mathbb P (X=\text{cruz})=\frac{1}{2}.$$

Una alternativa a escribir cara o cruz, es ponerle números a los lados de la moneda. Lo más simple es ponerles a las caras los valores $0$ y $1$, dando origen a la variable aleatoria Bernoulli positiva, o ponerles $-1$ y $1$, dando origen a la variable aleatoria Bernoulli simétrica.

La ventaja de trabajar con la monedas con números en sus caras es que podemos realizar operaciones aritméticas para expresar de manera muy sencilla, fenomenos aleatorios más complicados que se construyan con estas monedas, como las variables aleatorias binomiales.

La Bernoulli positiva

Bernoulli simétrica:

Binomial.

Dado

Suma de 2 dados

Suma de 3 dados

Multinomial.

Poisson.

Paradoja de St. Petersburgo


Figs: Ruleta dado. Ruleta cara o cruz.  Ruleta Bernoulli simétrica. Ruleta Bernoulli positiva. Ruletas binomiales. Ruleta lotería. Ruleta infinita discreta. Ruleta contínua \vspace{4cm}









Por ejemplo, consideremos que las variable aleatorias $X, Y, Z$ representadas por las siguientes ruletas: 

Figura: Tres ruletas.

La variable aleatoria $X$ es el resultado (aleatorio) de girar una ruleta con probabilidades $(0.1, 0.3, 0.6)$ y respectivos valores (pagos) $(35, -5, -3)$.

¿Me conviene jugar esta ruleta?. Una respuesta parcial objetiva nos la otorgan la media $\mathbb E (X)$ y la varianza $\mathrm{Var}(X)$, de la variable aleatoria $X$. Para una variable aleatorias discreta, como nuestra ruleta, la esperanza, media o valor esperado se define como la suma de los posibles resultados, ponderados por la probabilidad de que ocurran. Por ejemplo, la ruleta anterior tenemos 
$$\mathbb E (X)=p_1v_1+p_2v_2+p_3v_3=(.1)(35)+(.3)(-5)+(.6)(-3)=3.5-1.5-1.8=0.2.$$

La varianza se define como la suma ponderada de los cuadrados de las desviaciones de la media.

$$\mathbb E (X)=p_1v_1+p_2v_2+p_3v_3=(.1)(35)+(.3)(-5)+(.6)(-3)=3.5-1.5-1.8=0.2.$$

La decisión de si me conviene o no desde algo sencillo como juego de azar simple, o algo más sofisticado, como la contratación de algún seguro, además de parámetros objetivos, como la media, dependen de mi situación personal.

Por ejemplo. La media de la ruleta $X$ es positiva, lo que en principio me sugiere que es buena idea aceptar la apuesta porque la media me favorece a largo plazo. Sin embargo, la probabilidad de perder $()$ es muy alta en comparación a la probabilidad de ganar.  

Sin embargo, los problemas probabilísticos, como los de la vida real, lamentablemente dependen del ancho de la cartera. Si se me presenta la ruleta $X$ pero yo solo tengo 

Ejemplos de distribuciones discretas famosas básicas.



\subsection{El juego de Cardano y la importancia de la media a largo plazo}

Un ejemplo sobre un juego de azar que se jugaba en el siglo XVI era el siguiente.

\begin{ejercicio}
Se lanzan tres dados justos y se suman los valores de los tres dados. ¿Qué es mas probable: obtener como suma un $10$ o un $11$, o que la suma de $9$ o $12$?
\end{ejercicio}
%Sug: Hay la misma cantidad de configuraciones distintas (seis) para obtener 9 y 10, pero como estas configuraciones no ocurren con la misma frecuencia, resulta más probable obtener un 10 que un 9.

\subsection{La variabilidad a largo plazo}


\subsection{La variabilidad a largo plazo}


ver Geogebra: rknp4m6s

\definecolor{qqwuqq}{rgb}{0.,0.39215686274509803,0.}
\begin{tikzpicture}[line cap=round,line join=round,>=triangle 45,x=1.0cm,y=1.0cm]
\begin{axis}[
x=1.0cm,y=1.0cm,
axis lines=middle,
xmin=-0.7600000000000002,
xmax=4.820000000000001,
ymin=-1.24,
ymax=3.120000000000001,
xtick={-0.5,0.0,...,4.5},
ytick={-1.0,-0.5,...,3.0},]
\clip(-0.76,-1.24) rectangle (4.82,3.12);
\draw[line width=2.pt,color=qqwuqq] (-0.7600000000000002,0.0) -- (2.197399999999994,0.0);
\draw[line width=2.pt,color=qqwuqq] (2.197399999999994,0.0) -- (2.2671499999999933,0.0034262426287961457);
\draw[line width=2.pt,color=qqwuqq] (2.2671499999999933,0.0034262426287961457) -- (2.350849999999993,0.014546955347326071);
\draw[line width=2.pt,color=qqwuqq] (2.350849999999993,0.014546955347326071) -- (2.4205999999999923,0.042458356499313135);
\draw[line width=2.pt,color=qqwuqq] (2.4205999999999923,0.042458356499313135) -- (2.5042999999999918,0.1307578479219581);
\draw[line width=2.pt,color=qqwuqq] (2.5042999999999918,0.1307578479219581) -- (2.587999999999991,0.3379940353694778);
\draw[line width=2.pt,color=qqwuqq] (2.587999999999991,0.3379940353694778) -- (2.74144999999999,1.223201605605937);
\draw[line width=2.pt,color=qqwuqq] (2.74144999999999,1.223201605605937) -- (2.8251499999999896,1.9249816558376494);
\draw[line width=2.pt,color=qqwuqq] (2.8251499999999896,1.9249816558376494) -- (2.908849999999989,2.5426807510177);
\draw[line width=2.pt,color=qqwuqq] (2.908849999999989,2.5426807510177) -- (2.9925499999999885,2.8189914757160928);
\draw[line width=2.pt,color=qqwuqq] (2.903799999999991,2.512786934145086) -- (2.918699999999991,2.597246497716907);
\draw[line width=2.pt,color=qqwuqq] (2.918699999999991,2.597246497716907) -- (2.933599999999991,2.6696862856811285);
\draw[line width=2.pt,color=qqwuqq] (2.933599999999991,2.6696862856811285) -- (2.948499999999991,2.728957981318804);
\draw[line width=2.pt,color=qqwuqq] (2.948499999999991,2.728957981318804) -- (2.9633999999999907,2.7741058235073144);
\draw[line width=2.pt,color=qqwuqq] (2.9633999999999907,2.7741058235073144) -- (2.9782999999999906,2.8043922378766557);
\draw[line width=2.pt,color=qqwuqq] (2.9782999999999906,2.8043922378766557) -- (2.9931999999999905,2.819317880968331);
\draw[line width=2.pt,color=qqwuqq] (2.9931999999999905,2.819317880968331) -- (3.0080999999999904,2.818635336260315);
\draw[line width=2.pt,color=qqwuqq] (3.0080999999999904,2.818635336260315) -- (3.0229999999999904,2.802355937113414);
\draw[line width=2.pt,color=qqwuqq] (3.0229999999999904,2.802355937113414) -- (3.0378999999999903,2.7707494537073027);
\draw[line width=2.pt,color=qqwuqq] (3.0378999999999903,2.7707494537073027) -- (3.05279999999999,2.724336656050884);
\draw[line width=2.pt,color=qqwuqq] (3.05279999999999,2.724336656050884) -- (3.06769999999999,2.6638750395693704);
\draw[line width=2.pt,color=qqwuqq] (3.06769999999999,2.6638750395693704) -- (3.076249999999988,2.623205880463206);
\draw[line width=2.pt,color=qqwuqq] (3.076249999999988,2.623205880463206) -- (3.1599499999999874,2.0488382767690796);
\draw[line width=2.pt,color=qqwuqq] (3.1599499999999874,2.0488382767690796) -- (3.243649999999987,1.3431350061101264);
\draw[line width=2.pt,color=qqwuqq] (3.243649999999987,1.3431350061101264) -- (3.3273499999999863,0.7390406803304986);
\draw[line width=2.pt,color=qqwuqq] (3.3273499999999863,0.7390406803304986) -- (3.4110499999999857,0.34131369027116126);
\draw[line width=2.pt,color=qqwuqq] (3.4110499999999857,0.34131369027116126) -- (3.494749999999985,0.13230484837061976);
\draw[line width=2.pt,color=qqwuqq] (3.494749999999985,0.13230484837061976) -- (3.5784499999999846,0.04304616813369);
\draw[line width=2.pt,color=qqwuqq] (3.5784499999999846,0.04304616813369) -- (3.662149999999984,0.011755197312267002);
\draw[line width=2.pt,color=qqwuqq] (3.662149999999984,0.011755197312267002) -- (3.7458499999999835,0.002694399790310957);
\draw[line width=2.pt,color=qqwuqq] (3.7458499999999835,0.002694399790310957) -- (3.7597999999999834,0.0020722350280527164);
\draw[line width=2.pt,color=qqwuqq] (3.7597999999999834,0.0020722350280527164) -- (3.7737499999999833,0.0015859997526303379);
\draw[line width=2.pt,color=qqwuqq] (3.7737499999999833,0.0015859997526303379) -- (3.787699999999983,0.0012079649974740475);
\draw[line width=2.pt,color=qqwuqq] (3.787699999999983,0.0012079649974740475) -- (3.801649999999983,0.0);
\draw[line width=2.pt,color=qqwuqq] (3.801649999999983,0.0) -- (3.815599999999983,0.0);
\draw[line width=2.pt,color=qqwuqq] (3.815599999999983,0.0) -- (4.806050000000002,0.0);
\draw[color=qqwuqq] (-0.6519366197183101,0.08374717832957149) node {$f$};
\end{axis}
\end{tikzpicture}

% 
\chapter{Geometría y Trigonometría}

Funciones trigonométricas.

En la sección anterior derivamos una fórmulas para el área del triángulo. 

Desde chicos conocemos la fórmula básica para el área de un triángulo: Base por altura entre dos.

Existe una terna adicional de fórmulas para el área de un triángulo bastante útiles, que involucran los \emph{senos} de los ángulos del triángulo.\\

\definecolor{qqwuqq}{rgb}{0,0.39215686274509803,0}
\begin{figure}[h]
\begin{subfigure}{.45 \textwidth}
\begin{tikzpicture}[line cap=round,line join=round,>=triangle 45,x=1cm,y=1cm]
\draw [shift={(3.62,-0.64)},line width=1pt,color=qqwuqq,fill=qqwuqq,fill opacity=0.10000000149011612] (0,0) -- (-13.981204311452633:0.6) arc (-13.981204311452633:42.69262700303779:0.6) -- cycle;
\draw [shift={(6.46,1.98)},line width=1pt,color=qqwuqq,fill=qqwuqq,fill opacity=0.10000000149011612] (0,0) -- (-137.30737299696221:0.6) arc (-137.30737299696221:-61.78829573131603:0.6) -- cycle;
\draw [shift={(8.52,-1.86)},line width=1pt,color=qqwuqq,fill=qqwuqq,fill opacity=0.10000000149011612] (0,0) -- (118.21170426868399:0.6) arc (118.21170426868399:166.01879568854736:0.6) -- cycle;
\draw [line width=1pt] (3.62,-0.64)-- (6.46,1.98);
\draw [line width=1pt] (3.62,-0.64)-- (8.52,-1.86);
\draw [line width=1pt] (8.52,-1.86)-- (6.46,1.98);
\draw [fill=black] (3.62,-0.64) circle (1.5pt);
\draw[color=black] (3.26,-0.63) node {A};
\draw [fill=black] (8.52,-1.86) circle (1.5pt);
\draw[color=black] (8.86,-2.09) node {B};
\draw [fill=black] (6.46,1.98) circle (1.5pt);
\draw[color=black] (6.68,2.51) node {C};
\draw[color=black] (4.8,1.17) node {b};
\draw[color=black] (6.06,-1.53) node {c};
\draw[color=black] (7.76,0.51) node {a};
\draw[color=qqwuqq] (4.8,-0.31) node {$\alpha$};
\draw[color=qqwuqq] (6.32,1.01) node {$\gamma$};
\draw[color=qqwuqq] (7.6,-1.17) node {$\beta$};
\end{tikzpicture}
\end{subfigure}
\begin{subfigure}{.45 \textwidth}
\begin{tikzpicture}[line cap=round,line join=round,>=triangle 45,x=1.5cm,y=1.5cm]
\draw [shift={(0,0)},line width=1pt,color=black,fill=black,fill opacity=0.10000000149011612] (0,0) -- (0:0.2580645161290323) arc (0:30:0.2580645161290323) -- cycle;
\draw [line width=1pt] (0,0)-- (2.598076211353316,1.5);
\draw [line width=1pt] (0,0)-- (4,0);
\draw [dashed, line width=1pt] (2.598076211353316,1.5)--(2.598076211353316,0);
\draw [line width=1pt] (2.598076211353316,1.5)-- (4,0);
\draw[color=black] (-0.2313802083333099,0.22449680779561737) node {$A$};
\draw[color=black] (4.190125168010777,0.03524949596765889) node {$C$};
\draw[color=black] (2.5299101142473357,1.7986903561827265) node {$B$};
\draw[color=black] (1.3944262432795937,1.1) node {$c=3$};
\draw[color=black] (2.03098538306454,-0.2) node {$b=4$};
\draw[color=black] (0.85,0.2) node {$\alpha = 30^{\circ}$};
\end{tikzpicture}
\end{subfigure}
\end{figure}

Para esto solo nos basta rotar el triángulo de manera que uno de los lados sea la base. La altura, entonces, está determinada

Si tengo un triángulo $ABC$, con lados $a,b,c$ y ángulos $\alpha, \beta, \gamma$, como en la figura, entonces el área del triángulo la puedo calcular con cualquiera de siguientes tres fórmulas $$\frac{b\cdot c\cdot \mathrm{sen}(\alpha)}{2}=\frac{a\cdot c\cdot \mathrm{sen}(\beta)}{2}=\frac{a\cdot b\cdot \mathrm{sen}(\gamma)}{2}.$$

Dependiendo de los datos que se me proporcionen, me puede ser útil alguna de las tres. Por ejemplo, si me dicen que $b=3$, $c=4$ y $\alpha=30^{\circ}$, el área del triángulo $ABC$ sería $$\frac{b\cdot c\cdot \mathrm{sen}(\alpha)}{2}=\frac{3\cdot 4 \cdot \sen(30^{\circ})}{2}=\frac{3\cdot 4}{4}=3,$$
sin que haya tenido la necesidad de medir ninguna alguna altura, como en la fórmula usual.\\

Se vienen a la mente dos preguntas inmediatas. ¿qué es el seno? y ¿por qué funciona esa fórmula?.

\subsection*{Funciones trigonométricas}

Hace veinte años, cuando cursé secundaria y preparatoria, me enseñaron las {\bf funciones trigonométricas}: seno, coseno, tangente, cotangente, secante y cosecante, como las 6 distintas \emph{proporciones entre la hipotenusa y los dos catetos}, en un triángulo rectángulo.\\

Durante medio año despejábamos lados y ángulos de triángulos, usando la ley de senos, el teorema de Pitágoras y la ley de cosenos. Después de muchos años dedicados a la enseñanza y la investigación en matemáticas, no he utilizado durante el ejercicio de esta profesión una sola secante, cosecante o cotangente.\\

La tangente tiene cierta utilidad porque se relaciona con el concepto de la {\bf derivada} (crucial en el cálculo), pero raramente se utiliza a la función tangente de manera directa. Las únicas funciones trigonométricas que son verdaderamente fundamentales en matemáticas, directamente, son el {\bf seno} y el {\bf coseno}.\\

Cuando cursé secundaria, se insistía demasiado en las definiciones $$\cos(\alpha)=\frac{CA}{H}, \quad \sen(\alpha)=\frac{CO}{H}$$ y las otras cuatro $$\mathrm{tan}(\alpha)=\frac{CO}{CA}, \quad \mathrm{cot}(\alpha)=\frac{CA}{CO}, \quad \mathrm{sec}(\alpha)=\frac{H}{CO}, \quad \mathrm{csc}(\alpha)=\frac{H}{CA}$$

\begin{wrapfigure}{R}{0.3\textwidth}
\begin{tikzpicture}[line cap=round,line join=round,>=triangle 45,x=3cm,y=3cm]
\draw [shift={(0,0)},line width=1pt,fill=black,fill opacity=0.10000000149011612] (0,0) -- (0:0.17777777777777792) arc (0:54.46232220802562:0.17777777777777792) -- cycle;
\draw[line width=1pt,fill=black,fill opacity=0.10000000149011612] (1,0.12570787221094187) -- (0.8742921277890581,0.1257078722109419) -- (0.8742921277890581,0) -- (1,0) -- cycle; 
\draw [line width=1pt] (0,0)-- (1,0);
\draw [line width=1pt] (0,0)-- (1,1.4);
\draw [line width=1pt] (1,1.4)-- (1,0);
\draw [fill=black] (0,0) circle (1.5pt);
\draw[color=black] (-0.11444444444444588,0.087037037037038) node {$A$};
\draw [fill=black] (1,0) circle (1.5pt);
\draw[color=black] (1.1240740740740738,-0.02555555555555454) node {$B$};
\draw [fill=black] (1,1.4) circle (1.5pt);
\draw[color=black] (1.0470370370370365,1.5033333333333336) node {$C$};
\draw[color=black] (0.5670370370370361,-0.08481481481481377) node {Cat. Ady.};
\draw[color=black] (0.3537037037037027,0.7388888888888895) node {Hip.};
\draw[color=black] (1.2544444444444443,0.697407407407408) node {Cat. Op.};
\draw[color=black] (0.31814814814814707,0.14629629629629723) node {$\alpha$};
\draw[color=black] (0.7462962962962957,0.24703703703703794) node {$\beta = 90^{\circ}$};
\end{tikzpicture}
\end{wrapfigure}


Estas definiciones, como proporciones $\frac{CA}{H}$, $\frac{CO}{H}$, no son tan concretas como pudieran presentarse. Hacen la falsa impresión de que el coseno el seno solo son útiles cuando se trabaja con triángulos rectángulos.\\ 

Una forma más tangible de apreciar al coseno y al seno es fijando el tamaño de la hipotenusa $H=1$ y observando que $(\cos\alpha, \sen\alpha)$ se convierten en \emph{las coordenadas  $(x,y)$ de los puntos en el círculo unitario}.\\

En bachillerato me hubiera gustado enterarme que el seno y el coseno se pueden definir como las siguientes series de potencias (más adelante):
$$\cos(x)=\frac{1}{0!}-\frac{x^2}{2!}+\frac{x^4}{4!}-\frac{x^6}{6!}+\dots,\quad   \sen(x)=\frac{x}{1!}-\frac{x^3}{3!}+\frac{x^5}{5!}+\dots$$

\subsection*{Una definición concreta del seno y el coseno}

\begin{wrapfigure}{R}{0.5\textwidth}
\centering
\definecolor{rvwvcq}{rgb}{0.26666666666666666,0.26666666666666666,0.26666666666666666}
\definecolor{dtsfsf}{rgb}{0.49019607843137253,0.49019607843137253,1}
\definecolor{wrwrwr}{rgb}{0.30196078431372547,0.30196078431372547,1}
\begin{tikzpicture}[line cap=round,line join=round,>=triangle 45,x=6cm,y=6cm]
%\clip(-0.23711934156378495,-0.45761316872428137) rectangle (1.279917695473252,1.2595884773662553);
\draw [shift={(0,0)},line width=1pt,fill=black,fill opacity=0.10000000149011612] (0,0) -- (0:0.07901234567901234) arc (0:66.34165192231299:0.07901234567901234) -- cycle;
\draw [line width=1pt] (0,0)-- (0.4012820175103667,0.9159545525968031);
\draw [line width=1pt] (0.4012820175103667,0.9159545525968031)-- (1,0);
\draw [shift={(0,0)},line width=1pt,color=black]  plot[domain=0:1.157880257256387,variable=\t]({1*1*cos(\t r)+0*1*sin(\t r)},{0*1*cos(\t r)+1*1*sin(\t r)});
\draw [line width=1pt] (0,0)-- (1,0);
\draw [line width=1pt] (0.4012820175103667,0.9159545525968031)-- (0.4012820175103667,0);
\draw [fill=rvwvcq] (0,0) circle (1.5pt);
\draw[color=rvwvcq] (0.01572016460905455,-0.06123456790123568) node {$A = (0, 0)$};
\draw [fill=rvwvcq] (1,0) circle (1.5pt);
\draw[color=rvwvcq] (1.0639506172839517,-0.06386831275720276) node {$B = (1, 0)$};
\draw [fill=rvwvcq] (0.4012820175103667,0.9159545525968031) circle (1.5pt);
\draw[color=rvwvcq] (0.43185185185185293,0.968559670781893) node {$C=(\cos(\alpha), \mathrm{sen}(\alpha))$};
\draw[color=black] (0.1,0.47341563786008173) node {b=1};
\draw[color=dtsfsf] (0.8611522633744867,0.5945679012345675) node {$d$};
\draw[color=black] (0.1421399176954743,0.08362139917695377) node {$\alpha$};
\draw[color=rvwvcq] (0.42131687242798466,-0.05728395061728296) node {$H$};
\draw [fill=wrwrwr] (0.4012820175103667,0) circle (1.5pt);
\draw[color=black] (0.5293004115226349,-0.1) node {c=1};
\draw[color=black] (0.5767078189300423,0.43127572016460847) node {h=$|\mathrm{sen}(\alpha)|$};
\end{tikzpicture}
\end{wrapfigure}

Si abro un arco $BAC$, desde el centro $A=(0,0)$, con $B=(1,0)$, de ángulo $\alpha$ (como en la figura), el punto $C$ sobre la circunferencia tiene coordenadas $(x,y)=(\cos \alpha,\mathrm{sen}\alpha)$.\\

Vista la definición desde esta perspectiva, tiene mucho sentido medir los ángulos en radianes. La medida de un ángulo en radianes es exactamente la longitud del arco $d$ que tiende dicho ángulo. Recordemos que el perímetro del círculo unitario es $2\pi$, por lo que $2\pi$ corresponde a la vuelta completa, o $360^{\circ}$. \href{https://www.geogebra.org/calculator/sh3uynfv}{Figura dinámica} sh3uynfv).\\

Se hace una regla de tres para convertir los otros arcos. El ángulo recto, por ejemplo, mide un cuarto de circunferencia, que es igual a $\frac{\pi}{2}$ radianes, o noventa grados. \\

La elección de que una vuelta mida $360^{\circ}$ se basa en una unidad de medida de ángulo completamente arbitraria (influenciada por la astronomía y por las matemáticas de Babilonia, en las que se utiliza base $60$). Por otro lado, $2r\pi $ es la longitud del círculo completo, algo bastante concreto. \\

Por esta razón, es más común en matemáticas profesionales medir los ángulos y evaluar las funciones trigonométricas en radianes (y no en grados, que resultan un tanto arbitrarios).\\ 

Se recomienda al estudiante usar radianes pero sobre todo a poner mucho cuidado, y verificar siempre que se esté utilizando/interpretando en la escala correcta, \emph{nunca olvidar} el símbolo ° cuando se trate de grados.\\ 

\begin{ejercicio}
¿Cuánto miden (en grados y en radianes):
\begin{itemize} 
\item una media circunferencia? 
\item un tercio de circunferencia? 
\item la mitad de un ángulo recto? 
\item un séptimo de circunferencia? 
\item un noveno de circunferencia?
\end{itemize}
 \end{ejercicio}

El seno, entonces, en vez de definirse como una \emph{proporción} $\frac{CO}{H}$, abstracta por naturaleza, se puede entender mucho más concretamente, como la \emph{longitud} de la altura $CH$, del triángulo $ABC$ en la figura.\\

Esta definición (como la coordenada $y$ de un punto en el círculo) nos regala intuición y respuestas inmediatas a varias preguntas: \\

\definecolor{rvwvcq}{rgb}{0.08235294117647059,0.396078431372549,0.7529411764705882}
\definecolor{sexdts}{rgb}{0.1803921568627451,0.49019607843137253,0.19607843137254902}
\begin{tikzpicture}[line cap=round,line join=round,>=triangle 45,x=3cm,y=3cm]
\begin{axis}[
x=2cm,y=2cm,
axis lines=middle,
xmin=-1.2777777777777772,
xmax=7.273333333333342,
ymin=-1.4888888888888887,
ymax=2.680000000000001,
xtick={-1,-0,...,7},
ytick={-1,0,...,2},]
\draw[line width=1pt,color=sexdts,smooth,samples=100,domain=-1.2777777777777772:7.273333333333342] plot(\x,{sin(((\x))*180/pi)});
\draw[line width=1pt,color=rvwvcq,smooth,samples=100,domain=-1.2777777777777772:7.273333333333342] plot(\x,{cos(((\x))*180/pi)});
\draw[color=sexdts] (+2.5,1.2) node {$sen(x)$};
\draw[color=rvwvcq] (+5.5,1.2) node {$cos(x)$};
\end{axis}
\end{tikzpicture}


\begin{ejercicio}
¿Cuál es la relación entre: 
\begin{itemize}
    \item $\sen(\alpha)$ y $\sen(-\alpha)$?
    \item $\sen(\alpha)$ y $\sen(180^{\circ}-\alpha)$?
    \item $\cos(\alpha)$ y $\cos(\alpha)$?
    \item $\sen(\alpha)$ y $\sen(180^{\circ}-\alpha)$?
\end{itemize}
\end{ejercicio}

\begin{ejercicio}
Demuestra la identidad trigonométrica $(\sen(\alpha))^2+ (\cos(\alpha))^2=1$.
\end{ejercicio}

\begin{ejercicio}
¿Por qué son ambas funciones periódicas? 

¿De que tamaño es el periodo?
\end{ejercicio}

\subsection*{El seno y el área}

¿Cuánto vale el área del triángulo ABC?. Usando la fórmula de base por altura entre dos, tengo que en este caso la base es $c=1$, mientras que la altura es la coordenada $y$ del punto $C$, es decir $h=\mathrm{sen}(\alpha)$. Entonces el área resulta ser simplemente $\frac{1}{2}\mathrm{sen}(\alpha)$ \\

Concluímos que, al menos para triángulos que son sectores triangulares del círculo unitario, como el de la figura, se verifica la fórmula extraña de arriba: $\frac{bc}{2}\sen(\alpha)$.
En efecto, este caso, $c=1$, $b=1$ y la altura es exactamente $h=\sen(\alpha)$.\\

\begin{wrapfigure}{R}{0.4\textwidth}
\centering
\begin{tikzpicture}[line cap=round,line join=round,>=triangle 45,x=4cm,y=4cm]
\draw [shift={(0,0)},line width=1pt,fill=black,fill opacity=0.10000000149011612] (0,0) -- (0:0.11851851851851836) arc (0:53.09503941018474:0.11851851851851836) -- cycle;
\draw [line width=1pt] (0,0)-- (0.6004894586737243,0.7996326719323927);
\draw [line width=1pt] (0.6004894586737243,0.7996326719323927)-- (1,0);
\draw [shift={(0,0)},line width=1pt]  plot[domain=0:0.9266832541838718,variable=\t]({1*1*cos(\t r)+0*1*sin(\t r)},{0*1*cos(\t r)+1*1*sin(\t r)});
\draw [line width=1pt] (0,0)-- (1,0);
\draw [line width=1pt] (0.6004894586737243,0.7996326719323927)-- (0.6004894586737243,0);
\draw [line width=1pt] (0.6004894586737243,0.7996326719323927)-- (1.7864197530864234,0);
\draw [line width=1pt] (0.9605776276417528,1.2791386158320652)-- (1.7864197530864234,0);
\draw [line width=1pt] (0.6004894586737243,0.7996326719323927)-- (0.9605776276417528,1.2791386158320652);
\draw [line width=1pt] (1,0)-- (1.7864197530864234,0);
\draw [fill=rvwvcq] (0,0) circle (1.5pt);
\draw[color=rvwvcq] (0.024938271604938195,-0.0908641975308668) node {$A = (0, 0)$};
\draw [fill=rvwvcq] (1,0) circle (1.5pt);
\draw[color=rvwvcq] (1.095555555555554,-0.09481481481481743) node {$B = (1, 0)$};
\draw [fill=rvwvcq] (0.6004894586737243,0.7996326719323927) circle (1.5pt);
\draw[color=rvwvcq] (0.2192592592592576,0.845432098765431) node {($\sen(\alpha),\cos(\alpha))=$  C};
\draw[color=black] (0.15851851851851817,0.08851851851851623) node {$\alpha$};
\draw [fill=rvwvcq] (0.6004894586737243,0) circle (1.5pt);
\draw[color=rvwvcq] (0.597777777777777,-0.07111111111111369) node {H};
\draw[color=black] (0.4624691358024681,0.35950617283950426) node {h=|sen$\alpha$|};
\draw [fill=rvwvcq] (1.7864197530864234,0) circle (1.5pt);
\draw[color=rvwvcq] (1.8343209876543185,0.08691358024691125) node {B'};
\draw [fill=rvwvcq] (0.9605776276417528,1.2791386158320652) circle (1.5pt);
\draw[color=rvwvcq] (1.056049382716048,1.3234567901234564) node {C'};
\end{tikzpicture}
\end{wrapfigure}

¿Cómo paso de una fórmula del área para un triangulo como ABC a la fórmula para el área de un triángulo A'B'C' en general? Muy sencillo, solo se escalan las áreas, en dos pasos, como se ilustra en la siguiente figura:\\

Ya vimos que el área del triángulo $ABC$ es $\frac{1}{2}\sen(\alpha)$. Observemos que el triángulo $AB'C$ comparte la misma altura $CH$, pero la base ahora es $AB'=c$, en vez de $AB=1$. Por lo tanto, el área sólo se multiplica por $c$ y obtenemos $c\sen(\alpha)/2$ \\

De la misma manera, el triángulo $AB'C$ y el $AB'C'$ (cuya área queremos calcular), comparten la misma base.  Las alturas se encuentran en la misma proporción que los lados $\frac{AC'}{AC}=\frac{b}{1}=b$. Por lo tanto, la altura sólo se multiplica por $b$.\\

Entonces, como la base no cambia, el área también se multiplica por $c$ y concluímos que el área de $AB'C'$ es $$\frac{1}{2}bc \sen(\alpha).$$\\

\newpage

\section{Algunas consecuencias}

\begin{multicols}{2}
\begin{problema}[Ley de senos]

a). Usando las tres nuevas fórmulas del área, con los senos del triángulo, demuestra la \emph{ley de senos}:

$$\frac{a}{sen(\alpha)}=\frac{b}{sen(\beta)}=\frac{c}{sen(\gamma)}.$$

b). Demuestra que si el circuncírculo del triángulo $ABC$ tiene radio $R$, entonces 
$$\frac{a}{sen(\alpha)}=\frac{b}{sen(\beta)}=\frac{c}{sen(\gamma)}=2R.$$
\end{problema}

\columnbreak
\definecolor{qqwuqq}{rgb}{0,0.39215686274509803,0}
\begin{tikzpicture}[line cap=round,line join=round,>=triangle 45,x=1cm,y=1cm]
\draw [shift={(3.62,-0.64)},line width=1pt,color=qqwuqq,fill=qqwuqq,fill opacity=0.10000000149011612] (0,0) -- (-13.981204311452633:0.6) arc (-13.981204311452633:42.69262700303779:0.6) -- cycle;
\draw [line width=1pt] (3.62,-0.64)-- (6.46,1.98);
\draw [line width=1pt] (3.62,-0.64)-- (8.52,-1.86);
\draw [line width=1pt] (8.52,-1.86)-- (6.46,1.98);
\draw [line width=1pt] (6.227540054469171,-0.6172571582795595) circle (2.6076392335809424cm);
\draw [fill=black] (3.62,-0.64) circle (1.5pt);
\draw[color=black] (3.26,-0.63) node {A};
\draw [fill=black] (8.52,-1.86) circle (1.5pt);
\draw[color=black] (8.86,-2.09) node {B};
\draw [fill=black] (6.46,1.98) circle (1.5pt);
\draw[color=black] (6.68,2.51) node {C};
\draw[color=black] (4.32,0.67) node {b};
\draw[color=black] (6.06,-1.43) node {c};
\draw[color=black] (7.84,0.45) node {a};
\draw[color=qqwuqq] (4.5,-0.40) node {$\alpha$};
\end{tikzpicture}
\end{multicols}

\begin{multicols}{2}
\begin{problema}[Ley de cosenos]
Muestra que en un triángulo con lados de longitudes $a, b, c$ se tiene que $b^2=a^2+c^2-2ac\cos(\beta)$, donde $\beta$ es el ángulo opuesto al lado $b$.
\end{problema}

Sug: Aquí sí conviene la definición clásica de $\cos\beta=\frac{CA}{H}$

\begin{problema}

Sean $a,b,c$ lados de un triángulo y $\alpha, \beta, \gamma$ sus ángulos. \\

Demuestra que si sabes tres datos de $\{a,b,c,\alpha, \beta, \gamma\}$, de los al menos uno es un lado, entonces conoces los seis datos.

\columnbreak
\definecolor{qqwuqq}{rgb}{0,0.39215686274509803,0}
\begin{tikzpicture}[line cap=round,line join=round,>=triangle 45,x=1cm,y=1cm]
\draw [shift={(3.62,-0.64)},line width=1pt,color=qqwuqq,fill=qqwuqq,fill opacity=0.10000000149011612] (0,0) -- (-13.981204311452633:0.6) arc (-13.981204311452633:42.69262700303779:0.6) -- cycle;
\draw [shift={(6.46,1.98)},line width=1pt,color=qqwuqq,fill=qqwuqq,fill opacity=0.10000000149011612] (0,0) -- (-137.30737299696221:0.6) arc (-137.30737299696221:-61.78829573131603:0.6) -- cycle;
\draw [shift={(8.52,-1.86)},line width=1pt,color=qqwuqq,fill=qqwuqq,fill opacity=0.10000000149011612] (0,0) -- (118.21170426868399:0.6) arc (118.21170426868399:166.01879568854736:0.6) -- cycle;
\draw [line width=1pt] (3.62,-0.64)-- (6.46,1.98);
\draw [line width=1pt] (3.62,-0.64)-- (8.52,-1.86);
\draw [line width=1pt] (8.52,-1.86)-- (6.46,1.98);
\draw [fill=black] (3.62,-0.64) circle (1.5pt);
\draw[color=black] (3.26,-0.63) node {A};
\draw [fill=black] (8.52,-1.86) circle (1.5pt);
\draw[color=black] (8.86,-2.09) node {B};
\draw [fill=black] (6.46,1.98) circle (1.5pt);
\draw[color=black] (6.68,2.51) node {C};
\draw[color=black] (4.8,1.17) node {b};
\draw[color=black] (6.06,-1.53) node {c};
\draw[color=black] (7.76,0.51) node {a};
\draw[color=qqwuqq] (4.8,-0.31) node {$\alpha$};
\draw[color=qqwuqq] (6.32,1.01) node {$\gamma$};
\draw[color=qqwuqq] (7.6,-1.17) node {$\beta$};
\end{tikzpicture}
\end{problema}

\end{multicols}



%Ejecicios (opcional).

\begin{problema}
Calcula el área de un cuadrilátero convexo $ABCD$ en términos de sus diagonales $AC$, $BD$ y el ángulo que forman.
\end{problema}

% R= \frac{1}{2}(AC)(BD)\sen(\alpha)

\begin{problema}
Demuestra que el área de un n-ágono regular inscrito en un círculo de radio $r$ está dada por $nr^2 \sen \left(\frac{2\pi}{n} \right)$.
\end{problema}

\begin{problema}


\begin{multicols}{2}
Calcula el área de caracol.\\

Los ángulos entre rayos consecutivos desde el centro del caracol son de $30^{\circ}$.\\

Las longitudes de los trece rayos están en progresión aritmética: \\

$l_1=1, l_2=2, \dots, l_{13}=13$. 

\columnbreak

\begin{tikzpicture}[line cap=round,line join=round,>=triangle 45,x=.3cm,y=.3cm]
\draw [line width=1pt] (0,0)-- (1,0);
\draw [line width=1pt] (1,0)-- (1.7320508075688774,1);
\draw [line width=1pt] (1.7320508075688774,1)-- (1.5,2.598076211353316);
\draw [line width=1pt] (1.5,2.598076211353316)-- (0,4);
\draw [line width=1pt] (0,4)-- (-2.5,4.3301270189221945);
\draw [line width=1pt] (-2.5,4.3301270189221945)-- (-5.196152422706631,3);
\draw [line width=1pt] (-5.196152422706631,3)-- (-7,0);
\draw [line width=1pt] (-7,0)-- (-6.928203230275514,-4);
\draw [line width=1pt] (-6.928203230275514,-4)-- (-4.5,-7.794228634059941);
\draw [line width=1pt] (-4.5,-7.794228634059941)-- (0,-10);
\draw [line width=1pt] (0,-10)-- (5.5,-9.526279441628834);
\draw [line width=1pt] (5.5,-9.526279441628834)-- (10.392304845413255,-6);
\draw [line width=1pt] (10.392304845413255,-6)-- (13,0);
\draw [line width=1pt] (0,0)-- (1.7320508075688774,1);
\draw [line width=1pt] (0,0)-- (1.5,2.598076211353316);
\draw [line width=1pt] (0,0)-- (0,4);
\draw [line width=1pt] (0,0)-- (-2.5,4.3301270189221945);
\draw [line width=1pt] (0,0)-- (-5.196152422706631,3);
\draw [line width=1pt] (0,0)-- (-7,0);
\draw [line width=1pt] (0,0)-- (-6.928203230275514,-4);
\draw [line width=1pt] (0,0)-- (-4.5,-7.794228634059941);
\draw [line width=1pt] (0,0)-- (0,-10);
\draw [line width=1pt] (0,0)-- (5.5,-9.526279441628834);
\draw [line width=1pt] (0,0)-- (10.392304845413255,-6);
\draw [line width=1pt] (1,0)-- (13,0);
\draw [fill=black] (1,0) circle (1.5pt);
\draw [fill=black] (1.7320508075688774,1) circle (1.5pt);
\draw [fill=black] (1.5,2.598076211353316) circle (1.5pt);
\draw [fill=black] (0,4) circle (1.5pt);
\draw [fill=black] (-2.5,4.3301270189221945) circle (1.5pt);
\draw [fill=black] (-5.196152422706631,3) circle (1.5pt);
\draw [fill=black] (-7,0) circle (1.5pt);
\draw [fill=black] (-6.928203230275514,-4) circle (1.5pt);
\draw [fill=black] (-4.5,-7.794228634059941) circle (1.5pt);
\draw [fill=black] (0,-10) circle (1.5pt);
\draw [fill=black] (5.5,-9.526279441628834) circle (1.5pt);
\draw [fill=black] (10.392304845413255,-6) circle (1.5pt);
\draw [fill=black] (13,0) circle (1.5pt);
\draw [fill=black] (0,0) circle (1.5pt);
\draw [fill=black] (1,0) circle (1.5pt);
\draw [fill=black] (1.7320508075688774,1) circle (1.5pt);
\draw [fill=black] (1.5,2.598076211353316) circle (1.5pt);
\draw [fill=black] (0,4) circle (1.5pt);
\draw [fill=black] (-2.5,4.3301270189221945) circle (1.5pt);
\draw [fill=black] (-5.196152422706631,3) circle (1.5pt);
\draw [fill=black] (-7,0) circle (1.5pt);
\draw [fill=black] (-6.928203230275514,-4) circle (1.5pt);
\draw [fill=black] (-4.5,-7.794228634059941) circle (1.5pt);
\draw [fill=black] (0,-10) circle (1.5pt);
\draw [fill=black] (5.5,-9.526279441628834) circle (1.5pt);
\draw [fill=black] (10.392304845413255,-6) circle (1.5pt);
\draw [fill=black] (13,0) circle (1.5pt);
\draw[color=black] (-4.3475,-4.305) node {$\ddots$};
\draw[color=black] (-1.7575,-6.06) node {$l_{9}=9$};
\draw[color=black] (0.375,-7.635) node {$l_{10}=10$};
\draw[color=black] (5.4875,-5.475) node {$l_{11}=11$};
\draw[color=black] (6.7625,-2.325) node {$l_{12}=12$};
\draw[color=black] (7.3975,0.69) node {$l_{13}=13$};
\end{tikzpicture}

\end{multicols}

\end{problema}
\newpage

\begin{problema}[Teorema de la bisectriz]

Demuestra el teorema de la bisectriz:

\begin{tikzpicture}[line cap=round,line join=round,>=triangle 45,x=5cm,y=5cm]
\draw [shift={(0.2707407407407396,0.6885185185185192)},line width=1pt,fill=black,fill opacity=0.10000000149011612] (0,0) -- (-111.46585191085832:0.07901234567901248) arc (-111.46585191085832:-77.40993720664714:0.07901234567901248) -- cycle;
\draw [shift={(0.2707407407407396,0.6885185185185192)},line width=1pt,fill=black,fill opacity=0.10000000149011612] (0,0) -- (-77.40993720664713:0.07901234567901248) arc (-77.40993720664713:-43.354022502435924:0.07901234567901248) -- cycle;
\draw [line width=1pt] (0,0)-- (0.2707407407407396,0.6885185185185192);
\draw [line width=1pt] (0,0)-- (1,0);
\draw [line width=1pt] (1,0)-- (0.2707407407407396,0.6885185185185192);
\draw [line width=1pt] (0.2707407407407396,0.6885185185185192)-- (0.4245174874452492,0);
\draw[color=black] (-0.050905349794240086,0.03831275720164729) node {$A$};
\draw[color=black] (0.2783127572016453,0.7599588477366268) node {$B$};
\draw[color=black] (1.0210288065843625,0.046213991769548525) node {$C$};
\draw[color=black] (0.4468724279835385,0.05674897119341684) node {$D$};
\draw[color=black] (0.3600823045267485,0.5076954732510301) node {$\beta$};
\draw[color=black] (0.25209876543209816,0.5055967078189314) node {$\beta$};

\draw[color=black] (.8,.6) node {\LARGE{$\frac{AD}{DC}=\frac{BA}{BC}$}};

\end{tikzpicture}


\end{problema}

%Sug: comparar áreas

\begin{problema}[Teorema generalizado de la bisectriz]
Demuestra el teorema generalizado de la bisectriz: 

\end{problema}

\begin{tikzpicture}[line cap=round,line join=round,>=triangle 45,x=5cm,y=5
cm]
\draw [shift={(0.2707407407407396,0.6885185185185192)},line width=1pt,fill=black,fill opacity=0.10000000149011612] (0,0) -- (-111.46585191085832:0.07901234567901248) arc (-111.46585191085832:-82.64370893112309:0.07901234567901248) -- cycle;
\draw [shift={(0.2707407407407396,0.6885185185185192)},line width=1pt,fill=black,fill opacity=0.10000000149011612] (0,0) -- (-82.64370893112309:0.07901234567901248) arc (-82.64370893112309:-43.35402250243593:0.07901234567901248) -- cycle;
\draw [line width=1pt] (0,0)-- (0.2707407407407396,0.6885185185185192);
\draw [line width=1pt] (0,0)-- (1,0);
\draw [line width=1pt] (1,0)-- (0.2707407407407396,0.6885185185185192);
\draw [line width=1pt] (0.2707407407407396,0.6885185185185192)-- (0.3596296296296286,0);
\draw[color=black] (-0.050905349794240086,0.03831275720164729) node {$A$};
\draw[color=black] (0.2783127572016453,0.7599588477366268) node {$B$};
\draw[color=black] (1.0210288065843625,0.046213991769548525) node {$C$};
\draw[color=black] (0.38102880658436145,0.046213991769548525) node {$D$};
\draw[color=black] (0.3600823045267485,0.5076954732510301) node {$\beta$};
\draw[color=black] (0.25209876543209816,0.5055967078189314) node {$\alpha$};

\draw[color=black] (.8,.6) node {\LARGE{$\frac{AD}{DC}=\frac{BA}{BC}\frac{\sen\alpha}{\sen\beta}$}};

\end{tikzpicture}



\begin{problema}[Teorema de Ceva]
Demuestra el teorema de Ceva:

\begin{tikzpicture}[line cap=round,line join=round,>=triangle 45,x=6cm,y=6cm]
\draw [line width=1pt] (0,0)-- (1,0);
\draw [line width=1pt] (1,0)-- (0.46,0.6);
\draw [line width=1pt] (0.46,0.6)-- (0,0);
\draw [line width=1pt] (0.46,0.6)-- (0.47604938271604985,0);
\draw [line width=1pt] (0,0)-- (0.7141231839574385,0.3176409067139572);
\draw [line width=1pt] (1,0)-- (0.2856594873389836,0.3725993313117177);
\draw [line width=1pt] (1,0)-- (0.23251449386558373,0.3032797746072831);
\draw[color=black] (-0.06650205761316873,-0.026995884773662552) node {$A$};
\draw[color=black] (1.071275720164609,-0.013827160493827161) node {$B$};
\draw[color=black] (0.45234567901234574,0.6735802469135803) node {$C$};
\draw[color=black] (0.47341563786008234,-0.05069958847736626) node {$D$};
\draw[color=black] (0.7341563786008231,0.3627983539094651) node {$E$};
\draw[color=black] (0.26534979423868316,0.4418106995884774) node {$F'$};
\draw[color=black] (0.15209876543209877,0.3627983539094651) node {$F$};

\draw[color=black] (1.3,.6) node {\LARGE{$\frac{AD}{DB}\frac{BE}{EC}\frac{CF}{FA}=1\neq \frac{AD}{DB}\frac{BE}{EC}\frac{CF'}{F'A}.$}};
\end{tikzpicture}


\begin{tikzpicture}[line cap=round,line join=round,>=triangle 45,x=6cm,y=6cm]
\draw [line width=1pt] (0,0)-- (1,0);
\draw [line width=1pt] (1,0)-- (0.46,0.6);
\draw [line width=1pt] (0.46,0.6)-- (0,0);
\draw [line width=1pt] (0.46,0.6)-- (0.47604938271604985,0);
\draw [line width=1pt] (0,0)-- (0.7141231839574385,0.3176409067139572);
%\draw [line width=1pt] (1,0)-- (0.2856594873389836,0.3725993313117177);
\draw [line width=1pt] (1,0)-- (0.23251449386558373,0.3032797746072831);
\draw[color=black] (-0.06650205761316873,-0.026995884773662552) node {$A$};
\draw[color=black] (1.071275720164609,-0.013827160493827161) node {$B$};
\draw[color=black] (0.45234567901234574,0.6735802469135803) node {$C$};
\draw[color=black] (0.47341563786008234,-0.05069958847736626) node {$D$};
\draw[color=black] (0.7341563786008231,0.3627983539094651) node {$E$};
\draw[color=black] (0.15209876543209877,0.3627983539094651) node {$F$};

\end{tikzpicture}


%Sug1: Comparar áreas (AOB, BOC, COA).

%Sug2: Teo gral. bis X3 (AOB, BOC, COA).
\end{problema}

\begin{problema}
Demuestra la versión trigonométrica del teorema de Ceva.
\end{problema}

\begin{multicols}{2}
\begin{tikzpicture}[line cap=round,line join=round,>=triangle 45,x=4cm,y=4cm]
\draw [line width=1pt] (0,0)-- (1,0);
\draw [line width=1pt] (1,0)-- (0.46,0.6);
\draw [line width=1pt] (0.46,0.6)-- (0,0);
\draw [line width=1pt] (0.46,0.6)-- (0.47604938271604985,0);
\draw [line width=1pt] (0,0)-- (0.7141231839574385,0.3176409067139572);
\draw [line width=1pt] (1,0)-- (0.2856594873389836,0.3725993313117177);
\draw [line width=1pt] (1,0)-- (0.23251449386558373,0.3032797746072831);
\draw [fill=black] (0,0) circle (1.5pt);
\draw[color=black] (-0.06650205761316873,-0.026995884773662552) node {$A$};
\draw [fill=black] (1,0) circle (1.5pt);
\draw[color=black] (1.071275720164609,-0.013827160493827161) node {$B$};
\draw [fill=black] (0.46,0.6) circle (1.5pt);
\draw[color=black] (0.45234567901234574,0.6735802469135803) node {$C$};
\draw [fill=black] (0.47604938271604985,0) circle (1.5pt);
\draw[color=black] (0.47341563786008234,-0.05069958847736626) node {$D$};
\draw [fill=black] (0.7141231839574385,0.3176409067139572) circle (1.5pt);
\draw[color=black] (0.7341563786008231,0.3627983539094651) node {$E$};
\draw [fill=black] (0.2856594873389836,0.3725993313117177) circle (1.5pt);
\draw[color=black] (0.26534979423868316,0.4418106995884774) node {$F'$};
\draw [fill=black] (0.470451991158082,0.2092563305517134) circle (1.5pt);
\draw[color=black] (0.5102880658436214,0.14683127572016463) node {$O$};
\draw [fill=black] (0.23251449386558373,0.3032797746072831) circle (1.5pt);
\draw[color=black] (0.15209876543209877,0.3627983539094651) node {$F$};
\end{tikzpicture}
\columnbreak
$$\frac{\sen(DCA)}{\sen(CAE)}\frac{\sen(EAB)}{\sen(ABF)}\frac{\sen(FBC)}{\sen(BCD)}=1,$$

$$\frac{\sen(DCA)}{\sen(CAE)}\frac{\sen(EAB)}{\sen(ABF')}\frac{\sen(F'BC)}{\sen(BCD)}\neq 1,$$
\end{multicols}

%Sug: Teo gral. bis. X3 (a ABC)


\begin{ejercicio}
Utiliza Ceva para demostrar que las medianas, las bisectrices y las alturas de un triángulo son concurrentes.

¿Te sabes alguna demostración más sencilla?
\end{ejercicio}


\begin{problema}
Sobre los lados $BC$, $CA$ y $AB$ de un triángulo se colocan los puntos $L, M, N$, de tal forma que $AB+BL=LC+CA$,  $BC+CM=MA+AB$ y $CA+AN=NB+BA$.

Demuestra que $AL$, $BM$ y $CN$ son concurrentes.
\end{problema}

\begin{problema}
En los lados $BC$, $CA$, y $AB$ de un triángulo $ABC$ se toman puntos $A_1$, $B_1$, $C_1$ de tal forma que $AA_1$, $BB_1$ y $CC_1$ concurren. Luego, en los lados $B_1C_1$, $C_1A_1$, y $A_1B_1$ del triángulo $A_1B_1C_1$ se toman puntos $A_2$, $B_2$, $C_2$ de tal forma que $A_1A_2$, $B_1B_2$ y $C_1C_2$ concurren.

Demuestre que $AA_2$, $BB_2$ y $CC_2$ concurren.
\end{problema}

%OMM año?

\begin{problema}
Demuestra que las simedianas de un triángulo concurren. Una simediana es la reflejada de la mediana con respecto a la bisectriz.
\end{problema}
\newpage

\newpage





















\begin{problema}
Dos circunferencias se intersectan en los puntos $A$ y $B$. Las rectas $L_1$ y $L_2$
son paralelas, con la particularidad de que $L_1$ pasa por el punto $A$ y corta las circunferencias
en los puntos $E$ y $K$, mientras que $L_2$ pasa por el punto $B$ y corta las circunferencias
en los puntos $M$ y $P$. Demuestre que el cuadril\'atero $EKMP$ es un paralelogramo.

\begin{tikzpicture}[line cap=round,line join=round,>=triangle 45,x=1.3cm,y=1.3cm]
\draw [line width=1.pt] (1.34,0.513333333333335) circle (1.3*1.3979667775419817cm);
\draw [line width=1.pt] (4.446666666666667,0.32666666666666827) circle (1.3*2.1526624341859906cm);
\draw [line width=1.pt] (0.5559044016253842,-0.6440366177716114)-- (0.002111677305102866,0.9187540371802734);
\draw [line width=1.pt] (6.596101311351841,0.2088266743776298)-- (6.04230858703156,1.7716173293295143);
\draw [line width=1.pt] (-0.7242465935523759,0.8161937538945099)-- (6.9179238245269925,1.895252387137624);
\draw [line width=1.pt] (7.30824366115857,0.3093796990575103)-- (-0.5840021523968433,-0.8049890620272606);
\draw[color=black] (2.5,1.573333333333335) node {$A$};
\draw[color=black] (2.3666666666666676,-0.5066666666666652) node {$B$};
\draw[color=black] (-0.1,1.1066666666666685) node {$E$};
\draw[color=black] (0.54,-0.76) node {$F$};
\draw[color=black] (6.7,0.1066666666666682) node {$G$};
\draw[color=black] (6.073333333333334,2.0133333333333354) node {$H$};
\end{tikzpicture}

(GusievMIRclubsigma)
\end{problema}

\begin{problema}
Dos circunferencias se intersectan en los puntos $A$ y $B$. Estos yacen por diferentes
lados de la recta $l$ que corta las circunferencias en los puntos $C$, $D$, $E$ y $M$,
respectivamente. Demuestre que la suma de los \'angulos $\angle DBE$ y $\angle CAM$ es igual 
a $180^\circ$.


\begin{tikzpicture}[line cap=round,line join=round,>=triangle 45,x=1.0cm,y=1.0cm]
\draw [line width=1.pt] (1.34,0.513333333333335) circle (1.6226179121681383cm);
\draw [line width=1.pt] (4.446666666666667,0.32666666666666827) circle (2.314975401837157cm);
\draw [line width=1.pt] (-0.26501383145183266,0.7517017241430133)-- (6.7605431543523995,0.39798784757567285);
\draw [line width=1.pt] (-0.26501383145183266,0.7517017241430133)-- (2.5266666666666677,1.62);
\draw [line width=1.pt] (2.5266666666666677,1.62)-- (6.7605431543523995,0.39798784757567285);
\draw [line width=1.pt] (2.151654340410889,0.6300303649648458)-- (2.3856174482273413,-0.7274619925973483);
\draw [line width=1.pt] (2.9608391005397086,0.5892905518877172)-- (2.3856174482273413,-0.7274619925973483);
\draw[color=black] (2.5266666666666677,1.8533333333333344) node {$A$};
\draw[color=black] (2.3266666666666675,-0.8533333333333327) node {$D$};
\draw[color=black] (-0.4466666666666663,0.7466666666666676) node {$E$};
\draw[color=black] (1.993333333333334,0.5733333333333342) node {$F$};
\draw[color=black] (3.06,0.52) node {$G$};
\draw[color=black] (6.86,0.56) node {$H$};
\end{tikzpicture}
(GusievMIRclubsigma)
\end{problema}

\begin{problema}
Por el punto $A$ de la cuerda com\'un $AB$ de dos circunferencias se ha trazado
una recta que cruza la primera circunferencia en el punto $C$ y la segunda en el punto
$D$. La tangente a la primera circunferencia en el punto $C$ y la tangente a la segunda
circunferencia en el punto $D$ concurren en el punto $M$. Demostrar que los puntos $M$,
$C$, $B$ y $D$ yacen en una circunferencia.

\begin{tikzpicture}[line cap=round,line join=round,>=triangle 45,x=1.0cm,y=1.0cm]
\draw [line width=1.pt] (0.7533333333333334,0.6333333333333332) circle (1.8857123617113807cm);
\draw [line width=1.pt] (3.86,0.4466666666666666) circle (2.3544190130239957cm);
\draw [line width=1.pt] (-0.8938882191430002,1.551250687003352)-- (1.9025873787892702,-0.8617005291976386);
\draw [line width=1.pt] (1.9025873787892702,-0.8617005291976386)-- (5.139529313030776,2.4230506440313566);
\draw [line width=1.pt] (-0.8938882191430002,1.551250687003352)-- (5.139529313030776,2.4230506440313566);
\draw [line width=1.pt] (-0.8938882191430002,1.551250687003352)-- (1.0627160912475522,5.062417326197354);
\draw [line width=1.pt] (1.0627160912475522,5.062417326197354)-- (5.139529313030776,2.4230506440313566);
\draw[color=black] (2.0666666666666678,2.3366666666666673) node {$A$};
\draw[color=black] (1.8266666666666675,-1.133333333333328) node {$C$};
\draw[color=black] (-1.1333333333333335,1.8166666666666673) node {$D$};
\draw[color=black] (5.306666666666669,2.6166666666666676) node {$B$};
\draw[color=black] (1.0066666666666673,5.356666666666667) node {$M$};
\end{tikzpicture}
(GusievMIRclubsigma)
\end{problema}

\begin{problema}
En el punto $A$ dos circunferencias son tangentes exteriores, y $BC$ es su tangente
com\'un externa. Demuestre que $\angle BAC=90^\circ$.


\begin{tikzpicture}[line cap=round,line join=round,>=triangle 45,x=1.0cm,y=1.0cm]
\draw [line width=1.pt] (5.5066666666666695,-0.1733333333333326) circle (3.147368913381952cm);
\draw [line width=1.pt] (0.1066666666666669,0.12666666666666743) circle (2.2609579998140346cm);
\draw [line width=1.pt] (-0.1690468123184532,2.3864805321521656)-- (2.3641435897435907,0.0012512820512828315);
\draw [line width=1.pt] (2.3641435897435907,0.0012512820512828315)-- (5.172841041434405,2.9562819032191383);
\draw [line width=1.pt] (-0.1690468123184532,2.3864805321521656)-- (5.172841041434405,2.9562819032191383);
\draw[color=black] (2.5666666666666678,-0.04333333333333256) node {$A$};
\draw[color=black] (5.166666666666669,3.3766666666666674) node {$B$};
\draw[color=black] (-0.2533333333333332,2.7966666666666673) node {$C$};
\end{tikzpicture}

(GusievMIRclubsigma)
\end{problema}

\newpage


\begin{problema}
En el plano hay 5 puntos, $A, B, C, D$ y $E$ situados de tal manera que $ABC$ es
un tri\'angulo equil\'atero, $B$ es el punto medio de $AD$ y $E$ es el punto m\'as alejado
de $C$ para el cual los segmentos $DE$ y $AB$ miden lo mismo. ?`Cu\'anto mide el \'angulo
$\angle BED$?


%\includegraphics[width=0.55\textwidth]{figuras/MLPS_Niv_Olimp_P90}
(P90 Nivel Ol\'impico MatPreolimpicasMLPS)
\end{problema}

\begin{problema}
Sean $C$ y $C'$ dos c\'irculos con centros en $O$ y $O'$, respectivamente,
y tales que se intersectan en dos puntos distintos $P$ y $Q$. Sea $\mathcal{L}$
una recta por $P$ que intersecta a $C$ y $C'$ en dos puntos $B$ y $B'$, respectivamente.
Sea $A$ el circuncentro de $BB'Q$. Probar que $A$ est\'a en el circunc\'irculo de $O$,
$O'$ y $Q$.
%\includegraphics[width=0.55\textwidth]{figuras/MLPS_Niv_Final_P12}
(P12 Nivel Final MatPreolimpicasMLPS)


\end{problema}

\begin{problema}
Sean $A$ y $B$ dos puntos fijos en el plano y sea $\mathcal{L}$ una recta que
pasa por $A$ pero no por $B$. Para $P$ y $Q$ puntos de $\mathcal{L}$ (distintos de $A$)
sean $O_P$ y $O_Q$ los centros de las circunferencias circunscritas a $APB$ y $AQB$,
respectivamente. Demostrar que los \'angulos $\angle O_P PB$ y $\angle O_Q QB$ son
iguales.

\begin{tikzpicture}[line cap=round,line join=round,>=triangle 45,x=3*1.0cm,y=3*1.0cm]
\draw [line width=1.pt] (0.,0.)-- (1.3902748774292129,-0.009084041275863961);
\draw [line width=1.pt] (0.4332940580495208,0.31524833890999726) circle (3*0.5358407001400081cm);
\draw [line width=1.pt] (0.6935451428961725,-0.24823629795108226) circle (3*0.7366316072877479cm);
\draw[color=black] (-0.06930581431126241,0.006980526086059501) node {$A$};
\draw[color=black] (1.0047252607430497,0.5118669288893684) node {$B$};
\draw[color=black] (1.440763517709544,0.03910966080990643) node {$Q$};
\draw[color=black] (0.899158103793267,-0.05268786697251336) node {$P$};
\draw [fill=black] (0.4332940580495208,0.31524833890999726) circle (1.0pt);
\draw[color=black] (0.4309907121029257,0.2202439747094719) node {$O_Q$};
\draw [fill=black] (0.6935451428961725,-0.24823629795108226) circle (1.0pt);
\draw[color=black] (0.7017934190610643,-0.3243108211524097) node {$O_P$};
\end{tikzpicture}

%\includegraphics[width=0.65\textwidth]{figuras/MLPS_Niv_Final_P14}
(P14 Nivel Final MatPreolimpicasMLPS)

\end{problema}


\begin{problema}
Sea $ABC$ un tri\'angulo y $AD$ la altura sobre el lado $BC$. Tomando a $D$ como
centro y a $AD$ como radio, se traza una circunferencia que corta a la recta $AB$ en
$P$, y corta a la recta $AC$ en $Q$. Muestra que el tri\'angulo $AQP$ es semejante
al tri\'angulo $ABC$.

\begin{tikzpicture}[line cap=round,line join=round,>=triangle 45,x=4*1.0cm,y=4*1.0cm]
\draw [line width=1.pt] (0.863807405686497,-0.14503467816191534) circle (4*0.4939628904042216cm);
\draw [line width=1.pt] (0.23362602737072305,-0.14678033294949364)-- (1.8905714038434012,-0.14219045656037266);
\draw [line width=1.pt] (0.23362602737072305,-0.14678033294949364)-- (0.862439092680299,0.3489263170755732);
\draw [line width=1.pt] (0.862439092680299,0.3489263170755732)-- (1.8905714038434012,-0.14219045656037266);
\draw [line width=1.pt] (0.862439092680299,0.3489263170755732)-- (0.863807405686497,-0.14503467816191534);
\draw[color=black] (0.853259339902057,0.4338390302743115) node {$A$};
\draw[color=black] (0.16477788153390815,-0.13989551836581215) node {$B$};
\draw[color=black] (1.936470167734611,-0.1628449003114171) node {$C$};
\draw[color=black] (0.857849216291178,-0.19038415864614305) node {$D$};
\draw[color=black] (0.34378306070962683,0.03451978442078544) node {$Q$};
\draw[color=black] (1.2938874732576722,0.227294592763867) node {$P$};
\end{tikzpicture}

(P1 OMM 2009)%23aOMM
\end{problema}

\begin{problema}
En el tri\'angulo is\'osceles $ABC$ se tiene que $\angle C>90$. Sean $O$ el circuncentro del tri\'angulo,
$I$ el incentro y $D$ el punto sobre $BC$ de tal manera que las l\'ineas $OD$ y $BI$ son perpendiculares.
Prueba que $ID$ y $AC$ son paralelas.

\begin{tikzpicture}[line cap=round,line join=round,>=triangle 45,x=4*1.0cm,y=4*1.0cm]
\draw [line width=1.pt] (0.7499871211468343,-0.879578056618647) circle (4*0.81082562228421cm);
\draw [line width=1.pt] (0.04085121902764139,-0.4864311857444469)-- (1.4591230232660277,-0.4864311857444469);
\draw [line width=1.pt] (0.04085121902764139,-0.4864311857444469)-- (0.7499871211468345,-0.06875243433443678);
\draw [line width=1.pt] (0.7499871211468345,-0.06875243433443678)-- (1.4591230232660277,-0.4864311857444469);
\draw [line width=1.pt] (0.7499871211468343,-0.879578056618647)-- (0.9404463682306589,-0.18093231449384145);
\draw [line width=1.pt] (1.4591230232660277,-0.4864311857444469)-- (0.7499871211468344,-0.2931121946532464);
\draw[color=black] (-0.018817174030931506,-0.45200711282603945) node {$A$};
\draw[color=black] (1.5096116635463588,-0.43823748365867643) node {$B$};
\draw[color=black] (0.7431023065631532,0.01616027886430148) node {$C$};
\draw [fill=black] (0.7499871211468343,-0.879578056618647) circle (1.0pt);
\draw[color=black] (0.6650744079480964,-0.8421466059013235) node {$O$};
\draw [fill=black] (0.7499871211468344,-0.2931121946532464) circle (1.0pt);
\draw[color=black] (0.6880237898937013,-0.2775918100394419) node {$I$};
\draw[color=black] (0.9771860024083238,-0.14153601280932821) node {$D$};
\end{tikzpicture}

(PRE 1999)
\end{problema}

\begin{problema}
.
\end{problema}

\begin{problema}
.
\end{problema}

\begin{problema}
.
\end{problema}



\begin{tikzpicture}[line cap=round,line join=round,>=triangle 45,x=3*1.0cm,y=3*1.0cm]
\draw [line width=1.pt] (-0.13330614542676317,-0.4291592113950537) circle (3*0.8418908652468511cm);
\draw [line width=1.pt] (-0.5296296296296298,1.234074074074074)-- (-1.1429629629629634,-1.1659259259259263);
\draw [line width=1.pt] (-1.1429629629629634,-1.1659259259259263)-- (1.6748148148148159,-1.1748148148148152);
\draw [line width=1.pt] (1.6748148148148159,-1.1748148148148152)-- (-0.5296296296296298,1.234074074074074);
\draw [line width=1.pt] (-0.5296296296296298,1.234074074074074)-- (-0.5372066356335443,-1.1678368291667132);
\draw [line width=1.pt] (1.6748148148148159,-1.1748148148148152)-- (-0.9723517166488544,-0.4983167012185435);
\draw [line width=1.pt] (-1.1429629629629634,-1.1659259259259263)-- (0.3861267649508784,0.23338865902843856);
\draw [line width=1.pt] (0.26592592592592623,-1.1703703703703707)-- (0.268834513075749,-0.24834824387656002);
\draw [line width=1.pt] (0.268834513075749,-0.24834824387656002)-- (0.572592592592593,0.02962962962962945);
\draw [line width=1.pt] (0.268834513075749,-0.24834824387656002)-- (-0.8362962962962965,0.03407407407407392);
\draw[color=black] (-0.6007407407407409,1.3285185185185187) node {$A$};
\draw[color=black] (-1.2585185185185193,-1.2059259259259263) node {$B$};
\draw[color=black] (1.7725925925925938,-1.152592592592593) node {$C$};
\draw [fill=black] (-0.8362962962962965,0.03407407407407392) circle (1.0pt);
\draw[color=black] (-0.9474074074074077,0.10962962962962948) node {$M_3$};
\draw [fill=black] (0.572592592592593,0.02962962962962945) circle (1.0pt);
\draw[color=black] (0.6348148148148154,0.1718518518518517) node {$M_2$};
\draw [fill=black] (0.26592592592592623,-1.1703703703703707) circle (1.0pt);
\draw[color=black] (0.30592592592592627,-1.232592592592593) node {$M_1$};
\draw [fill=black] (0.3861267649508784,0.23338865902843856) circle (1.0pt);
\draw[color=black] (0.4214814814814819,0.37629629629629624) node {$E$};
\draw [fill=black] (-0.9723517166488544,-0.4983167012185435) circle (1.0pt);
\draw[color=black] (-1.0896296296296302,-0.47703703703703726) node {$F$};
\draw [fill=black] (-0.5372066356335443,-1.1678368291667132) circle (1.0pt);
\draw[color=black] (-0.5296296296296297,-1.232592592592593) node {$D$};
\draw [fill=black] (-0.5354468039292756,-0.6099701789135471) circle (1.0pt);
\draw[color=black] (-0.41407407407407415,-0.573703703703704) node {$H$};
\draw [fill=black] (0.268834513075749,-0.24834824387656002) circle (1.0pt);
\draw[color=black] (0.1370370370370373,-0.28148148148148167) node {$O$};
\draw [fill=black] (-0.5325382167794527,0.31205194758026333) circle (1.0pt);
\draw [fill=black] (0.5696840054427702,-0.8923924968641812) circle (1.0pt);
\draw [fill=black] (-0.8392048834461193,-0.8879480524197367) circle (1.0pt);
\end{tikzpicture}

\begin{tikzpicture}[line cap=round,line join=round,>=triangle 45,x=4*1.0cm,y=4*1.0cm]
\draw [line width=1.pt] (0.05111111111111182,0.7718518518518515)-- (-0.7488888888888887,-0.6859259259259256);
\draw [line width=1.pt] (-0.7488888888888887,-0.6859259259259256)-- (1.4496296296296314,-0.68);
\draw [line width=1.pt] (1.4496296296296314,-0.68)-- (0.05111111111111182,0.7718518518518515);
\draw [line width=1.pt] (0.35037037037037133,-0.6829629629629628)-- (0.05111111111111182,0.7718518518518515);
\draw [line width=1.pt] (0.05111111111111182,0.7718518518518515)-- (0.05503459054003266,-0.6837590162778693);
\draw [line width=1.pt] (-0.7488888888888887,-0.6859259259259256)-- (0.3944533835691012,0.4154160181560577);
\draw [line width=1.pt] (0.05295851271490794,0.0864658568434467)-- (0.34944666956847326,-0.3402699654587605);
\draw [line width=1.pt] (1.4496296296296314,-0.68)-- (-0.23753140041401427,0.24588105307273342);
\draw [line width=1.pt] (0.35037037037037133,-0.6829629629629628)-- (0.34944666956847326,-0.3402699654587605);
\draw [line width=1.pt] (0.7503703703703715,0.04592592592592572)-- (0.34944666956847326,-0.3402699654587605);
\draw[color=black] (0.09259259259259334,0.8814814814814811) node {$A$};
\draw[color=black] (-0.8318518518518517,-0.6948148148148146) node {$B$};
\draw[color=black] (1.5088888888888907,-0.7125925925925922) node {$C$};
\draw [fill=black] (0.05295851271490794,0.0864658568434467) circle (1.pt);
\draw[color=black] (-0.043703703703703065,0.09111111111111105) node {$H$};
\draw[color=black] (0.3533333333333342,-0.736296296296296) node {$M$};
\draw [fill=black] (0.34944666956847326,-0.3402699654587605) circle (1.pt);
\draw[color=black] (0.41259259259259357,-0.3392592592592591) node {$O$};
\draw [fill=black] (0.2506172839506183,-0.19802469135802492) circle (1.pt);
\draw[color=black] (0.29407407407407493,-0.12) node {$G$};
\end{tikzpicture}



\begin{tikzpicture}[line cap=round,line join=round,>=triangle 45,x=3*1.0cm,y=3*1.0cm]
\draw [line width=1.pt] (0.29056581856682023,-0.3269839728432721) circle (3*1.1246238155739503cm);
\draw [line width=1.pt] (-0.5474074074074076,-1.0770370370370375)-- (1.1177777777777793,-1.0888888888888884);
\draw [line width=1.pt] (1.1177777777777793,-1.0888888888888884)-- (0.05111111111111182,0.7718518518518515);
\draw [line width=1.pt] (-0.5474074074074076,-1.0770370370370375)-- (0.05111111111111182,0.7718518518518515);
\draw [dashed, line width=1.pt] (1.3704976124706072,-0.6408725630657572)-- (-0.23768623011828507,-0.12027458838153174);
\draw [dashed, line width=1.pt] (1.3704976124706072,-0.6408725630657572)-- (0.9869817222381007,-0.8607224364474491);
%\draw [line width=1.pt] (-0.23768623011828507,-0.12027458838153174)-- (1.367296244086496,-1.0906648210334342);
\draw [line width=1.pt] (-0.8406626189286864,-1.0749498113322598)-- (1.6213083527394943,-1.0924727363975124);
\draw [dashed, line width=1.pt] (1.3748561424656531,-0.6408725630657572)-- (1.367296244086496,-1.0906648210334342);
\draw[color=black] (0.03925925925925945,0.9362962962962963) node {$A$};
\draw[color=black] (-0.6451851851851853,-1.1348148148148152) node {$B$};
\draw[color=black] (1.2037037037037046,-1.1348148148148152) node {$C$};
\draw [fill=black] (1.3704976124706072,-0.6408725630657572) circle (1.0pt);
\draw[color=black] (1.505925925925927,-0.5748148148148151) node {$P$};
\draw [fill=black] (-0.23768623011828507,-0.12027458838153174) circle (1.0pt);
\draw[color=black] (-0.3518518518518519,-0.04148148148148166) node {$C'$};
\draw [fill=black] (0.9869817222381007,-0.8607224364474491) circle (1.0pt);
\draw[color=black] (1.0348148148148155,-0.7170370370370374) node {$B'$};
\draw [fill=black] (1.367296244086496,-1.0906648210334342) circle (1.0pt);
\draw[color=black] (1.3725925925925935,-1.1614814814814818) node {$A'$};
\end{tikzpicture}


% \chapter{Cuatro principios en matemáticas}

\section{Principio de las casillas (principio de la pichonera)}

\begin{ejercicio}
Demuestra que de tres números enteros, hay dos cuya suma es par.
\end{ejercicio}
\vspace{2cm}

El principio de las casillas establece que si hay $n+1$ elementos repartidos en $n$ casillas, entonces hay una casilla que tiene al menos dos elementos.

Una versión más general nos dice que si hay $nk+1$ elementos repartidos en $n$ clases distintas, entonces debe haber por lo menos una clase con al menos $k+1$ elementos.

\begin{ejercicio}
Muestra que en un conjunto de 25 personas hay tres que tienen el mismo signo zodiacal.
\end{ejercicio}
\vspace{2cm}

\begin{ejercicio}
A un auditorio asistieron 367 personas. Demuestra que hay dos que tienen el mismo cumpleaños
\end{ejercicio}
\vspace{2cm}

Nota: Comparar con el problema probabilístico. Bastan 23 personas para que la probabilidad de que dos de ellas tengan el mismo cumpleaños supere $\frac{1}{2}$.

\begin{ejercicio}
La ciudad de México tiene más de 16 millones de habitantes. Es conocido que una persona tiene menos de un millón de cabellos. Demuestra que hay al menos 17 personas con la misma cantidad de cabellos 
\end{ejercicio}
\vspace{2cm}

\begin{ejercicio}
Una bolsa contiene suficientes canicas negras y blancas. ¿Cuál es el mínimo número de pelotas que se tienen que extraer para garantizar que se hayan elegido al menos dos de un mismo color?

¿y para garantizar diez de un mismo color?

¿Y en la bolsa también hubieran suficientes canicas rojas?
\end{ejercicio}
\vspace{2cm}

El principio de las casillas también admite otras variaciones:

Si se reparten $n-1$ elementos en $n$ casillas, entonces necesariamente hay una casilla vacía.

O bien, más generalmente, si se reparten $kn-1$ elementos en $n$ casillas, entonces necesariamente hay una casilla con a lo más $k-1$ elementos.
\vspace{2cm}


El principio es bastante sencillo, lo que a veces es muy ingenioso es determinar cuales son las casillas que nos ayudan a resolver el problema.

\begin{ejercicio}
Se te otorga una lista de doce números enteros. Muestra que hay dos cuya diferencia es divisible entre $11$.
\end{ejercicio}
\vspace{2cm}

\begin{ejercicio}
Se te otorga una colección de $5$ puntos en el plano con coordenadas enteras. Muestra que algún punto medio de dos de esos puntos también tiene coordenadas enteras.
\end{ejercicio}
\vspace{2cm}

\begin{problema} [OMM '90]
Se tienen 19 puntos en el plano con coordenadas enteras. Muestra que se pueden elegir tres puntos cuyo centro de gravedad también tiene coordenadas enteras.
\end{problema}
\vspace{2cm}
\newpage

\begin{ejercicio}
¿Cuál es el máximo número de reyes que se pueden colocar en un tablero de ajedrez sin que estos se ataquen entre sí? 
\end{ejercicio}
%Sug: Se pueden poner $16$ con un acomodo regular. Casillas de $2\times 2$ muestra que no es posible con $17$.

  \begin{tikzpicture}[x=0.5cm,y=0.5cm]
     \foreach \i in {0,...,8} \draw[black] (0,\i) -- (8,\i);
    \foreach \i in {0,...,8} \draw[black] (\i,0) -- (\i,8);       
  \end{tikzpicture}
  
  

\begin{ejercicio}
¿Cuál es el máximo número de casillas que se pueden colorear en un tablero de ajedrez de tal manera que si se coloca cualquier triminó sobre el tablero, cubra al menos un cuadrito sin colorear?
\end{ejercicio}
%Sug: Se pueden colorear $32$ (ajedrez). Casillas de $2\times 2$ muestran que no es posible con $33$.

\begin{proposicion}
[Lema de los saludos I]
Demuestra que en una reunión de $n$ personas, en todo momento hay dos personas que han saludado a la misma cantidad de personas.
\end{proposicion}
\vspace{2cm}

\begin{ejercicio}
Los cuadritos de una cuadrícula de $3\times 7$ se colorean de rojo o azul. Muestra que hay un rectángulo con sus 4 esquinas del mismo color. 
\end{ejercicio}
  \begin{tikzpicture}[x=0.5cm,y=0.5cm]
     \foreach \i in {0,...,3} \draw[black] (0,\i) -- (7,\i);
    \foreach \i in {0,...,7} \draw[black] (\i,0) -- (\i,3);       
  \end{tikzpicture}

\begin{ejercicio}
Los cuadritos de una cuadrícula de $19\times 4$ se colorean de rojo, azul o verde. Muestra que hay un rectángulo con sus 4 esquinas del mismo color. 
\end{ejercicio}
%Sug: En cada columna domina un color. Combinaciones de 4 en 2 da seis.
  \begin{tikzpicture}[x=0.5cm,y=0.5cm]
     \foreach \i in {0,...,4} \draw[black] (0,\i) -- (19,\i);
    \foreach \i in {0,...,19} \draw[black] (\i,0) -- (\i,4);       
  \end{tikzpicture}

\begin{problema}
En una cuadrícula de $8\times 8$ se han escogido arbitrariamente diez cuadritos y se han marcado sus centros. Demuestra que, o bien existen dos puntos marcados con distancia menor o igual que $\sqrt{2}$, o bien algún punto marcado se encuentra a distancia $\frac{1}{2}$ del borde de la cuadrícula. 
\end{problema}
  \begin{tikzpicture}[x=0.5cm,y=0.5cm]
     \foreach \i in {0,...,8} \draw[black] (0,\i) -- (8,\i);
    \foreach \i in {0,...,8} \draw[black] (\i,0) -- (\i,8);       
  \end{tikzpicture}
  
 
\begin{ejercicio}
Cinco puntos se colocan en un triángulo equilátero de lado $2$. Demuestra que hay dos puntos a distancia menor o igual $1$.
\end{ejercicio}
\vspace{2cm}

\begin{ejercicio}
Demuestra que un triángulo equilátero de lado $1$ no puede cubrirse completamente con dos triángulos equiláteros de lado menor.
\end{ejercicio}
\vspace{2cm}

\begin{ejercicio}
Se colocan siete puntos en un disco de radio $1$, de tal manera que la distancia entre cada par de ellos es mayor o igual a $1$. Demuestra que uno de los puntos tiene que ser el centro del círculo.
\end{ejercicio}
\vspace{2cm}

\begin{ejercicio}
Se colocan 501 tarjetas de tamaño $1\times 1$ sobre un recipiente en forma de prisma con base cuadrangular de $10\times 10$. Muestra que hay un punto de la base cuadrada que está cubierto por al menos 6 tarjetas.
\end{ejercicio}
\vspace{2cm}


\begin{ejercicio}
Se colocan 101 puntos en un cuadrado de lado $10$. Muestra que hay tres puntos que forman un triángulo de área menor o igual que 1.
\end{ejercicio}
\vspace{2cm}

\begin{ejercicio}
Se toman 8 números diferentes del conjunto $\{1, 2, 3, \dots , 14, 15 \}$. Demuestra que hay tres parejas usando esos números que tienen la misma diferencia (positiva) 
\end{ejercicio}
\vspace{2cm}

\newpage
\begin{problema}
Dado un conjunto de diez números de dos dígitos, mostrar que existen dos subconjuntos disjuntos con la misa suma.
\end{problema}
Sug: Las posibles sumas van de $0$ hasta $99+98+97+\cdots +90 < 1000$. ¿Cuántos subconjuntos hay?
\vspace{2cm}


\begin{ejercicio}
Hay 650 puntos dentro de un círculo de radio 16. Muestra que existe un anillo de radio interior 2 y radio exterior 3 que cubre al menos 10 puntos.
\end{ejercicio}
\vspace{2cm}

%Sug: Un anillo $R$ tapa a un punto $P$ si y solo si el anillo con centro en $P$ tapa al centro de $R$. 

\begin{problema}
Hay $33$ torres en un tablero de ajedrez. Muestra que hay un subconjunto de $5$ torres que no se atacan entre sí.
\end{problema}
%Sug: Si hay 9 torres hay dos distinta columna y dos en distinta fila. Si hay 17 torres, hay tres que no se atacan...


\newpage

Una {\bf gráfica} o {\bf grafo} (simple) es un conjunto de puntos $V=\{v_1, v_2,\dots, v_n \}$ llamados {\bf vértices} y un subconjunto de pares (no-ordenados) de vértices, llamados {\bf aristas}. 

A la gráfica de $n$-vértices que contiene a todas las aristas se le llama {\bf gráfica completa} o $K_n$.

\begin{tikzpicture}[line cap=round,line join=round,>=triangle 45,x=1.5cm,y=1.5cm]
\draw [line width=1.pt] (1.,0.)-- (-0.5,0.8660254037844387);
\draw [line width=1.pt] (-0.5,0.8660254037844387)-- (-0.5,-0.8660254037844385);
\draw [line width=1.pt] (-0.5,-0.8660254037844385)-- (1.,0.);
\draw [fill=black] (-0.5,0.8660254037844387) circle (1.5pt);
\draw [fill=black] (-0.5,-0.8660254037844385) circle (1.5pt);
\draw [fill=black] (1.,0.) circle (1.5pt);
\draw [fill=black] (1.,0.) circle (1.5pt);
\draw [fill=black] (-0.5,0.8660254037844387) circle (1.5pt);
\draw [fill=black] (-0.5,-0.8660254037844385) circle (1.5pt);
\end{tikzpicture}
\begin{tikzpicture}[line cap=round,line join=round,>=triangle 45,x=1.5cm,y=1.5cm]
\draw [line width=1.pt] (0.,-1.)-- (0.,1.);
\draw [line width=1.pt] (0.,1.)-- (-1.,0.);
\draw [line width=1.pt] (-1.,0.)-- (0.,-1.);
\draw [line width=1.pt] (0.,-1.)-- (1.,0.);
\draw [line width=1.pt] (1.,0.)-- (0.,1.);
\draw [line width=1.pt] (1.,0.)-- (-1.,0.);
\draw [fill=black] (0.,1.) circle (1.5pt);
\draw [fill=black] (-1.,0.) circle (1.5pt);
\draw [fill=black] (0.,-1.) circle (1.5pt);
\draw [fill=black] (0.,-1.) circle (1.5pt);
\draw [fill=black] (0.,1.) circle (1.5pt);
\draw [fill=black] (-1.,0.) circle (1.5pt);
\draw [fill=black] (1.,0.) circle (1.5pt);
\draw [fill=black] (1.,0.) circle (1.5pt);
\end{tikzpicture}
\begin{tikzpicture}[line cap=round,line join=round,>=triangle 45,x=1.5cm,y=1.5cm]
\draw [line width=1.pt] (-0.8090169943749475,-0.587785252292473)-- (0.30901699437494745,0.9510565162951535);
\draw [line width=1.pt] (0.30901699437494745,0.9510565162951535)-- (-0.8090169943749473,0.5877852522924732);
\draw [line width=1.pt] (-0.8090169943749473,0.5877852522924732)-- (-0.8090169943749475,-0.587785252292473);
\draw [line width=1.pt] (-0.8090169943749475,-0.587785252292473)-- (0.30901699437494723,-0.9510565162951536);
\draw [line width=1.pt] (0.30901699437494723,-0.9510565162951536)-- (0.30901699437494745,0.9510565162951535);
\draw [line width=1.pt] (0.30901699437494723,-0.9510565162951536)-- (-0.8090169943749473,0.5877852522924732);
\draw [line width=1.pt] (0.30901699437494745,0.9510565162951535)-- (1.,0.);
\draw [line width=1.pt] (1.,0.)-- (-0.8090169943749473,0.5877852522924732);
\draw [line width=1.pt] (1.,0.)-- (-0.8090169943749475,-0.587785252292473);
\draw [line width=1.pt] (1.,0.)-- (0.30901699437494723,-0.9510565162951536);
\draw [fill=black] (0.30901699437494745,0.9510565162951535) circle (1.5pt);
\draw [fill=black] (-0.8090169943749473,0.5877852522924732) circle (1.5pt);
\draw [fill=black] (-0.8090169943749475,-0.587785252292473) circle (1.5pt);
\draw [fill=black] (-0.8090169943749475,-0.587785252292473) circle (1.5pt);
\draw [fill=black] (0.30901699437494745,0.9510565162951535) circle (1.5pt);
\draw [fill=black] (-0.8090169943749473,0.5877852522924732) circle (1.5pt);
\draw [fill=black] (0.30901699437494723,-0.9510565162951536) circle (1.5pt);
\draw [fill=black] (0.30901699437494723,-0.9510565162951536) circle (1.5pt);
\draw [fill=black] (1.,0.) circle (1.5pt);
\draw [fill=black] (1.,0.) circle (1.5pt);
\end{tikzpicture}
\begin{tikzpicture}[line cap=round,line join=round,>=triangle 45,x=1.5cm,y=1.5cm]
\draw [line width=1.pt] (-1.,0.)-- (0.5,0.8660254037844386);
\draw [line width=1.pt] (0.5,0.8660254037844386)-- (-0.5,0.8660254037844387);
\draw [line width=1.pt] (-0.5,0.8660254037844387)-- (-1.,0.);
\draw [line width=1.pt] (-1.,0.)-- (-0.5,-0.8660254037844385);
\draw [line width=1.pt] (-0.5,-0.8660254037844385)-- (0.5,0.8660254037844386);
\draw [line width=1.pt] (-0.5,-0.8660254037844385)-- (-0.5,0.8660254037844387);
\draw [line width=1.pt] (0.5,0.8660254037844386)-- (1.,0.);
\draw [line width=1.pt] (1.,0.)-- (-0.5,0.8660254037844387);
\draw [line width=1.pt] (1.,0.)-- (-1.,0.);
\draw [line width=1.pt] (1.,0.)-- (-0.5,-0.8660254037844385);
\draw [line width=1.pt] (0.5,-0.866025403784439)-- (-1.,0.);
\draw [line width=1.pt] (0.5,-0.866025403784439)-- (-0.5,0.8660254037844387);
\draw [line width=1.pt] (0.5,-0.866025403784439)-- (0.5,0.8660254037844386);
\draw [line width=1.pt] (0.5,-0.866025403784439)-- (-0.5,-0.8660254037844385);
\draw [line width=1.pt] (0.5,-0.866025403784439)-- (1.,0.);
\draw [fill=black] (0.5,0.8660254037844386) circle (1.5pt);
\draw [fill=black] (-0.5,0.8660254037844387) circle (1.5pt);
\draw [fill=black] (-1.,0.) circle (1.5pt);
\draw [fill=black] (-1.,0.) circle (1.5pt);
\draw [fill=black] (0.5,0.8660254037844386) circle (1.5pt);
\draw [fill=black] (-0.5,0.8660254037844387) circle (1.5pt);
\draw [fill=black] (-0.5,-0.8660254037844385) circle (1.5pt);
\draw [fill=black] (-0.5,-0.8660254037844385) circle (1.5pt);
\draw [fill=black] (1.,0.) circle (1.5pt);
\draw [fill=black] (1.,0.) circle (1.5pt);
\draw [fill=black] (0.5,-0.866025403784439) circle (1.5pt);
\draw [fill=black] (0.5,-0.866025403784439) circle (1.5pt);
\end{tikzpicture}

\begin{ejercicio}
Colorea las aristas del $K_5$ de manera que no se forme un triángulo monocromático.
\end{ejercicio}

\begin{ejercicio}
Demuestra que si se colorean las aristas del $K_6$ de dos colores, entonces forzosamente se forma un triángulo monocromático.
\end{ejercicio}
\vspace{4cm}

Para $n,m\geq 1$, el {\bf número de Ramsey} $R(n,m)=R$ se define como el número natural más pequeño tal que, de colorearse las aristas de un $K_R$ con rojo y azul, entonces se garantiza la existencia de un $K_n$ rojo o bien de un $K_m$ azul.

Los dos ejercicios anteriores juntos significan justamente que el número de Ramsey $R(3,3)=6$.
\newpage

\begin{ejercicio}
Para todo $n,m\geq 1$, $R(n,1)= 1$ y $R(1,m)=1$.
\end{ejercicio}
\vspace{2cm}

\begin{ejercicio}
Sea $m\geq 2$. Demuestra que $R(2,m)= m$.
\end{ejercicio}
\vspace{2cm}

\begin{ejercicio}
Demuestra que si se colorean las aristas del $K_{10}$ de dos colores, entonces forzosamente se forma un $K_3$ rojo, o un $K_4$ azul.
\end{ejercicio}
\vspace{2cm}

Observa que el ejercicio anterior muestra que $R(3,4)\leq 10$. Más adelante mostraremos que $R(3,4)\leq 9$. Como es posible encontrar una coloración del $K_8$ sin un triángulo rojo ni un $K_4$ azul, se concluye que $R(3,4)=9$.

En general se pueden encontrar cotas sencillas para los números de Ramsey:

\begin{ejercicio}
Demuestra que $R(m,n)\leq R(m-1,n)+ R(m,n-1)$.
\end{ejercicio}
\vspace{2cm}

Se pueden considerar más colores:

\begin{ejercicio}
Demuestra que si se colorean las aristas del $K_{17}$ de tres colores, entonces forzosamente se forma un triángulo monocromático.
\end{ejercicio}
\vspace{2cm}

\begin{proposicion}
[Lema de los saludos II / Handshaking Lemma]
Demuestra que en una reunión de $n$ personas, en todo momento hay siempre una cantidad par de personas que han saludado a un número impar de personas.
\end{proposicion}
\vspace{2cm}

\begin{ejercicio}
Demuestra que si se colorean las aristas del $K_{9}$ de dos colores, entonces forzosamente se forma un triángulo rojo, o un $K_4$ azul.
\end{ejercicio}
\vspace{2cm}

De hecho, el resultado anterior es un caso particular del siguiente:

\begin{proposicion}
Si tanto $R(m-1,n)$ como $R(m,n-1)$ son pares, entonces se cumple que $R(m,n)\leq R(m-1,n)+R(m,n-1)-1$
\end{proposicion}
La demostración se deja como ejercicio.
\vspace{2cm}

Usando los resultados anteriores se puede deducir que $R(4,4)\leq R(4,3)+R(3,4)=18$.

De todas las $6.4\times 10^{22}$ posibles bi-coloraciones de las aristas del $K_{16}$, solamente hay dos que no contienen un $K_4$ monocromático. De todas las $2.46 \times 10^{26}$ posibles bi-coloraciones de aristas del $K_{17}$, solamente hay una que no tiene un $K_4$ monocromático, estableciendo así que $R(4,4)=18$ (1979). En 1995 se demostró que $R(4,5)=25$. De $R(5,5)$ solamente sabemos que $43\leq R(5,5)\leq 48$

Sobre la complejidad de calcular numeros de Ramsey:

\epigraph{Paul Erdös nos pide imaginarnos que una civilización extraterrestre, mucho más poderosa que la nuestra, ha aterrizado en nuestro planeta, demandándonos calcular $R(5,5)$ para mostrarles que somos vida inteligente y no destruirnos. En ese caso, nos sugiere poner a todos nuestros científicos y a todas nuestras computadoras a intentar calcular ese valor. Si en cambio, nos exigieran calcular $R(6,6)$, él considera mucho más viable intentar destruir a los invasores.}{Joel Spencer}

\begin{problema}
[OMM '98] Se colorean todas las aristas de un $K_8$ con dos colores. Mostrar que existen al menos siete triángulos monocromáticos.
\end{problema}
\vspace{2cm}

En lugar de colorear aristas, el teorema de Ramsey también permite colorear caras de dimensiones mayores. El caso que ya estudiamos $R^d(m,n)$ corresponde a $d=2$. Para dimensión $d=3$ los números de Ramsey se definen de la siguiente manera.

Una $3$-hipergráfica completa $K^3_n$ es un conjunto de $n$ vértices junto con todas sus 3-caras (subconjuntos de 3 vértices).

\begin{ejercicio}
Si $m\geq 3$, $R^{d=3}(m,3)=m=R^{d=3}(3,m)$
\end{ejercicio}
\vspace{2cm}

\begin{problema}
Muestra que $R^{d=3}(4,4)\leq R^{d=2}(4,4)+1$
\end{problema}
\vspace{2cm}

\begin{problema}
Muestra que $R^{d=4}(5,5)\leq R^{d=3}(5,5)+1$
\end{problema}
\vspace{2cm}

\begin{problema}
Demuestra que para todo número natural $k\geq 1$ existe un $N$ suficientemente grande tal que para cualquier configuración de $N$ puntos en el plano en posición general, hay un subconjunto de $k$ puntos que forman un polígono convexo.
\end{problema}
Sug: $N=R^{d=4}(k,5)$ ¿Puede haber un subconjunto de 5 puntos donde todos los sub-cuadriláteros no sean convexos?.
\vspace{2cm}

\newpage

\section{Principio de inclusión y exclusión}

\section{Principio de inducción matemática}

\begin{ejercicio} Fibonacci:
¿Identidad de Cassini?, 

¿Identidad de Catalan?, 

¿Identidad de d'Ocagne?
\end{ejercicio}

%Revisar si tienen solución elemental, si no mover a inducción.


\section{Principio de invarianza}

\subsection*{Problemas de tableros}

\begin{problema}
A un tablero de ajedréz de $8\times 8$ se le remueven dos esquinas opuestas. ¿Se pueden cubrir todas las casillas con $31$ dominós de $2\times 1$?
\end{problema}
%Sug: de qué color son las esquinas que se removieron?.

\begin{problema}
¿Se puede formar un rectángulo usando una sola vez, sin traslaparse, cada una de las seis fichas del tetris?

\begin{tikzpicture}[line cap=round,line join=round,>=triangle 45,x=.5cm,y=.5cm]
\draw [line width=1pt] (0,0)-- (2,0);
\draw [line width=1pt] (0,0)-- (0,2);
\draw [line width=1pt] (0,2)-- (2,2);
\draw [line width=1pt] (2,2)-- (2,0);
\draw [line width=1pt] (0,1)-- (2,1);
\draw [line width=1pt] (1,2)-- (1,0);
\draw [line width=1pt] (5.6,2.6)-- (7.6,2.6);
\draw [line width=1pt] (7.6,2.6)-- (7.6,0.6);
\draw [line width=1pt] (6.6,0.6)-- (8.6,0.6);
\draw [line width=1pt] (8.6,0.6)-- (8.6,1.6);
\draw [line width=1pt] (8.6,1.6)-- (5.6,1.6);
\draw [line width=1pt] (5.6,2.6)-- (5.6,1.6);
\draw [line width=1pt] (6.6,2.6)-- (6.6,0.6);
\draw [line width=1pt] (0,4)-- (0,7);
\draw [line width=1pt] (0,7)-- (1,7);
\draw [line width=1pt] (1,7)-- (1,4);
\draw [line width=1pt] (1,4)-- (0,4);
\draw [line width=1pt] (0,5)-- (1,5);
\draw [line width=1pt] (1,6)-- (0,6);
\draw [line width=1pt] (5.9,4.7)-- (5.9,6.7);
\draw [line width=1pt] (5.9,6.7)-- (6.9,6.7);
\draw [line width=1pt] (6.9,6.7)-- (6.9,4.7);
\draw [line width=1pt] (7.9,5.7)-- (7.9,4.7);
\draw [line width=1pt] (8.9,5.7)-- (8.9,4.7);
\draw [line width=1pt] (8.9,5.7)-- (5.9,5.7);
\draw [line width=1pt] (5.9,4.7)-- (8.9,4.7);
\draw [line width=1pt] (4.5,1.2)-- (4.5,4.2);
\draw [line width=1pt] (3.5,4.2)-- (3.5,1.2);
\draw [line width=1pt] (2.5,3.2)-- (4.5,3.2);
\draw [line width=1pt] (2.5,2.2)-- (4.5,2.2);
\draw [line width=1pt] (2.5,3.2)-- (2.5,2.2);
\draw [line width=1pt] (3.5,4.2)-- (4.5,4.2);
\draw [line width=1pt] (3.5,1.2)-- (4.5,1.2);
\draw [line width=1pt] (2,7)-- (2,5);
\draw [line width=1pt] (3,5)-- (3,7);
\draw [line width=1pt] (2,7)-- (5,7);
\draw [line width=1pt] (5,6)-- (2,6);
\draw [line width=1pt] (2,5)-- (3,5);
\draw [line width=1pt] (4,7)-- (4,6);
\draw [line width=1pt] (5,7)-- (5,6);
\draw [line width=1pt] (0,4)-- (0,3);
\draw [line width=1pt] (0,3)-- (1,3);
\draw [line width=1pt] (1,3)-- (1,4);
\draw [fill=black] (0,0) circle (1.5pt);
\draw [fill=black] (1,0) circle (1.5pt);
\draw [fill=black] (2,0) circle (1.5pt);
\draw [fill=black] (0,1) circle (1.5pt);
\draw [fill=black] (1,1) circle (1.5pt);
\draw [fill=black] (2,1) circle (1.5pt);
\draw [fill=black] (2,2) circle (1.5pt);
\draw [fill=black] (1,2) circle (1.5pt);
\draw [fill=black] (0,2) circle (1.5pt);
\draw [fill=black] (0,4) circle (1.5pt);
\draw [fill=black] (0,5) circle (1.5pt);
\draw [fill=black] (0,6) circle (1.5pt);
\draw [fill=black] (0,7) circle (1.5pt);
\draw [fill=black] (1,6) circle (1.5pt);
\draw [fill=black] (1,5) circle (1.5pt);
\draw [fill=black] (1,4) circle (1.5pt);
\draw [fill=black] (1,7) circle (1.5pt);
\draw [fill=black] (2,6) circle (1.5pt);
\draw [fill=black] (2,7) circle (1.5pt);
\draw [fill=black] (3,7) circle (1.5pt);
\draw [fill=black] (4,7) circle (1.5pt);
\draw [fill=black] (5,7) circle (1.5pt);
\draw [fill=black] (5,6) circle (1.5pt);
\draw [fill=black] (4,6) circle (1.5pt);
\draw [fill=black] (3,6) circle (1.5pt);
\draw [fill=black] (2,5) circle (1.5pt);
\draw [fill=black] (3,5) circle (1.5pt);
\draw [fill=black] (5.9,6.7) circle (1.5pt);
\draw [fill=black] (5.9,5.7) circle (1.5pt);
\draw [fill=black] (6.9,5.7) circle (1.5pt);
\draw [fill=black] (6.9,6.7) circle (1.5pt);
\draw [fill=black] (5.9,4.7) circle (1.5pt);
\draw [fill=black] (6.9,4.7) circle (1.5pt);
\draw [fill=black] (7.9,4.7) circle (1.5pt);
\draw [fill=black] (8.9,4.7) circle (1.5pt);
\draw [fill=black] (8.9,5.7) circle (1.5pt);
\draw [fill=black] (7.9,5.7) circle (1.5pt);
\draw [fill=black] (5.6,2.6) circle (1.5pt);
\draw [fill=black] (6.6,2.6) circle (1.5pt);
\draw [fill=black] (7.6,2.6) circle (1.5pt);
\draw [fill=black] (6.6,1.6) circle (1.5pt);
\draw [fill=black] (5.6,1.6) circle (1.5pt);
\draw [fill=black] (6.6,0.6) circle (1.5pt);
\draw [fill=black] (7.6,0.6) circle (1.5pt);
\draw [fill=black] (8.6,0.6) circle (1.5pt);
\draw [fill=black] (8.6,1.6) circle (1.5pt);
\draw [fill=black] (7.6,1.6) circle (1.5pt);
\draw [fill=black] (2.5,3.2) circle (1.5pt);
\draw [fill=black] (3.5,4.2) circle (1.5pt);
\draw [fill=black] (4.5,4.2) circle (1.5pt);
\draw [fill=black] (3.5,3.2) circle (1.5pt);
\draw [fill=black] (3.5,2.2) circle (1.5pt);
\draw [fill=black] (4.5,2.2) circle (1.5pt);
\draw [fill=black] (2.5,2.2) circle (1.5pt);
\draw [fill=black] (3.5,1.2) circle (1.5pt);
\draw [fill=black] (4.5,1.2) circle (1.5pt);
\draw [fill=black] (4.5,3.2) circle (1.5pt);
\draw [fill=black] (0,3) circle (1.5pt);
\draw [fill=black] (1,3) circle (1.5pt);
\end{tikzpicture}

\end{problema}

%Sug: Tablero de ajedréz

\begin{problema}
¿Se puede cubrir un tablero de $7\times 7$ al que se le removieron tres esquinas, con $24$ dominós?
\end{problema}

%Sug: Tablero de ajedréz


\begin{problema}
¿Se puede cubrir un tablero $8\times 8$ con 15 tetraminos T y un tetramino cuadrado?.
\end{problema}

%Sug: Tablero de ajedréz


\begin{problema}
¿Se puede cubrir un tablero de $10\times 10$ con $25$ fichas de $1\times 4$?
\end{problema}

%Sug: Tablero de ajedréz doble, con cuadros negros y blacos de 2\times 2

\begin{problema}
¿Para qué valores de $n$ se puede cubrir un tablero de $n\times n$ sin las cuatro esquinas con L-tetraminos que no se traslapen?.
\end{problema}

%Sug: Se pueden los n=4k+2 (Mostrar construcción inductiva). Para descartar n=4k usar coloración por columnas + paridad.

\begin{problema}
Demuestra que no se puede cubrir un tablero de $4\times 11$ con once L-tetraminos.
\end{problema}

%Sug: Idem. Columnas o filas (una de las dos es la que sirve).

\begin{problema}
%[IWYMIC]
Se tienen muchas fichas de seis cuadritos como en el de la figura.

¿Es posible rellenar un rectángulo con una cantidad impar de esas fichas?. Justifica tu respuesta.

\begin{tikzpicture}[line cap=round,line join=round,>=triangle 45,x=.7cm,y=.7cm]
\draw [line width=1pt] (0,0)-- (0,2);
\draw [line width=1pt] (0,2)-- (2,2);
\draw [line width=1pt] (2,2)-- (2,0);
\draw [line width=1pt] (1,2)-- (1,0);
\draw [line width=1pt] (0,1)-- (4,1);
\draw [line width=1pt] (0,0)-- (4,0);
\draw [line width=1pt] (4,1)-- (4,0);
\draw [line width=1pt] (3,1)-- (3,0);
\draw [fill=black] (0,0) circle (1.5pt);
\draw [fill=black] (1,0) circle (1.5pt);
\draw [fill=black] (2,0) circle (1.5pt);
\draw [fill=black] (0,1) circle (1.5pt);
\draw [fill=black] (1,1) circle (1.5pt);
\draw [fill=black] (2,1) circle (1.5pt);
\draw [fill=black] (2,2) circle (1.5pt);
\draw [fill=black] (1,2) circle (1.5pt);
\draw [fill=black] (0,2) circle (1.5pt);
\draw [fill=black] (3,1) circle (1.5pt);
\draw [fill=black] (3,0) circle (1.5pt);
\draw [fill=black] (4,1) circle (1.5pt);
\draw [fill=black] (4,0) circle (1.5pt);
\end{tikzpicture}

\end{problema}

%Sug: Sí se puede. Sug2: Ej. con 15 fichas el de 9\times 10. ¿Mínimo??

\begin{problema}
Se puede cubrir un tablero de $5\times 5$, exceptuando un cuadrito de $1\times 1$, con $6$ tetraminos L.
¿Cuáles son las posibles posiciones del cuadrito de $1\times 1$?
\end{problema}

\begin{problema}
Una tablero de $7\times 7$ se puede cubrir con 16 fichas de $3\times 1$ y una ficha de $1\times 1$
¿Cuáles son las posibles posiciones del cuadrito de $1\times 1$?
\end{problema}

\begin{problema}
Para qué valores de $n, m$ se puede cubrir un tablero de $m\times n$ con fichas de $1\times k$
\end{problema}

\begin{problema}
Se tiene un tableros de $9\times 9$ con un insecto en cada casilla, cada insecto vuela a alguna casilla vecina con la que comparte solo un vértice (algunos insectos se mueven a la misma casilla). ¿Cuál es el mínimo número de casillas desocupadas?
\end{problema}

\begin{problema}
[OMM]
1999 fichas, 2 jugadores.
\end{problema}

\begin{problema}
[IMO]
Ganchos.
\end{problema}


% \setcounter{chapter}{8}

\chapter{Gráficas}

\section{Definiciones}

Las gráficas son de suma importancia en las matemáticas, ciencias de la computación y, más recientemente, en ciencias de datos. Su estudio moderno sistemático comenzó con König en los 30's.

No raramente aparecen problemas en las olimpiadas nacionales o internacionales de matemáticas que tienen directamente que ver con gráficas o que tienen una solución directa, o un atajo, usando herramientas de gráficas.

Además de la importancia intrínseca de la teoría de gráficas, aprovecharemos su riqueza para repasar algunos problemas de conteo (en particular el principio de inclusión y exclusión).

Comenzamos con las principales definiciones.


\begin{definicion}
\begin{enumerate}
 
\item Una {\bf gráfica} es un par $G=(V,A)$ que consta de un conjunto de {\bf vértices} $V$ y un conjunto de {\bf aristas} $A$. 

\item El {\bf grado} de un vértice $v$ es el número de aristas $d(v)$ que son incidentes al vértice $v$. Por ejemplo:

\begin{tikzpicture}[line cap=round,line join=round,>=triangle 45,x=1.0cm,y=1.0cm]
\draw [line width=1.pt] (1.8,3.26)-- (3.5,2.34);
\draw [line width=1.pt] (3.5,2.34)-- (3.98,0.54);
\draw [line width=1.pt] (3.98,0.54)-- (5.82,1.8);
\draw [line width=1.pt] (5.82,1.8)-- (3.5,2.34);
\draw (7.18,3.12) node[anchor=north west] {$V=\{v_1,v_2,v_3,v_4\}$};
\draw (7.18,2.28) node[anchor=north west] {$A=\{v_1v_2, v_2v_3,v_3v_4,v_2v_4\}$};

\draw (7.02,1.1) node[anchor=north west] {$d(v_1)=1,\quad d(v_2)=3,\quad  d(v_3)=2=d(v_4)$};
\draw [fill=green] (1.8,3.26) circle (1.8pt);
\draw[color=black] (1.48,3.43) node {$v_1$};
\draw [fill=green] (3.5,2.34) circle (1.8pt);
\draw[color=black] (3.56,2.77) node {$v_2$};
\draw [fill=green] (5.82,1.8) circle (1.8pt);
\draw[color=black] (6.1,1.95) node {$v_3$};
\draw [fill=green] (3.98,0.54) circle (1.8pt);
\draw[color=black] (3.88,0.33) node {$v_4$};
\end{tikzpicture}

\item Una gráfica se llama $d$-{\bf regular} si todos sus vértices tienen el mismo grado $d$. A las gráficas $1$-regulares también se le llama {\bf emparejamientos}.

\item Ejemplos de gráficas regulares son las {\bf gráficas completas} de $n$ vértices. Éstas son $(n-1)$-regulares y se denotan por $K_n$.


\begin{tikzpicture}[line cap=round,line join=round,>=triangle 45,x=1.5cm,y=1.5cm]
\draw [line width=1.pt] (0,1)-- (0,-1);
\draw [fill=green] (0,-1) circle (1.8pt);
\draw [fill=green] (0,1) circle (1.8pt);
\draw[color=black] (0,-1.5) node {$K_2$};
\end{tikzpicture}
\hspace{.5cm}
\begin{tikzpicture}[line cap=round,line join=round,>=triangle 45,x=1.5cm,y=1.5cm]
\draw [line width=1.pt] (1.,0.)-- (-0.5,0.8660254037844387);
\draw [line width=1.pt] (-0.5,0.8660254037844387)-- (-0.5,-0.8660254037844385);
\draw [line width=1.pt] (-0.5,-0.8660254037844385)-- (1.,0.);
\draw [fill=green] (-0.5,0.8660254037844387) circle (1.8pt);
\draw [fill=green] (-0.5,-0.8660254037844385) circle (1.8pt);
\draw [fill=green] (1.,0.) circle (1.8pt);
\draw[color=black] (0,-1.5) node {$K_3$};
\end{tikzpicture}
\hspace{.5cm}
\begin{tikzpicture}[line cap=round,line join=round,>=triangle 45,x=1.5cm,y=1.5cm]
\draw [line width=1.pt] (0.,-1.)-- (0.,1.);
\draw [line width=1.pt] (0.,1.)-- (-1.,0.);
\draw [line width=1.pt] (-1.,0.)-- (0.,-1.);
\draw [line width=1.pt] (0.,-1.)-- (1.,0.);
\draw [line width=1.pt] (1.,0.)-- (0.,1.);
\draw [line width=1.pt] (1.,0.)-- (-1.,0.);
\draw [fill=green] (0.,1.) circle (1.8pt);
\draw [fill=green] (-1.,0.) circle (1.8pt);
\draw [fill=green] (0.,-1.) circle (1.8pt);
\draw [fill=green] (0.,-1.) circle (1.8pt);
\draw [fill=green] (0.,1.) circle (1.8pt);
\draw [fill=green] (-1.,0.) circle (1.8pt);
\draw [fill=green] (1.,0.) circle (1.8pt);
\draw [fill=green] (1.,0.) circle (1.8pt);
\draw[color=black] (0,-1.5) node {$K_4$};
\end{tikzpicture}
\hspace{.5cm}
\begin{tikzpicture}[line cap=round,line join=round,>=triangle 45,x=1.5cm,y=1.5cm]
\draw [line width=1.pt] (-0.8090169943749475,-0.587785252292473)-- (0.30901699437494745,0.9510565162951535);
\draw [line width=1.pt] (0.30901699437494745,0.9510565162951535)-- (-0.8090169943749473,0.5877852522924732);
\draw [line width=1.pt] (-0.8090169943749473,0.5877852522924732)-- (-0.8090169943749475,-0.587785252292473);
\draw [line width=1.pt] (-0.8090169943749475,-0.587785252292473)-- (0.30901699437494723,-0.9510565162951536);
\draw [line width=1.pt] (0.30901699437494723,-0.9510565162951536)-- (0.30901699437494745,0.9510565162951535);
\draw [line width=1.pt] (0.30901699437494723,-0.9510565162951536)-- (-0.8090169943749473,0.5877852522924732);
\draw [line width=1.pt] (0.30901699437494745,0.9510565162951535)-- (1.,0.);
\draw [line width=1.pt] (1.,0.)-- (-0.8090169943749473,0.5877852522924732);
\draw [line width=1.pt] (1.,0.)-- (-0.8090169943749475,-0.587785252292473);
\draw [line width=1.pt] (1.,0.)-- (0.30901699437494723,-0.9510565162951536);
\draw [fill=green] (0.30901699437494745,0.9510565162951535) circle (1.8pt);
\draw [fill=green] (-0.8090169943749473,0.5877852522924732) circle (1.8pt);
\draw [fill=green] (-0.8090169943749475,-0.587785252292473) circle (1.8pt);
\draw [fill=green] (-0.8090169943749475,-0.587785252292473) circle (1.8pt);
\draw [fill=green] (0.30901699437494745,0.9510565162951535) circle (1.8pt);
\draw [fill=green] (-0.8090169943749473,0.5877852522924732) circle (1.8pt);
\draw [fill=green] (0.30901699437494723,-0.9510565162951536) circle (1.8pt);
\draw [fill=green] (0.30901699437494723,-0.9510565162951536) circle (1.8pt);
\draw [fill=green] (1.,0.) circle (1.8pt);
\draw [fill=green] (1.,0.) circle (1.8pt);
\draw[color=black] (0,-1.5) node {$K_5$};
\end{tikzpicture}
\hspace{.5cm}
\begin{tikzpicture}[line cap=round,line join=round,>=triangle 45,x=1.5cm,y=1.5cm]
\draw [line width=1.pt] (-1.,0.)-- (0.5,0.8660254037844386);
\draw [line width=1.pt] (0.5,0.8660254037844386)-- (-0.5,0.8660254037844387);
\draw [line width=1.pt] (-0.5,0.8660254037844387)-- (-1.,0.);
\draw [line width=1.pt] (-1.,0.)-- (-0.5,-0.8660254037844385);
\draw [line width=1.pt] (-0.5,-0.8660254037844385)-- (0.5,0.8660254037844386);
\draw [line width=1.pt] (-0.5,-0.8660254037844385)-- (-0.5,0.8660254037844387);
\draw [line width=1.pt] (0.5,0.8660254037844386)-- (1.,0.);
\draw [line width=1.pt] (1.,0.)-- (-0.5,0.8660254037844387);
\draw [line width=1.pt] (1.,0.)-- (-1.,0.);
\draw [line width=1.pt] (1.,0.)-- (-0.5,-0.8660254037844385);
\draw [line width=1.pt] (0.5,-0.866025403784439)-- (-1.,0.);
\draw [line width=1.pt] (0.5,-0.866025403784439)-- (-0.5,0.8660254037844387);
\draw [line width=1.pt] (0.5,-0.866025403784439)-- (0.5,0.8660254037844386);
\draw [line width=1.pt] (0.5,-0.866025403784439)-- (-0.5,-0.8660254037844385);
\draw [line width=1.pt] (0.5,-0.866025403784439)-- (1.,0.);
\draw [fill=green] (0.5,0.8660254037844386) circle (1.8pt);
\draw [fill=green] (-0.5,0.8660254037844387) circle (1.8pt);
\draw [fill=green] (-1.,0.) circle (1.8pt);
\draw [fill=green] (-1.,0.) circle (1.8pt);
\draw [fill=green] (0.5,0.8660254037844386) circle (1.8pt);
\draw [fill=green] (-0.5,0.8660254037844387) circle (1.8pt);
\draw [fill=green] (-0.5,-0.8660254037844385) circle (1.8pt);
\draw [fill=green] (-0.5,-0.8660254037844385) circle (1.8pt);
\draw [fill=green] (1.,0.) circle (1.8pt);
\draw [fill=green] (1.,0.) circle (1.8pt);
\draw [fill=green] (0.5,-0.866025403784439) circle (1.8pt);
\draw [fill=green] (0.5,-0.866025403784439) circle (1.8pt);
\draw[color=black] (0,-1.5) node {$K_6$};
\end{tikzpicture}

Ejemplo: Varias gráficas regulares de doce vértices. Por conveniencia arreglamos los doce vértices en el círculo (el punto medio del círculo donde a veces se cruzan las aristas NO es un vértice): 

\begin{tikzpicture}[line cap=round,line join=round,>=triangle 45,x=2.0cm,y=2.0cm]
\draw [line width=1.pt] (1.,0.)-- (0.9103703703703715,0.5348148148148145);
\draw [line width=1.pt] (0.9103703703703715,0.5348148148148145)-- (0.5653414267719156,0.9531632159498997);
\draw [line width=1.pt] (0.5653414267719156,0.9531632159498997)-- (0.057363396007201695,1.1429490871662584);
\draw [line width=1.pt] (0.057363396007201695,1.1429490871662584)-- (-0.4774514188076126,1.05331945753663);
\draw [line width=1.pt] (-0.4774514188076126,1.05331945753663)-- (-0.895799819942698,0.7082905139381741);
\draw [line width=1.pt] (-0.895799819942698,0.7082905139381741)-- (-1.0855856911590567,0.20031248317346018);
\draw [line width=1.pt] (-1.0855856911590567,0.20031248317346018)-- (-0.9959560615294283,-0.3345023316413541);
\draw [line width=1.pt] (-0.9959560615294283,-0.3345023316413541)-- (-0.6509271179309726,-0.7528507327764395);
\draw [line width=1.pt] (-0.6509271179309726,-0.7528507327764395)-- (-0.14294908716625856,-0.9426366039927982);
\draw [line width=1.pt] (-0.14294908716625856,-0.9426366039927982)-- (0.39186572764855543,-0.8530069743631701);
\draw [line width=1.pt] (0.39186572764855543,-0.8530069743631701)-- (0.8102141287836412,-0.5079780307647142);
\draw [line width=1.pt] (0.8102141287836412,-0.5079780307647142)-- (1.,0.);
\draw [line width=1.pt] (1.,0.)-- (0.9103703703703715,0.5348148148148145);
\draw [line width=1.pt] (1.,0.)-- (0.8102141287836412,-0.5079780307647142);
\draw [line width=1.pt] (0.8102141287836412,-0.5079780307647142)-- (0.39186572764855543,-0.8530069743631701);
\draw [line width=1.pt] (0.39186572764855543,-0.8530069743631701)-- (-0.14294908716625856,-0.9426366039927982);
\draw [line width=1.pt] (-0.14294908716625856,-0.9426366039927982)-- (-0.6509271179309726,-0.7528507327764395);
\draw [line width=1.pt] (-0.6509271179309726,-0.7528507327764395)-- (-0.9959560615294283,-0.3345023316413541);
\draw [line width=1.pt] (-0.9959560615294283,-0.3345023316413541)-- (-1.0855856911590567,0.20031248317346018);
\draw [line width=1.pt] (-1.0855856911590567,0.20031248317346018)-- (-0.895799819942698,0.7082905139381741);
\draw [line width=1.pt] (-0.4774514188076126,1.05331945753663)-- (0.057363396007201695,1.1429490871662584);
\draw [line width=1.pt] (0.057363396007201695,1.1429490871662584)-- (0.5653414267719156,0.9531632159498997);
\draw [line width=1.pt] (0.5653414267719156,0.9531632159498997)-- (0.9103703703703715,0.5348148148148145);
\draw [line width=1.pt] (0.057363396007201695,1.1429490871662584)-- (-0.14294908716625856,-0.9426366039927982);
\draw [line width=1.pt] (-0.4774514188076126,1.05331945753663)-- (0.39186572764855543,-0.8530069743631701);
\draw [line width=1.pt] (-0.895799819942698,0.7082905139381741)-- (0.8102141287836412,-0.5079780307647142);
\draw [line width=1.pt] (-1.0855856911590567,0.20031248317346018)-- (1.,0.);
\draw [line width=1.pt] (-0.9959560615294283,-0.3345023316413541)-- (0.9103703703703715,0.5348148148148145);
\draw [line width=1.pt] (-0.6509271179309726,-0.7528507327764395)-- (0.5653414267719156,0.9531632159498997);
\draw [fill=green] (1.,0.) circle (1.8pt);
\draw [fill=green] (0.9103703703703715,0.5348148148148145) circle (1.8pt);
\draw [fill=green] (0.5653414267719156,0.9531632159498997) circle (1.8pt);
\draw [fill=green] (0.057363396007201695,1.1429490871662584) circle (1.8pt);
\draw [fill=green] (-0.4774514188076126,1.05331945753663) circle (1.8pt);
\draw [fill=green] (-0.895799819942698,0.7082905139381741) circle (1.8pt);
\draw [fill=green] (-1.0855856911590567,0.20031248317346018) circle (1.8pt);
\draw [fill=green] (-0.9959560615294283,-0.3345023316413541) circle (1.8pt);
\draw [fill=green] (-0.6509271179309726,-0.7528507327764395) circle (1.8pt);
\draw [fill=green] (-0.14294908716625856,-0.9426366039927982) circle (1.8pt);
\draw [fill=green] (0.39186572764855543,-0.8530069743631701) circle (1.8pt);
\draw [fill=green] (0.8102141287836412,-0.5079780307647142) circle (1.8pt);
\end{tikzpicture}
\hspace{1cm}
\begin{tikzpicture}[line cap=round,line join=round,>=triangle 45,x=2.0cm,y=2.0cm]
\draw [line width=1.pt] (1.,0.)-- (0.9103703703703715,0.5348148148148145);
\draw [line width=1.pt] (1.,0.)-- (0.5653414267719156,0.9531632159498997);
\draw [line width=1.pt] (0.5653414267719156,0.9531632159498997)-- (-0.4774514188076126,1.05331945753663);
\draw [line width=1.pt] (-0.4774514188076126,1.05331945753663)-- (-1.0855856911590567,0.20031248317346018);
\draw [line width=1.pt] (-1.0855856911590567,0.20031248317346018)-- (-0.6509271179309726,-0.7528507327764395);
\draw [line width=1.pt] (-0.6509271179309726,-0.7528507327764395)-- (0.39186572764855543,-0.8530069743631701);
\draw [line width=1.pt] (0.39186572764855543,-0.8530069743631701)-- (1.,0.);
\draw [line width=1.pt] (0.8102141287836412,-0.5079780307647142)-- (0.9103703703703715,0.5348148148148145);
\draw [line width=1.pt] (0.9103703703703715,0.5348148148148145)-- (0.057363396007201695,1.1429490871662584);
\draw [line width=1.pt] (0.057363396007201695,1.1429490871662584)-- (-0.895799819942698,0.7082905139381741);
\draw [line width=1.pt] (-0.895799819942698,0.7082905139381741)-- (-0.9959560615294283,-0.3345023316413541);
\draw [line width=1.pt] (-0.9959560615294283,-0.3345023316413541)-- (-0.14294908716625856,-0.9426366039927982);
\draw [line width=1.pt] (-0.14294908716625856,-0.9426366039927982)-- (0.8102141287836412,-0.5079780307647142);
\draw [line width=1.pt] (-1.0855856911590567,0.20031248317346018)-- (1.,0.);
\draw [line width=1.pt] (0.057363396007201695,1.1429490871662584)-- (-0.14294908716625856,-0.9426366039927982);
\draw [line width=1.pt] (-0.895799819942698,0.7082905139381741)-- (0.8102141287836412,-0.5079780307647142);
\draw [line width=1.pt] (-0.9959560615294283,-0.3345023316413541)-- (0.9103703703703715,0.5348148148148145);
\draw [line width=1.pt] (0.5653414267719156,0.9531632159498997)-- (-0.6509271179309726,-0.7528507327764395);
\draw [line width=1.pt] (-0.4774514188076126,1.05331945753663)-- (0.39186572764855543,-0.8530069743631701);
\draw [fill=green] (1.,0.) circle (1.8pt);
\draw [fill=green] (0.9103703703703715,0.5348148148148145) circle (1.8pt);
\draw [fill=green] (0.5653414267719156,0.9531632159498997) circle (1.8pt);
\draw [fill=green] (0.057363396007201695,1.1429490871662584) circle (1.8pt);
\draw [fill=green] (-0.4774514188076126,1.05331945753663) circle (1.8pt);
\draw [fill=green] (-0.895799819942698,0.7082905139381741) circle (1.8pt);
\draw [fill=green] (-1.0855856911590567,0.20031248317346018) circle (1.8pt);
\draw [fill=green] (-0.9959560615294283,-0.3345023316413541) circle (1.8pt);
\draw [fill=green] (-0.6509271179309726,-0.7528507327764395) circle (1.8pt);
\draw [fill=green] (-0.14294908716625856,-0.9426366039927982) circle (1.8pt);
\draw [fill=green] (0.39186572764855543,-0.8530069743631701) circle (1.8pt);
\draw [fill=green] (0.8102141287836412,-0.5079780307647142) circle (1.8pt);
\end{tikzpicture}
\hspace{1cm}
\begin{tikzpicture}[line cap=round,line join=round,>=triangle 45,x=2.0cm,y=2.0cm]
\draw [line width=1.pt] (1.,0.)-- (-0.8660254037844386,-0.5);
\draw [line width=1.pt] (-0.8660254037844386,-0.5)-- (-0.5,0.8660254037844386);
\draw [line width=1.pt] (-0.5,0.8660254037844386)-- (0.5,-0.8660254037844388);
\draw [line width=1.pt] (0.5,-0.8660254037844388)-- (1.,0.);
\draw [line width=1.pt] (0.8660254037844384,-0.5)-- (-0.5,-0.8660254037844384);
\draw [line width=1.pt] (-0.5,-0.8660254037844384)-- (0.,-1.);
\draw [line width=1.pt] (0.,-1.)-- (-0.8660254037844383,0.5);
\draw [line width=1.pt] (-0.8660254037844383,0.5)-- (0.8660254037844384,-0.5);
\draw [line width=1.pt] (0.,1.)-- (0.5,0.8660254037844386);
\draw [line width=1.pt] (0.5,0.8660254037844386)-- (0.8660254037844387,0.5);
\draw [line width=1.pt] (0.8660254037844387,0.5)-- (-1.,0.);
\draw [line width=1.pt] (-1.,0.)-- (0.,1.);
\draw [line width=1.pt] (-0.5,-0.8660254037844384)-- (-0.8660254037844386,-0.5);
\draw [line width=1.pt] (-1.,0.)-- (0.,-1.);
\draw [line width=1.pt] (0.5,-0.8660254037844388)-- (0.8660254037844384,-0.5);
\draw [line width=1.pt] (1.,0.)-- (0.5,0.8660254037844386);
\draw [line width=1.pt] (0.8660254037844387,0.5)-- (-0.5,0.8660254037844386);
\draw [line width=1.pt] (0.,1.)-- (-0.8660254037844383,0.5);
\draw [fill=green] (0.8660254037844387,0.5) circle (1.8pt);
\draw [fill=green] (0.5,0.8660254037844386) circle (1.8pt);
\draw [fill=green] (0.,1.) circle (1.8pt);
\draw [fill=green] (-0.5,0.8660254037844386) circle (1.8pt);
\draw [fill=green] (-0.8660254037844383,0.5) circle (1.8pt);
\draw [fill=green] (-1.,0.) circle (1.8pt);
\draw [fill=green] (-0.8660254037844386,-0.5) circle (1.8pt);
\draw [fill=green] (-0.5,-0.8660254037844384) circle (1.8pt);
\draw [fill=green] (0.,-1.) circle (1.8pt);
\draw [fill=green] (0.5,-0.8660254037844388) circle (1.8pt);
\draw [fill=green] (0.8660254037844384,-0.5) circle (1.8pt);
\end{tikzpicture}

i). Dos gráficas $3$-regulares simétricas (notar que la segunda no es conexa) y una gráfica $3$-regular aleatoria (ver ejercicio ?).
  

\begin{tikzpicture}[line cap=round,line join=round,>=triangle 45,x=2.0cm,y=2.0cm]
\draw [line width=1.pt] (1.,0.)-- (0.9103703703703715,0.5348148148148145);
\draw [line width=1.pt] (0.9103703703703715,0.5348148148148145)-- (0.5653414267719156,0.9531632159498997);
\draw [line width=1.pt] (0.5653414267719156,0.9531632159498997)-- (0.057363396007201695,1.1429490871662584);
\draw [line width=1.pt] (0.057363396007201695,1.1429490871662584)-- (-0.4774514188076126,1.05331945753663);
\draw [line width=1.pt] (-0.4774514188076126,1.05331945753663)-- (-0.895799819942698,0.7082905139381741);
\draw [line width=1.pt] (-0.895799819942698,0.7082905139381741)-- (-1.0855856911590567,0.20031248317346018);
\draw [line width=1.pt] (-1.0855856911590567,0.20031248317346018)-- (-0.9959560615294283,-0.3345023316413541);
\draw [line width=1.pt] (-0.9959560615294283,-0.3345023316413541)-- (-0.6509271179309726,-0.7528507327764395);
\draw [line width=1.pt] (-0.6509271179309726,-0.7528507327764395)-- (-0.14294908716625856,-0.9426366039927982);
\draw [line width=1.pt] (-0.14294908716625856,-0.9426366039927982)-- (0.39186572764855543,-0.8530069743631701);
\draw [line width=1.pt] (0.39186572764855543,-0.8530069743631701)-- (0.8102141287836412,-0.5079780307647142);
\draw [line width=1.pt] (0.8102141287836412,-0.5079780307647142)-- (1.,0.);
\draw [line width=1.pt] (1.,0.)-- (0.9103703703703715,0.5348148148148145);
\draw [line width=1.pt] (1.,0.)-- (0.5653414267719156,0.9531632159498997);
\draw [line width=1.pt] (1.,0.)-- (0.8102141287836412,-0.5079780307647142);
\draw [line width=1.pt] (0.8102141287836412,-0.5079780307647142)-- (0.39186572764855543,-0.8530069743631701);
\draw [line width=1.pt] (0.39186572764855543,-0.8530069743631701)-- (-0.14294908716625856,-0.9426366039927982);
\draw [line width=1.pt] (-0.14294908716625856,-0.9426366039927982)-- (-0.6509271179309726,-0.7528507327764395);
\draw [line width=1.pt] (-0.6509271179309726,-0.7528507327764395)-- (-0.9959560615294283,-0.3345023316413541);
\draw [line width=1.pt] (-0.9959560615294283,-0.3345023316413541)-- (-1.0855856911590567,0.20031248317346018);
\draw [line width=1.pt] (-1.0855856911590567,0.20031248317346018)-- (-0.895799819942698,0.7082905139381741);
\draw [line width=1.pt] (-0.895799819942698,0.7082905139381741)-- (-0.4774514188076126,1.05331945753663);
\draw [line width=1.pt] (-0.4774514188076126,1.05331945753663)-- (0.057363396007201695,1.1429490871662584);
\draw [line width=1.pt] (0.057363396007201695,1.1429490871662584)-- (0.5653414267719156,0.9531632159498997);
\draw [line width=1.pt] (0.5653414267719156,0.9531632159498997)-- (0.9103703703703715,0.5348148148148145);
\draw [line width=1.pt] (0.057363396007201695,1.1429490871662584)-- (0.9103703703703715,0.5348148148148145);
\draw [line width=1.pt] (-0.4774514188076126,1.05331945753663)-- (0.5653414267719156,0.9531632159498997);
\draw [line width=1.pt] (-0.895799819942698,0.7082905139381741)-- (0.057363396007201695,1.1429490871662584);
\draw [line width=1.pt] (-1.0855856911590567,0.20031248317346018)-- (-0.4774514188076126,1.05331945753663);
\draw [line width=1.pt] (-0.895799819942698,0.7082905139381741)-- (-0.9959560615294283,-0.3345023316413541);
\draw [line width=1.pt] (-1.0855856911590567,0.20031248317346018)-- (-0.6509271179309726,-0.7528507327764395);
\draw [line width=1.pt] (-0.9959560615294283,-0.3345023316413541)-- (-0.14294908716625856,-0.9426366039927982);
\draw [line width=1.pt] (-0.6509271179309726,-0.7528507327764395)-- (0.39186572764855543,-0.8530069743631701);
\draw [line width=1.pt] (-0.14294908716625856,-0.9426366039927982)-- (0.8102141287836412,-0.5079780307647142);
\draw [line width=1.pt] (0.39186572764855543,-0.8530069743631701)-- (1.,0.);
\draw [line width=1.pt] (0.8102141287836412,-0.5079780307647142)-- (0.9103703703703715,0.5348148148148145);
\draw [fill=green] (1.,0.) circle (1.8pt);
\draw [fill=green] (0.9103703703703715,0.5348148148148145) circle (1.8pt);
\draw [fill=green] (0.5653414267719156,0.9531632159498997) circle (1.8pt);
\draw [fill=green] (0.057363396007201695,1.1429490871662584) circle (1.8pt);
\draw [fill=green] (-0.4774514188076126,1.05331945753663) circle (1.8pt);
\draw [fill=green] (-0.895799819942698,0.7082905139381741) circle (1.8pt);
\draw [fill=green] (-1.0855856911590567,0.20031248317346018) circle (1.8pt);
\draw [fill=green] (-0.9959560615294283,-0.3345023316413541) circle (1.8pt);
\draw [fill=green] (-0.6509271179309726,-0.7528507327764395) circle (1.8pt);
\draw [fill=green] (-0.14294908716625856,-0.9426366039927982) circle (1.8pt);
\draw [fill=green] (0.39186572764855543,-0.8530069743631701) circle (1.8pt);
\draw [fill=green] (0.8102141287836412,-0.5079780307647142) circle (1.8pt);
\end{tikzpicture}
\begin{tikzpicture}[line cap=round,line join=round,>=triangle 45,x=2.0cm,y=2.0cm]
\draw [line width=1.pt] (1.,0.)-- (0.9103703703703715,0.5348148148148145);
\draw [line width=1.pt] (0.9103703703703715,0.5348148148148145)-- (0.5653414267719156,0.9531632159498997);
\draw [line width=1.pt] (0.5653414267719156,0.9531632159498997)-- (0.057363396007201695,1.1429490871662584);
\draw [line width=1.pt] (0.057363396007201695,1.1429490871662584)-- (-0.4774514188076126,1.05331945753663);
\draw [line width=1.pt] (-0.4774514188076126,1.05331945753663)-- (-0.895799819942698,0.7082905139381741);
\draw [line width=1.pt] (-0.895799819942698,0.7082905139381741)-- (-1.0855856911590567,0.20031248317346018);
\draw [line width=1.pt] (-1.0855856911590567,0.20031248317346018)-- (-0.9959560615294283,-0.3345023316413541);
\draw [line width=1.pt] (-0.9959560615294283,-0.3345023316413541)-- (-0.6509271179309726,-0.7528507327764395);
\draw [line width=1.pt] (-0.6509271179309726,-0.7528507327764395)-- (-0.14294908716625856,-0.9426366039927982);
\draw [line width=1.pt] (-0.14294908716625856,-0.9426366039927982)-- (0.39186572764855543,-0.8530069743631701);
\draw [line width=1.pt] (0.39186572764855543,-0.8530069743631701)-- (0.8102141287836412,-0.5079780307647142);
\draw [line width=1.pt] (0.8102141287836412,-0.5079780307647142)-- (1.,0.);
\draw [line width=1.pt] (1.,0.)-- (0.9103703703703715,0.5348148148148145);
\draw [line width=1.pt] (1.,0.)-- (0.8102141287836412,-0.5079780307647142);
\draw [line width=1.pt] (0.8102141287836412,-0.5079780307647142)-- (0.39186572764855543,-0.8530069743631701);
\draw [line width=1.pt] (0.39186572764855543,-0.8530069743631701)-- (-0.14294908716625856,-0.9426366039927982);
\draw [line width=1.pt] (-0.14294908716625856,-0.9426366039927982)-- (-0.6509271179309726,-0.7528507327764395);
\draw [line width=1.pt] (-0.6509271179309726,-0.7528507327764395)-- (-0.9959560615294283,-0.3345023316413541);
\draw [line width=1.pt] (-0.9959560615294283,-0.3345023316413541)-- (-1.0855856911590567,0.20031248317346018);
\draw [line width=1.pt] (-1.0855856911590567,0.20031248317346018)-- (-0.895799819942698,0.7082905139381741);
\draw [line width=1.pt] (-0.4774514188076126,1.05331945753663)-- (0.057363396007201695,1.1429490871662584);
\draw [line width=1.pt] (0.057363396007201695,1.1429490871662584)-- (0.5653414267719156,0.9531632159498997);
\draw [line width=1.pt] (0.5653414267719156,0.9531632159498997)-- (0.9103703703703715,0.5348148148148145);
\draw [line width=1.pt] (0.057363396007201695,1.1429490871662584)-- (-0.9959560615294283,-0.3345023316413541);
\draw [line width=1.pt] (0.057363396007201695,1.1429490871662584)-- (0.8102141287836412,-0.5079780307647142);
\draw [line width=1.pt] (-0.9959560615294283,-0.3345023316413541)-- (0.8102141287836412,-0.5079780307647142);
\draw [line width=1.pt] (-1.0855856911590567,0.20031248317346018)-- (0.5653414267719156,0.9531632159498997);
\draw [line width=1.pt] (0.5653414267719156,0.9531632159498997)-- (0.39186572764855543,-0.8530069743631701);
\draw [line width=1.pt] (0.39186572764855543,-0.8530069743631701)-- (-1.0855856911590567,0.20031248317346018);
\draw [line width=1.pt] (-0.895799819942698,0.7082905139381741)-- (0.9103703703703715,0.5348148148148145);
\draw [line width=1.pt] (0.9103703703703715,0.5348148148148145)-- (-0.14294908716625856,-0.9426366039927982);
\draw [line width=1.pt] (-0.14294908716625856,-0.9426366039927982)-- (-0.895799819942698,0.7082905139381741);
\draw [line width=1.pt] (-0.4774514188076126,1.05331945753663)-- (1.,0.);
\draw [line width=1.pt] (1.,0.)-- (-0.6509271179309726,-0.7528507327764395);
\draw [line width=1.pt] (-0.6509271179309726,-0.7528507327764395)-- (-0.4774514188076126,1.05331945753663);
\draw [fill=green] (1.,0.) circle (1.8pt);
\draw [fill=green] (0.9103703703703715,0.5348148148148145) circle (1.8pt);
\draw [fill=green] (0.5653414267719156,0.9531632159498997) circle (1.8pt);
\draw [fill=green] (0.057363396007201695,1.1429490871662584) circle (1.8pt);
\draw [fill=green] (-0.4774514188076126,1.05331945753663) circle (1.8pt);
\draw [fill=green] (-0.895799819942698,0.7082905139381741) circle (1.8pt);
\draw [fill=green] (-1.0855856911590567,0.20031248317346018) circle (1.8pt);
\draw [fill=green] (-0.9959560615294283,-0.3345023316413541) circle (1.8pt);
\draw [fill=green] (-0.6509271179309726,-0.7528507327764395) circle (1.8pt);
\draw [fill=green] (-0.14294908716625856,-0.9426366039927982) circle (1.8pt);
\draw [fill=green] (0.39186572764855543,-0.8530069743631701) circle (1.8pt);
\draw [fill=green] (0.8102141287836412,-0.5079780307647142) circle (1.8pt);
\end{tikzpicture}

ii). Dos gráficas $4$-regulares simétricas.



\item Una gráfica se llama {\bf bipartita} si se puede elegir una partición del conjunto de vértices en dos conjuntos (disjuntos) $V=V_1\cup V_2$, donde no hay aristas entre vértices de $V_1$ ni aristas entre vértices de $V_2$. Las únicas posibles aristas unen un elemento de cada conjunto de vértices.

\begin{tikzpicture}[line cap=round,line join=round,>=triangle 45,x=1.0cm,y=1.0cm]
\draw [line width=1.pt] (0.,0.)-- (3.22,1.14);
\draw [line width=1.pt] (0.02,1.58)-- (3.16,2.52);
\draw [line width=1.pt] (0.02,1.58)-- (3.28,-0.34);
\draw [line width=1.pt] (0.,0.)-- (3.16,2.52);
\draw [fill=green] (0.,0.) circle (1.8pt);
\draw [fill=green] (0.02,1.58) circle (1.8pt);
\draw [fill=green] (3.16,2.52) circle (1.8pt);
\draw [fill=green] (3.22,1.14) circle (1.8pt);
\draw [fill=green] (3.28,-0.34) circle (1.8pt);
\end{tikzpicture}
\hspace{1cm}
\begin{tikzpicture}[line cap=round,line join=round,>=triangle 45,x=4.0cm,y=4.0cm]
\draw [line width=1.pt] (0.,0.4)-- (1.,0.2);
\draw [line width=1.pt] (0.,0.4)-- (1.,0.);
\draw [line width=1.pt] (0.,0.2)-- (1.,-0.2);
\draw [line width=1.pt] (0.,-0.4)-- (1.,-0.2);
\draw [line width=1.pt] (0.,-0.2)-- (1.,0.);
\draw [line width=1.pt] (0.,0.)-- (1.,0.2);
\draw [line width=1.pt] (0.,0.4)-- (1.,-0.2);
\draw [line width=1.pt] (0.,-0.4)-- (1.,0.);
\draw [line width=1.pt] (0.,-0.2)-- (1.,0.2);
\draw [line width=1.pt] (0.,0.2)-- (1.,0.2);
\draw [line width=1.pt] (0.,0.)-- (1.,0.);
\draw [line width=1.pt] (0.,0.2)-- (1.,0.);
\draw [line width=1.pt] (0.,-0.2)-- (1.,-0.2);
\draw [line width=1.pt] (0.,-0.4)-- (1.,0.2);
\draw [line width=1.pt] (0.,0.)-- (1.,-0.2);
\draw [fill=green] (0.,0.) circle (1.8pt);
\draw [fill=green] (1.,0.) circle (1.8pt);
\draw [fill=green] (0.,0.2) circle (1.8pt);
\draw [fill=green] (0.,0.4) circle (1.8pt);
\draw [fill=green] (0.,-0.2) circle (1.8pt);
\draw [fill=green] (0.,-0.4) circle (1.8pt);
\draw [fill=green] (1.,-0.2) circle (1.8pt);
\draw [fill=green] (1.,0.2) circle (1.8pt);
\end{tikzpicture}

Ejemplo: Una gráfica bipartita, una gráfica bipartita completa $K_{5,3}$ y una gráfica bipartita aleatoria. 

\item Un {\bf camino} de longitud $n$ en una gráfica $G=(V,A)$ es una sucesión de $n$ aristas adyacentes. Otra forma de caracterizar a un camino es como una sucesión de $(n+1)$ vértices $(v_0,v_1,v_2,\dots ,v_n)$, de tal forma que $v_i$ es adyacente a $v_{i+1}$, $i=0,1,2,\dots, n$.

\item Un {\bf ciclo} es un camino cerrado (es decir, $v_0=v_n$).

\item Una {\bf trayectoria} es un camino que no repite vértices. Un {\bf paseo} es un camino que no repite aristas. Un paseo puede pasar varias veces por el mismo vértice.

Figura: Gráfica, caminos, trayectorias, ciclos.

\item Una gráfica es {\bf conexa} si para todo par de vértices $v_i,v_j\in V$ existe un camino de $v_i$ a $v_j$.

\item A una gráfica conexa sin ciclos se le llama {\bf árbol}. A los vértices de grado $1$ de un árbol se les llama {\bf hojas}.

\item Un {\bf paseo Euleriano} es un paseo que pasa por todas las aristas de una gráfica. Un {\bf ciclo Euleriano} es un paseo Euleriano que además es un ciclo. Una {\bf gráfica Euleriana} es toda aquella que contiene un ciclo Euleriano. 
\end{enumerate}
\end{definicion}

\newpage

\section{Ejercicios}

\begin{ejercicio}
Muestra que la suma de los grados de los vértices es igual al doble del número de aristas.
\end{ejercicio}

\begin{ejercicio}
Muestra que en una gráfica con un número impar de vértices siempre hay un vértice con grado par.
\end{ejercicio}

\begin{ejercicio}[Lema de los saludos]
Muestra que en una fiesta de $n$ invitados en todo momento hay dos invitados que han saludado a la misma cantidad de personas.
\end{ejercicio}

\begin{ejercicio}
Una gráfica regular tiene $26$ aristas. ¿Cuántos vértices puede tener? Para cada posible $n=|V|$ incluye o describe un ejemplo de una gráfica que cumpla las condiciones.
\end{ejercicio}
\vspace{3cm}

\begin{ejercicio}
¿Puede haber una gráfica $5$-regular de $23$ vértices? Dibuja una o muestra que no se puede.
\end{ejercicio}
\vspace{3cm}

\begin{ejercicio}
Muestra que existe una gráfica $4$-regular de $n$ vértices para todo $n=5,6,7,\dots$ 
\end{ejercicio}


\begin{ejercicio}
Muestra que una gráfica $2$-regular es necesariamente la unión de ciclos disjuntos.
\end{ejercicio}

\begin{ejercicio}
¿Cuántas gráficas $1$-regulares (emparejamientos) de $12$ vértices hay? 
\end{ejercicio}
\vspace{3cm}

%La respuesta en general es cero si $n$ es impar y  $(2k-1)(2k-3)\cdots 3\cdot 1$ si $n=2k$.

% (3-5) ejercicios más (Toñus o Malú)

\begin{ejercicio}
¿Cuántas gráficas $2$-regulares de $7$ vértices hay? 
\end{ejercicio}
\vspace{5cm}

%Sug: Hay una correspondencia (suprayectiva pero no inductuva) entre permutaciones sin ciclos de tamaño uno o dos y gráficas $2$-regulares simples.

Pregunta: ¿Cómo se ve una gráfica $d$-regular?

\newpage 


Figuras: Emparejamientos aleatorios: Todos los emparejamientos son gráficas isomorfas

\begin{tikzpicture}[line cap=round,line join=round,>=triangle 45,x=0.6784*2.0cm,y=0.6784*2.0cm]
\draw [line width=1.pt] (1.,0.)-- (-0.8660254037844383,0.5);
\draw [line width=1.pt] (-0.5,0.8660254037844386)-- (0.5,0.8660254037844386);
\draw [line width=1.pt] (0.,1.)-- (0.8660254037844387,0.5);
\draw [line width=1.pt] (-1.,0.)-- (0.8660254037844384,-0.5);
\draw [line width=1.pt] (-0.8660254037844386,-0.5)-- (0.5,-0.8660254037844388);
\draw [line width=1.pt] (-0.5,-0.8660254037844384)-- (0.,-1.);
\draw [fill=green] (1.,0.) circle (1.8pt);
\draw[color=black] (1.11037037037037,0.057777777777778094) node {$1$};
\draw [fill=green] (0.8660254037844387,0.5) circle (1.8pt);
\draw[color=black] (0.9681481481481479,0.5970370370370371) node {$2$};
\draw [fill=green] (0.5,0.8660254037844386) circle (1.8pt);
\draw[color=black] (0.5770370370370369,1.017777777777778) node {$3$};
\draw [fill=green] (0.,1.) circle (1.8pt);
\draw[color=black] (0.014074074074074072,1.1481481481481481) node {$4$};
\draw [fill=green] (-0.5,0.8660254037844386) circle (1.8pt);
\draw[color=black] (-0.5785185185185183,1.005925925925926) node {$5$};
\draw [fill=green] (-0.8660254037844383,0.5) circle (1.8pt);
\draw[color=black] (-0.987407407407407,0.5851851851851853) node {$6$};
\draw [fill=green] (-1.,0.) circle (1.8pt);
\draw[color=black] (-1.1,0.045925925925926245) node {$7$};
\draw [fill=green] (-0.8660254037844386,-0.5) circle (1.8pt);
\draw[color=black] (-0.987407407407407,-0.5051851851851847) node {$8$};
\draw [fill=green] (-0.5,-0.8660254037844384) circle (1.8pt);
\draw[color=black] (-0.6140740740740738,-0.8251851851851846) node {$9$};
\draw [fill=green] (0.,-1.) circle (1.8pt);
\draw[color=black] (0.0022222222222222227,-1.13) node {$10$};
\draw [fill=green] (0.5,-0.8660254037844388) circle (1.8pt);
\draw[color=black] (0.61,-0.8962962962962957) node {$11$};
\draw [fill=green] (0.8660254037844384,-0.5) circle (1.8pt);
\draw[color=black] (0.98,-0.49) node {$12$};
\end{tikzpicture}
\hspace{1cm}
\begin{tikzpicture}[line cap=round,line join=round,>=triangle 45,x=0.6784*2.0cm,y=0.6784*2.0cm]
\draw [line width=1.pt] (-0.5,0.8660254037844386)-- (0.5,0.8660254037844386);
\draw [line width=1.pt] (-0.5,-0.8660254037844384)-- (0.8660254037844384,-0.5);
\draw [line width=1.pt] (-1.,0.)-- (0.8660254037844387,0.5);
\draw [line width=1.pt] (1.,0.)-- (0.,1.);
\draw [line width=1.pt] (-0.8660254037844383,0.5)-- (0.,-1.);
\draw [line width=1.pt] (-0.8660254037844386,-0.5)-- (0.5,-0.8660254037844388);
\draw [fill=green] (1.,0.) circle (1.8pt);
\draw[color=black] (1.11037037037037,0.057777777777778094) node {$1$};
\draw [fill=green] (0.8660254037844387,0.5) circle (1.8pt);
\draw[color=black] (0.9681481481481479,0.5970370370370371) node {$2$};
\draw [fill=green] (0.5,0.8660254037844386) circle (1.8pt);
\draw[color=black] (0.5770370370370369,1.017777777777778) node {$3$};
\draw [fill=green] (0.,1.) circle (1.8pt);
\draw[color=black] (0.014074074074074072,1.1481481481481481) node {$4$};
\draw [fill=green] (-0.5,0.8660254037844386) circle (1.8pt);
\draw[color=black] (-0.5785185185185183,1.005925925925926) node {$5$};
\draw [fill=green] (-0.8660254037844383,0.5) circle (1.8pt);
\draw[color=black] (-0.987407407407407,0.5851851851851853) node {$6$};
\draw [fill=green] (-1.,0.) circle (1.8pt);
\draw[color=black] (-1.1,0.045925925925926245) node {$7$};
\draw [fill=green] (-0.8660254037844386,-0.5) circle (1.8pt);
\draw[color=black] (-0.987407407407407,-0.5051851851851847) node {$8$};
\draw [fill=green] (-0.5,-0.8660254037844384) circle (1.8pt);
\draw[color=black] (-0.6140740740740738,-0.8251851851851846) node {$9$};
\draw [fill=green] (0.,-1.) circle (1.8pt);
\draw[color=black] (0.0022222222222222227,-1.13) node {$10$};
\draw [fill=green] (0.5,-0.8660254037844388) circle (1.8pt);
\draw[color=black] (0.61,-0.8962962962962957) node {$11$};
\draw [fill=green] (0.8660254037844384,-0.5) circle (1.8pt);
\draw[color=black] (0.98,-0.49) node {$12$};
\end{tikzpicture}
\hspace{1cm}
\begin{tikzpicture}[line cap=round,line join=round,>=triangle 45,x=0.6784*2.0cm,y=0.6784*2.0cm]
\draw [line width=1.pt] (1.,0.)-- (0.5,-0.8660254037844388);
\draw [line width=1.pt] (0.,1.)-- (0.5,0.8660254037844386);
\draw [line width=1.pt] (0.8660254037844384,-0.5)-- (-0.8660254037844383,0.5);
\draw [line width=1.pt] (0.,-1.)-- (-0.5,-0.8660254037844384);
\draw [line width=1.pt] (-0.8660254037844386,-0.5)-- (-0.5,0.8660254037844386);
\draw [line width=1.pt] (-1.,0.)-- (0.8660254037844387,0.5);
\draw [fill=green] (1.,0.) circle (1.8pt);
\draw[color=black] (1.11037037037037,0.057777777777778094) node {$1$};
\draw [fill=green] (0.8660254037844387,0.5) circle (1.8pt);
\draw[color=black] (0.9681481481481479,0.5970370370370371) node {$2$};
\draw [fill=green] (0.5,0.8660254037844386) circle (1.8pt);
\draw[color=black] (0.5770370370370369,1.017777777777778) node {$3$};
\draw [fill=green] (0.,1.) circle (1.8pt);
\draw[color=black] (0.014074074074074072,1.1481481481481481) node {$4$};
\draw [fill=green] (-0.5,0.8660254037844386) circle (1.8pt);
\draw[color=black] (-0.5785185185185183,1.005925925925926) node {$5$};
\draw [fill=green] (-0.8660254037844383,0.5) circle (1.8pt);
\draw[color=black] (-0.987407407407407,0.5851851851851853) node {$6$};
\draw [fill=green] (-1.,0.) circle (1.8pt);
\draw[color=black] (-1.1,0.045925925925926245) node {$7$};
\draw [fill=green] (-0.8660254037844386,-0.5) circle (1.8pt);
\draw[color=black] (-0.987407407407407,-0.5051851851851847) node {$8$};
\draw [fill=green] (-0.5,-0.8660254037844384) circle (1.8pt);
\draw[color=black] (-0.6140740740740738,-0.8251851851851846) node {$9$};
\draw [fill=green] (0.,-1.) circle (1.8pt);
\draw[color=black] (0.0022222222222222227,-1.13) node {$10$};
\draw [fill=green] (0.5,-0.8660254037844388) circle (1.8pt);
\draw[color=black] (0.61,-0.8962962962962957) node {$11$};
\draw [fill=green] (0.8660254037844384,-0.5) circle (1.8pt);
\draw[color=black] (0.98,-0.49) node {$12$};
\end{tikzpicture}
\hspace{1cm}
\begin{tikzpicture}[line cap=round,line join=round,>=triangle 45,x=0.6784*2.0cm,y=0.6784*2.0cm]
\draw [line width=1.pt] (1.,0.)-- (-0.8660254037844386,-0.5);
\draw [line width=1.pt] (0.8660254037844387,0.5)-- (0.5,0.8660254037844386);
\draw [line width=1.pt] (0.,1.)-- (-1.,0.);
\draw [line width=1.pt] (-0.5,0.8660254037844386)-- (0.5,-0.8660254037844388);
\draw [line width=1.pt] (-0.8660254037844383,0.5)-- (0.,-1.);
\draw [line width=1.pt] (0.8660254037844384,-0.5)-- (-0.5,-0.8660254037844384);
\draw [fill=green] (1.,0.) circle (1.8pt);
\draw[color=black] (1.11037037037037,0.057777777777778094) node {$1$};
\draw [fill=green] (0.8660254037844387,0.5) circle (1.8pt);
\draw[color=black] (0.9681481481481479,0.5970370370370371) node {$2$};
\draw [fill=green] (0.5,0.8660254037844386) circle (1.8pt);
\draw[color=black] (0.5770370370370369,1.017777777777778) node {$3$};
\draw [fill=green] (0.,1.) circle (1.8pt);
\draw[color=black] (0.014074074074074072,1.1481481481481481) node {$4$};
\draw [fill=green] (-0.5,0.8660254037844386) circle (1.8pt);
\draw[color=black] (-0.5785185185185183,1.005925925925926) node {$5$};
\draw [fill=green] (-0.8660254037844383,0.5) circle (1.8pt);
\draw[color=black] (-0.987407407407407,0.5851851851851853) node {$6$};
\draw [fill=green] (-1.,0.) circle (1.8pt);
\draw[color=black] (-1.1,0.045925925925926245) node {$7$};
\draw [fill=green] (-0.8660254037844386,-0.5) circle (1.8pt);
\draw[color=black] (-0.987407407407407,-0.5051851851851847) node {$8$};
\draw [fill=green] (-0.5,-0.8660254037844384) circle (1.8pt);
\draw[color=black] (-0.6140740740740738,-0.8251851851851846) node {$9$};
\draw [fill=green] (0.,-1.) circle (1.8pt);
\draw[color=black] (0.0022222222222222227,-1.13) node {$10$};
\draw [fill=green] (0.5,-0.8660254037844388) circle (1.8pt);
\draw[color=black] (0.61,-0.8962962962962957) node {$11$};
\draw [fill=green] (0.8660254037844384,-0.5) circle (1.8pt);
\draw[color=black] (0.98,-0.49) node {$12$};
\end{tikzpicture}
\hspace{1cm}
\begin{tikzpicture}[line cap=round,line join=round,>=triangle 45,x=0.6784*2.0cm,y=0.6784*2.0cm]
\draw [line width=1.pt] (1.,0.)-- (-1.,0.);
\draw [line width=1.pt] (-0.8660254037844383,0.5)-- (-0.5,-0.8660254037844384);
\draw [line width=1.pt] (0.8660254037844387,0.5)-- (0.,1.);
\draw [line width=1.pt] (0.5,0.8660254037844386)-- (-0.8660254037844386,-0.5);
\draw [line width=1.pt] (-0.5,0.8660254037844386)-- (0.5,-0.8660254037844388);
\draw [line width=1.pt] (0.,-1.)-- (0.8660254037844384,-0.5);
\draw [fill=green] (1.,0.) circle (1.8pt);
\draw[color=black] (1.11037037037037,0.057777777777778094) node {$1$};
\draw [fill=green] (0.8660254037844387,0.5) circle (1.8pt);
\draw[color=black] (0.9681481481481479,0.5970370370370371) node {$2$};
\draw [fill=green] (0.5,0.8660254037844386) circle (1.8pt);
\draw[color=black] (0.5770370370370369,1.017777777777778) node {$3$};
\draw [fill=green] (0.,1.) circle (1.8pt);
\draw[color=black] (0.014074074074074072,1.1481481481481481) node {$4$};
\draw [fill=green] (-0.5,0.8660254037844386) circle (1.8pt);
\draw[color=black] (-0.5785185185185183,1.005925925925926) node {$5$};
\draw [fill=green] (-0.8660254037844383,0.5) circle (1.8pt);
\draw[color=black] (-0.987407407407407,0.5851851851851853) node {$6$};
\draw [fill=green] (-1.,0.) circle (1.8pt);
\draw[color=black] (-1.1,0.045925925925926245) node {$7$};
\draw [fill=green] (-0.8660254037844386,-0.5) circle (1.8pt);
\draw[color=black] (-0.987407407407407,-0.5051851851851847) node {$8$};
\draw [fill=green] (-0.5,-0.8660254037844384) circle (1.8pt);
\draw[color=black] (-0.6140740740740738,-0.8251851851851846) node {$9$};
\draw [fill=green] (0.,-1.) circle (1.8pt);
\draw[color=black] (0.0022222222222222227,-1.13) node {$10$};
\draw [fill=green] (0.5,-0.8660254037844388) circle (1.8pt);
\draw[color=black] (0.61,-0.8962962962962957) node {$11$};
\draw [fill=green] (0.8660254037844384,-0.5) circle (1.8pt);
\draw[color=black] (0.98,-0.49) node {$12$};
\end{tikzpicture}
\hspace{1cm}
\begin{tikzpicture}[line cap=round,line join=round,>=triangle 45,x=0.6784*2.0cm,y=0.6784*2.0cm]
\draw [line width=1.pt] (1.,0.)-- (0.,1.);
\draw [line width=1.pt] (-0.5,0.8660254037844386)-- (0.5,0.8660254037844386);
\draw [line width=1.pt] (-0.5,-0.8660254037844384)-- (-0.8660254037844386,-0.5);
\draw [line width=1.pt] (0.8660254037844387,0.5)-- (0.,-1.);
\draw [line width=1.pt] (-0.8660254037844383,0.5)-- (0.8660254037844384,-0.5);
\draw [line width=1.pt] (-1.,0.)-- (0.5,-0.8660254037844388);
\draw [fill=green] (1.,0.) circle (1.8pt);
\draw[color=black] (1.11037037037037,0.057777777777778094) node {$1$};
\draw [fill=green] (0.8660254037844387,0.5) circle (1.8pt);
\draw[color=black] (0.9681481481481479,0.5970370370370371) node {$2$};
\draw [fill=green] (0.5,0.8660254037844386) circle (1.8pt);
\draw[color=black] (0.5770370370370369,1.017777777777778) node {$3$};
\draw [fill=green] (0.,1.) circle (1.8pt);
\draw[color=black] (0.014074074074074072,1.1481481481481481) node {$4$};
\draw [fill=green] (-0.5,0.8660254037844386) circle (1.8pt);
\draw[color=black] (-0.5785185185185183,1.005925925925926) node {$5$};
\draw [fill=green] (-0.8660254037844383,0.5) circle (1.8pt);
\draw[color=black] (-0.987407407407407,0.5851851851851853) node {$6$};
\draw [fill=green] (-1.,0.) circle (1.8pt);
\draw[color=black] (-1.1,0.045925925925926245) node {$7$};
\draw [fill=green] (-0.8660254037844386,-0.5) circle (1.8pt);
\draw[color=black] (-0.987407407407407,-0.5051851851851847) node {$8$};
\draw [fill=green] (-0.5,-0.8660254037844384) circle (1.8pt);
\draw[color=black] (-0.6140740740740738,-0.8251851851851846) node {$9$};
\draw [fill=green] (0.,-1.) circle (1.8pt);
\draw[color=black] (0.0022222222222222227,-1.13) node {$10$};
\draw [fill=green] (0.5,-0.8660254037844388) circle (1.8pt);
\draw[color=black] (0.61,-0.8962962962962957) node {$11$};
\draw [fill=green] (0.8660254037844384,-0.5) circle (1.8pt);
\draw[color=black] (0.98,-0.49) node {$12$};
\end{tikzpicture}

Seis emparejamientos aleatorios.

\begin{tikzpicture}[line cap=round,line join=round,>=triangle 45,x=2.0cm,y=2.0cm]
%G3
\draw [line width=1.pt] (1.,0.)-- (0.5,-0.8660254037844388);
\draw [line width=1.pt] (0.,1.)-- (0.5,0.8660254037844386);
\draw [line width=1.pt] (0.8660254037844384,-0.5)-- (-0.8660254037844383,0.5);
\draw [line width=1.pt] (0.,-1.)-- (-0.5,-0.8660254037844384);
\draw [line width=1.pt] (-0.8660254037844386,-0.5)-- (-0.5,0.8660254037844386);
\draw [line width=1.pt] (-1.,0.)-- (0.8660254037844387,0.5);
%G5
\draw [line width=1.pt] (1.,0.)-- (-1.,0.);
\draw [line width=1.pt] (-0.8660254037844383,0.5)-- (-0.5,-0.8660254037844384);
\draw [line width=1.pt] (0.8660254037844387,0.5)-- (0.,1.);
\draw [line width=1.pt] (0.5,0.8660254037844386)-- (-0.8660254037844386,-0.5);
\draw [line width=1.pt] (-0.5,0.8660254037844386)-- (0.5,-0.8660254037844388);
\draw [line width=1.pt] (0.,-1.)-- (0.8660254037844384,-0.5);
\draw [fill=green] (1.,0.) circle (1.8pt);
\draw[color=black] (1.11037037037037,0.057777777777778094) node {$1$};
\draw [fill=green] (0.8660254037844387,0.5) circle (1.8pt);
\draw[color=black] (0.9681481481481479,0.5970370370370371) node {$2$};
\draw [fill=green] (0.5,0.8660254037844386) circle (1.8pt);
\draw[color=black] (0.5770370370370369,1.017777777777778) node {$3$};
\draw [fill=green] (0.,1.) circle (1.8pt);
\draw[color=black] (0.014074074074074072,1.1481481481481481) node {$4$};
\draw [fill=green] (-0.5,0.8660254037844386) circle (1.8pt);
\draw[color=black] (-0.5785185185185183,1.005925925925926) node {$5$};
\draw [fill=green] (-0.8660254037844383,0.5) circle (1.8pt);
\draw[color=black] (-0.987407407407407,0.5851851851851853) node {$6$};
\draw [fill=green] (-1.,0.) circle (1.8pt);
\draw[color=black] (-1.1,0.045925925925926245) node {$7$};
\draw [fill=green] (-0.8660254037844386,-0.5) circle (1.8pt);
\draw[color=black] (-0.987407407407407,-0.5051851851851847) node {$8$};
\draw [fill=green] (-0.5,-0.8660254037844384) circle (1.8pt);
\draw[color=black] (-0.6140740740740738,-0.8251851851851846) node {$9$};
\draw [fill=green] (0.,-1.) circle (1.8pt);
\draw[color=black] (0.0022222222222222227,-1.13) node {$10$};
\draw [fill=green] (0.5,-0.8660254037844388) circle (1.8pt);
\draw[color=black] (0.61,-0.8962962962962957) node {$11$};
\draw [fill=green] (0.8660254037844384,-0.5) circle (1.8pt);
\draw[color=black] (0.98,-0.49) node {$12$};
\end{tikzpicture}
%G3+G5
\hspace{1cm}
\begin{tikzpicture}[line cap=round,line join=round,>=triangle 45,x=2.0cm,y=2.0cm]
%G4
\draw [line width=1.pt] (1.,0.)-- (-0.8660254037844386,-0.5);
\draw [line width=1.pt] (0.8660254037844387,0.5)-- (0.5,0.8660254037844386);
\draw [line width=1.pt] (0.,1.)-- (-1.,0.);
\draw [line width=1.pt] (-0.5,0.8660254037844386)-- (0.5,-0.8660254037844388);
\draw [line width=1.pt] (-0.8660254037844383,0.5)-- (0.,-1.);
\draw [line width=1.pt] (0.8660254037844384,-0.5)-- (-0.5,-0.8660254037844384);
%G6
\draw [line width=1.pt] (1.,0.)-- (0.,1.);
\draw [line width=1.pt] (-0.5,0.8660254037844386)-- (0.5,0.8660254037844386);
\draw [line width=1.pt] (-0.5,-0.8660254037844384)-- (-0.8660254037844386,-0.5);
\draw [line width=1.pt] (0.8660254037844387,0.5)-- (0.,-1.);
\draw [line width=1.pt] (-0.8660254037844383,0.5)-- (0.8660254037844384,-0.5);
\draw [line width=1.pt] (-1.,0.)-- (0.5,-0.8660254037844388);
\draw [fill=green] (1.,0.) circle (1.8pt);
\draw[color=black] (1.11037037037037,0.057777777777778094) node {$1$};
\draw [fill=green] (0.8660254037844387,0.5) circle (1.8pt);
\draw[color=black] (0.9681481481481479,0.5970370370370371) node {$2$};
\draw [fill=green] (0.5,0.8660254037844386) circle (1.8pt);
\draw[color=black] (0.5770370370370369,1.017777777777778) node {$3$};
\draw [fill=green] (0.,1.) circle (1.8pt);
\draw[color=black] (0.014074074074074072,1.1481481481481481) node {$4$};
\draw [fill=green] (-0.5,0.8660254037844386) circle (1.8pt);
\draw[color=black] (-0.5785185185185183,1.005925925925926) node {$5$};
\draw [fill=green] (-0.8660254037844383,0.5) circle (1.8pt);
\draw[color=black] (-0.987407407407407,0.5851851851851853) node {$6$};
\draw [fill=green] (-1.,0.) circle (1.8pt);
\draw[color=black] (-1.1,0.045925925925926245) node {$7$};
\draw [fill=green] (-0.8660254037844386,-0.5) circle (1.8pt);
\draw[color=black] (-0.987407407407407,-0.5051851851851847) node {$8$};
\draw [fill=green] (-0.5,-0.8660254037844384) circle (1.8pt);
\draw[color=black] (-0.6140740740740738,-0.8251851851851846) node {$9$};
\draw [fill=green] (0.,-1.) circle (1.8pt);
\draw[color=black] (0.0022222222222222227,-1.13) node {$10$};
\draw [fill=green] (0.5,-0.8660254037844388) circle (1.8pt);
\draw[color=black] (0.61,-0.8962962962962957) node {$11$};
\draw [fill=green] (0.8660254037844384,-0.5) circle (1.8pt);
\draw[color=black] (0.98,-0.49) node {$12$};
\end{tikzpicture}
%G4+G6
\hspace{1cm}
\begin{tikzpicture}[line cap=round,line join=round,>=triangle 45,x=2.0cm,y=2.0cm]
%G3
\draw [line width=1.pt] (1.,0.)-- (0.5,-0.8660254037844388);
\draw [line width=1.pt] (0.,1.)-- (0.5,0.8660254037844386);
\draw [line width=1.pt] (0.8660254037844384,-0.5)-- (-0.8660254037844383,0.5);
\draw [line width=1.pt] (0.,-1.)-- (-0.5,-0.8660254037844384);
\draw [line width=1.pt] (-0.8660254037844386,-0.5)-- (-0.5,0.8660254037844386);
\draw [line width=1.pt] (-1.,0.)-- (0.8660254037844387,0.5);
%G4
\draw [line width=1.pt] (1.,0.)-- (-0.8660254037844386,-0.5);
\draw [line width=1.pt] (0.8660254037844387,0.5)-- (0.5,0.8660254037844386);
\draw [line width=1.pt] (0.,1.)-- (-1.,0.);
\draw [line width=1.pt] (-0.5,0.8660254037844386)-- (0.5,-0.8660254037844388);
\draw [line width=1.pt] (-0.8660254037844383,0.5)-- (0.,-1.);
\draw [line width=1.pt] (0.8660254037844384,-0.5)-- (-0.5,-0.8660254037844384);
\draw [fill=green] (1.,0.) circle (1.8pt);
\draw[color=black] (1.11037037037037,0.057777777777778094) node {$1$};
\draw [fill=green] (0.8660254037844387,0.5) circle (1.8pt);
\draw[color=black] (0.9681481481481479,0.5970370370370371) node {$2$};
\draw [fill=green] (0.5,0.8660254037844386) circle (1.8pt);
\draw[color=black] (0.5770370370370369,1.017777777777778) node {$3$};
\draw [fill=green] (0.,1.) circle (1.8pt);
\draw[color=black] (0.014074074074074072,1.1481481481481481) node {$4$};
\draw [fill=green] (-0.5,0.8660254037844386) circle (1.8pt);
\draw[color=black] (-0.5785185185185183,1.005925925925926) node {$5$};
\draw [fill=green] (-0.8660254037844383,0.5) circle (1.8pt);
\draw[color=black] (-0.987407407407407,0.5851851851851853) node {$6$};
\draw [fill=green] (-1.,0.) circle (1.8pt);
\draw[color=black] (-1.1,0.045925925925926245) node {$7$};
\draw [fill=green] (-0.8660254037844386,-0.5) circle (1.8pt);
\draw[color=black] (-0.987407407407407,-0.5051851851851847) node {$8$};
\draw [fill=green] (-0.5,-0.8660254037844384) circle (1.8pt);
\draw[color=black] (-0.6140740740740738,-0.8251851851851846) node {$9$};
\draw [fill=green] (0.,-1.) circle (1.8pt);
\draw[color=black] (0.0022222222222222227,-1.13) node {$10$};
\draw [fill=green] (0.5,-0.8660254037844388) circle (1.8pt);
\draw[color=black] (0.61,-0.8962962962962957) node {$11$};
\draw [fill=green] (0.8660254037844384,-0.5) circle (1.8pt);
\draw[color=black] (0.98,-0.49) node {$12$};
\end{tikzpicture}
%G3+G4

Uniones de dos emparejamientos aleatorios sin aristas repetidas:

\hspace{1cm}
\begin{tikzpicture}[line cap=round,line join=round,>=triangle 45,x=2.0cm,y=2.0cm]
\draw [line width=1.pt] (1.,0.)-- (-0.8660254037844386,-0.5);
\draw [line width=1.pt] (-0.8660254037844386,-0.5)-- (-0.5,0.8660254037844386);
\draw [line width=1.pt] (-0.5,0.8660254037844386)-- (0.5,-0.8660254037844388);
\draw [line width=1.pt] (0.5,-0.8660254037844388)-- (1.,0.);
\draw [line width=1.pt] (0.8660254037844384,-0.5)-- (-0.5,-0.8660254037844384);
\draw [line width=1.pt] (-0.5,-0.8660254037844384)-- (0.,-1.);
\draw [line width=1.pt] (0.,-1.)-- (-0.8660254037844383,0.5);
\draw [line width=1.pt] (-0.8660254037844383,0.5)-- (0.8660254037844384,-0.5);
\draw [line width=1.pt] (0.,1.)-- (0.5,0.8660254037844386);
\draw [line width=1.pt] (0.5,0.8660254037844386)-- (0.8660254037844387,0.5);
\draw [line width=1.pt] (0.8660254037844387,0.5)-- (-1.,0.);
\draw [line width=1.pt] (-1.,0.)-- (0.,1.);
\draw [line width=1.pt] (-0.5,-0.8660254037844384)-- (-0.8660254037844386,-0.5);
\draw [line width=1.pt] (-1.,0.)-- (0.,-1.);
\draw [line width=1.pt] (0.5,-0.8660254037844388)-- (0.8660254037844384,-0.5);
\draw [line width=1.pt] (1.,0.)-- (0.5,0.8660254037844386);
\draw [line width=1.pt] (0.8660254037844387,0.5)-- (-0.5,0.8660254037844386);
\draw [line width=1.pt] (0.,1.)-- (-0.8660254037844383,0.5);
\draw[color=black] (1.11037037037037,0.057777777777778094) node {$1$};
\draw [fill=green] (0.8660254037844387,0.5) circle (1.8pt);
\draw[color=black] (0.9681481481481479,0.5970370370370371) node {$2$};
\draw [fill=green] (0.5,0.8660254037844386) circle (1.8pt);
\draw[color=black] (0.5770370370370369,1.017777777777778) node {$3$};
\draw [fill=green] (0.,1.) circle (1.8pt);
\draw[color=black] (0.014074074074074072,1.1481481481481481) node {$4$};
\draw [fill=green] (-0.5,0.8660254037844386) circle (1.8pt);
\draw[color=black] (-0.5785185185185183,1.005925925925926) node {$5$};
\draw [fill=green] (-0.8660254037844383,0.5) circle (1.8pt);
\draw[color=black] (-0.987407407407407,0.5851851851851853) node {$6$};
\draw [fill=green] (-1.,0.) circle (1.8pt);
\draw[color=black] (-1.1,0.045925925925926245) node {$7$};
\draw [fill=green] (-0.8660254037844386,-0.5) circle (1.8pt);
\draw[color=black] (-0.987407407407407,-0.5051851851851847) node {$8$};
\draw [fill=green] (-0.5,-0.8660254037844384) circle (1.8pt);
\draw[color=black] (-0.6140740740740738,-0.8251851851851846) node {$9$};
\draw [fill=green] (0.,-1.) circle (1.8pt);
\draw[color=black] (0.0022222222222222227,-1.13) node {$10$};
\draw [fill=green] (0.5,-0.8660254037844388) circle (1.8pt);
\draw[color=black] (0.61,-0.8962962962962957) node {$11$};
\draw [fill=green] (0.8660254037844384,-0.5) circle (1.8pt);
\draw[color=black] (0.98,-0.49) node {$12$};
\end{tikzpicture}

Unión de tres emparejamientos sin aristas repetidas:
%G4+G6+G11

Tarea (Charli): Programar gráficas 4-regulares aleatorias.

\newpage

\newpage
\section{Gráficas regulares aleatorias}

\begin{problema}
Se toman dos emparejamientos al azar de $6$ vértices (uniformemente). ¿Cuál es la probabilidad de que ninguna arista se repita?
\end{problema}
\vspace{3cm}

%2 formas: Verificar primera con Inc-Exc.

\begin{problema}
Se toman dos emparejamientos al azar de $8$ vértices. ¿Cuál es la probabilidad de que ninguna arista se repita?
\end{problema}
\vspace{3cm}

%2 formas: verificar Inc-Exc

%\begin{problema} ¿Cuántas gráficas $2$-regulares de $n$ vértices hay?  \end{problema}

\newpage

\section{Problemas de gráficas}

\begin{problema}
En una gráfica de $n$ vértices todos los vértices tienen grado mayor o igual a $\frac{n}{2}$. Muestra que la gráfica es conexa.
\end{problema}

\begin{problema}
Al final de un torneo de fútbol en el que cada dos equipos se encontraron una vez, donde no hubo empates, se observó que para cada tres equipos $A, B$ y $C$, si $A$ venció a $B$ y $B$ venció a  $C$, entonces $A$ venció a $C$. 

Cada equipo calculó la diferencia (positiva) entre el número de juegos que ganó y perdió. La suma de todas estas diferencias da $5000$. ¿Cuántos equipos jugaron en el torneo? Encuentra todas las posibles respuestas.
%Sug: Conjunto totalmente ordenado, n=50.
\end{problema}

\begin{problema}
Muestra que una gráfica conexa es Euleriana si y solo si todos sus vértices tienen grado par.
\end{problema}


\begin{problema}
En una mesa redonda se sientan $2n$ peruanos, $2n$ bolivanos y $2n$ ecuatorianos. Si se pide que se pongan de pie a todos los que estén sentados junto a dos personas de la misma nacionalidad. ¿Cuál es la máxima cantidad de personas que se pueden parar? 

%Sug: Gráfica de 2-distancia 
\end{problema}

\begin{problema}
En una fiesta con $n$ invitados se cumple que para cualesquiera cuatro personas, hay un subconjunto de tres personas que, o se conocen todas entre sí o ninguna se conoce.

Mostrar que los $n$ invitados se pueden separar en dos cuartos, de tal manera que en un cuarto todos se conocen y en el otro ninguno se conoce.
\end{problema}

\begin{problema}
Muestra que la matriz de adyacencia de una gráfica regular de grado $d$ tiene un valor propio igual a $d$.
\end{problema}



% \setcounter{chapter}{9}

\chapter{Recursiones, Matrices y Gráficas}

Consideremos la sucesión de Fibonacci $F_0=0$, $F_1=1$, $F_{n+1}=F_{n}+F_{n-1}$, y las siguientes preguntas:

\begin{teorema}\label{Teo:Fibo}
Si $\phi_+:=\frac{1+\sqrt{5}}{2}$ y $\phi_-:=\frac{1-\sqrt{5}}{2}$, entonces 
\begin{equation}\label{eq:FibonacciGeneral}
F_n=\frac{(\phi_+)^n-(\phi_-)^n}{\sqrt{5}}
\end{equation}
\end{teorema}

\begin{itemize}
    \item P1. ¿De donde viene la fórmula \ref{eq:FibonacciGeneral}?
\end{itemize}

\begin{ejercicio}
Muestra que $\phi_+$ y $\phi_-$ son las dos raíces del polinomio $x^2-x-1$.

Muestra que $$\phi_+^2=1+\phi_+,\quad \phi_-^2=1+\phi_-,\quad \phi_+\phi_-=-1.$$
\end{ejercicio}

Supongamos que se cambian las condiciones iniciales de los números de Fibonacci:
$$\hat F_0=x,\quad  \hat F_1=y,\quad  \hat F_n=\hat F_{n-1}+\hat F_{n-2} \quad (n\geq 1).$$

\begin{itemize}
    \item P2. ¿Cómo cambian los primeros números de la recursión si cambiamos las condiciones iniciales  $(x,y)$?

    \item P3. ¿Cómo cambia la fórmula del teorema \ref{Teo:Fibo} al cambiarse las condiciones iniciales  $(x,y)$?
\end{itemize}

En esta sesión de ejercicios responderemos a las preguntas anteriores. Comenzamos con la segunda y la tercera pregunta, que tienen respuestas más o menos intuitivas.

\begin{ejercicio} Completa la siguiente tabla sobre números de Fibonacci con distintas condiciones iniciales

  \begin{center}
    \renewcommand{\arraystretch}{1.5}
    \begin{tabular}{llll}
      \hline
      $(x,y)$ & Primeros ocho términos $\hat F_0$ a $\hat F_7$ & Fórmula General \\
      \hline
      $(0,0)$ &  &  \\
      \hline
      $(1,0)$ &  &  \\
            \hline
      $(1,1)$ &  &  \\
            \hline
      $(0,-1)$ &  &  \\
            \hline
      $(3,5)$ &  &  \\
            \hline
      $(4,4)$ &  &  \\
            \hline
      $(7,8)$ &  &  \\
                  \hline
      $(2022,2202)$ &  &  \\
                  \hline
      $(\pi,2022)$ &  &  \\
                  \hline
    \end{tabular}
  \end{center}
\end{ejercicio}

A priori, la fórmula general del Teorema \ref{Teo:Fibo} nos resulta bastante misteriosa. Al involucrar raíces de cinco y la razón dorada $\phi$, ni siquiera es obvio que la fórmula arroja un número entero, en contraste con la fórmula recursiva. Si ya se conoce la fórmula, verificarla es un buen ejercicio de inducción.

\begin{ejercicio}
Demuestra el Teorema \ref{Teo:Fibo} usando inducción matemática.
\end{ejercicio}

Es un poco artificial suponer que a alguno de nosotros se nos ocurriría dicha fórmula de la nada. Sin embargo, la fórmula se deriva de una manera bastante natural y estándar si se ataca con con algunas herramientas básicas del álgebra superior (matrices y vectores de dimensión pequeña) 

\section{Multiplicación de matrices}

Una {\bf matriz} es un arreglo rectangular de números (reales o complejos), por ejemplo:
$$
\left(\begin{array}{ccc}
a_{11}&a_{12} \\
a_{21}&a_{22} 
\end{array}\right)
, \quad
\left(\begin{array}{ccc}
a_{11}&a_{12}&a_{13} \\
a_{21}&a_{22}&a_{23} \\
a_{31}&a_{32}&a_{33}  
\end{array}\right),\quad 
\left(\begin{array}{cc}
a_{11}&a_{12} \\
a_{21}&a_{22} \\
a_{31}&a_{32}  
\end{array}\right)
$$

Dos matrices cuadradas del mismo tamaño se pueden sumar y multiplicar de acuerdo a las siguientes reglas: Si

$$A=\left(\begin{array}{cc}
a_{11}&a_{12} \\
a_{21}&a_{22} 
\end{array}\right),\quad 
B=\left(\begin{array}{cc}
b_{11}&b_{12} \\
b_{21}&b_{22} 
\end{array}\right),$$
entonces $$A+B=\left(\begin{array}{cc}
a_{11}&a_{12} \\
a_{21}&a_{22} 
\end{array}\right)
+
\left(\begin{array}{ccc}
b_{11}&b_{12} \\
b_{21}&b_{22}
\end{array}\right)=
\left(\begin{array}{ccc}
a_{11}+b_{11}&a_{12}+b_{12} \\
a_{21}+b_{21}&a_{22}+b_{22}
\end{array}\right),$$
y
$$A\cdot B=\left(\begin{array}{cc}
a_{11}&a_{12} \\
a_{21}&a_{22} 
\end{array}\right)
\cdot
\left(\begin{array}{ccc}
b_{11}&b_{12} \\
b_{21}&b_{22}
\end{array}\right)=
\left(\begin{array}{ccc}
a_{11}b_{11}+a_{12}b_{21}&a_{11}b_{12}+a_{12}b_{22} \\
a_{21}b_{11}+a_{22}b_{21}&a_{21}b_{12}+a_{22}b_{22}
\end{array}\right)$$


Para matrices de $3\times 3$ y $n\times n$ la regla es similar:
$$C=\left(\begin{array}{ccc}
c_{11}&c_{12}&c_{13} \\
c_{21}&c_{22}&c_{23} \\
c_{31}&c_{32}&c_{33}  
\end{array}\right),
\quad 
D=\left(\begin{array}{ccc}
d_{11}&d_{12}&d_{13} \\
d_{21}&d_{22}&d_{23} \\
d_{31}&d_{32}&d_{33}  
\end{array}\right),$$
entonces 
$$C+D=\left(\begin{array}{ccc}
c_{11}&c_{12}&c_{13} \\
c_{21}&c_{22}&c_{23} \\
c_{31}&c_{32}&c_{33}  
\end{array}\right)
+
\left(\begin{array}{ccc}
d_{11}&d_{12}&d_{13} \\
d_{21}&d_{22}&d_{23} \\
d_{31}&d_{32}&d_{33}  
\end{array}\right)= 
\left(\begin{array}{ccc}
c_{11}+d_{11}&c_{12}+d_{12}&c_{13}+d_{13} \\
c_{21}+d_{21}&c_{22}+d_{22}&c_{23}+d_{23} \\
c_{31}+d_{31}&c_{32}+d_{32}&c_{33}+d_{33}  
\end{array}\right)
,$$
La entrada $(i,j)$ del producto se obtiene apareando las entradas del i-ésimo renglón de $C$, $(c_{i1},c_{i2},c_{i3})$ con las entradas de la $j$-ésima columna de $D$, $(d_{1j},d_{2j}, d_{3j})$. Por ejemplo, la entrada $(2,3)$ da $c_{21}d_{13}+c_{22}d_{23}+c_{23}d_{33}$ y similarmente 
$$C\cdot 
D=\left(\begin{array}{ccc}
c_{11}&c_{12}&c_{13} \\
c_{21}&c_{22}&c_{23} \\
c_{31}&c_{32}&c_{33}  
\end{array}\right)\cdot
\left(\begin{array}{ccc}
d_{11}&d_{12}&d_{13} \\
d_{21}&d_{22}&d_{23} \\
d_{31}&d_{32}&d_{33}  
\end{array}\right)$$
$$
=\left(\begin{array}{ccc}
c_{11}d_{11}+c_{12}d_{21}+c_{13}d_{31}  &  c_{11}d_{12}+c_{12}d_{22}+c_{13}d_{32}  &  c_{11}d_{13}+c_{12}d_{23}+c_{13}d_{33} \\
c_{21}d_{11}+c_{22}d_{21}+c_{23}d_{31}  &  c_{21}d_{12}+c_{22}d_{22}+c_{23}d_{32}  &  c_{21}d_{13}+c_{22}d_{23}+c_{23}d_{33}   \\
c_{31}d_{11}+c_{32}d_{21}+c_{33}d_{31}  &  c_{31}d_{12}+c_{32}d_{22}+c_{33}d_{32}  &  c_{31}d_{13}+c_{32}d_{23}+c_{33}d_{33}  
\end{array}\right)
$$

De manera análoga se definen las sumas y los productos de matrices para $n\geq 3$. La entrada $(i,j)$ del producto $A\cdot B$ de dos matrices involucra únicamente apareamientos de las entradas del $i$-ésimo renglón de $A$ y las entradas de la $j$-ésima columna de $B$.

Como ocurre con variables en álgebra, se conviene que dos matrices escritas una detrás de la otra se están multiplicando y normalmente se omite el puntito, entendiendo que $AB=A\cdot B$ y $A^n=A\cdot A\cdot A\cdot \cdots A$.

El elemento neutro multiplicativo en las matrices de $n\times n$ es la matriz identidad $I_n$ que consiste de unos en la diagonal y ceros en el resto de las entradas, por ejemplo: 
$$ I_2=
\left(\begin{array}{cc}
1&0  \\
0&1
\end{array}\right),
\quad 
I_3=
\left(\begin{array}{ccc}
1&0&0  \\
0&1&0  \\
0&0&1
\end{array}\right),
\quad 
I_4=
\left(\begin{array}{cccc}
1&0&0&0  \\
0&1&0&0  \\
0&0&1&0  \\
0&0&0&1
\end{array}\right)
$$

\begin{ejercicio}
Verifica que $I_n^k=I_n$ y calcula los siguientes productos de matrices

\begin{enumerate}
    \item     $$\left(\begin{array}{cc}
1&0 \\
0&1 
\end{array}\right)
\left(\begin{array}{cc}
a_{11}&a_{12} \\
a_{21}&a_{22} 
\end{array}\right),
\quad
\left(\begin{array}{cc}
a_{11}&a_{12} \\
a_{21}&a_{22} 
\end{array}\right)
\left(\begin{array}{cc}
1&0 \\
0&1 
\end{array}\right)
$$
    \item $$\left(\begin{array}{ccc}
1&0&0  \\
0&1&0  \\
0&0&1
\end{array}\right)
\left(\begin{array}{ccc}
a_{11}& a_{12}& a_{13} \\
a_{21}& a_{22}& a_{23} \\
a_{31}& a_{32}& a_{33}  
\end{array}\right),
\quad
\left(\begin{array}{ccc}
a_{11}& a_{12}& a_{13} \\
a_{21}& a_{22}& a_{23} \\
a_{31}& a_{32}& a_{33}  
\end{array}\right)
\left(\begin{array}{ccc}
1&0&0  \\
0&1&0  \\
0&0&1
\end{array}\right)
$$
\end{enumerate}
       
\end{ejercicio}

El objetivo principal de esta sesion de ejercicios es entender las potencias de la matriz de Fibonacci:
$$\mathbb F:=\left(\begin{array}{cc}
0&1 \\
1&1 
\end{array}\right)$$
\begin{ejercicio}
Muestra que las potencias de la matriz de Fibonacci estan dadas por la fórmula

$$\mathbb F^n=\left(\begin{array}{cc}
0&1 \\
1&1 
\end{array}\right)^n
=
\left(\begin{array}{cc}
F_n&F_{n+1} \\
F_{n+1}&F_{n+2} 
\end{array}\right),
\quad
n\geq 0
$$
Donde $F_n$, $n\geq 0$ son los números de Fibonacci.

Sug: Inducción.
\end{ejercicio}

Las matrices diagonales son fáciles de multiplicar.

\begin{ejercicio}
Calcula los siguientes productos de matrices

$$\left(\begin{array}{cc}
a&0 \\
0&b 
\end{array}\right)
\left(\begin{array}{cc}
c&0 \\
0&d 
\end{array}\right)
,\quad
\left(\begin{array}{cc}
\lambda_1&0 \\
0&\lambda_2 
\end{array}\right)^n$$
\end{ejercicio}

Recuerda que la suma y la multiplicación de dos números complejos $z=a+b\mathrm i$ y $w=c+d\mathrm i$ se define como $$z+w=(a+c)+(b+d)\mathrm i, \quad z w=(ac-bd)+(ad+bc)\mathrm i$$

El siguiente ejercicio muestra que el campo de números complejos está escondido dentro del anillo de matrices reales de $2\times 2$.
\begin{ejercicio}
Calcula los siguientes productos y sumas de matrices.
$$\left(\begin{array}{cc}
0&1 \\
-1&0 
\end{array}\right)
\left(\begin{array}{cc}
0&1 \\
-1&0 
\end{array}\right) 
,\quad
\left(\begin{array}{cc}
a&b \\
-b&a 
\end{array}\right)
+
\left(\begin{array}{cc}
c&d \\
-d&c 
\end{array}\right)
,\quad
\left(\begin{array}{cc}
a&b \\
-b&a 
\end{array}\right)
\left(\begin{array}{cc}
c&d \\
-d&c 
\end{array}\right)$$
\end{ejercicio}


\begin{ejercicio}
En particular, los complejos de norma $1$ del ejercicio anterior corresponden a {\bf matrices de rotación} reales. Calcula:
$$\left(\begin{array}{cc}
\cos\alpha&\sen\alpha \\
-\sen\alpha&\cos\alpha 
\end{array}\right)
\left(\begin{array}{cc}
\cos\beta&\sen\beta \\
-\sen\beta&\cos\beta 
\end{array}\right).
$$

\end{ejercicio}



\begin{ejercicio}
La suma y la multiplicación de complejos son conmutativas. Por el contrario la multiplicación de matrices no es conmutativa en general. Verifica lo anterior calculando:
$$\left(\begin{array}{cc}
1&2 \\
3&4 
\end{array}\right)
\left(\begin{array}{cc}
5&6 \\
7&8 
\end{array}\right),
\quad
\left(\begin{array}{cc}
5&6 \\
7&8 
\end{array}\right)
\left(\begin{array}{cc}
1&2 \\
3&4 
\end{array}\right)
.
$$
\end{ejercicio}

\begin{ejercicio}
¿Cuántas matrices distintas de $n\times n$ existen con exactamente $n$ unos y $n^2-n$ ceros, de tal forma que haya exactamente un uno en cada columna y en cada fila?
\end{ejercicio}

A las matrices como las del ejercicio anterior se les llama {\bf matrices de permutación} (más adelante, cuando apliquemos matrices a vectores veremos por qué se les llama así).

\begin{ejercicio}
Calcula las potencias de las siguientes matrices de permutación:

$$\left(\begin{array}{ccc}
0&1&0 \\
1&0&0 \\
0&0&1
\end{array}\right)
,\quad 
\left(\begin{array}{ccc}
0&1&0 \\
0&0&1 \\
1&0&0  
\end{array}\right)
,\quad 
\left(\begin{array}{ccccc}
0 & 1 & 0 & 0 & 0 \\
0 & 0 & 1 & 0 & 0 \\
0 & 0 & 0 & 1 & 0 \\
0 & 0 & 0 &  0 & 1 \\
1 & 0 & 0 & 0 & 0
\end{array}\right)
,\quad 
\left(\begin{array}{ccccc}
0 & 0 & 0 & 0 & 1 \\
0 & 0 & 1 & 0 & 0 \\
0 & 1 & 0 & 0 & 0 \\
1 & 0 & 0 & 0 & 0 \\
0 & 0 & 0 & 1 & 0
\end{array}\right)
$$
\end{ejercicio}

\begin{ejercicio}
Para las matrices de permutación del ejercicio anterior. ¿Puedes encontrar los inversos multiplicativos?
\end{ejercicio}


Para verificar productos de matrices de forma rápida, se pueden ingresar las matrices en GeoGebra renglón por renglón, separados por comas.

Por ejemplo, para ingresar las matrices 
$$A=\left(\begin{array}{cc}
a& b \\
c& d 
\end{array}\right),
\quad 
B=\left(\begin{array}{ccc}
a& b& c \\
d& e& f \\
g& h& i  
\end{array}\right),$$
en GeoGebra se escribiría $\{\{a,b\},\{c,d\}\}$ para la matriz $A$ y  $\{\{a,b,c\},\{d,e,f\},\{g,h,i\}\}$ para la matriz $B$.

\begin{ejercicio}
Utiliza Geogebra para calcular la veinteava potencia de la matriz de Fibonacci .
\end{ejercicio}
%geogebra:e8q8rtrr
Antes de continuar con el problema de entender mejor la matriz de Fibonacci, revisaremos otro tipo de matrices que describen la estructura de una gráfica.
\newpage

\section{Matrices de adyacencia}


\begin{definicion}
La {\bf matriz de adyacencia} de una gráfica $G=(V,A)$ es la matriz de unos y ceros que tiene ceros en la diagonal y un $1$ en las entradas $(i,j)$ y $(j,i)$ si y solo si la arista $v_iv_j\in A$. 
\end{definicion}

\begin{tikzpicture}[line cap=round,line join=round,>=triangle 45,x=.7*1.0cm,y=.7*1.0cm]
\draw [line width=1.pt] (4.16,1.18)-- (3.16,-0.78);
\draw [line width=1.pt] (4.16,1.18)-- (2.62,2.92);
\draw [line width=1.pt] (4.16,1.18)-- (-0.32,0.08);
\draw [line width=1.pt] (-1.92,3.)-- (0.54,3.82);
\draw [line width=1.pt] (0.54,3.82)-- (-0.32,0.08);
\draw [line width=1.pt] (-0.32,0.08)-- (2.62,2.92);
\draw [line width=1.pt] (0.54,3.82)-- (2.62,2.92);
\draw [fill=black] (0.54,3.82) circle (1.5pt);
\draw[color=black] (0.56,4.27) node {$v_2$};
\draw [fill=black] (-0.32,0.08) circle (1.5pt);
\draw[color=black] (-0.6,-0.13) node {$v_3$};
\draw [fill=black] (3.16,-0.78) circle (1.5pt);
\draw[color=black] (3.08,-1.27) node {$v_6$};
\draw [fill=black] (4.16,1.18) circle (1.5pt);
\draw[color=black] (4.42,1.47) node {$v_5$};
\draw [fill=black] (2.62,2.92) circle (1.5pt);
\draw[color=black] (2.98,3.21) node {$v_4$};
\draw [fill=black] (-1.92,3.) circle (1.5pt);
\draw[color=black] (-2.3,3.17) node {$v_1$};
\draw[color=black] (.5,-3.5) node 
{$A_{G_1}=\left(\begin{array}{cccccc}
0 & 1 & 0 & 0 & 0 & 0 \\
1 & 0 & 1 & 1 & 0 & 0\\
0 & 1 & 0 & 1 & 1 & 0\\
0 & 1 & 1 & 0 & 1 & 0\\
0 & 0 & 1 & 1 & 0 & 1\\
0 & 0 & 0 & 0 & 1 & 0
\end{array}\right)
$};
\end{tikzpicture}
\begin{tikzpicture}[line cap=round,line join=round,>=triangle 45,x=1.0cm,y=1.0cm]
\draw [line width=1.pt] (-1.5,2.52)-- (1.88,0.3);
\draw [line width=1.pt] (1.88,0.3)-- (-0.24,3.66);
\draw [line width=1.pt] (-0.24,3.66)-- (-0.8,0.22);
\draw [line width=1.pt] (-0.8,0.22)-- (2.24,2.7);
\draw [line width=1.pt] (2.24,2.7)-- (-1.5,2.52);
\draw [fill=black] (-1.5,2.52) circle (1.5pt);
\draw[color=black] (-1.82,2.69) node {$v_1$};
\draw [fill=black] (-0.24,3.66) circle (1.5pt);
\draw[color=black] (-0.2,4.11) node {$v_2$};
\draw [fill=black] (2.24,2.7) circle (1.5pt);
\draw[color=black] (2.52,2.87) node {$v_3$};
\draw [fill=black] (1.88,0.3) circle (1.5pt);
\draw[color=black] (2.16,0.15) node {$v_4$};
\draw [fill=black] (-0.8,0.22) circle (1.5pt);
\draw[color=black] (-0.96,-0.01) node {$v_5$};
\draw[color=black] (.5,-1.5) node {$A_{G_2}=\left(\begin{array}{ccccc}
0 & 0 & 1 & 1 & 0 \\
0 & 0 & 0 & 1 & 1 \\
1 & 0 & 0 & 0 & 1 \\
1 & 1 & 0 & 0 & 0 \\
0 & 1 & 1 & 0 & 0
\end{array}\right)
$};
\end{tikzpicture}
\begin{tikzpicture}[line cap=round,line join=round,>=triangle 45,x=1.0cm,y=1.0cm]
\draw [line width=1.pt] (0.,0.)-- (3.22,1.14);
\draw [line width=1.pt] (0.02,1.58)-- (3.16,2.52);
\draw [line width=1.pt] (0.02,1.58)-- (3.28,-0.34);
\draw [line width=1.pt] (0.,0.)-- (3.16,2.52);
\draw [fill=black] (0.,0.) circle (1.5pt);
\draw[color=black] (-0.3,0.03) node {$v_1$};
\draw [fill=black] (0.02,1.58) circle (1.5pt);
\draw[color=black] (-0.28,1.67) node {$v_2$};
\draw [fill=black] (3.16,2.52) circle (1.5pt);
\draw[color=black] (3.44,2.63) node {$v_3$};
\draw [fill=black] (3.22,1.14) circle (1.5pt);
\draw[color=black] (3.48,1.23) node {$v_4$};
\draw [fill=black] (3.28,-0.34) circle (1.5pt);
\draw[color=black] (3.54,-0.27) node {$v_5$};
\draw[color=black] (1.5,-2.1) node {$A_{G_3}=\left(\begin{array}{ccccc}
0 & 0 & 1 & 1 & 0 \\
0 & 0 & 1 & 0 & 1 \\
1 & 1 & 0 & 0 & 0 \\
1 & 0 & 0 & 0 & 0 \\
0 & 1 & 0 & 0 & 0
\end{array}\right)
$};
\end{tikzpicture}

\begin{ejercicio} 
Utiliza GeoGebra para calcular las primeras cuatro potencias de las matrices de adyacencia anteriores.
\end{ejercicio}

\begin{ejercicio} 
Demuestra que las entradas de la diagonal de la matriz $(A_G)^2$ son los grados de los vértices.
\end{ejercicio}

\begin{ejercicio}
Demuestra que la entrada $(i,j)$ de $A_G^k$ cuenta el número de caminos de tamaño $k$ en $G$ que van desde el vértice $v_i$ al vértice $v_j$.

En particular las entradas de la diagonal $(i,i)$ cuentan el número de ciclos de tamaño $k$ desde $v_i$ 
\end{ejercicio}

\begin{tikzpicture}[line cap=round,line join=round,>=triangle 45,x=.7*1.0cm,y=.7*1.0cm]
\draw [line width=1.pt] (4.16,1.18)-- (3.16,-0.78);
\draw [line width=1.pt] (4.16,1.18)-- (2.62,2.92);
\draw [line width=1.pt] (4.16,1.18)-- (-0.32,0.08);
\draw [line width=1.pt] (-1.92,3.)-- (0.54,3.82);
\draw [line width=1.pt] (0.54,3.82)-- (-0.32,0.08);
\draw [line width=1.pt] (-0.32,0.08)-- (2.62,2.92);
\draw [line width=1.pt] (0.54,3.82)-- (2.62,2.92);
\draw [fill=black] (0.54,3.82) circle (1.5pt);
\draw[color=black] (0.56,4.27) node {$v_2$};
\draw [fill=black] (-0.32,0.08) circle (1.5pt);
\draw[color=black] (-0.6,-0.13) node {$v_3$};
\draw [fill=black] (3.16,-0.78) circle (1.5pt);
\draw[color=black] (3.08,-1.27) node {$v_6$};
\draw [fill=black] (4.16,1.18) circle (1.5pt);
\draw[color=black] (4.42,1.47) node {$v_5$};
\draw [fill=black] (2.62,2.92) circle (1.5pt);
\draw[color=black] (2.98,3.21) node {$v_4$};
\draw [fill=black] (-1.92,3.) circle (1.5pt);
\draw[color=black] (-2.3,3.17) node {$v_1$};
\draw[color=black] (.5,-3.5) node 
{$(A_{G_1})^2=\left(\begin{array}{cccccc}
1 & 0 & 1 & 1 & 0 & 0 \\
0 & 3 & 1 & 1 & 2 & 0\\
1 & 1 & 3 & 2 & 1 & 1\\
1 & 1 & 2 & 3 & 1 & 1\\
0 & 2 & 1 & 1 & 3 & 0\\
0 & 0 & 1 & 1 & 0 & 1
\end{array}\right)
$};
\end{tikzpicture}
\begin{tikzpicture}[line cap=round,line join=round,>=triangle 45,x=1.0cm,y=1.0cm]
\draw [line width=1.pt] (-1.5,2.52)-- (1.88,0.3);
\draw [line width=1.pt] (1.88,0.3)-- (-0.24,3.66);
\draw [line width=1.pt] (-0.24,3.66)-- (-0.8,0.22);
\draw [line width=1.pt] (-0.8,0.22)-- (2.24,2.7);
\draw [line width=1.pt] (2.24,2.7)-- (-1.5,2.52);
\draw [fill=black] (-1.5,2.52) circle (1.5pt);
\draw[color=black] (-1.82,2.69) node {$v_1$};
\draw [fill=black] (-0.24,3.66) circle (1.5pt);
\draw[color=black] (-0.2,4.11) node {$v_2$};
\draw [fill=black] (2.24,2.7) circle (1.5pt);
\draw[color=black] (2.52,2.87) node {$v_3$};
\draw [fill=black] (1.88,0.3) circle (1.5pt);
\draw[color=black] (2.16,0.15) node {$v_4$};
\draw [fill=black] (-0.8,0.22) circle (1.5pt);
\draw[color=black] (-0.96,-0.01) node {$v_5$};
\draw[color=black] (.5,-1.5) node {$(A_{G_2})^2=\left(\begin{array}{ccccc}
2 & 1 & 0 & 0 & 1 \\
1 & 2 & 1 & 0 & 0 \\
0 & 1 & 2 & 1 & 0 \\
0 & 0 & 1 & 2 & 1 \\
1 & 0 & 0 & 1 & 2
\end{array}\right)
$};
\end{tikzpicture}
\begin{tikzpicture}[line cap=round,line join=round,>=triangle 45,x=1.0cm,y=1.0cm]
\draw [line width=1.pt] (0.,0.)-- (3.22,1.14);
\draw [line width=1.pt] (0.02,1.58)-- (3.16,2.52);
\draw [line width=1.pt] (0.02,1.58)-- (3.28,-0.34);
\draw [line width=1.pt] (0.,0.)-- (3.16,2.52);
\draw [fill=black] (0.,0.) circle (1.5pt);
\draw[color=black] (-0.3,0.03) node {$v_1$};
\draw [fill=black] (0.02,1.58) circle (1.5pt);
\draw[color=black] (-0.28,1.67) node {$v_2$};
\draw [fill=black] (3.16,2.52) circle (1.5pt);
\draw[color=black] (3.44,2.63) node {$v_3$};
\draw [fill=black] (3.22,1.14) circle (1.5pt);
\draw[color=black] (3.48,1.23) node {$v_4$};
\draw [fill=black] (3.28,-0.34) circle (1.5pt);
\draw[color=black] (3.54,-0.27) node {$v_5$};
\draw[color=black] (1.5,-2.1) node {$(A_{G_3})^2=\left(\begin{array}{ccccc}
2 & 1 & 0 & 0 & 0 \\
1 & 2 & 0 & 0 & 0 \\
0 & 0 & 2 & 1 & 1 \\
0 & 0 & 1 & 1 & 0 \\
0 & 0 & 1 & 0 & 1
\end{array}\right)
$};
\end{tikzpicture}

%%%cubos

\begin{tikzpicture}[line cap=round,line join=round,>=triangle 45,x=.7*1.0cm,y=.7*1.0cm]
\draw [line width=1.pt] (4.16,1.18)-- (3.16,-0.78);
\draw [line width=1.pt] (4.16,1.18)-- (2.62,2.92);
\draw [line width=1.pt] (4.16,1.18)-- (-0.32,0.08);
\draw [line width=1.pt] (-1.92,3.)-- (0.54,3.82);
\draw [line width=1.pt] (0.54,3.82)-- (-0.32,0.08);
\draw [line width=1.pt] (-0.32,0.08)-- (2.62,2.92);
\draw [line width=1.pt] (0.54,3.82)-- (2.62,2.92);
\draw [fill=black] (0.54,3.82) circle (1.5pt);
\draw[color=black] (0.56,4.27) node {$v_2$};
\draw [fill=black] (-0.32,0.08) circle (1.5pt);
\draw[color=black] (-0.6,-0.13) node {$v_3$};
\draw [fill=black] (3.16,-0.78) circle (1.5pt);
\draw[color=black] (3.08,-1.27) node {$v_6$};
\draw [fill=black] (4.16,1.18) circle (1.5pt);
\draw[color=black] (4.42,1.47) node {$v_5$};
\draw [fill=black] (2.62,2.92) circle (1.5pt);
\draw[color=black] (2.98,3.21) node {$v_4$};
\draw [fill=black] (-1.92,3.) circle (1.5pt);
\draw[color=black] (-2.3,3.17) node {$v_1$};
\draw[color=black] (.5,-3.5) node 
{$(A_{G_1})^3=\left(\begin{array}{cccccc}
0 & 3 & 1 & 1 & 2 & 0 \\
3 & 2 & 6 & 6 & 2 & 2\\
1 & 6 & 4 & 5 & 6 & 1\\
1 & 6 & 5 & 4 & 6 & 1\\
2 & 2 & 6 & 6 & 2 & 3\\
0 & 2 & 1 & 1 & 3 & 0
\end{array}\right)
$};
\end{tikzpicture}
\begin{tikzpicture}[line cap=round,line join=round,>=triangle 45,x=1.0cm,y=1.0cm]
\draw [line width=1.pt] (-1.5,2.52)-- (1.88,0.3);
\draw [line width=1.pt] (1.88,0.3)-- (-0.24,3.66);
\draw [line width=1.pt] (-0.24,3.66)-- (-0.8,0.22);
\draw [line width=1.pt] (-0.8,0.22)-- (2.24,2.7);
\draw [line width=1.pt] (2.24,2.7)-- (-1.5,2.52);
\draw [fill=black] (-1.5,2.52) circle (1.5pt);
\draw[color=black] (-1.82,2.69) node {$v_1$};
\draw [fill=black] (-0.24,3.66) circle (1.5pt);
\draw[color=black] (-0.2,4.11) node {$v_2$};
\draw [fill=black] (2.24,2.7) circle (1.5pt);
\draw[color=black] (2.52,2.87) node {$v_3$};
\draw [fill=black] (1.88,0.3) circle (1.5pt);
\draw[color=black] (2.16,0.15) node {$v_4$};
\draw [fill=black] (-0.8,0.22) circle (1.5pt);
\draw[color=black] (-0.96,-0.01) node {$v_5$};
\draw[color=black] (.5,-1.5) node {$(A_{G_2})^3=\left(\begin{array}{ccccc}
0 & 1 & 3 & 3 & 1 \\
1 & 0 & 1 & 3 & 3 \\
3 & 1 & 0 & 1 & 3 \\
3 & 3 & 1 & 0 & 1 \\
1 & 3 & 3 & 1 & 0
\end{array}\right)
$};
\end{tikzpicture}
\begin{tikzpicture}[line cap=round,line join=round,>=triangle 45,x=1.0cm,y=1.0cm]
\draw [line width=1.pt] (0.,0.)-- (3.22,1.14);
\draw [line width=1.pt] (0.02,1.58)-- (3.16,2.52);
\draw [line width=1.pt] (0.02,1.58)-- (3.28,-0.34);
\draw [line width=1.pt] (0.,0.)-- (3.16,2.52);
\draw [fill=black] (0.,0.) circle (1.5pt);
\draw[color=black] (-0.3,0.03) node {$v_1$};
\draw [fill=black] (0.02,1.58) circle (1.5pt);
\draw[color=black] (-0.28,1.67) node {$v_2$};
\draw [fill=black] (3.16,2.52) circle (1.5pt);
\draw[color=black] (3.44,2.63) node {$v_3$};
\draw [fill=black] (3.22,1.14) circle (1.5pt);
\draw[color=black] (3.48,1.23) node {$v_4$};
\draw [fill=black] (3.28,-0.34) circle (1.5pt);
\draw[color=black] (3.54,-0.27) node {$v_5$};
\draw[color=black] (1.5,-2.1) node {$(A_{G_3})^3=\left(\begin{array}{ccccc}
0 & 0 & 3 & 2 & 1 \\
0 & 0 & 3 & 1 & 2 \\
3 & 3 & 0 & 0 & 0 \\
2 & 1 & 0 & 0 & 0 \\
1 & 2 & 0 & 0 & 0
\end{array}\right)
$};
\end{tikzpicture}

Sug: Inducción.

Una gráfica se llama {\bf bipartita} si se puede elegir una partición del conjunto de vértices en dos conjuntos (disjuntos) $V=V_1\cup V_2$, donde no hay aristas entre vértices de $V_1$ ni aristas entre vértices de $V_2$. Las únicas posibles aristas unen un elemento de cada conjunto de vértices.

\begin{ejercicio} 
Demuestra que una gráfica es bipartita si y solo si no tiene ciclos impares.
\end{ejercicio}

\newpage

\section{Matrices y vectores propios}

Se pueden multiplicar matrices rectangulares $A$ de $(n\times m)$ y $B$ $(p\times q)$ siempre y cuando $m=p$ (es decir, cuando el número de columnas de $A$ coincide con el número de filas de $B$). 

El resultado $AB$ es una matriz de $n\times q$ ($n$ filas y $q$ columnas).

Por ejemplo: 

$$\left(\begin{array}{ccc}
a_{11}&a_{12}&a_{13} \\
a_{21}&a_{22}&a_{23} 
\end{array}\right)
\left(\begin{array}{cccc}
b_{11}&b_{12}&b_{13}&b_{14} \\
b_{21}&b_{22}&b_{23}&b_{24} \\
b_{31}&b_{32}&b_{33}&b_{34}
\end{array}\right)$$ 
$$=
\left(\begin{array}{cccc}
a_{11}b_{11}+a_{12}b_{21}+a_{13}b_{31} & a_{11}b_{12}+a_{12}b_{22}+a_{13}b_{32} & a_{11}b_{13}+a_{12}b_{23}+a_{13}b_{33} & a_{11}b_{14}+a_{12}b_{24}+a_{13}b_{34} \\
a_{21}b_{11}+a_{22}b_{21}+a_{23}b_{31} & a_{21}b_{12}+a_{22}b_{22}+a_{23}b_{32} & a_{21}b_{13}+a_{22}b_{23}+a_{23}b_{33} & a_{21}b_{14}+a_{22}b_{24}+a_{23}b_{34}
\end{array}\right)$$
En esta sesión nos interesa principalmente el comportamiento de las potencias de una matriz, por lo que esto solamente tiene sentido con matrices cuadradas.

Sin embargo, un caso particular que sí que nos interesa es cuando $A$ es de $n\times n$ y $B$ es de $n\times 1$. En este caso decimos que $B$ es un {\bf vector}, y escribimos $$v=\left(\begin{array}{c}
    x_{1}\\
    x_{2}\\
    \vdots \\
    x_{n}
\end{array} \right)\quad \text{en lugar de } B=\left(\begin{array}{c}
    b_{11}\\
    b_{21}\\
    \vdots \\
    b_{n1}
\end{array} \right).$$

Una matriz aplicada a un vector es nuevamente un vector. La regla es la siguiente:

$$ \left(\begin{array}{cccc}
    a_{11}& a_{12} & \dots & a_{1n} \\
    a_{21}& a_{22} & \dots &  a_{2n}\\

    \vdots& \vdots &\ddots &  \\
    a_{n1}& a_{n2} & & a_{nn}
\end{array} \right)
\left(\begin{array}{c}
    x_{1}\\
    x_{2}\\
    \vdots \\
    x_{n}
\end{array} \right) 
=
\left(\begin{array}{c}
    a_{11}x_1+ a_{12}x_2 + \dots + a_{1n}x_n\\
    a_{21}x_1+ a_{22}x_2 + \dots + a_{2n}x_n\\
    \vdots \\
    a_{n1}x_1+ a_{n2}x_2 + \dots + a_{nn}x_n
\end{array} \right)
$$

Un vector se puede estirar/encoger multiplicando todas las entradas por un mismo factor, por ejemplo. 
$$
\lambda
\left(\begin{array}{c}
x \\
y
\end{array}\right)
=
\left(\begin{array}{c}
\lambda x \\
\lambda y
\end{array}\right),
\quad 
\lambda
\left(\begin{array}{c}
x \\
y \\
z
\end{array}\right)
=
\left(\begin{array}{c}
\lambda x \\
\lambda y \\
\lambda z
\end{array}\right)
$$

\begin{ejercicio}
Las matrices son {\bf transformaciones lineales}: Demuestra que $A(\alpha v+\beta w)=\alpha A(v)+\beta A(w)$. 
\end{ejercicio}

\begin{ejercicio}
Muestra que las potencias de la matriz de Fibonacci aplicadas al vector columna $(x,y)$ arrojan los números de Fibonacci con condiciones iniciales $\hat F_0=x$, $\hat F_1=y$, $\hat F_{n+1}=\hat F_n+\hat F_{n-1}$:
$$
\left(\begin{array}{cc}
0&1 \\
1&1
\end{array}\right)^n
\left(\begin{array}{c}
x \\
y
\end{array}\right)
=
\left(\begin{array}{c}
\hat F_n \\
\hat F_{n+1}
\end{array}\right)
$$
\end{ejercicio}

A veces una matriz solo estira o contrae algunos vectores. A este tipo de vectores se les llama {\bf vectores propios} o {\bf eigenvectores} (de la matriz):  $$Av=\lambda v$$. Al factor $\lambda$ (que puede ser negativo) con el que una matriz $A$ estira a uno de sus vector propios se le denomina el {\bf valor propio}  $\lambda$ o {\bf eigenvalor} (del vector propio/eigenvector $v$).

\begin{ejercicio}
Sea $A$ una matriz y $v$ un vector propio de $A$ con valor propio $\lambda$. Muestra que para cualquier número real $\alpha\neq 0$, el vector $\alpha v$ es nuevamente un vector propio con el mismo valor propio.   
\end{ejercicio}

El siguiente ejercicio es un botón de muestra de la utilidad de los vectores propios.

\begin{ejercicio}
Sea $A$ una matriz y $v$ un vector. 

Supón que $v_1$ y $v_2$ son vectores tales que $$v=\alpha v_1+\beta v_2,\quad Av_1=\lambda_1 v_1,\quad Av=\lambda_2 v_2.$$ Calcula $A^n(v)$. 
\end{ejercicio}


Nuestro objetivo es encontrar vectores propios para la matriz de Fibonacci. Pero antes revisemos qué sucede con matrices más sencillas.

\begin{ejercicio} Matrices diagonales
$$
\left(\begin{array}{cc}
\lambda_1&0 \\
0&\lambda_2
\end{array}\right)^n
\left(\begin{array}{c}
x \\
0
\end{array}\right),
\quad
\left(\begin{array}{cc}
\lambda_1&0 \\
0&\lambda_2
\end{array}\right)^n
\left(\begin{array}{c}
0 \\
y
\end{array}\right),
\quad
\left(\begin{array}{cc}
\lambda_1&0 \\
0&\lambda_2
\end{array}\right)^n
\left(\begin{array}{c}
x \\
y
\end{array}\right)
$$

$$\left(\begin{array}{ccc}
\lambda_1&0&0 \\
0&\lambda_2&0 \\
0&0&\lambda_3
\end{array}\right)
\left(\begin{array}{c}
x \\
0 \\
0
\end{array}\right),
\quad 
\left(\begin{array}{ccc}
\lambda_1&0&0 \\
0&\lambda_2&0 \\
0&0&\lambda_3
\end{array}\right)
\left(\begin{array}{c}
x \\
y \\
z
\end{array}\right)
$$

\end{ejercicio}

\begin{ejercicio}
Matrices de Permutación.
$$
\left(\begin{array}{cc}
0&1 \\
1&0 
\end{array}\right)
\left(\begin{array}{c}
x \\
y
\end{array}\right) 
,\quad
\left(\begin{array}{cc}
0&\pi \\
\pi&0 
\end{array}\right)
\left(\begin{array}{c}
1 \\
1
\end{array}\right)
,\quad
\left(\begin{array}{cc}
0&2 \\
2&0 
\end{array}\right)
\left(\begin{array}{c}
1 \\
-1
\end{array}\right) 
$$

$$\left(\begin{array}{ccc}
0&1&0 \\
1&0&0 \\
0&0&1
\end{array}\right)
\left(\begin{array}{c}
x \\
y \\
z
\end{array}\right),
\quad
\left(\begin{array}{ccc}
0&1&0 \\
1&0&0 \\
0&0&1
\end{array}\right)
\left(\begin{array}{c}
0 \\
0 \\
1
\end{array}\right),
$$

$$
\left(\begin{array}{ccc}
0&1&0 \\
0&0&1 \\
1&0&0  
\end{array}\right)
\left(\begin{array}{c}
x \\
y \\
z
\end{array}\right)
,\quad
\left(\begin{array}{ccc}
0&1&0 \\
0&0&1 \\
1&0&0  
\end{array}\right)
\left(\begin{array}{c}
1 \\
1 \\
1
\end{array}\right)
,\quad
\left(\begin{array}{ccc}
0&1&0 \\
0&0&1 \\
1&0&0  
\end{array}\right)
\left(\begin{array}{c}
\omega \\
\omega^2 \\
\omega^3
\end{array}\right),
$$
donde $\omega=-\frac{1}{2}+\frac{\sqrt{3}}{2}\mathrm{i}$ es una raíz cúbica de $1$.
$$
\left(\begin{array}{ccccc}
0 & 1 & 0 & 0 & 0 \\
0 & 0 & 1 & 0 & 0 \\
0 & 0 & 0 & 1 & 0 \\
0 & 0 & 0 &  0 & 1 \\
1 & 0 & 0 & 0 & 0
\end{array}\right)
\left(\begin{array}{c}
x_1 \\
x_2 \\
x_3 \\
x_4 \\
x_5 
\end{array}\right)
$$

$$\left(\begin{array}{ccccc}
0 & 0 & 0 & 0 & 1 \\
0 & 0 & 1 & 0 & 0 \\
0 & 1 & 0 & 0 & 0 \\
1 & 0 & 0 & 0 & 0 \\
0 & 0 & 0 & 1 & 0
\end{array}\right)
\left(\begin{array}{c}
x_1 \\
x_2 \\
x_3 \\
x_4 \\
x_5 
\end{array}\right)
$$
\end{ejercicio}

Para encontrar los distintos valores y vectores propios de una matriz real simétrica $A$ de $n\times n$ se tienen que calcular las raíces del polinomio característico de la matriz: 
$$\mathrm{det}(A-xI_n)$$

Más adelante revisaremos la definición general del determinante y el polinomio característico.
\newpage

Por lo pronto utilizaremos que en el caso de matrices simétricas de $(2\times 2)$, tanto el determinante como el polinomio característico son muy simples: $$\mathrm{det}
\left(\begin{array}{cc}
a&b \\
c&d 
\end{array}\right)=ad-bc,
\quad
\mathrm{det}
\left(\begin{array}{cc}
p-x&q \\
q&r-x 
\end{array}\right)
=
(p-x)(r-x)-q^2
=x^2-(p+r)x-q^2+pr.
$$

\begin{ejercicio}
Muestra que los valores propios de una matriz simétrica 
$$
\left(\begin{array}{cc}
p&q \\
q&r 
\end{array}\right)
$$
estan dados por la fórmula $$\frac{1}{2}((r+p)\pm \sqrt{(r-p)^2+4q^2})$$
\end{ejercicio}

\begin{ejercicio}
Calcula los valores propios de la siguiente matriz y un vector propio para cada valor propio.
$$
\left(\begin{array}{cc}
-1&4 \\
4&5 
\end{array}\right)
$$
\end{ejercicio}
\vspace{2cm}

\begin{ejercicio}
Usando el ejercicio anterior, calcula
$$
\left(\begin{array}{cc}
-1&4 \\
4&5 
\end{array}\right)^n
\left(\begin{array}{c}
7 \\
1 
\end{array}\right)
$$
\end{ejercicio}

Finalmente terminaremos con nuestro objetivo:

\begin{ejercicio}
Encuentra un vector propio con valor propio $\phi_+=\frac{1+\sqrt{5}}{2}$ y un vector propio con valor propio $\phi_-=\frac{1-\sqrt{5}}{2}$ Recuerda que $\phi_+^2=1+\phi_+$ y $\phi_-^2=1+\phi_-$ y que $\phi_+\phi_-=-1$.

$$
\left(\begin{array}{cc}
0&1 \\
1&1
\end{array}\right)^n
\left(\begin{array}{c}
x \\
y
\end{array}\right)
= \phi_+
\left(\begin{array}{c}
x \\
y
\end{array}\right)
,\quad
\left(\begin{array}{cc}
0&1 \\
1&1
\end{array}\right)^n
\left(\begin{array}{c}
x \\
y
\end{array}\right)
= \phi_-
\left(\begin{array}{c}
x \\
y
\end{array}\right)
$$
\end{ejercicio}

\begin{ejercicio}
Expresa el vector inicial $(0,1)$ como combinación lineal de los vectores propios del ejercicio anterior para mostrar la fórmula del Teorema \ref{Teo:Fibo}: 
\begin{equation}
F_n=\frac{(\phi_+)^n-(\phi_-)^n}{\sqrt{5}}
\end{equation}
\end{ejercicio}

\newpage

\section{Productos interiores y magnitudes}

La {\bf transpuesta $A^t$} y la {\bf adjunta} $A^*$ de una matriz $A$ de $(n\times m)$ son matrices de $(m\times n)$ que se definen como sigue:
$$A=
\left(\begin{array}{cccc}
    a_{11}& a_{12} & \dots & a_{1m} \\
    a_{21}& a_{22} & \dots &  a_{2m}\\
    \vdots& \vdots &\ddots &  \\
    a_{n1}& a_{n2} & & a_{nm}
\end{array} \right),
\quad
A^t:=
\left(\begin{array}{cccc}
    a_{11}& a_{21} & \dots & a_{n1} \\
    a_{12}& a_{22} & \dots &  a_{n2}\\
    \vdots& \vdots &\ddots &  \\
    a_{1m}& a_{2m} & & a_{nm}
\end{array} \right),
\quad 
A^*:=
\left(\begin{array}{cccc}
    \bar a_{11}& \bar a_{21} & \dots & \bar a_{n1} \\
    \bar a_{12}& \bar a_{22} & \dots &  \bar a_{n2}\\
    \vdots& \vdots &\ddots &  \\
    \bar a_{1n}& \bar a_{2n} & & \bar a_{nn}
\end{array} \right)
$$

Te recordamos que $\bar z=a-b\mathrm{i}$ denota el conjugado de un número complejo $z=a+b\mathrm{i}$. La matriz adjunta es una especie de generalización del conjugado para matrices.

\begin{ejercicio}
Muestra que $(A^t)^t=A=(A^*)^*$
\end{ejercicio}
\begin{ejercicio}
Muestra que $(AA^*)$ y $(A^*A)$ es una matriz real simétrica (¿de qué dimensión?). 
\end{ejercicio}

Cuando multiplicamos un complejo $z=a+b\mathrm{i}$ por su adjunto, obtenemos su norma al cuadrado: $$z\bar z=a^2+b^2,$$ que no solo es un número real, sino positivo. De manera análoga, la matriz $(AA^*)$ no solo es real simétrica, sino positiva: Una {\bf matriz positiva} es aquella todos sus valores propios son números reales positivos. Esto no lo discutiremos aquí.

Si las entradas son reales $A^*=A^t$ es simplemente la matriz transpuesta.

En particular, la matriz transpuesta/adjunta de un vector columna son {\bf vectores fila}: 
$$
v=
\left(\begin{array}{c}
    x_{1}\\
    x_{2}\\
    \vdots \\
    x_{n}
\end{array} \right)
,\quad
v^t=(x_{1}, x_{2}, \dots, x_{n})
,\quad
v^*=(\bar x_{1},\bar x_{2}, \dots, \bar x_{n})
$$

Consideremos ahora dos vectores columna de la misma dimensión:
$$
v=
\left(\begin{array}{c}
    x_{1}\\
    x_{2}\\
    \vdots \\
    x_{n}
\end{array} \right)
,\quad
w=
\left(\begin{array}{c}
    y_{1}\\
    y_{2}\\
    \vdots \\
    y_{n}
\end{array} \right)
$$
Como matrices rectangulares con dimensiones compatibles, se pueden considerar las multiplicaciones de vectores con vectores adjuntos en cualquier orden: $v^*w, w^*v, vw^*,wv^*$. Si se multiplica un vector renglón con un vector columna, se obtiene una matriz de $1\times 1$, es decir, un número (real o complejo, según sea el caso):
$$v^*w=x_{1}\bar y_{1}+x_{2}\bar y_{2}+\dots +x_{n}\bar y_{n}$$

$$w^*v=y_{1}\bar x_{1}+y_{2}\bar x_{2}+\dots +y_{n}\bar x_{n},$$

\begin{ejercicio}
Sean $z$ y $w$ números complejos, muestra que $\overline {z+w}=\bar z+ \bar w$
\end{ejercicio}

\begin{ejercicio}
Sean $z$ y $w$ números complejos, muestra que $\overline {z+w}=\bar z+ \bar w$
\end{ejercicio}

\begin{ejercicio}
Sean $z$ y $w$ números complejos, entonces $z\bar w=\overline{\bar z w}$
\end{ejercicio}

\begin{ejercicio}
Muestra que $v^*w$ es el conjugado de $w^*v$.
\end{ejercicio}

A $v^*w$ se le denomina el {\bf producto interior} de los vectores $v,w$ y a veces se le denota por $\langle v,w\rangle:=v^*w$. Para el caso $v=w$ (ya sean vectores reales o complejos), obtenemos la magnitud del vector al cuadrado:
$$\langle v,v\rangle=z_1\bar z_1+z_2\bar z_2+ \dots +z_n\bar z_n, \quad \langle v,v\rangle=x_1^2+x_2^2+ \dots+ x_n^2, $$

\begin{definicion}
Si $\langle v,w\rangle=0$ decimos que los vectores $v$ y $w$ son {\bf ortogonales}.

Si $\langle v,v\rangle=1$ decimos que el vector $v$ es {\bf unitario}.
\end{definicion}

Si multiplicamos los vectores en el orden inverso $vw^*, vw^*$, obtenemos matrices de $n\times n$. Estas son matrices de rango $1$ y aparecerán al final cuando diagonalicemos una matriz.

\section{Volúmenes y Determinantes}

Una matriz de $n\times n$ se puede pensar como un arreglo de $n$ vectores columna cada uno de dimensión $n$. Equivalentemente se puede pensar como un arreglo de $n$ vectores renglón.

\begin{definicion}
El determinante de una matriz de $2\times 2$ se define como
\vspace{2cm}

El determinante de una matriz de $3\times 3$ se define como
\vspace{3cm}
\end{definicion}

Interpretación: Volúmen del sólido determinado por los vectores de la matriz es igual al valor absoluto del determinante.
%GeoGebra: bmxfaxvq (2x2)
%GeoGebra: jjfxxtsq (3x3)

\begin{definicion}
El determinante de una matriz de $n\times n$ se define como
\vspace{4cm}
\end{definicion}

Propiedades del determinante
\vspace{4cm}

\newpage

\begin{definicion}
Polinomio característico.
\end{definicion}

El polinomio característico es la herramienta principal para encontrar los valores propios de una matriz.


\section{Lectura: Diagonalización de matrices normales}

Una matriz se llama {\bf normal}, si conmuta con su matriz adjunta:

Ejemplos de matrices normales.
\begin{enumerate}
    \item Matrices unitarias ($UU^*=I_n=U^*U$). 
    \item Entre ellas se encuentran las matrices de permutación.
    \item Matrices autoadjuntas. Son las matrices que cumplen $A=A^*$.
    \item Entre ellas las matrices simétricas reales.
\end{enumerate}

En general una matriz al azar no es normal.

Matrices normales y su descomposición singular.

\begin{teorema}[Descomposición singular de matrices normales]
Para toda matriz normal $A$, existen $D$ matriz diagonal y $U$ matriz unitaria tales que $U$ {\bf diagonaliza} simultáneamente a $A$ y a su adjunta $A^*$. 

Es decir: $A=UDU^*\quad A^*=UD^*U^*$, 
\end{teorema}

\begin{corolario}[Diagonalización de matrices reales simétricas]
Toda matriz real simétrica se puede escribir como $A=UDU^*$, donde $D$ es una matriz diagonal y $U$ es una matriz unitaria.
\end{corolario}

La diagonalización de una matriz simétrica es muy útil porque nos permite calcular potencias fácilmente: Si $A=UDU^*$ donde $U$ es unitaria y $D$ es diagonal, tenemos que
$$A^n=(UDU^*)^n=UDU^*UDU^*\dots UDU^*UDU^*=UD^nU^*.$$

\newpage

\section{Problemas}


% 
\begin{thebibliography}{99}

\bibitem{Ari13} O. Arizmendi. Convergence of the fourth moment and infinite divisibility 
\emph{Probab. Math. Statist.} {\bf 33} (2013), no. 2, 201--212.

\bibitem{Ari18} O. Arizmendi. $k$-divisible random variables and  
\emph{Adv. App. Math.} {\bf 93} (2018), 1--68.

\bibitem{AHLV15} O. Arizmendi, T. Hasebe, F. Lehner, C. Vargas. Relations between cumulants in non-commutative probability 
\emph{Adv. Math.} {\bf 282} (2015), 10, 56--92.

\bibitem{AV12} O. Arizmendi, C. Vargas. Products of free random variables and $k$-divisible partitions  
\emph{Electron. Commun. Probab.} {\bf 17} (2012), 11, 13 pp.

\bibitem{BS91} M. Bozejko, R. Speicher, $\psi$-simmetrized and independent white noises, \emph{Quantum probability and related topics (L. Accardi Ed.)} vol {\bf VI}, World Scientific, Singapore, 219--236, 1991.

\bibitem{BLS96} M. Bozejko, M. Leinert and R. Speicher. Convolution and limit theorems for conditionally
free random variables. \emph{Pac. J. Math.} {\bf 175}, 357--388, (1996).

\bibitem{Fra06} U. Franz. Multiplicative monotone convolutions \emph{Quantum Probability, M. Bozejko, W. Mlotkowski and J. Wysoczanski (eds.), Banach Center Publications} {\bf 73}, 153--166, (2006).

\bibitem{HS11} T. Hasebe, H.Saigo. The monotone cumulants. \emph{Ann. Inst. H. Poincaré Prob. Statist.} Vol {\bf 47}, No 4 (2011), 1160--1170.

\bibitem{Has11} T. Hasebe. Conditionally monotone independence I: Independence, additive convolutions and related convolutions. \emph{Infin. Dimens. Anal. Quantum. Probab. Relat. Top.} {\bf 14}, no. 3, 465--516, (2011).

\bibitem{Has13} T. Hasebe. Conditionally monotone independence II: Multiplicative convolutions and infinite divisibility. \emph{Complex Anal. Op. Th.} {\bf 7}, no. 1, 115--134, (2013).

\bibitem{Hud73} R. L. Hudson. A quantum mechanical central limit theorem for anticommuting observables.
\emph{J. Appl. Prob.} {\bf 10}, 502--509, (1973).

\bibitem{Kre72} G. Kreweras, Sur les partitions non-croisées d’un cycle, \emph{Discrete Mathematics} {\bf 1} (1972), 333--350.

\bibitem{Mal11} C. Male. Traffic distributions and independence: permutation invariant random matrices and the three notions of independence. Preprint, (2011).

\bibitem{MP67} V. Mar\v cenko and L. Pastur. Distribution of eigenvalues for some
sets of random matrices, \emph{Math. USSR-Sbornik} {\bf 1}, 457--483, (1967).

\bibitem{Mlo02} W. Mlotkowski, Operator-valued version of conditionally free product, \emph{Studia Mathematica} {\bf 153} no.1, 2012.
% Segundo funcional es lineal.

\bibitem{NS06} A. Nica, R. Speicher, \emph{Lectures on the combinatorics of free probability}, London Mathematical Society Lecture Note Series, vol. {\bf 335}, Cambridge University Press, Cambridge, 2006.

\bibitem{Spe94} R. Speicher. Multiplicative functions on the lattice of noncrossing partitions
and free convolution, \emph{Math. Ann.} {\bf 298}, no. 4, 611--628, (1994).

\bibitem{PVW15} M. Popa, V. Vinnikov, J.-C. Wang, On the multiplication of operator-valued C-free random variables, \emph{Colloquium Mathematicum} {\bf 153}(2), 2015.
% S.transform twisted multiplicativity, etc

\bibitem{PW11} M. Popa, J.-C. Wang, On multiplicative conditionally free convolution, \emph{Trans. Amer. Math. Soc. 363 (2011)} {\bf 363} no. 12, 6309--6335 2011.
% caso escalar

\bibitem{Pop08} M. Popa, Multilinear function series in conditionally free probability with amalgamation, \emph{Communications on Stochastic Analysis} Vol. {\bf 2}, No. 2 (2008) 307--322.
% positividad, caso inverso al que nosotros presentamos.

\bibitem{Spe98} R. Speicher. \emph{Combinatorial theory of the free product with amalgamation
and operator-valued free probability theory}. Memoirs of the American Math.
Society, vol. {\bf 132}, (1998).

\bibitem{Voi85} D. Voiculescu. Symmetries of some reduced free product $C^*$-algebras,\emph{Operator
algebras and their connections with topology and ergodic theory (Busteni,
1983), Lecture Notes in Math.}, vol. {\bf 1132}, Springer, Berlin, 556--588, (1985).

\bibitem{Voi91} D. Voiculescu. Limit laws for random matrices and free products, \emph{Invent. Math.} {\bf 104}, 201--220, (1991).

\bibitem{VDN92} D. Voiculescu, K. Dykema and A. Nica. \emph{Free Random Variables}, CRM Monograph Series, Vol. {\bf 1}, AMS, (1992).

\bibitem{Voi95} D. Voiculescu. Operations on certain non-commutative operator-valued random
variables. \emph{Recent advances in operator algebras (Orleans, 1992). Asterisque} {\bf 232}, (1995).

\bibitem{Wig58} E. Wigner. On the distribution of the roots of certain symmetric matrices, \emph{Ann. of Math.} {\bf 67}, 325--327, (1958).

\end{thebibliography}

hg
agregar Ben-G schurman y Belinschi -N- S 
SW Booolean
Bozeko solo 96?


\chapter{Elementary counting problems}

\section{Sums and sequences}

Let us call $I_n$ the sum of the first $n$ odd numbers. For example:
$$I_1=1, \quad I_2=1+3=4, \quad I_3=1+3+5=9.$$%
\begin{exercise}\label{Prob:SumImp}
Compute these sums of odd numbers:
  \begin{itemize}  
  \item $I_5=1+3+5+7+9$
  \item $I_{50}=1+3+5+7+\dots+99$
  \item $I_{500}=1+3+5+7+\dots+999$
  \item What is the value of $I_{1000}$ (the sum of the first thousand odd numbers)?
  \item What is the value of $I_n$ (the sum of the first $n$ odd numbers?)
\end{itemize}
\hint{Find a pattern. Draw a square}
\end{exercise}
\tutpagebreak

\begin{exercise}\label{Prob:SumGau}
Compute the following sums of \emph{consecutive} numbers:
\begin{itemize} 
    \item $1+2+3+4+5+\dots+99+100$.
    \item $1+2+3+4+5+\dots+999+1000$.
    \item $1+2+3+4+5+\dots+(n-2)+(n-1)+n$.    
\end{itemize}
\hint{Write the sum again in a new line, in reverse order, to compute \emph{twice the required sum}}
\end{exercise}

\begin{exercise}
Compute these sums of numbers in \emph{arithmetic progression}:
\begin{itemize} 
    \item $2+4+6+\cdots +2k$.
    \item $7+14+21+\cdots +7k$.
    \item $-16-9-2+5+12+19+26+\cdots +145$.
    \item $\frac{-21}{7} + \frac{-17}{7} + \frac{-13}{7}+ \frac{-9}{7} + \cdots + \frac{11}{7} + \frac{15}{7}$.  
\end{itemize}
\end{exercise}
\tutpagebreak

\begin{exercise}\label{ex:powers-of-two}
What is the value of these sums of numbers in \emph{geometric progression}?:
\begin{itemize}
    \item $1+2+4$
    \item $1+2+4+8+16+32+64$
    \item $1+2+4+8+16+\cdots+1024$
\end{itemize}
\hint{Find a pattern. Try using binary numbers}
\end{exercise}

\begin{exercise}
What about these fractions (also in \emph{geometric progression})?:
\begin{itemize}
    \item $\frac{1}{2}+\frac{1}{4}+\frac{1}{8}$
    \item $\frac{1}{2}+\frac{1}{4}+\frac{1}{8}+\frac{1}{16}+\frac{1}{32}$
    \item $\frac{1}{2}+\frac{1}{4}+\frac{1}{8}+\dots +\left(\frac{1}{2}\right)^{15}$    
    \end{itemize}
\hint{Can you relate these sums to the ones in the previous exercise?}
\end{exercise}
\tutpagebreak

% \begin{exercise}
% Simplifica las las siguientes sumas de fracciones:
% \begin{itemize} 
%     \item $\frac{1}{2}+\frac{1}{4}+\frac{1}{8}+\dots +\left(\frac{1}{2^{10}}\right)$
%     \item $\frac{2}{3}+\frac{2}{9}+\frac{2}{27}+...+\left(\frac{2}{3^{10}}\right)$
%     \item $\frac{3}{4}+\frac{3}{16}+\frac{3}{64}+...+\left(\frac{3}{4^{10}}\right)$
%     \item $\frac{4}{5}+\frac{4}{25}+\frac{4}{125}+...+\left(\frac{4}{5^{10}}\right)$
%     \end{itemize}
% \end{exercise}

\begin{exercise}
Compute these sums of numbers in \emph{geometric progression}:
\begin{itemize}
    \item $1+2+4+8+16+\dots +2^{10}$.
    \item $1+3+9+27+81+\dots +3^{10}$.
    \item $1+k+k^2+k^3+\dots +k^{10}$.
    \end{itemize}
\hint{Write the sums in binary/ternary/$k$-nary numbers. Multiply by $k-1$}
\end{exercise}

\begin{exercise}
What is the value of the following sums of fractions in \emph{geometric progression}?:
\begin{itemize} 
    \item $1+\frac{1}{2}+\frac{1}{4}+\frac{1}{8}+\dots +\left(\frac{1}{2}\right)^{10}$.
    \item $1+\frac{1}{3}+\frac{1}{9}+\frac{1}{27}+\dots+\left(\frac{1}{3}\right)^{10}$
    \item $1+\frac{1}{k}+\frac{1}{k^2}+\frac{1}{k^3}+\dots+\left(\frac{1}{k}\right)^{10}$
    \end{itemize}
\end{exercise}
\hint{Can you relate these sums to the ones in the previous exercise?}
\tutpagebreak

Consider the sequence of numbers: $$c_n:=\frac{1}{6}n(n+1)(2n+1), \quad n=1,2,3,\dots $$

For example, to obtain $c_5$, we replace $n$ by $5$ in the expression and we get 
$$c_5 = \frac{1}{6}5(5+1)(2\cdot5+1) = \frac{1}{6}5(6)(11) = 55.$$ 

\begin{exercise} 
Compute the values of $c_1,c_2,c_3$, and $c_4$.
\end{exercise}

\begin{exercise} 
For some reason $c_n$ is always an integer. 
This means that the expression $n(n+1)(2n+1)$ is always divisible by $6$. 
Can you explain why this is the case?
\hint{Separate by prime numbers: show that $n(n+1)(2n+1)$ is divisible by $2$ and by $3$. 
To show that it is divisible by $3$, 
separate in three cases: $n= 3k$, $n= 3k+1$ and $n= 3k+2$} 
\end{exercise}

\begin{exercise} 
What is the value of the difference between two consecutive terms in the sequence? 
More concretely, what is the value of $c_k-c_{k-1}$?
\end{exercise}
$$c_k-c_{k-1} = \frac{1}{6}k(k+1)(2k+1)-\frac{1}{6}(k-1)(k)(2k-1) = $$
\tutpagebreak

In the last exercise we obtained that the difference $c_k-c_{k-1} = k^2$ for all $k=2,3,4,\dots$. 
Equivalently, we obtain that $c_{k}+(k+1)^2=c_{k+1}$ for all $k=1,2,3,4,5,\dots$. 
With this result we can obtain a sophisticated formula:
\begin{exercise}
Show that the formula for the sum of the first $n$ squares is given by: 
$$1^2+2^2+3^2+\cdots+n^2=\frac{1}{6}n(n+1)(2n+1).$$ 
Test it for $n=1,2,3,4$.

Use the previous exercise anterior to conclude that the formula is valid for \emph{all} $n=1,2,3, \dots$.
\end{exercise}
\tutpagebreak

The formula for the sums of cubes is a bit simpler than the sum of squares.
\begin{exercise}
\begin{enumerate}
    \item Compute the sum: $$1^3+2^3+3^3+\cdots+10^3.$$
    \item Compute the sum: $$1^3+2^3+3^3+\cdots+100^3.$$ Hint: Test for smaller cases and find a pattern.
    \item Make a conjecture for the formula to compute $c_n = 1^3+2^3+3^3+\cdots+n^3$. 
    \item What is the difference $c_k-c_{k-1}$ between two consecutive elements from your formula?
    \item Conclude that your formula is valid for all $n\geq 1$.
\end{enumerate}
\end{exercise}
\tutpagebreak

\begin{exercise}
Encuentra una fórmula para calcular la suma $$1\cdot2+2\cdot3+3\cdot4+\cdots+n\cdot(n+1).$$
\hint{Can you decompose this sum into two sums that we have already solved?}
\end{exercise}
\tutpagebreak

\begin{exercise}
  Consider $n$ points in the circle and draw the $n(n-1)/2$ chords defined by each pair of points. 
  Suppose that the points are arranged in such a way that no three chords intersect in a single point, 
  as shown in the figures.

\begin{center}
  \begin{tikzpicture}[line cap=round,line join=round,>=triangle 45,x=1.5cm,y=1.5cm]
    \clip(-1.1,-1.1) rectangle (1.1,1.1);
    \draw [line width=1.pt] (0.,0.) circle (1.5cm);
    \draw [line width=1.pt] (-0.6842733454009037,0.7292256089673864)-- (0.6995716435123726,-0.7145624644447803);
    \draw [line width=1.pt] (-0.3147519836675201,-0.9491739507473649)-- (0.09504515717703704,0.9954729620121243);
    \draw [line width=1.pt] (1.,0.)-- (-0.9052766399383212,-0.4248225572894913);
    \draw [fill=black] (1.,0.) circle (1.5pt);
    \draw [fill=black] (0.09504515717703704,0.9954729620121243) circle (1.5pt);
    \draw [fill=black] (-0.6842733454009037,0.7292256089673864) circle (1.5pt);
    \draw [fill=black] (-0.9052766399383212,-0.4248225572894913) circle (1.5pt);
    \draw [fill=black] (-0.3147519836675201,-0.9491739507473649) circle (1.5pt);
    \draw [fill=black] (0.6995716435123726,-0.7145624644447803) circle (1.5pt);
    \end{tikzpicture}
    \hspace{1cm}
    \begin{tikzpicture}[line cap=round,line join=round,>=triangle 45,x=1.5cm,y=1.5cm]
      \clip(-1.1,-1.1) rectangle (1.1,1.1);
      \draw [line width=1.pt] (0.,0.) circle (1.5cm);
      \draw [line width=1.pt] (-0.6842733454009037,0.7292256089673864)-- (0.6995716435123726,-0.7145624644447803);
      \draw [line width=1.pt] (-0.3147519836675201,-0.9491739507473649)-- (0.7421026427511382,0.67028625796877);
      \draw [line width=1.pt] (1.,0.)-- (-0.9052766399383212,-0.4248225572894913);
      \draw [fill=black] (1.,0.) circle (1.5pt);
      \draw [fill=black] (0.7421026427511382,0.67028625796877) circle (1.5pt);
      \draw [fill=black] (-0.6842733454009037,0.7292256089673864) circle (1.5pt);
      \draw [fill=black] (-0.9052766399383212,-0.4248225572894913) circle (1.5pt);
      \draw [fill=black] (-0.3147519836675201,-0.9491739507473649) circle (1.5pt);
      \draw [fill=black] (0.6995716435123726,-0.7145624644447803) circle (1.5pt);
      \end{tikzpicture}

  Valid configuration (left) vs invalid configuration (right).    

\end{center} 

In how many regions is the disk split by the chords? 
  
Compute the cases for $n=1,2,3,4,5$ and formulate a conjecture.
\end{exercise}
  
\begin{exercise}
  Now compute the case $n=6$.
\end{exercise}
\tutpagebreak
  
\begin{exercise}
  Consider the sequence of numbers $$r_n = {n \choose 0} + {n \choose 2} + {n \choose 4}$$
  Recall that the \emph{binomial coefficient} is defined as:
  \begin{itemize}
    \item   ${n \choose k} = \frac{n!}{k! (n-k)!}$,
    where $n!=n(n-1)(n-2) \cdots (2)(1)$ is \emph{$n$ factorial} if $k\leq n$,
    \item ${n \choose k} =0$, if $k>n$.
  \end{itemize}

  Do you wish to change your conjecture?
\end{exercise}
\tutpagebreak

\begin{exercise}
Compute the sum $$\frac{1}{1 \cdot 2}+\frac{1}{2 \cdot 3}+\frac{1}{3 \cdot 4}+\cdots+\frac{1}{999 \cdot 1000}.$$
\hint{Simplify your result and find a pattern.}
\end{exercise}

\begin{exercise}
Calcula la suma $$\frac{1}{1 \cdot 3}+\frac{1}{3 \cdot 5}+\frac{1}{5 \cdot 7}+\cdots+\frac{1}{999 \cdot 1001}.$$
\hint{Simplify your result and find a pattern. Factorize.}
\end{exercise}
\vspace{1.5cm}
\tutpagebreak

The famous \emph{Fibonacci numbers} are defined as follows:
$F_0=0$, $F_1=1$, and for all $n\geq 2$, $$F_n=F_{n-1}+F_{n-2}.$$

\begin{exercise}
Compute the first 11 Fibonacci numbers $(F_0,F_1,F_2,\dots,F_{10})$.
\end{exercise}

\begin{exercise}
Show that $$1+F_0+F_1+F_2+\dots +F_n=F_{n+2}.$$
\hint{Simple induction argument.}
\end{exercise}

\begin{exercise}
Show that for all $n\geq 0$, we have $$F_0^2+F_1^2+F_2^2+\dots +F_n^2=F_{n}F_{n+1}.$$
\hint{Simple induction argument.}
\end{exercise}
\tutpagebreak

\begin{exercise}
Consider now the odd Fibonacci numbers $$(J_1,J_2,J_3,J_4,\dots)=(F_1,F_3,F_5,F_7,\dots)=(1,2,5,13,\dots).$$
Show that $$J_1+J_2+J_3+\cdots + J_{n-2}+J_{n-1}+ 2\cdot J_{n}=J_{n+1}.$$
\hint{Simple induction argument.}
\end{exercise}

\begin{exercise}\begin{enumerate}
    \item Compute the sum: $$4\cdot 1+3\cdot 2+2\cdot 3 + 1\cdot4.$$
    \item Compute the sum: $$6\cdot 1+5\cdot 2+4\cdot 3 + 3\cdot4+ 2 \cdot 5 + 1 \cdot 6.$$
    \item Compute the sum: $$1000\cdot 1+999\cdot 2+998\cdot 3+ \cdots + 998\cdot3+ 999 \cdot 2 + 1000 \cdot 1$$
\end{enumerate}
\hint{Compute initial cases. Draw a Pascal-Tartaglia triangle. Find the computed numbers inside the triangle.}
\end{exercise}

\section{Principios Fundamentales de Conteo}

Los siguientes ejercicios tienen que ver con los {\bf principios fundamentales de conteo}: 
\begin{itemize}
    \item Si debo elegir un elemento de un conjunto $X$ {\bf o} un elemento de un conjunto $Y$, las posibilidades se suman ({\bf principio  aditivo}).
    \item Si debo elegir un elemento de un conjunto $X$ {\bf y} un elemento del conjunto $Y$, esto es lo mismo que elegir una pareja $(x,y)$, $x\in X$ $y\in Y$, por lo que las posibilidades se multiplican ({\bf principio multiplicativo}).
\end{itemize}

\begin{ejercicio}
De la ciudad A hay tres caminos a la ciudad B y cuatro caminos a la ciudad C. De la ciudad C a la D hay dos caminos, y de la ciudad B a la D hay dos caminos. ¿De cuantas formas se puede viajar de A a D?.
\end{ejercicio}
\vspace{4cm}

\begin{ejercicio}
Juan tiene tres sombreros y cuatro gorras de béisbol, siete camisetas, dos shorts, tres pantalones, dos pares de tenis y tres pares de zapatos.

¿De cuántas formas se podría vestir Juan el lunes?
\end{ejercicio}
\vspace{4cm}

\section{Factoriales, coeficientes binomiales y multinomiales}

El siguiente grupito de problemas tiene que ver con los números factoriales. Al número $n!:=n(n-1)(n-2)\cdots (3)(2)(1)$ se le llama $n$ {\bf factorial}. Por ejemplo. $5!=120$ y $6!=720$. Por convención, se establece que un producto sobre un conjunto vacío de factores es $1$, por lo que $0!=1$. Esta convención es de gran ayuda para muchas identidades de conteo.

\begin{ejercicio}
¿De cuántas formas puedo ordenar a seis personas en una fila?
\end{ejercicio}
\vspace{4cm}

%Sug: $6!$.

\begin{ejercicio}
Un equipo de futbol consta de 11 jugadoras. ¿De cuántas formas se puede elegir la lista ordenada de cinco jugadoras que tiraran los penales?
\end{ejercicio}
\vspace{4cm}

Un anagrama de una palabra es otra palabra con las mismas letras pero posiblemente en otro orden. Por ejemplo, de la palabra OSA salen seis anagramas distintos. OSA, OAS, ASO, AOS, SAO y SOA.

\begin{ejercicio}
¿De cuántas formas diferentes se pueden reordenar las letras de las siguientes palabras? 

i). OSO, ii). COSA, iii). COCO, iv). CALACA, v). CALABAZA, vi). MURCIELAGO, vii). SALCHIPAPAS.
\end{ejercicio}

%Sug: Si las letras no se repiten son $n!$. Hay que dividir entre el número de repeticiones.

Todas las respuestas del ejercicio anterior son de la forma $${n\choose k_1, k_2, k_3,\dots, k_r}:= \frac{n!}{k_1!k_2!\dots k_r!},\quad  p_1+p_2+\cdots +p_r = n.$$ A estos números se les llama coeficientes multinomiales. También se les conoce como permutaciones con repetición en algunos textos, y a veces se denotan como  $P _{k_1,k_2,\dots, k_r}$, o bien $P^n_{k_1,k_2,\dots, k_r}$. Sugerimos la notación con los paréntesis

Por ejemplo, para SALCHIPAPAS, $n=11$ porque son once letras en total, $k_1 = 3$ (por que hay  tres A's) $k_2=2, k_3=2,$ (porque hay dos letras P's y dos S') $1=k_4=k_5=k_6=k_7$ (porque L,C,H,I solo aparecen una vez) y obtenemos que el número de anagramas derivados de SALCHIPAPAS es $$\frac{11!}{3!2!2!1!1!1!1!}=\frac{11!}{12} = {11\choose 3, 2, 2, 1, 1, 1, 1}.$$ 


Ahora prestemos atención particular a cuando solo tenemos dos caracteres distintos. 

\begin{ejercicio}
¿Cuántos anagramas hay del fatality de Lui Knag: $(\uparrow,\uparrow,\rightarrow,\uparrow,\uparrow,\rightarrow)$?
\end{ejercicio}
\vspace{1cm}

Una {\bf biyección} es una correspondencia uno-a-uno entre dos conjuntos $X$ y $Y$ con la misma cantidad de elementos. Es decir, a cada elemento del conjunto $X$ le toca uno y solo un elemento del conjunto $Y$.

\begin{ejercicio}
¿Puedes encontrar una biyección entre un conjunto de especial de caminos en el plano y el conjunto de anagramas del fatality de Lui Knag? (Pista: considera una cuadricula de $2 \times 4$)
\end{ejercicio}
\vspace{3cm}

Cuando consideramos únicamente dos tipos de caracteres, con $k_1=k$ repeticiones de uno y $k_2= n-k$ repeticiones del otro, obtenemos los coeficientes binomiales:
$${n\choose k}=\frac{n!}{k!(n-k)!}$$

\section{Contando subconjuntos}

Ahora revisaremos algunos problemas básicos sobre contar conjuntos y subconjuntos. Para esto primero repasemos las definiciones y la notación básica.

\subsection{Definiciones de conjuntos, subconjuntos y ejemplos básicos}

Un {\bf conjunto} $X$ es una colección de elementos. Para describir a un conjunto ponemos a sus elementos dentro de los signos de llave.

Por ejemplo $$X:=\{1,2,3 \}$$ es el conjunto que contiene a los primeros tres números naturales. Los elementos del conjunto $X$ son $1$, $2$ y $3$.

El conjunto $$X:=\{\clubsuit,\vardiamondsuit,\spadesuit,\varheartsuit\}$$ tiene cuatro elementos, los elementos son cada uno de los símbolos de la baraja inglesa.

En un conjunto no importa el orden en que se listan sus elementos. Es decir, $$\{\clubsuit,\vardiamondsuit,\spadesuit,\varheartsuit\} =\{\spadesuit,\clubsuit,\vardiamondsuit,\varheartsuit\}$$ son el mismo conjunto.

Un {\bf subconjunto} $Y$ de otro conjunto $X$  se puede pensar intuitivamente como una subcolección de elementos de $X$. Escribimos $$Y\subseteq X.$$ En un sentido mas concreto, $Y$ es un subconjunto de $X$ si cada elemento de $Y$ está contenido en $X$.

Por ejemplo, tanto el conjunto formado por las figuras negras $Y_1 = \{\clubsuit,\spadesuit\}$ como el conjunto formado por las figuras rojas $Y_2 ={\{\vardiamondsuit,\varheartsuit\} }$ son subconjuntos del conjunto $X=\{\clubsuit,\vardiamondsuit,\spadesuit,\varheartsuit\}$.

En principio se vale considerar conjuntos infinitos. Por ejemplo, el conjunto de números enteros positivos $\mathbb N$ está contenido en el conjunto de números reales $\mathbb R$ y podemos escribir $\mathbb N\subseteq \mathbb R$. Pero no nos espantemos: esto es solo para establecer la notación y por ahora solo estaremos trabajando con conjuntos finitos!

\subsection{El tamaño de un conjunto}

Al numero de elementos de un conjunto $X$ se le llama la cardinalidad de $X$. Si un conjunto $X$ tiene $n$ elementos escribimos $$|X|=n$$

Por ejemplo $$|\{1,2,3\}| = 3,\quad\{0,1,2,3\}| = 4,\quad |\{-1,0,1\}| = 3,\quad |\{\clubsuit\}| = 1$$ 

Hay un conjunto un poco raro pero bastante fundamental que se llama el {\bf conjunto vacío} $\{\}$. Se define como el conjunto que no tiene ningún elemento, es decir $$|\{\}|=0.$$ El conjunto vacío es el único conjunto con cero elementos. Se denota muy frecuentemente con el símbolo $\emptyset$, o bien con las llaves vacías $\{\}$. 

En estas notas vamos a usar las llaves vacías porque nos parece más intuitivo a la vista, que se trata del subconjunto que obtengo cuando no elijo a ningún elemento de cualquier conjunto $X$. 

\subsection{Contando subconjuntos}

Entonces, por ejemplo

\begin{itemize}
    \item El conjunto $\{\clubsuit\}$ solo tiene dos subconjuntos: el  conjunto vacío $\{\}$  y el conjunto $\{\clubsuit\}$. 
    \item El conjunto $[2]:=\{1,2\}$ tiene $4$ subconjuntos: $\{\}$, $\{1\}$, $\{2\}$, y $\{1,2\}$.
    \item El conjunto $[3]:=\{1,2,3\}$ tiene $8$ subconjuntos: $\{\}$, $\{1\}$, $\{2\}$, $\{3\}$, $\{1,2\}$, $\{1,3\}$, $\{2,3\}$, $\{1,2,3\}$.

    \item Como podemos observar de los ejemplos anteriores, todo conjunto $X$ tiene siempre dos subconjuntos "triviales": el conjunto "total" $Y=X$ (cuando se incluyen todos los elementos de $X$) y el conjunto vacío $Y = \{\}$ (cuando no se toma ningún elemento del conjunto).

\end{itemize}


Una forma de entender a todos los subconjuntos de $\{1,2,3\}$ es como ternas de taches y O's, donde ponemos una O si incluímos al elemento en el subconjunto y un tache si no lo incluímos. 

Por ejemplo: podemos representar al subconjunto $\{1,3\}$ como la terna $OXO$, porque se eligió incluir al uno y al tres, pero no al dos. Los subconjuntos $\{3\}$, $\{1,2,3\}$ y el conjunto vacío $\emptyset$, se representan con las ternas $XXO$, $OOO$ y $XXX$, respectivamente.

\begin{ejercicio}
¿Cuántos subconjuntos tiene el conjunto $[4]:=\{1,2,3,4\}$?

¿Cómo representarías al subconjunto $\{2,4\}$ y al subconjunto $\{1,2,3\}$ con $O$'s y $X$'s?
\end{ejercicio}
\vspace{1.5cm}

\begin{ejercicio}
¿Cuántos subconjuntos tiene el conjunto $[6]:=\{1,2,3,4,5,6\}$? (No es necesario listarlos, solo calcular cuántos hay).
\end{ejercicio}
\vspace{1cm}





\begin{ejercicio}
¿Cuántos subconjuntos de tamaño $4$ tiene el conjunto $[6]:=\{1,2,3,4,5,6\}$?

¿Cuántos subconjuntos de tamaño $3$ tiene el conjunto $[6]:=\{1,2,3,4,5,6\}$?

¿Cuántos subconjuntos de tamaño $2$ tiene el conjunto $[6]:=\{1,2,3,4,5,6\}$?

¿Cuántos subconjuntos de tamaño $0$ tiene el conjunto $[6]:=\{1,2,3,4,5,6\}$?
\end{ejercicio}
\vspace{2cm}
%Sug: anagramas OOXXOO con cierta cantidad de O's.

\begin{ejercicio} Demuestra la identidad.
$${n \choose 0}+{n \choose 1}+{n \choose 2}+\cdots + {n \choose n}=2^n.$$
\end{ejercicio}
\vspace{2cm}


\begin{ejercicio}
Un equipo completo de baloncesto consta de nueve jugadoras en total. 

a) ¿De cuántas maneras se puede elegir al conjunto de cinco jugadoras titulares que 
jugarán el siguiente partido?

b) ¿De cuántas maneras se puede elegir una alineación con una capitana y 4 jugadoras titulares?
\end{ejercicio}
\vspace{2cm}

%Sug: a) anagramas de titulares y suplentes TTTTTSSSS. b) anagramas de CTTTTSSSS

% Obs: Se puede contar de dos maneras. Primero elegir la capitana y luego a las titulares, o primero a las titulares y de entre ellas a la capitana.

\section{Separadores}

\begin{ejercicio}
¿De cuántas formas se puede elegir una quíntupla ordenada $(n_1, n_2, n_3, n_4, n_5)$ de cinco enteros no-negativos que sumen $16$?
\end{ejercicio}

\begin{ejercicio}
¿De cuántas formas se puede elegir una quíntupla ordenada $(n_1, n_2, n_3, n_4, n_5)$ de cinco números naturales que sumen $16$?
\end{ejercicio}

Nota: Si en lugar de preguntar por quíntuplas ordenadas  por quíntuplas no-ordenadas , no existe una fórmula cerrada.

% hipervínculo.

\begin{ejercicio}
Si tengo veinte canicas idénticas y voy a repartirlas todas entre mis cinco sobrinos. ¿De cuántas maneras distintas puedo repartir \emph{todas} las canicas?.

¿De cuántas maneras puedo hacerlo, si no puedo dejar a ninguno con cero canicas?

¿De cuántas formas puedo repartir las canicas, si se permite que me quede con algunas canicas?

¿De cuántas formas puedo repartir las canicas, si al sobrino más grande le tengo que dar al menos tres?
\end{ejercicio}
\vspace{4cm}

\section{Permutaciones en círculos}

\begin{ejercicio}
¿De cuantas formas distintas se pueden sentar ocho personas en una mesa redonda?
\end{ejercicio}
\vspace{4cm}

\begin{ejercicio}
¿De cuántas formas se pueden pintar las seis caras de un cubo de seis colores distintos, usando todos los seis colores?

¿Y si tuviéramos ocho colores?
\end{ejercicio}


\section{Contando caminos: Coeficientes binomiales}

Según las reglas para multiplicar polinomios, el binomio $(x+y)$ elevado a la $n=1,2,3$ nos da:

$$(x+y)^0=1,\quad (x+y)^1=x+y,\quad (x+y)^2=x^2+2xy+y^2, \quad (x+y)^3=x^3+3x^2y+3xy^2+y^3.$$

%Obs: Revisar simplificación de términos semejantes.

\begin{ejercicio}
¿Cuál es el coeficiente que acompaña $x^5y^4$ al simplificar $(x+y)^9$?
\end{ejercicio}
%Sug: Comparar xxxyxyxyy con los XXXOXOXOO del problema anterior.
\vspace{2cm}

Los coeficientes de $(x+y)^n$ son los números en el $n$-ésimo renglón del {\bf triángulo de Pascal} (donde el primer renglón corresponde a $n=0$):

$$\begin{array}{ccccccccccccccc}
     &&&&&&& 1 &&&&&&&  \\
    &&&&&& 1 && 1 &&&&&& \\
   &&&&& 1 && 2 && 1 &&&&& \\
  &&&& 1 && 3 && 3 && 1 &&&& \\
 &&& 1 && 4 && 6 && 4 && 1 &&& \\
&& 1 && 5 && 10 && 10 && 5 && 1 &&\\
& 1 && 6 && 15 && 20 && 15 && 6 && 1 &\\
1 && 7 && 21 && 35 && 35 && 21 && 7 && 1\\
&& \vdots && &\vdots && && \vdots && &\vdots
\end{array}$$

Entonces, por ejemplo: $$(x+y)^6=x^6+6x^5y+15x^4y^2+20x^3y^3+15x^2 y^4+6xy^5+y^6.$$

Por esta razón los números que aparecen en el triángulo de Pascal se llaman {\bf coeficientes binomiales}, y se denotan por el símbolo ${n \choose k}$.

$$\begin{array}{ccccccccccccccc}
     &&&&&&& {0 \choose 0} &&&&&&&  \\
    &&&&&& {1 \choose 0} && {1 \choose 1} &&&&&& \\
   &&&&& {2 \choose 0} && {2 \choose 1} && {2 \choose 2} &&&&& \\
  &&&& {3 \choose 0} && {3 \choose 1} && {3 \choose 2} && {3 \choose 3} &&&& \\
 &&& {4 \choose 0} && {4 \choose 1} && {4 \choose 2} && {4 \choose 3} && {4 \choose 4} &&& \\
&& {5 \choose 0} && {5 \choose 1} && {5 \choose 2} && {5 \choose 3} && {5 \choose 4} && {5 \choose 5} &&\\
& {6 \choose 0} && {6 \choose 1} && {6 \choose 2} && {6 \choose 3} && {6 \choose 4} && {6 \choose 5} && {6 \choose 6} &\\
{7 \choose 0} && {7 \choose 1} && {7 \choose 2} && {7 \choose 3} && {7 \choose 4} && {7 \choose 5} && {7 \choose 6} && {7 \choose 7}\\
&& \vdots && &\vdots && && \vdots &&& \vdots
\end{array},$$

La fórmula para el coeficiente binomial es muy sencilla en términos de factoriales:
$${n \choose k}=\frac{n!}{k!(n-k)!}$$

\begin{ejercicio} Demuestra la recursión
$${n \choose k}={n-1 \choose k-1}+{n-1\choose k}$$
\end{ejercicio}

\begin{ejercicio} Considera una cuadrícula de $2\times k$ (como en la Figura).
¿Cuántos caminos distintos hay, que solo avancen una unidad hacia la derecha o una unidad hacia arriba, desde el punto $(0,0)$ al $(k,2)$, para $k=0,1,2,3,\dots$?
\end{ejercicio}

\begin{center}
  \begin{tikzpicture}[x=0.75cm,y=0.75cm]
    \foreach \i in {0,...,2} \draw[black!60] (0,\i) -- (7,\i);
    \foreach \i in {0,...,7} \draw[black!60] (\i,0) -- (\i,2);

    \draw (0,0) node[circle,fill,inner sep=1pt,label=below left:{$(0,0)$}] {};
    \draw (7,2) node[circle,fill,inner sep=1pt,label=above right:{$(7,2)$}] {};
    \draw[->,line width=1pt] (0,0) -- (2,0);
    \draw[->,line width=1pt] (2,0) -- (2,1);
    \draw[->,line width=1pt] (2,1) -- (5,1);
    \draw[->,line width=1pt] (5,1) -- (5,2);
    \draw[->,line width=1pt] (5,2) -- (7,2);
  \end{tikzpicture}

  $k=7$ y un camino.
\end{center}

%Sug: ¿Cuántos caminos utilizan la subida $(j,0)\uparrow (j,1)$, para cada $j=1,2,3,\dots ,k$?.

\begin{ejercicio} Si se parte del $(0,0)$ y solo se puede mover en cada paso una unidad hacia arriba o hacia la derecha,
\begin{itemize}
    \item ¿Cuántos caminos hay del punto $(0,0)$ al $(2,5)$?.
    \item ¿Cuántos caminos hay del punto $(0,0)$ al $(3,4)$?.
    \item ¿Cuántos caminos hay del punto $(0,0)$ al $(3,5)$?.
\end{itemize}
\end{ejercicio}

\begin{ejercicio}
%ref% [IWYMICI '19]
Mismas reglas: 

¿Cuántos caminos distintos hay del punto $A$ al $B$?

¿Cuántos caminos distintos hay del punto $A$ al $B$, que no pasen por el punto $C$?

\begin{center}
  \begin{tikzpicture}[x=0.75cm,y=0.75cm]
    \foreach \i in {0,...,5} \draw[black!60] (0,\i) -- (10,\i);
    \foreach \i in {0,...,10} \draw[black!60] (\i,0) -- (\i,5);

    \draw (0,0) node[circle,fill,inner sep=1pt,label=below left:{$A$}] {};
    \draw (5,3) node[circle,fill,inner sep=1pt,label=above right:{$C$}] {};
    \draw (10,5) node[circle,fill,inner sep=1pt,label=above right:{$B$}] {};
  \end{tikzpicture}
\end{center}
\end{ejercicio}

\begin{ejercicio}
Sea $n$ un número natural. Demostrar siguiente identidad para coeficientes binomiales:
$${2n\choose n}={n \choose 0}^2+{n \choose 1}^2+{n \choose 2}^2+\cdots +{n \choose n}^2.$$
\end{ejercicio}
Sug: ¡Cuenta caminos!
\vspace{4cm}

\begin{ejercicio} Demuestra que para todo $n\geq 1$ se cumple la identidad:
$${n \choose 0}-{n \choose 1}+{n \choose 2}-{n \choose 3}+\cdots +{n \choose n-1}(-1)^{n-1} +{n \choose n}(-1)^n=0.$$
\end{ejercicio}
\vspace{2cm}

\newpage

Varios de los problemas anteriores se pueden resolver directamente usando el teorema del binomio

\begin{teorema}[Teorema del binomio]

Primera versión: $x,y$ números reales o complejos. Entonces se cumple que
$$(x+y)^n={n\choose 0}x^0y^n+{n\choose 1}x^1y^{n-1}+{n\choose 2}x^2y^{n-2}+\dots +{n\choose n}x^ny^0.$$

Versión simplificada ($y=1$): Se tiene que $$(1+x)^n={n\choose 0}x^0+{n\choose 1}x^1+{n\choose 2}x^2+\dots +{n\choose n}x^n.$$
\end{teorema}

\section{Cálculo de probabilidades}

Contar casos favorables entre casos totales es la esencia de la probabilidad discreta. Debido a su aparición frecuente en problemas conteo, los coeficientes binomiales aparecen en muchos problemas elementales de probabilidad.

A veces las probabilidades se presentan en términos de porcentajes. Aunque esto es bastante común, resulta mucho más útil pensar en las probabilidades como números reales entre $0$ y $1$, donde la probabilidad igual a $1$ es equivalente al $100\%$ mientras que la probabilidad igual a $0$ es equivalente al $0\%$.

Por ejemplo, si $X$ es el resultado de lanzar una moneda no truqueada, sabemos que la probabilidad de obtener tanto cara como cruz es de $50\%$ o de $\frac{1}{2}$. De la misma manera, la probabilidad de obtener un $5$ al lanzar un dado es $\frac{1}{6}$.

Es más fácil trabajar con números reales que con porcentajes porque las probabilidades de eventos independientes se multiplican.

Supón que se lanzo una moneda y un dado. ¿Cuál es la probabilidad de obtener una cara y un $2$?

Si lanzo un dado, luego una moneda, luego un dado y luego una moneda, ¿Cuál es la probabilidad de obtener un número par, luego una cara, luego no obtener 3 y luego cruz? 
\vspace{4cm}

\begin{ejercicio} 
En una baraja inglesa de $52$ cartas (con trece cartas de cuatro figuras distintas) calcula la probabilidad de obtener: 
\begin{itemize}
    \item Pokar (cuatro cartas con el mismo número y un número distinto, de una mano de cinco cartas)
    \item Dos pares (distintos y otra carta distinta, en una mano de cinco cartas).
    \item Un full (es decir, una tercia y un par en una mano de cinco cartas).
\end{itemize}.
\end{ejercicio}
\vspace{4cm}

\begin{ejercicio} 
En una baraja inglesa de $52$ cartas (con trece cartas de cuatro figuras distintas) se toman quince cartas.

Se dice que la mano de quince cartas es un churrisflais, si tiene dos tercias, tres pares distintos y tres números distintos.

¿Cuantas conjuntos distintos de quince cartas forman un churrisflais?

¿Cual es la probabilidad de obtener un churrisflais tomando quince cartas al azar?



\end{ejercicio}
\vspace{4cm}

\begin{ejercicio} 
Se lanza una moneda (justa) ocho veces. ¿Cual es la probabilidad de obtener exactamente tres caras?
\end{ejercicio}
\vspace{4cm}

\begin{ejercicio} 
En un casino se encuentran Juan y Sarita, ambos aficionados de la numerología mística. Su número de la suerte es el cuatro. En un arranque, deciden arrojar, cada uno, una moneda justa cuatro veces. Si ambos sacan el mismo número de caras, se casarán de inmediato en la capilla del casino. ¿Con qué probabilidad habrá boda?
\end{ejercicio}
\vspace{4cm}

\begin{ejercicio} 
¿Cuál es la probabilidad de obtener el mismo número de caras si Juan lanza cinco veces una moneda y Sarita la lanza tres veces?
\end{ejercicio}
\vspace{4cm}

\begin{ejercicio} 
Alicia se ubica el punto $(0,0)$ en el plano. Cada minuto lanza un dado y se mueve de acuerdo a la siguiente regla: Si le sale $1$ se mueve a la derecha. Si le sale $2$ o $3$ se mueve hacia la izquierda, y si le sale $4,5$ o $6$ se mueve hacia arriba. ¿Cuál es la probabilidad de que Alicia pise la posición $(1,1)$ por primera vez a los $4$ movimientos? 
\end{ejercicio}
\vspace{4cm}


\begin{ejercicio} 
En la urna A hay una bola blanca y una negra. En la urna B hay tres bolas negras y una blanca. En la urna C hay dos blancas y una negra.

Supón que se elige al azar una bola de la urna A y se mete en la B. Luego se toma a azar una bola de la urna B y se mete en la urna C. Finalmente se extrae una bola al azar de la urna C. ¿Cuál es la probabilidad de que se obtenga una bola negra?
\end{ejercicio}
\vspace{4cm}

\newpage 

\section{Principio de Inclusión y Exclusión}

\begin{ejercicio}
¿Cuántos caminos distintos hay?:
\begin{enumerate}
    \item ¿de $A$ a $B$?
    \item ¿de $A$ a $B$ pasando por $M$ y $N$?
    \item ¿de $A$ a $B$ sin pasar por $M$ ni $N$?
\end{enumerate}
\begin{center}
  \begin{tikzpicture}[x=1cm,y=0.75cm]
    \foreach \i in {0,...,5} \draw[black!60] (0,\i) -- (7,\i);
    \foreach \i in {0,...,7} \draw[black!60] (\i,0) -- (\i,5);

    \draw (0,0) node[circle,fill,inner sep=1pt,label=below left:{$A$}] {};
    \draw (7,5) node[circle,fill,inner sep=1pt,label=above right:{$B$}] {};
    \draw (3,2) node[circle,fill,inner sep=1pt,label=below left:{$M$}] {};
    \draw (5,3) node[circle,fill,inner sep=1pt,label=below left:{$N$}] {};
  \end{tikzpicture}
\end{center}
\end{ejercicio}
\newpage

\begin{ejercicio}
En un instituto de investigación científica trabajan 67 personas. De estas, 47
hablan inglés, 35 alemán y 23 ambos idiomas. ¿Cuántas personas en el instituto no hablan ni inglés ni alemán?
\end{ejercicio}

\begin{ejercicio}
En un centro de investigación en matemáticas trabajan varias personas, y cada
una de ellas habla por lo menos una lengua extranjera. Seis hablan inglés; seis, alemán; siete, francés. Cuatro hablan inglés y alemán; tres, alemán y francés; dos, francés e inglés. Una persona habla los tres idiomas. 
%Una persona únicamente habla elfo e inglés, pero tiene SNI 3.
¿Cuántas personas trabajan en el centro? ¿Cuántas de ellas hablan solamente inglés?
¿Y solamente francés?
\end{ejercicio}

\begin{tikzpicture}[line cap=round,line join=round,>=triangle 45,x=1.5*1.0cm,y=1.5*1.0cm]
\draw [line width=1.pt] (0.,0.) circle (1.5*1.46cm);
\draw [line width=1.pt] (2.5,0.) circle (1.5*2.28cm);
\draw [line width=1.pt] (0.42,-2.18) circle (1.5*1.9720040567909594cm);
\draw[color=black] (-0.2,0.37) node {$A$};
\draw[color=black] (1.04,0.15) node {$A\cap B$};
\draw[color=black] (2.72,0.51) node {$B$};
\draw[color=black] (.78,-0.53) node {$A\cap B\cap C$};
\draw[color=black] (0.8,-2.01) node {$C$};
\draw[color=black] (1.6,-1.19) node {$B\cap C$};
\draw[color=black] (-0.2,-0.75) node {$A\cap C$};
\end{tikzpicture}

\begin{ejercicio}
Calcula la siguiente suma alternante de coeficientes binomiales:
$${r \choose 1}-{r \choose 2}+{r \choose 3}-{r \choose 4}+\dots \pm {r \choose r}$$
\end{ejercicio}

\begin{teorema}[Principio de inclusión y exclusión]
Sean $A_1,A_2,\dots A_n$ conjuntos, posiblemente con elementos en común.

Entonces 
\begin{equation} \label{Eq:IncExc}
|A_1\cup A_2 \cup \cdots \cup  A_n|=k_1-k_2+k_3-k_4+\dots \pm k_n,
\end{equation}
donde 

$k_1=|A_1|+|A_2|+\dots +|A_n|$,

$k_2=|A_1\cap A_2|+|A_1\cap A_3|+ \dots |A_{n-1}\cap A_n|$

$k_3=|A_1\cap A_2 \cap A_3|+|A_1\cap A_2 \cap A_4|+ \dots |A_{n-2}\cap A_{n-1}\cap A_n|$

$\vdots$

$k_n=|A_1\cap A_2\cap \dots  \cap A_n|$
\end{teorema}

Sugerencia: Primero resuelve el siguiente ejercicio:

\begin{ejercicio}
Supón que un elemento $x\in A_1\cup A_2 \cup \dots \cup A_n$ se encuentra exactamente en $r$ de los conjuntos, digamos $x\in B_1 \cap B_2 \cap B_3 \cdots \cap B_r$, donde los $B_i$'s son algunos de los $A_i$'s.

¿Cuántas veces se cuenta al elemento $x$ en la suma de la ecuación (\ref{Eq:IncExc})?
\end{ejercicio}

\newpage 

\begin{ejercicio}
La función $\varphi(n)$ de Euler cuenta el número enteros positivos $k\leq n$ primos relativos con $n$. Calcula  $\varphi(40)$, $\varphi(120)$, $\varphi(330)$. 
\end{ejercicio}
\vspace{6cm}




%%%%%%%%%%%%%%%%%%%%%%%%%%%%%%%%%%%%%%%%%%%%%%%

\section{Ejercicios}

\begin{ejercicio}
Juan tiene tres sombreros y cuatro gorras de béisbol, siete camisetas, dos shorts, tres pantalones, dos pares de tenis y tres pares de zapatos. Juan es fresa y piensa que <<qué oso>> repetir alguna prenda durante la semana, de lunes a viernes. 

¿De cuantas formas se puede vestir el martes?

¿De cuantas formas puede elegir su outfit de la semana?
\end{ejercicio}

\begin{ejercicio}
¿Cuál es el coeficiente de $x^3y^4z^5$ al expandir el trinomio $(x+y+z)^{12}$?
\end{ejercicio}

\begin{ejercicio}
¿Puedes pensar en un objeto tridimensional análogo al triángulo de Pascal?
\end{ejercicio}

\begin{ejercicio}
Formula el teorema del trinomio: \[(x+y+z)^n=\text{¿qué?}\]
\end{ejercicio}

\begin{ejercicio}
Calcula las potencias $z^n$ del número complejo $z=1+\mathrm i$, para $n=1,2,3,4,5$.
\end{ejercicio}

\begin{ejercicio}
Calcula las potencias $z^n$ del número complejo $z=a+ b\mathrm i$, para $n=1,2,3,4,5$.
\end{ejercicio}

\begin{ejercicio}[Identidad de Vandermonde]
Mostrar que si $m\leq n$ son números naturales, entonces
$${m+n\choose n}={m \choose 0}{n \choose n}+{m \choose 1}{n \choose n-1}+{m \choose 2}{n \choose n-2}+\cdots +{m \choose m-1}{n \choose n-m-1}+{m \choose m}{n \choose n-m}.$$
\end{ejercicio}
%Sug: ¡Cuenta caminos!

\begin{ejercicio}
%[IWYMICT '19]
Hay veinte canicas rojas idénticas, quince canicas negras idénticas y veinte canicas blancas idénticas. Se reparten entre dos niños y una niña. Las condiciones que se deben cumplir son que cada niño reciba al menos dos canicas de cada color y que la niña reciba al menos tres canicas de cada color. 

¿De cuántas maneras se pueden repartir las canicas?
\end{ejercicio}
\newpage

%%%%%%%%%%%%%%%%%%%%%%%%%%%%%%%%%%%%%%%%%%%%%%%
\section{Problemas}

\begin{problema}
%[OMM '87]
¿De cuántas formas se pueden acomodar en una línea recta siete pelotas blancas y cinco negras, de manera que no haya dos pelotas negras juntas?
\end{problema}

\begin{problema}
%[OMM '88]
¿De cuántas formas puedo elegir ocho enteros no necesariamente distintos $a_1, a_2, a_3,\dots ,a_7, a_8$, de tal manera que $1\leq a_1\leq a_2 \leq a_3  \leq  \dots \leq a_7 \leq a_8$?
\end{problema}

\begin{problema}
%[OMM '89]
¿De cuántas formas se puede ir de $A$ a $B$, si solamente se permite ir hacia abajo o hacia los lados, sin pasar dos veces por el mismo punto?

\begin{center}
  \begin{tikzpicture}[x=.5cm,y=.5cm]
    \foreach \i in {0,...,7} \draw[black] (\i,0) -- (\i*.5,\i*0.866025);
    \foreach \i in {0,...,7} \draw[black] (7-\i,0) -- (7-\i*.5,\i*0.866025);
    \foreach \i in {0,...,7} \draw[black] (0+\i*.5,\i*0.866025) -- (7-\i*.5,\i*0.866025);
    
    \draw[color=black] (-.5,0) node {$B$};
    \draw[color=black] (3.5,6.7) node {$A$};
  \end{tikzpicture}
\end{center}

\end{problema}
\vspace{2cm}

%Sug: Solo importa seleccionar las bajadas, estas determinan las $\to, \leftarrow$ de manera unívoca.

\begin{problema}
%[OMM '00]
Considera un arreglo triangular de números como el siguiente
$$\begin{array}{ccccccccc}
     1 && 2 && 3 && 4 && 5\\
     & 3 && 5 && 7 && 9 &  \\
   && 8 && 12 && 16 &&  \\
  &&& 20 && 28 &&& \\
 &&&& 48 &&&&
\end{array}$$
pero con los números del $1$ al $2000$ en el primer renglón. 

¿Qué número aparece en el vértice inferior del arreglo?
\end{problema}
\vspace{2cm}

%Sug: Dado un número x en en el arreglo,  el número justo dos renglones abajo, en la misma posición, es 4x. 

\begin{problema}
Considera el siguiente tablero de $11 \times 11$: 
\begin{center}
  \begin{tikzpicture}[x=1cm,y=1cm]
    \foreach \i in {0,...,11} \draw[black!60] (0,\i) -- (11,\i);
    \foreach \i in {0,...,11} \draw[black!60] (\i,0) -- (\i,11);

    \draw (5.5,5.5) node[circle,fill,inner sep=1pt,label=below left:{$A$}] {};
  \end{tikzpicture}
\end{center}
Alicia se encuentra en la casilla del centro. En cada turno, Alicia arroja un dado en forma de tetraedro con los símbolos $N, S, E, O$ y dependiendo del símbolo que obtenga en el dado se mueve una casilla en esa dirección (Norte, Sur, Este, Oeste).

a) ¿Puede regresar Alicia a la casilla del centro después de exactamente cinco turnos?

b) En cuantas casillas del tablero puede encontrarse Alicia después de exactamente cinco lanzamientos. ¿De cuantas formas distintas puede llegar a cada casilla?

c) Después de cinco lanzamientos Alicia noto que en ninguno de los turnos el paso que dio la dejo mas cercana a la casilla del centro.
¿De cuantas formas distintas pudo moverse Alicia en esos cinco movimientos?
\end{problema}
\vspace{2cm}


\begin{problema}
%[OMM '98]
En el primer paso tomamos un número $n$ y se suman los cuadrados de sus dígitos. En el siguiente paso, se considera el número resultante y se suman los cuadrados de sus dígitos nuevamente, y así sucesivamente. Un número se llama \emph{suertudo} si después de algunos pasos se llega al $1$. 

Por ejemplo, el $1900$ es suertudo pues $$1900 \to 82 \to 68 \to 100 \to 1.$$ 

Encuentra una infinidad de parejas de números consecutivos $(k,k+1)$ donde ambos sean suertudos.
\end{problema}

\begin{problema}
%[OMM '91]
Evalúa la suma de todas las fracciones positivas irreducibles que sean menores que $1$ y que tengan como denominador al $1991$.
\end{problema}

\begin{problema}
%[OMM '93]
La función $f(n,k)$ se define de la siguiente manera:

(1) $f(n,0) = f(n,n) = 1$, y

(2) $f(n,k) = f(n-1,k-1) + f(n-1,k)$ para $0 < k < n$.

¿Cuántas veces se necesita usar la regla (2) para calcular $f(3991,1993)$?
\end{problema}


\begin{problema}
La sucesión $$1, 2, 4, 5, 7, 9 ,10, 12, 14, 16, 17,\dots $$ se forma de la siguiente manera. Primero consideramos un número impar, luego dos pares, luego tres impares, luego cuatro numeros pares, y así sucesivamente. Encuentra el número en la sucesión que se encuentre más cercano a $1994$.
\end{problema}
\vspace{2cm}

\begin{problema}
%[OMM '94]
Los $12$ números en la carátula de un reloj se reacomodan. Muestra que aún se pueden encontrar tres números adyacentes cuya suma es mayor o igual que $21$.
\end{problema}

\begin{problema}
%[OMM '94]
Un matemático bastante payaso, que acababa de leer <<Rayuela>>, escribió un libro con páginas numeradas del $2$ al $400$. Las páginas deben leerse en el siguiente orden. Se toma el número la de última página no leída hasta el momento (al inicio es la $400$) y se leen (en el orden usual) todas las páginas que no son primas relativas a ella, que no se hayan leído antes. El proceso se repite hasta que se leen todas las páginas. Entonces el orden sería $$2, 4, 5,\dots , 400, 3, 7, 9,\dots , 399, \dots$$ ¿Qué página se lee al final?
\end{problema}


\begin{problema}
%[OMM '95]

$N$ estudiantes se sientan en un arreglo de $m \times  n$ donde $m, n \ge  3$. Cada estudiante saluda a cada estudiante vecino horizontalmente, verticalmente y diagonalmente. Si en total hubo $1020$ saludos, ¿cuánto vale $N$?
\end{problema}


\begin{problema}
%[OMM '95]

Encuentra $26$ elementos del conjunto $\{1, 2, 3, ... , 40\}$ de tal forma que el producto de cada dos de ellos nunca sea un cuadrado.
Muestra que no se pueden elegir $27$ elementos con la misma propiedad.
\end{problema}







\end{document}

