\chapter*{Invitación al curso de resolución de problemas de Matemáticas}

El curso está dirigido tanto para estudiantes de secundaria como de preparatoria.

Las matemáticas: 
\begin{itemize}
    \item Son ineludibles desde casi cualquier carrera académica, actividad científica, tecnológica o industrial.
    \item Tienen un fuerte peso al ingresar a la universidad, en particular dentro de los exámenes de admisión.
    \item Con alta probabilidad aparecen en el tronco común de múltiples carreras (e.g. Economía, Ingenierías, Física, Química, Ciencias de la computación, Actuaría, Ciencia de Datos, etc). A veces, desde el primer semestre.
    \item En otros casos (como en las Ciencias de la Salud y en la industria), aparecen más para el final de la carrera, debido a la naturaleza estadística del método científico y de la optimización de procesos industriales.
\end{itemize}

Los principales objetivos del curso son los siguientes:
\begin{enumerate}[I.]
    \item Regularizarte en tus materias de Matemáticas, de secundaria o bachillerato.
    \item Prepararte para resolver cualquier tipo de examen de admisión, optimizar tu rendimiento y maximizar tus posibilidades de ingresar en la carrera de tu preferencia (incluyendo admisiones a programas internacionales).
   \item Dotarte de herramientas de matemáticas suficientes para confrontar los retos matemáticos de tu carrera con mayor confianza, mejor aprovechamiento, menos rezagos y menos angustias.
   \item Mostrarte la belleza de las matemáticas, cuando se estudian sin temas de relleno, y sin ejercicios repetitivos.
   \item Si te interesa participar en olimpiadas de matemáticas, el curso te será de gran utilidad.
\end{enumerate}
 
A lo largo del curso, se introducen herramientas de software libre (R, Python, LaTeX, Geogebra, etc.) esenciales hoy en día en el ámbito académico, la industria y la investigación.

Es casi seguro que en el futuro, los asistentes al curso se confronten con problemas que se simplifican de manera parcial, substancial o crucial, utilizando ideas y métodos aquí presentados.
\vspace{.5cm}

Te invitamos a intentar los {\bf retos de matemáticas} que se encuentran en las páginas siguientes (la última página está dirigida principalmente a estudiantes de últimos semestres).
\vspace{.5cm}

Te deseamos mucha suerte en tu preparación para lograr tus metas (ya sea en exámenes de admisión, de asignatura o de olimpiadas científicas).
\newpage

El curso se ofrece también en modalidad multilingüe, en caso de que el grupo interesado acredite un nivel básico de comprensión del idioma y desee perfeccionar su manejo del lenguaje científico-matemático.


Nuestro equipo consta de profesionales en matemáticas, tanto en su investigación como en su enseñanza y pedagogía. 
\vspace{.5cm}

\begin{flushright}
Informes: 

ei.turbomath@gmail.com 

473-740-9385

(precios accesibles a grupos pequeños).
\end{flushright}