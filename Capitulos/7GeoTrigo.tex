
\chapter{Geometría y Trigonometría}

Funciones trigonométricas.

En la sección anterior derivamos una fórmulas para el área del triángulo. 

Desde chicos conocemos la fórmula básica para el área de un triángulo: Base por altura entre dos.

Existe una terna adicional de fórmulas para el área de un triángulo bastante útiles, que involucran los \emph{senos} de los ángulos del triángulo.\\

\definecolor{qqwuqq}{rgb}{0,0.39215686274509803,0}
\begin{figure}[h]
\begin{subfigure}{.45 \textwidth}
\begin{tikzpicture}[line cap=round,line join=round,>=triangle 45,x=1cm,y=1cm]
\draw [shift={(3.62,-0.64)},line width=1pt,color=qqwuqq,fill=qqwuqq,fill opacity=0.10000000149011612] (0,0) -- (-13.981204311452633:0.6) arc (-13.981204311452633:42.69262700303779:0.6) -- cycle;
\draw [shift={(6.46,1.98)},line width=1pt,color=qqwuqq,fill=qqwuqq,fill opacity=0.10000000149011612] (0,0) -- (-137.30737299696221:0.6) arc (-137.30737299696221:-61.78829573131603:0.6) -- cycle;
\draw [shift={(8.52,-1.86)},line width=1pt,color=qqwuqq,fill=qqwuqq,fill opacity=0.10000000149011612] (0,0) -- (118.21170426868399:0.6) arc (118.21170426868399:166.01879568854736:0.6) -- cycle;
\draw [line width=1pt] (3.62,-0.64)-- (6.46,1.98);
\draw [line width=1pt] (3.62,-0.64)-- (8.52,-1.86);
\draw [line width=1pt] (8.52,-1.86)-- (6.46,1.98);
\draw [fill=black] (3.62,-0.64) circle (1.5pt);
\draw[color=black] (3.26,-0.63) node {A};
\draw [fill=black] (8.52,-1.86) circle (1.5pt);
\draw[color=black] (8.86,-2.09) node {B};
\draw [fill=black] (6.46,1.98) circle (1.5pt);
\draw[color=black] (6.68,2.51) node {C};
\draw[color=black] (4.8,1.17) node {b};
\draw[color=black] (6.06,-1.53) node {c};
\draw[color=black] (7.76,0.51) node {a};
\draw[color=qqwuqq] (4.8,-0.31) node {$\alpha$};
\draw[color=qqwuqq] (6.32,1.01) node {$\gamma$};
\draw[color=qqwuqq] (7.6,-1.17) node {$\beta$};
\end{tikzpicture}
\end{subfigure}
\begin{subfigure}{.45 \textwidth}
\begin{tikzpicture}[line cap=round,line join=round,>=triangle 45,x=1.5cm,y=1.5cm]
\draw [shift={(0,0)},line width=1pt,color=black,fill=black,fill opacity=0.10000000149011612] (0,0) -- (0:0.2580645161290323) arc (0:30:0.2580645161290323) -- cycle;
\draw [line width=1pt] (0,0)-- (2.598076211353316,1.5);
\draw [line width=1pt] (0,0)-- (4,0);
\draw [dashed, line width=1pt] (2.598076211353316,1.5)--(2.598076211353316,0);
\draw [line width=1pt] (2.598076211353316,1.5)-- (4,0);
\draw[color=black] (-0.2313802083333099,0.22449680779561737) node {$A$};
\draw[color=black] (4.190125168010777,0.03524949596765889) node {$C$};
\draw[color=black] (2.5299101142473357,1.7986903561827265) node {$B$};
\draw[color=black] (1.3944262432795937,1.1) node {$c=3$};
\draw[color=black] (2.03098538306454,-0.2) node {$b=4$};
\draw[color=black] (0.85,0.2) node {$\alpha = 30^{\circ}$};
\end{tikzpicture}
\end{subfigure}
\end{figure}

Para esto solo nos basta rotar el triángulo de manera que uno de los lados sea la base. La altura, entonces, está determinada

Si tengo un triángulo $ABC$, con lados $a,b,c$ y ángulos $\alpha, \beta, \gamma$, como en la figura, entonces el área del triángulo la puedo calcular con cualquiera de siguientes tres fórmulas $$\frac{b\cdot c\cdot \mathrm{sen}(\alpha)}{2}=\frac{a\cdot c\cdot \mathrm{sen}(\beta)}{2}=\frac{a\cdot b\cdot \mathrm{sen}(\gamma)}{2}.$$

Dependiendo de los datos que se me proporcionen, me puede ser útil alguna de las tres. Por ejemplo, si me dicen que $b=3$, $c=4$ y $\alpha=30^{\circ}$, el área del triángulo $ABC$ sería $$\frac{b\cdot c\cdot \mathrm{sen}(\alpha)}{2}=\frac{3\cdot 4 \cdot \sen(30^{\circ})}{2}=\frac{3\cdot 4}{4}=3,$$
sin que haya tenido la necesidad de medir ninguna alguna altura, como en la fórmula usual.\\

Se vienen a la mente dos preguntas inmediatas. ¿qué es el seno? y ¿por qué funciona esa fórmula?.

\subsection*{Funciones trigonométricas}

Hace veinte años, cuando cursé secundaria y preparatoria, me enseñaron las {\bf funciones trigonométricas}: seno, coseno, tangente, cotangente, secante y cosecante, como las 6 distintas \emph{proporciones entre la hipotenusa y los dos catetos}, en un triángulo rectángulo.\\

Durante medio año despejábamos lados y ángulos de triángulos, usando la ley de senos, el teorema de Pitágoras y la ley de cosenos. Después de muchos años dedicados a la enseñanza y la investigación en matemáticas, no he utilizado durante el ejercicio de esta profesión una sola secante, cosecante o cotangente.\\

La tangente tiene cierta utilidad porque se relaciona con el concepto de la {\bf derivada} (crucial en el cálculo), pero raramente se utiliza a la función tangente de manera directa. Las únicas funciones trigonométricas que son verdaderamente fundamentales en matemáticas, directamente, son el {\bf seno} y el {\bf coseno}.\\

Cuando cursé secundaria, se insistía demasiado en las definiciones $$\cos(\alpha)=\frac{CA}{H}, \quad \sen(\alpha)=\frac{CO}{H}$$ y las otras cuatro $$\mathrm{tan}(\alpha)=\frac{CO}{CA}, \quad \mathrm{cot}(\alpha)=\frac{CA}{CO}, \quad \mathrm{sec}(\alpha)=\frac{H}{CO}, \quad \mathrm{csc}(\alpha)=\frac{H}{CA}$$

\begin{wrapfigure}{R}{0.3\textwidth}
\begin{tikzpicture}[line cap=round,line join=round,>=triangle 45,x=3cm,y=3cm]
\draw [shift={(0,0)},line width=1pt,fill=black,fill opacity=0.10000000149011612] (0,0) -- (0:0.17777777777777792) arc (0:54.46232220802562:0.17777777777777792) -- cycle;
\draw[line width=1pt,fill=black,fill opacity=0.10000000149011612] (1,0.12570787221094187) -- (0.8742921277890581,0.1257078722109419) -- (0.8742921277890581,0) -- (1,0) -- cycle; 
\draw [line width=1pt] (0,0)-- (1,0);
\draw [line width=1pt] (0,0)-- (1,1.4);
\draw [line width=1pt] (1,1.4)-- (1,0);
\draw [fill=black] (0,0) circle (1.5pt);
\draw[color=black] (-0.11444444444444588,0.087037037037038) node {$A$};
\draw [fill=black] (1,0) circle (1.5pt);
\draw[color=black] (1.1240740740740738,-0.02555555555555454) node {$B$};
\draw [fill=black] (1,1.4) circle (1.5pt);
\draw[color=black] (1.0470370370370365,1.5033333333333336) node {$C$};
\draw[color=black] (0.5670370370370361,-0.08481481481481377) node {Cat. Ady.};
\draw[color=black] (0.3537037037037027,0.7388888888888895) node {Hip.};
\draw[color=black] (1.2544444444444443,0.697407407407408) node {Cat. Op.};
\draw[color=black] (0.31814814814814707,0.14629629629629723) node {$\alpha$};
\draw[color=black] (0.7462962962962957,0.24703703703703794) node {$\beta = 90^{\circ}$};
\end{tikzpicture}
\end{wrapfigure}


Estas definiciones, como proporciones $\frac{CA}{H}$, $\frac{CO}{H}$, no son tan concretas como pudieran presentarse. Hacen la falsa impresión de que el coseno el seno solo son útiles cuando se trabaja con triángulos rectángulos.\\ 

Una forma más tangible de apreciar al coseno y al seno es fijando el tamaño de la hipotenusa $H=1$ y observando que $(\cos\alpha, \sen\alpha)$ se convierten en \emph{las coordenadas  $(x,y)$ de los puntos en el círculo unitario}.\\

En bachillerato me hubiera gustado enterarme que el seno y el coseno se pueden definir como las siguientes series de potencias (más adelante):
$$\cos(x)=\frac{1}{0!}-\frac{x^2}{2!}+\frac{x^4}{4!}-\frac{x^6}{6!}+\dots,\quad   \sen(x)=\frac{x}{1!}-\frac{x^3}{3!}+\frac{x^5}{5!}+\dots$$

\subsection*{Una definición concreta del seno y el coseno}

\begin{wrapfigure}{R}{0.5\textwidth}
\centering
\definecolor{rvwvcq}{rgb}{0.26666666666666666,0.26666666666666666,0.26666666666666666}
\definecolor{dtsfsf}{rgb}{0.49019607843137253,0.49019607843137253,1}
\definecolor{wrwrwr}{rgb}{0.30196078431372547,0.30196078431372547,1}
\begin{tikzpicture}[line cap=round,line join=round,>=triangle 45,x=6cm,y=6cm]
%\clip(-0.23711934156378495,-0.45761316872428137) rectangle (1.279917695473252,1.2595884773662553);
\draw [shift={(0,0)},line width=1pt,fill=black,fill opacity=0.10000000149011612] (0,0) -- (0:0.07901234567901234) arc (0:66.34165192231299:0.07901234567901234) -- cycle;
\draw [line width=1pt] (0,0)-- (0.4012820175103667,0.9159545525968031);
\draw [line width=1pt] (0.4012820175103667,0.9159545525968031)-- (1,0);
\draw [shift={(0,0)},line width=1pt,color=black]  plot[domain=0:1.157880257256387,variable=\t]({1*1*cos(\t r)+0*1*sin(\t r)},{0*1*cos(\t r)+1*1*sin(\t r)});
\draw [line width=1pt] (0,0)-- (1,0);
\draw [line width=1pt] (0.4012820175103667,0.9159545525968031)-- (0.4012820175103667,0);
\draw [fill=rvwvcq] (0,0) circle (1.5pt);
\draw[color=rvwvcq] (0.01572016460905455,-0.06123456790123568) node {$A = (0, 0)$};
\draw [fill=rvwvcq] (1,0) circle (1.5pt);
\draw[color=rvwvcq] (1.0639506172839517,-0.06386831275720276) node {$B = (1, 0)$};
\draw [fill=rvwvcq] (0.4012820175103667,0.9159545525968031) circle (1.5pt);
\draw[color=rvwvcq] (0.43185185185185293,0.968559670781893) node {$C=(\cos(\alpha), \mathrm{sen}(\alpha))$};
\draw[color=black] (0.1,0.47341563786008173) node {b=1};
\draw[color=dtsfsf] (0.8611522633744867,0.5945679012345675) node {$d$};
\draw[color=black] (0.1421399176954743,0.08362139917695377) node {$\alpha$};
\draw[color=rvwvcq] (0.42131687242798466,-0.05728395061728296) node {$H$};
\draw [fill=wrwrwr] (0.4012820175103667,0) circle (1.5pt);
\draw[color=black] (0.5293004115226349,-0.1) node {c=1};
\draw[color=black] (0.5767078189300423,0.43127572016460847) node {h=$|\mathrm{sen}(\alpha)|$};
\end{tikzpicture}
\end{wrapfigure}

Si abro un arco $BAC$, desde el centro $A=(0,0)$, con $B=(1,0)$, de ángulo $\alpha$ (como en la figura), el punto $C$ sobre la circunferencia tiene coordenadas $(x,y)=(\cos \alpha,\mathrm{sen}\alpha)$.\\

Vista la definición desde esta perspectiva, tiene mucho sentido medir los ángulos en radianes. La medida de un ángulo en radianes es exactamente la longitud del arco $d$ que tiende dicho ángulo. Recordemos que el perímetro del círculo unitario es $2\pi$, por lo que $2\pi$ corresponde a la vuelta completa, o $360^{\circ}$. \href{https://www.geogebra.org/calculator/sh3uynfv}{Figura dinámica} sh3uynfv).\\

Se hace una regla de tres para convertir los otros arcos. El ángulo recto, por ejemplo, mide un cuarto de circunferencia, que es igual a $\frac{\pi}{2}$ radianes, o noventa grados. \\

La elección de que una vuelta mida $360^{\circ}$ se basa en una unidad de medida de ángulo completamente arbitraria (influenciada por la astronomía y por las matemáticas de Babilonia, en las que se utiliza base $60$). Por otro lado, $2r\pi $ es la longitud del círculo completo, algo bastante concreto. \\

Por esta razón, es más común en matemáticas profesionales medir los ángulos y evaluar las funciones trigonométricas en radianes (y no en grados, que resultan un tanto arbitrarios).\\ 

Se recomienda al estudiante usar radianes pero sobre todo a poner mucho cuidado, y verificar siempre que se esté utilizando/interpretando en la escala correcta, \emph{nunca olvidar} el símbolo ° cuando se trate de grados.\\ 

\begin{ejercicio}
¿Cuánto miden (en grados y en radianes):
\begin{itemize} 
\item una media circunferencia? 
\item un tercio de circunferencia? 
\item la mitad de un ángulo recto? 
\item un séptimo de circunferencia? 
\item un noveno de circunferencia?
\end{itemize}
 \end{ejercicio}

El seno, entonces, en vez de definirse como una \emph{proporción} $\frac{CO}{H}$, abstracta por naturaleza, se puede entender mucho más concretamente, como la \emph{longitud} de la altura $CH$, del triángulo $ABC$ en la figura.\\

Esta definición (como la coordenada $y$ de un punto en el círculo) nos regala intuición y respuestas inmediatas a varias preguntas: \\

\definecolor{rvwvcq}{rgb}{0.08235294117647059,0.396078431372549,0.7529411764705882}
\definecolor{sexdts}{rgb}{0.1803921568627451,0.49019607843137253,0.19607843137254902}
\begin{tikzpicture}[line cap=round,line join=round,>=triangle 45,x=3cm,y=3cm]
\begin{axis}[
x=2cm,y=2cm,
axis lines=middle,
xmin=-1.2777777777777772,
xmax=7.273333333333342,
ymin=-1.4888888888888887,
ymax=2.680000000000001,
xtick={-1,-0,...,7},
ytick={-1,0,...,2},]
\draw[line width=1pt,color=sexdts,smooth,samples=100,domain=-1.2777777777777772:7.273333333333342] plot(\x,{sin(((\x))*180/pi)});
\draw[line width=1pt,color=rvwvcq,smooth,samples=100,domain=-1.2777777777777772:7.273333333333342] plot(\x,{cos(((\x))*180/pi)});
\draw[color=sexdts] (+2.5,1.2) node {$sen(x)$};
\draw[color=rvwvcq] (+5.5,1.2) node {$cos(x)$};
\end{axis}
\end{tikzpicture}


\begin{ejercicio}
¿Cuál es la relación entre: 
\begin{itemize}
    \item $\sen(\alpha)$ y $\sen(-\alpha)$?
    \item $\sen(\alpha)$ y $\sen(180^{\circ}-\alpha)$?
    \item $\cos(\alpha)$ y $\cos(\alpha)$?
    \item $\sen(\alpha)$ y $\sen(180^{\circ}-\alpha)$?
\end{itemize}
\end{ejercicio}

\begin{ejercicio}
Demuestra la identidad trigonométrica $(\sen(\alpha))^2+ (\cos(\alpha))^2=1$.
\end{ejercicio}

\begin{ejercicio}
¿Por qué son ambas funciones periódicas? 

¿De que tamaño es el periodo?
\end{ejercicio}

\subsection*{El seno y el área}

¿Cuánto vale el área del triángulo ABC?. Usando la fórmula de base por altura entre dos, tengo que en este caso la base es $c=1$, mientras que la altura es la coordenada $y$ del punto $C$, es decir $h=\mathrm{sen}(\alpha)$. Entonces el área resulta ser simplemente $\frac{1}{2}\mathrm{sen}(\alpha)$ \\

Concluímos que, al menos para triángulos que son sectores triangulares del círculo unitario, como el de la figura, se verifica la fórmula extraña de arriba: $\frac{bc}{2}\sen(\alpha)$.
En efecto, este caso, $c=1$, $b=1$ y la altura es exactamente $h=\sen(\alpha)$.\\

\begin{wrapfigure}{R}{0.4\textwidth}
\centering
\begin{tikzpicture}[line cap=round,line join=round,>=triangle 45,x=4cm,y=4cm]
\draw [shift={(0,0)},line width=1pt,fill=black,fill opacity=0.10000000149011612] (0,0) -- (0:0.11851851851851836) arc (0:53.09503941018474:0.11851851851851836) -- cycle;
\draw [line width=1pt] (0,0)-- (0.6004894586737243,0.7996326719323927);
\draw [line width=1pt] (0.6004894586737243,0.7996326719323927)-- (1,0);
\draw [shift={(0,0)},line width=1pt]  plot[domain=0:0.9266832541838718,variable=\t]({1*1*cos(\t r)+0*1*sin(\t r)},{0*1*cos(\t r)+1*1*sin(\t r)});
\draw [line width=1pt] (0,0)-- (1,0);
\draw [line width=1pt] (0.6004894586737243,0.7996326719323927)-- (0.6004894586737243,0);
\draw [line width=1pt] (0.6004894586737243,0.7996326719323927)-- (1.7864197530864234,0);
\draw [line width=1pt] (0.9605776276417528,1.2791386158320652)-- (1.7864197530864234,0);
\draw [line width=1pt] (0.6004894586737243,0.7996326719323927)-- (0.9605776276417528,1.2791386158320652);
\draw [line width=1pt] (1,0)-- (1.7864197530864234,0);
\draw [fill=rvwvcq] (0,0) circle (1.5pt);
\draw[color=rvwvcq] (0.024938271604938195,-0.0908641975308668) node {$A = (0, 0)$};
\draw [fill=rvwvcq] (1,0) circle (1.5pt);
\draw[color=rvwvcq] (1.095555555555554,-0.09481481481481743) node {$B = (1, 0)$};
\draw [fill=rvwvcq] (0.6004894586737243,0.7996326719323927) circle (1.5pt);
\draw[color=rvwvcq] (0.2192592592592576,0.845432098765431) node {($\sen(\alpha),\cos(\alpha))=$  C};
\draw[color=black] (0.15851851851851817,0.08851851851851623) node {$\alpha$};
\draw [fill=rvwvcq] (0.6004894586737243,0) circle (1.5pt);
\draw[color=rvwvcq] (0.597777777777777,-0.07111111111111369) node {H};
\draw[color=black] (0.4624691358024681,0.35950617283950426) node {h=|sen$\alpha$|};
\draw [fill=rvwvcq] (1.7864197530864234,0) circle (1.5pt);
\draw[color=rvwvcq] (1.8343209876543185,0.08691358024691125) node {B'};
\draw [fill=rvwvcq] (0.9605776276417528,1.2791386158320652) circle (1.5pt);
\draw[color=rvwvcq] (1.056049382716048,1.3234567901234564) node {C'};
\end{tikzpicture}
\end{wrapfigure}

¿Cómo paso de una fórmula del área para un triangulo como ABC a la fórmula para el área de un triángulo A'B'C' en general? Muy sencillo, solo se escalan las áreas, en dos pasos, como se ilustra en la siguiente figura:\\

Ya vimos que el área del triángulo $ABC$ es $\frac{1}{2}\sen(\alpha)$. Observemos que el triángulo $AB'C$ comparte la misma altura $CH$, pero la base ahora es $AB'=c$, en vez de $AB=1$. Por lo tanto, el área sólo se multiplica por $c$ y obtenemos $c\sen(\alpha)/2$ \\

De la misma manera, el triángulo $AB'C$ y el $AB'C'$ (cuya área queremos calcular), comparten la misma base.  Las alturas se encuentran en la misma proporción que los lados $\frac{AC'}{AC}=\frac{b}{1}=b$. Por lo tanto, la altura sólo se multiplica por $b$.\\

Entonces, como la base no cambia, el área también se multiplica por $c$ y concluímos que el área de $AB'C'$ es $$\frac{1}{2}bc \sen(\alpha).$$\\

\newpage

\section{Algunas consecuencias}

\begin{multicols}{2}
\begin{problema}[Ley de senos]

a). Usando las tres nuevas fórmulas del área, con los senos del triángulo, demuestra la \emph{ley de senos}:

$$\frac{a}{sen(\alpha)}=\frac{b}{sen(\beta)}=\frac{c}{sen(\gamma)}.$$

b). Demuestra que si el circuncírculo del triángulo $ABC$ tiene radio $R$, entonces 
$$\frac{a}{sen(\alpha)}=\frac{b}{sen(\beta)}=\frac{c}{sen(\gamma)}=2R.$$
\end{problema}

\columnbreak
\definecolor{qqwuqq}{rgb}{0,0.39215686274509803,0}
\begin{tikzpicture}[line cap=round,line join=round,>=triangle 45,x=1cm,y=1cm]
\draw [shift={(3.62,-0.64)},line width=1pt,color=qqwuqq,fill=qqwuqq,fill opacity=0.10000000149011612] (0,0) -- (-13.981204311452633:0.6) arc (-13.981204311452633:42.69262700303779:0.6) -- cycle;
\draw [line width=1pt] (3.62,-0.64)-- (6.46,1.98);
\draw [line width=1pt] (3.62,-0.64)-- (8.52,-1.86);
\draw [line width=1pt] (8.52,-1.86)-- (6.46,1.98);
\draw [line width=1pt] (6.227540054469171,-0.6172571582795595) circle (2.6076392335809424cm);
\draw [fill=black] (3.62,-0.64) circle (1.5pt);
\draw[color=black] (3.26,-0.63) node {A};
\draw [fill=black] (8.52,-1.86) circle (1.5pt);
\draw[color=black] (8.86,-2.09) node {B};
\draw [fill=black] (6.46,1.98) circle (1.5pt);
\draw[color=black] (6.68,2.51) node {C};
\draw[color=black] (4.32,0.67) node {b};
\draw[color=black] (6.06,-1.43) node {c};
\draw[color=black] (7.84,0.45) node {a};
\draw[color=qqwuqq] (4.5,-0.40) node {$\alpha$};
\end{tikzpicture}
\end{multicols}

\begin{multicols}{2}
\begin{problema}[Ley de cosenos]
Muestra que en un triángulo con lados de longitudes $a, b, c$ se tiene que $b^2=a^2+c^2-2ac\cos(\beta)$, donde $\beta$ es el ángulo opuesto al lado $b$.
\end{problema}

Sug: Aquí sí conviene la definición clásica de $\cos\beta=\frac{CA}{H}$

\begin{problema}

Sean $a,b,c$ lados de un triángulo y $\alpha, \beta, \gamma$ sus ángulos. \\

Demuestra que si sabes tres datos de $\{a,b,c,\alpha, \beta, \gamma\}$, de los al menos uno es un lado, entonces conoces los seis datos.

\columnbreak
\definecolor{qqwuqq}{rgb}{0,0.39215686274509803,0}
\begin{tikzpicture}[line cap=round,line join=round,>=triangle 45,x=1cm,y=1cm]
\draw [shift={(3.62,-0.64)},line width=1pt,color=qqwuqq,fill=qqwuqq,fill opacity=0.10000000149011612] (0,0) -- (-13.981204311452633:0.6) arc (-13.981204311452633:42.69262700303779:0.6) -- cycle;
\draw [shift={(6.46,1.98)},line width=1pt,color=qqwuqq,fill=qqwuqq,fill opacity=0.10000000149011612] (0,0) -- (-137.30737299696221:0.6) arc (-137.30737299696221:-61.78829573131603:0.6) -- cycle;
\draw [shift={(8.52,-1.86)},line width=1pt,color=qqwuqq,fill=qqwuqq,fill opacity=0.10000000149011612] (0,0) -- (118.21170426868399:0.6) arc (118.21170426868399:166.01879568854736:0.6) -- cycle;
\draw [line width=1pt] (3.62,-0.64)-- (6.46,1.98);
\draw [line width=1pt] (3.62,-0.64)-- (8.52,-1.86);
\draw [line width=1pt] (8.52,-1.86)-- (6.46,1.98);
\draw [fill=black] (3.62,-0.64) circle (1.5pt);
\draw[color=black] (3.26,-0.63) node {A};
\draw [fill=black] (8.52,-1.86) circle (1.5pt);
\draw[color=black] (8.86,-2.09) node {B};
\draw [fill=black] (6.46,1.98) circle (1.5pt);
\draw[color=black] (6.68,2.51) node {C};
\draw[color=black] (4.8,1.17) node {b};
\draw[color=black] (6.06,-1.53) node {c};
\draw[color=black] (7.76,0.51) node {a};
\draw[color=qqwuqq] (4.8,-0.31) node {$\alpha$};
\draw[color=qqwuqq] (6.32,1.01) node {$\gamma$};
\draw[color=qqwuqq] (7.6,-1.17) node {$\beta$};
\end{tikzpicture}
\end{problema}

\end{multicols}



%Ejecicios (opcional).

\begin{problema}
Calcula el área de un cuadrilátero convexo $ABCD$ en términos de sus diagonales $AC$, $BD$ y el ángulo que forman.
\end{problema}

% R= \frac{1}{2}(AC)(BD)\sen(\alpha)

\begin{problema}
Demuestra que el área de un n-ágono regular inscrito en un círculo de radio $r$ está dada por $nr^2 \sen \left(\frac{2\pi}{n} \right)$.
\end{problema}

\begin{problema}


\begin{multicols}{2}
Calcula el área de caracol.\\

Los ángulos entre rayos consecutivos desde el centro del caracol son de $30^{\circ}$.\\

Las longitudes de los trece rayos están en progresión aritmética: \\

$l_1=1, l_2=2, \dots, l_{13}=13$. 

\columnbreak

\begin{tikzpicture}[line cap=round,line join=round,>=triangle 45,x=.3cm,y=.3cm]
\draw [line width=1pt] (0,0)-- (1,0);
\draw [line width=1pt] (1,0)-- (1.7320508075688774,1);
\draw [line width=1pt] (1.7320508075688774,1)-- (1.5,2.598076211353316);
\draw [line width=1pt] (1.5,2.598076211353316)-- (0,4);
\draw [line width=1pt] (0,4)-- (-2.5,4.3301270189221945);
\draw [line width=1pt] (-2.5,4.3301270189221945)-- (-5.196152422706631,3);
\draw [line width=1pt] (-5.196152422706631,3)-- (-7,0);
\draw [line width=1pt] (-7,0)-- (-6.928203230275514,-4);
\draw [line width=1pt] (-6.928203230275514,-4)-- (-4.5,-7.794228634059941);
\draw [line width=1pt] (-4.5,-7.794228634059941)-- (0,-10);
\draw [line width=1pt] (0,-10)-- (5.5,-9.526279441628834);
\draw [line width=1pt] (5.5,-9.526279441628834)-- (10.392304845413255,-6);
\draw [line width=1pt] (10.392304845413255,-6)-- (13,0);
\draw [line width=1pt] (0,0)-- (1.7320508075688774,1);
\draw [line width=1pt] (0,0)-- (1.5,2.598076211353316);
\draw [line width=1pt] (0,0)-- (0,4);
\draw [line width=1pt] (0,0)-- (-2.5,4.3301270189221945);
\draw [line width=1pt] (0,0)-- (-5.196152422706631,3);
\draw [line width=1pt] (0,0)-- (-7,0);
\draw [line width=1pt] (0,0)-- (-6.928203230275514,-4);
\draw [line width=1pt] (0,0)-- (-4.5,-7.794228634059941);
\draw [line width=1pt] (0,0)-- (0,-10);
\draw [line width=1pt] (0,0)-- (5.5,-9.526279441628834);
\draw [line width=1pt] (0,0)-- (10.392304845413255,-6);
\draw [line width=1pt] (1,0)-- (13,0);
\draw [fill=black] (1,0) circle (1.5pt);
\draw [fill=black] (1.7320508075688774,1) circle (1.5pt);
\draw [fill=black] (1.5,2.598076211353316) circle (1.5pt);
\draw [fill=black] (0,4) circle (1.5pt);
\draw [fill=black] (-2.5,4.3301270189221945) circle (1.5pt);
\draw [fill=black] (-5.196152422706631,3) circle (1.5pt);
\draw [fill=black] (-7,0) circle (1.5pt);
\draw [fill=black] (-6.928203230275514,-4) circle (1.5pt);
\draw [fill=black] (-4.5,-7.794228634059941) circle (1.5pt);
\draw [fill=black] (0,-10) circle (1.5pt);
\draw [fill=black] (5.5,-9.526279441628834) circle (1.5pt);
\draw [fill=black] (10.392304845413255,-6) circle (1.5pt);
\draw [fill=black] (13,0) circle (1.5pt);
\draw [fill=black] (0,0) circle (1.5pt);
\draw [fill=black] (1,0) circle (1.5pt);
\draw [fill=black] (1.7320508075688774,1) circle (1.5pt);
\draw [fill=black] (1.5,2.598076211353316) circle (1.5pt);
\draw [fill=black] (0,4) circle (1.5pt);
\draw [fill=black] (-2.5,4.3301270189221945) circle (1.5pt);
\draw [fill=black] (-5.196152422706631,3) circle (1.5pt);
\draw [fill=black] (-7,0) circle (1.5pt);
\draw [fill=black] (-6.928203230275514,-4) circle (1.5pt);
\draw [fill=black] (-4.5,-7.794228634059941) circle (1.5pt);
\draw [fill=black] (0,-10) circle (1.5pt);
\draw [fill=black] (5.5,-9.526279441628834) circle (1.5pt);
\draw [fill=black] (10.392304845413255,-6) circle (1.5pt);
\draw [fill=black] (13,0) circle (1.5pt);
\draw[color=black] (-4.3475,-4.305) node {$\ddots$};
\draw[color=black] (-1.7575,-6.06) node {$l_{9}=9$};
\draw[color=black] (0.375,-7.635) node {$l_{10}=10$};
\draw[color=black] (5.4875,-5.475) node {$l_{11}=11$};
\draw[color=black] (6.7625,-2.325) node {$l_{12}=12$};
\draw[color=black] (7.3975,0.69) node {$l_{13}=13$};
\end{tikzpicture}

\end{multicols}

\end{problema}
\newpage

\begin{problema}[Teorema de la bisectriz]

Demuestra el teorema de la bisectriz:

\begin{tikzpicture}[line cap=round,line join=round,>=triangle 45,x=5cm,y=5cm]
\draw [shift={(0.2707407407407396,0.6885185185185192)},line width=1pt,fill=black,fill opacity=0.10000000149011612] (0,0) -- (-111.46585191085832:0.07901234567901248) arc (-111.46585191085832:-77.40993720664714:0.07901234567901248) -- cycle;
\draw [shift={(0.2707407407407396,0.6885185185185192)},line width=1pt,fill=black,fill opacity=0.10000000149011612] (0,0) -- (-77.40993720664713:0.07901234567901248) arc (-77.40993720664713:-43.354022502435924:0.07901234567901248) -- cycle;
\draw [line width=1pt] (0,0)-- (0.2707407407407396,0.6885185185185192);
\draw [line width=1pt] (0,0)-- (1,0);
\draw [line width=1pt] (1,0)-- (0.2707407407407396,0.6885185185185192);
\draw [line width=1pt] (0.2707407407407396,0.6885185185185192)-- (0.4245174874452492,0);
\draw[color=black] (-0.050905349794240086,0.03831275720164729) node {$A$};
\draw[color=black] (0.2783127572016453,0.7599588477366268) node {$B$};
\draw[color=black] (1.0210288065843625,0.046213991769548525) node {$C$};
\draw[color=black] (0.4468724279835385,0.05674897119341684) node {$D$};
\draw[color=black] (0.3600823045267485,0.5076954732510301) node {$\beta$};
\draw[color=black] (0.25209876543209816,0.5055967078189314) node {$\beta$};

\draw[color=black] (.8,.6) node {\LARGE{$\frac{AD}{DC}=\frac{BA}{BC}$}};

\end{tikzpicture}


\end{problema}

%Sug: comparar áreas

\begin{problema}[Teorema generalizado de la bisectriz]
Demuestra el teorema generalizado de la bisectriz: 

\end{problema}

\begin{tikzpicture}[line cap=round,line join=round,>=triangle 45,x=5cm,y=5
cm]
\draw [shift={(0.2707407407407396,0.6885185185185192)},line width=1pt,fill=black,fill opacity=0.10000000149011612] (0,0) -- (-111.46585191085832:0.07901234567901248) arc (-111.46585191085832:-82.64370893112309:0.07901234567901248) -- cycle;
\draw [shift={(0.2707407407407396,0.6885185185185192)},line width=1pt,fill=black,fill opacity=0.10000000149011612] (0,0) -- (-82.64370893112309:0.07901234567901248) arc (-82.64370893112309:-43.35402250243593:0.07901234567901248) -- cycle;
\draw [line width=1pt] (0,0)-- (0.2707407407407396,0.6885185185185192);
\draw [line width=1pt] (0,0)-- (1,0);
\draw [line width=1pt] (1,0)-- (0.2707407407407396,0.6885185185185192);
\draw [line width=1pt] (0.2707407407407396,0.6885185185185192)-- (0.3596296296296286,0);
\draw[color=black] (-0.050905349794240086,0.03831275720164729) node {$A$};
\draw[color=black] (0.2783127572016453,0.7599588477366268) node {$B$};
\draw[color=black] (1.0210288065843625,0.046213991769548525) node {$C$};
\draw[color=black] (0.38102880658436145,0.046213991769548525) node {$D$};
\draw[color=black] (0.3600823045267485,0.5076954732510301) node {$\beta$};
\draw[color=black] (0.25209876543209816,0.5055967078189314) node {$\alpha$};

\draw[color=black] (.8,.6) node {\LARGE{$\frac{AD}{DC}=\frac{BA}{BC}\frac{\sen\alpha}{\sen\beta}$}};

\end{tikzpicture}



\begin{problema}[Teorema de Ceva]
Demuestra el teorema de Ceva:

\begin{tikzpicture}[line cap=round,line join=round,>=triangle 45,x=6cm,y=6cm]
\draw [line width=1pt] (0,0)-- (1,0);
\draw [line width=1pt] (1,0)-- (0.46,0.6);
\draw [line width=1pt] (0.46,0.6)-- (0,0);
\draw [line width=1pt] (0.46,0.6)-- (0.47604938271604985,0);
\draw [line width=1pt] (0,0)-- (0.7141231839574385,0.3176409067139572);
\draw [line width=1pt] (1,0)-- (0.2856594873389836,0.3725993313117177);
\draw [line width=1pt] (1,0)-- (0.23251449386558373,0.3032797746072831);
\draw[color=black] (-0.06650205761316873,-0.026995884773662552) node {$A$};
\draw[color=black] (1.071275720164609,-0.013827160493827161) node {$B$};
\draw[color=black] (0.45234567901234574,0.6735802469135803) node {$C$};
\draw[color=black] (0.47341563786008234,-0.05069958847736626) node {$D$};
\draw[color=black] (0.7341563786008231,0.3627983539094651) node {$E$};
\draw[color=black] (0.26534979423868316,0.4418106995884774) node {$F'$};
\draw[color=black] (0.15209876543209877,0.3627983539094651) node {$F$};

\draw[color=black] (1.3,.6) node {\LARGE{$\frac{AD}{DB}\frac{BE}{EC}\frac{CF}{FA}=1\neq \frac{AD}{DB}\frac{BE}{EC}\frac{CF'}{F'A}.$}};
\end{tikzpicture}


\begin{tikzpicture}[line cap=round,line join=round,>=triangle 45,x=6cm,y=6cm]
\draw [line width=1pt] (0,0)-- (1,0);
\draw [line width=1pt] (1,0)-- (0.46,0.6);
\draw [line width=1pt] (0.46,0.6)-- (0,0);
\draw [line width=1pt] (0.46,0.6)-- (0.47604938271604985,0);
\draw [line width=1pt] (0,0)-- (0.7141231839574385,0.3176409067139572);
%\draw [line width=1pt] (1,0)-- (0.2856594873389836,0.3725993313117177);
\draw [line width=1pt] (1,0)-- (0.23251449386558373,0.3032797746072831);
\draw[color=black] (-0.06650205761316873,-0.026995884773662552) node {$A$};
\draw[color=black] (1.071275720164609,-0.013827160493827161) node {$B$};
\draw[color=black] (0.45234567901234574,0.6735802469135803) node {$C$};
\draw[color=black] (0.47341563786008234,-0.05069958847736626) node {$D$};
\draw[color=black] (0.7341563786008231,0.3627983539094651) node {$E$};
\draw[color=black] (0.15209876543209877,0.3627983539094651) node {$F$};

\end{tikzpicture}


%Sug1: Comparar áreas (AOB, BOC, COA).

%Sug2: Teo gral. bis X3 (AOB, BOC, COA).
\end{problema}

\begin{problema}
Demuestra la versión trigonométrica del teorema de Ceva.
\end{problema}

\begin{multicols}{2}
\begin{tikzpicture}[line cap=round,line join=round,>=triangle 45,x=4cm,y=4cm]
\draw [line width=1pt] (0,0)-- (1,0);
\draw [line width=1pt] (1,0)-- (0.46,0.6);
\draw [line width=1pt] (0.46,0.6)-- (0,0);
\draw [line width=1pt] (0.46,0.6)-- (0.47604938271604985,0);
\draw [line width=1pt] (0,0)-- (0.7141231839574385,0.3176409067139572);
\draw [line width=1pt] (1,0)-- (0.2856594873389836,0.3725993313117177);
\draw [line width=1pt] (1,0)-- (0.23251449386558373,0.3032797746072831);
\draw [fill=black] (0,0) circle (1.5pt);
\draw[color=black] (-0.06650205761316873,-0.026995884773662552) node {$A$};
\draw [fill=black] (1,0) circle (1.5pt);
\draw[color=black] (1.071275720164609,-0.013827160493827161) node {$B$};
\draw [fill=black] (0.46,0.6) circle (1.5pt);
\draw[color=black] (0.45234567901234574,0.6735802469135803) node {$C$};
\draw [fill=black] (0.47604938271604985,0) circle (1.5pt);
\draw[color=black] (0.47341563786008234,-0.05069958847736626) node {$D$};
\draw [fill=black] (0.7141231839574385,0.3176409067139572) circle (1.5pt);
\draw[color=black] (0.7341563786008231,0.3627983539094651) node {$E$};
\draw [fill=black] (0.2856594873389836,0.3725993313117177) circle (1.5pt);
\draw[color=black] (0.26534979423868316,0.4418106995884774) node {$F'$};
\draw [fill=black] (0.470451991158082,0.2092563305517134) circle (1.5pt);
\draw[color=black] (0.5102880658436214,0.14683127572016463) node {$O$};
\draw [fill=black] (0.23251449386558373,0.3032797746072831) circle (1.5pt);
\draw[color=black] (0.15209876543209877,0.3627983539094651) node {$F$};
\end{tikzpicture}
\columnbreak
$$\frac{\sen(DCA)}{\sen(CAE)}\frac{\sen(EAB)}{\sen(ABF)}\frac{\sen(FBC)}{\sen(BCD)}=1,$$

$$\frac{\sen(DCA)}{\sen(CAE)}\frac{\sen(EAB)}{\sen(ABF')}\frac{\sen(F'BC)}{\sen(BCD)}\neq 1,$$
\end{multicols}

%Sug: Teo gral. bis. X3 (a ABC)


\begin{ejercicio}
Utiliza Ceva para demostrar que las medianas, las bisectrices y las alturas de un triángulo son concurrentes.

¿Te sabes alguna demostración más sencilla?
\end{ejercicio}


\begin{problema}
Sobre los lados $BC$, $CA$ y $AB$ de un triángulo se colocan los puntos $L, M, N$, de tal forma que $AB+BL=LC+CA$,  $BC+CM=MA+AB$ y $CA+AN=NB+BA$.

Demuestra que $AL$, $BM$ y $CN$ son concurrentes.
\end{problema}

\begin{problema}
En los lados $BC$, $CA$, y $AB$ de un triángulo $ABC$ se toman puntos $A_1$, $B_1$, $C_1$ de tal forma que $AA_1$, $BB_1$ y $CC_1$ concurren. Luego, en los lados $B_1C_1$, $C_1A_1$, y $A_1B_1$ del triángulo $A_1B_1C_1$ se toman puntos $A_2$, $B_2$, $C_2$ de tal forma que $A_1A_2$, $B_1B_2$ y $C_1C_2$ concurren.

Demuestre que $AA_2$, $BB_2$ y $CC_2$ concurren.
\end{problema}

%OMM año?

\begin{problema}
Demuestra que las simedianas de un triángulo concurren. Una simediana es la reflejada de la mediana con respecto a la bisectriz.
\end{problema}
\newpage

\newpage





















\begin{problema}
Dos circunferencias se intersectan en los puntos $A$ y $B$. Las rectas $L_1$ y $L_2$
son paralelas, con la particularidad de que $L_1$ pasa por el punto $A$ y corta las circunferencias
en los puntos $E$ y $K$, mientras que $L_2$ pasa por el punto $B$ y corta las circunferencias
en los puntos $M$ y $P$. Demuestre que el cuadril\'atero $EKMP$ es un paralelogramo.

\begin{tikzpicture}[line cap=round,line join=round,>=triangle 45,x=1.3cm,y=1.3cm]
\draw [line width=1.pt] (1.34,0.513333333333335) circle (1.3*1.3979667775419817cm);
\draw [line width=1.pt] (4.446666666666667,0.32666666666666827) circle (1.3*2.1526624341859906cm);
\draw [line width=1.pt] (0.5559044016253842,-0.6440366177716114)-- (0.002111677305102866,0.9187540371802734);
\draw [line width=1.pt] (6.596101311351841,0.2088266743776298)-- (6.04230858703156,1.7716173293295143);
\draw [line width=1.pt] (-0.7242465935523759,0.8161937538945099)-- (6.9179238245269925,1.895252387137624);
\draw [line width=1.pt] (7.30824366115857,0.3093796990575103)-- (-0.5840021523968433,-0.8049890620272606);
\draw[color=black] (2.5,1.573333333333335) node {$A$};
\draw[color=black] (2.3666666666666676,-0.5066666666666652) node {$B$};
\draw[color=black] (-0.1,1.1066666666666685) node {$E$};
\draw[color=black] (0.54,-0.76) node {$F$};
\draw[color=black] (6.7,0.1066666666666682) node {$G$};
\draw[color=black] (6.073333333333334,2.0133333333333354) node {$H$};
\end{tikzpicture}

(GusievMIRclubsigma)
\end{problema}

\begin{problema}
Dos circunferencias se intersectan en los puntos $A$ y $B$. Estos yacen por diferentes
lados de la recta $l$ que corta las circunferencias en los puntos $C$, $D$, $E$ y $M$,
respectivamente. Demuestre que la suma de los \'angulos $\angle DBE$ y $\angle CAM$ es igual 
a $180^\circ$.


\begin{tikzpicture}[line cap=round,line join=round,>=triangle 45,x=1.0cm,y=1.0cm]
\draw [line width=1.pt] (1.34,0.513333333333335) circle (1.6226179121681383cm);
\draw [line width=1.pt] (4.446666666666667,0.32666666666666827) circle (2.314975401837157cm);
\draw [line width=1.pt] (-0.26501383145183266,0.7517017241430133)-- (6.7605431543523995,0.39798784757567285);
\draw [line width=1.pt] (-0.26501383145183266,0.7517017241430133)-- (2.5266666666666677,1.62);
\draw [line width=1.pt] (2.5266666666666677,1.62)-- (6.7605431543523995,0.39798784757567285);
\draw [line width=1.pt] (2.151654340410889,0.6300303649648458)-- (2.3856174482273413,-0.7274619925973483);
\draw [line width=1.pt] (2.9608391005397086,0.5892905518877172)-- (2.3856174482273413,-0.7274619925973483);
\draw[color=black] (2.5266666666666677,1.8533333333333344) node {$A$};
\draw[color=black] (2.3266666666666675,-0.8533333333333327) node {$D$};
\draw[color=black] (-0.4466666666666663,0.7466666666666676) node {$E$};
\draw[color=black] (1.993333333333334,0.5733333333333342) node {$F$};
\draw[color=black] (3.06,0.52) node {$G$};
\draw[color=black] (6.86,0.56) node {$H$};
\end{tikzpicture}
(GusievMIRclubsigma)
\end{problema}

\begin{problema}
Por el punto $A$ de la cuerda com\'un $AB$ de dos circunferencias se ha trazado
una recta que cruza la primera circunferencia en el punto $C$ y la segunda en el punto
$D$. La tangente a la primera circunferencia en el punto $C$ y la tangente a la segunda
circunferencia en el punto $D$ concurren en el punto $M$. Demostrar que los puntos $M$,
$C$, $B$ y $D$ yacen en una circunferencia.

\begin{tikzpicture}[line cap=round,line join=round,>=triangle 45,x=1.0cm,y=1.0cm]
\draw [line width=1.pt] (0.7533333333333334,0.6333333333333332) circle (1.8857123617113807cm);
\draw [line width=1.pt] (3.86,0.4466666666666666) circle (2.3544190130239957cm);
\draw [line width=1.pt] (-0.8938882191430002,1.551250687003352)-- (1.9025873787892702,-0.8617005291976386);
\draw [line width=1.pt] (1.9025873787892702,-0.8617005291976386)-- (5.139529313030776,2.4230506440313566);
\draw [line width=1.pt] (-0.8938882191430002,1.551250687003352)-- (5.139529313030776,2.4230506440313566);
\draw [line width=1.pt] (-0.8938882191430002,1.551250687003352)-- (1.0627160912475522,5.062417326197354);
\draw [line width=1.pt] (1.0627160912475522,5.062417326197354)-- (5.139529313030776,2.4230506440313566);
\draw[color=black] (2.0666666666666678,2.3366666666666673) node {$A$};
\draw[color=black] (1.8266666666666675,-1.133333333333328) node {$C$};
\draw[color=black] (-1.1333333333333335,1.8166666666666673) node {$D$};
\draw[color=black] (5.306666666666669,2.6166666666666676) node {$B$};
\draw[color=black] (1.0066666666666673,5.356666666666667) node {$M$};
\end{tikzpicture}
(GusievMIRclubsigma)
\end{problema}

\begin{problema}
En el punto $A$ dos circunferencias son tangentes exteriores, y $BC$ es su tangente
com\'un externa. Demuestre que $\angle BAC=90^\circ$.


\begin{tikzpicture}[line cap=round,line join=round,>=triangle 45,x=1.0cm,y=1.0cm]
\draw [line width=1.pt] (5.5066666666666695,-0.1733333333333326) circle (3.147368913381952cm);
\draw [line width=1.pt] (0.1066666666666669,0.12666666666666743) circle (2.2609579998140346cm);
\draw [line width=1.pt] (-0.1690468123184532,2.3864805321521656)-- (2.3641435897435907,0.0012512820512828315);
\draw [line width=1.pt] (2.3641435897435907,0.0012512820512828315)-- (5.172841041434405,2.9562819032191383);
\draw [line width=1.pt] (-0.1690468123184532,2.3864805321521656)-- (5.172841041434405,2.9562819032191383);
\draw[color=black] (2.5666666666666678,-0.04333333333333256) node {$A$};
\draw[color=black] (5.166666666666669,3.3766666666666674) node {$B$};
\draw[color=black] (-0.2533333333333332,2.7966666666666673) node {$C$};
\end{tikzpicture}

(GusievMIRclubsigma)
\end{problema}

\newpage


\begin{problema}
En el plano hay 5 puntos, $A, B, C, D$ y $E$ situados de tal manera que $ABC$ es
un tri\'angulo equil\'atero, $B$ es el punto medio de $AD$ y $E$ es el punto m\'as alejado
de $C$ para el cual los segmentos $DE$ y $AB$ miden lo mismo. ?`Cu\'anto mide el \'angulo
$\angle BED$?


%\includegraphics[width=0.55\textwidth]{figuras/MLPS_Niv_Olimp_P90}
(P90 Nivel Ol\'impico MatPreolimpicasMLPS)
\end{problema}

\begin{problema}
Sean $C$ y $C'$ dos c\'irculos con centros en $O$ y $O'$, respectivamente,
y tales que se intersectan en dos puntos distintos $P$ y $Q$. Sea $\mathcal{L}$
una recta por $P$ que intersecta a $C$ y $C'$ en dos puntos $B$ y $B'$, respectivamente.
Sea $A$ el circuncentro de $BB'Q$. Probar que $A$ est\'a en el circunc\'irculo de $O$,
$O'$ y $Q$.
%\includegraphics[width=0.55\textwidth]{figuras/MLPS_Niv_Final_P12}
(P12 Nivel Final MatPreolimpicasMLPS)


\end{problema}

\begin{problema}
Sean $A$ y $B$ dos puntos fijos en el plano y sea $\mathcal{L}$ una recta que
pasa por $A$ pero no por $B$. Para $P$ y $Q$ puntos de $\mathcal{L}$ (distintos de $A$)
sean $O_P$ y $O_Q$ los centros de las circunferencias circunscritas a $APB$ y $AQB$,
respectivamente. Demostrar que los \'angulos $\angle O_P PB$ y $\angle O_Q QB$ son
iguales.

\begin{tikzpicture}[line cap=round,line join=round,>=triangle 45,x=3*1.0cm,y=3*1.0cm]
\draw [line width=1.pt] (0.,0.)-- (1.3902748774292129,-0.009084041275863961);
\draw [line width=1.pt] (0.4332940580495208,0.31524833890999726) circle (3*0.5358407001400081cm);
\draw [line width=1.pt] (0.6935451428961725,-0.24823629795108226) circle (3*0.7366316072877479cm);
\draw[color=black] (-0.06930581431126241,0.006980526086059501) node {$A$};
\draw[color=black] (1.0047252607430497,0.5118669288893684) node {$B$};
\draw[color=black] (1.440763517709544,0.03910966080990643) node {$Q$};
\draw[color=black] (0.899158103793267,-0.05268786697251336) node {$P$};
\draw [fill=black] (0.4332940580495208,0.31524833890999726) circle (1.0pt);
\draw[color=black] (0.4309907121029257,0.2202439747094719) node {$O_Q$};
\draw [fill=black] (0.6935451428961725,-0.24823629795108226) circle (1.0pt);
\draw[color=black] (0.7017934190610643,-0.3243108211524097) node {$O_P$};
\end{tikzpicture}

%\includegraphics[width=0.65\textwidth]{figuras/MLPS_Niv_Final_P14}
(P14 Nivel Final MatPreolimpicasMLPS)

\end{problema}


\begin{problema}
Sea $ABC$ un tri\'angulo y $AD$ la altura sobre el lado $BC$. Tomando a $D$ como
centro y a $AD$ como radio, se traza una circunferencia que corta a la recta $AB$ en
$P$, y corta a la recta $AC$ en $Q$. Muestra que el tri\'angulo $AQP$ es semejante
al tri\'angulo $ABC$.

\begin{tikzpicture}[line cap=round,line join=round,>=triangle 45,x=4*1.0cm,y=4*1.0cm]
\draw [line width=1.pt] (0.863807405686497,-0.14503467816191534) circle (4*0.4939628904042216cm);
\draw [line width=1.pt] (0.23362602737072305,-0.14678033294949364)-- (1.8905714038434012,-0.14219045656037266);
\draw [line width=1.pt] (0.23362602737072305,-0.14678033294949364)-- (0.862439092680299,0.3489263170755732);
\draw [line width=1.pt] (0.862439092680299,0.3489263170755732)-- (1.8905714038434012,-0.14219045656037266);
\draw [line width=1.pt] (0.862439092680299,0.3489263170755732)-- (0.863807405686497,-0.14503467816191534);
\draw[color=black] (0.853259339902057,0.4338390302743115) node {$A$};
\draw[color=black] (0.16477788153390815,-0.13989551836581215) node {$B$};
\draw[color=black] (1.936470167734611,-0.1628449003114171) node {$C$};
\draw[color=black] (0.857849216291178,-0.19038415864614305) node {$D$};
\draw[color=black] (0.34378306070962683,0.03451978442078544) node {$Q$};
\draw[color=black] (1.2938874732576722,0.227294592763867) node {$P$};
\end{tikzpicture}

(P1 OMM 2009)%23aOMM
\end{problema}

\begin{problema}
En el tri\'angulo is\'osceles $ABC$ se tiene que $\angle C>90$. Sean $O$ el circuncentro del tri\'angulo,
$I$ el incentro y $D$ el punto sobre $BC$ de tal manera que las l\'ineas $OD$ y $BI$ son perpendiculares.
Prueba que $ID$ y $AC$ son paralelas.

\begin{tikzpicture}[line cap=round,line join=round,>=triangle 45,x=4*1.0cm,y=4*1.0cm]
\draw [line width=1.pt] (0.7499871211468343,-0.879578056618647) circle (4*0.81082562228421cm);
\draw [line width=1.pt] (0.04085121902764139,-0.4864311857444469)-- (1.4591230232660277,-0.4864311857444469);
\draw [line width=1.pt] (0.04085121902764139,-0.4864311857444469)-- (0.7499871211468345,-0.06875243433443678);
\draw [line width=1.pt] (0.7499871211468345,-0.06875243433443678)-- (1.4591230232660277,-0.4864311857444469);
\draw [line width=1.pt] (0.7499871211468343,-0.879578056618647)-- (0.9404463682306589,-0.18093231449384145);
\draw [line width=1.pt] (1.4591230232660277,-0.4864311857444469)-- (0.7499871211468344,-0.2931121946532464);
\draw[color=black] (-0.018817174030931506,-0.45200711282603945) node {$A$};
\draw[color=black] (1.5096116635463588,-0.43823748365867643) node {$B$};
\draw[color=black] (0.7431023065631532,0.01616027886430148) node {$C$};
\draw [fill=black] (0.7499871211468343,-0.879578056618647) circle (1.0pt);
\draw[color=black] (0.6650744079480964,-0.8421466059013235) node {$O$};
\draw [fill=black] (0.7499871211468344,-0.2931121946532464) circle (1.0pt);
\draw[color=black] (0.6880237898937013,-0.2775918100394419) node {$I$};
\draw[color=black] (0.9771860024083238,-0.14153601280932821) node {$D$};
\end{tikzpicture}

(PRE 1999)
\end{problema}

\begin{problema}
.
\end{problema}

\begin{problema}
.
\end{problema}

\begin{problema}
.
\end{problema}



\begin{tikzpicture}[line cap=round,line join=round,>=triangle 45,x=3*1.0cm,y=3*1.0cm]
\draw [line width=1.pt] (-0.13330614542676317,-0.4291592113950537) circle (3*0.8418908652468511cm);
\draw [line width=1.pt] (-0.5296296296296298,1.234074074074074)-- (-1.1429629629629634,-1.1659259259259263);
\draw [line width=1.pt] (-1.1429629629629634,-1.1659259259259263)-- (1.6748148148148159,-1.1748148148148152);
\draw [line width=1.pt] (1.6748148148148159,-1.1748148148148152)-- (-0.5296296296296298,1.234074074074074);
\draw [line width=1.pt] (-0.5296296296296298,1.234074074074074)-- (-0.5372066356335443,-1.1678368291667132);
\draw [line width=1.pt] (1.6748148148148159,-1.1748148148148152)-- (-0.9723517166488544,-0.4983167012185435);
\draw [line width=1.pt] (-1.1429629629629634,-1.1659259259259263)-- (0.3861267649508784,0.23338865902843856);
\draw [line width=1.pt] (0.26592592592592623,-1.1703703703703707)-- (0.268834513075749,-0.24834824387656002);
\draw [line width=1.pt] (0.268834513075749,-0.24834824387656002)-- (0.572592592592593,0.02962962962962945);
\draw [line width=1.pt] (0.268834513075749,-0.24834824387656002)-- (-0.8362962962962965,0.03407407407407392);
\draw[color=black] (-0.6007407407407409,1.3285185185185187) node {$A$};
\draw[color=black] (-1.2585185185185193,-1.2059259259259263) node {$B$};
\draw[color=black] (1.7725925925925938,-1.152592592592593) node {$C$};
\draw [fill=black] (-0.8362962962962965,0.03407407407407392) circle (1.0pt);
\draw[color=black] (-0.9474074074074077,0.10962962962962948) node {$M_3$};
\draw [fill=black] (0.572592592592593,0.02962962962962945) circle (1.0pt);
\draw[color=black] (0.6348148148148154,0.1718518518518517) node {$M_2$};
\draw [fill=black] (0.26592592592592623,-1.1703703703703707) circle (1.0pt);
\draw[color=black] (0.30592592592592627,-1.232592592592593) node {$M_1$};
\draw [fill=black] (0.3861267649508784,0.23338865902843856) circle (1.0pt);
\draw[color=black] (0.4214814814814819,0.37629629629629624) node {$E$};
\draw [fill=black] (-0.9723517166488544,-0.4983167012185435) circle (1.0pt);
\draw[color=black] (-1.0896296296296302,-0.47703703703703726) node {$F$};
\draw [fill=black] (-0.5372066356335443,-1.1678368291667132) circle (1.0pt);
\draw[color=black] (-0.5296296296296297,-1.232592592592593) node {$D$};
\draw [fill=black] (-0.5354468039292756,-0.6099701789135471) circle (1.0pt);
\draw[color=black] (-0.41407407407407415,-0.573703703703704) node {$H$};
\draw [fill=black] (0.268834513075749,-0.24834824387656002) circle (1.0pt);
\draw[color=black] (0.1370370370370373,-0.28148148148148167) node {$O$};
\draw [fill=black] (-0.5325382167794527,0.31205194758026333) circle (1.0pt);
\draw [fill=black] (0.5696840054427702,-0.8923924968641812) circle (1.0pt);
\draw [fill=black] (-0.8392048834461193,-0.8879480524197367) circle (1.0pt);
\end{tikzpicture}

\begin{tikzpicture}[line cap=round,line join=round,>=triangle 45,x=4*1.0cm,y=4*1.0cm]
\draw [line width=1.pt] (0.05111111111111182,0.7718518518518515)-- (-0.7488888888888887,-0.6859259259259256);
\draw [line width=1.pt] (-0.7488888888888887,-0.6859259259259256)-- (1.4496296296296314,-0.68);
\draw [line width=1.pt] (1.4496296296296314,-0.68)-- (0.05111111111111182,0.7718518518518515);
\draw [line width=1.pt] (0.35037037037037133,-0.6829629629629628)-- (0.05111111111111182,0.7718518518518515);
\draw [line width=1.pt] (0.05111111111111182,0.7718518518518515)-- (0.05503459054003266,-0.6837590162778693);
\draw [line width=1.pt] (-0.7488888888888887,-0.6859259259259256)-- (0.3944533835691012,0.4154160181560577);
\draw [line width=1.pt] (0.05295851271490794,0.0864658568434467)-- (0.34944666956847326,-0.3402699654587605);
\draw [line width=1.pt] (1.4496296296296314,-0.68)-- (-0.23753140041401427,0.24588105307273342);
\draw [line width=1.pt] (0.35037037037037133,-0.6829629629629628)-- (0.34944666956847326,-0.3402699654587605);
\draw [line width=1.pt] (0.7503703703703715,0.04592592592592572)-- (0.34944666956847326,-0.3402699654587605);
\draw[color=black] (0.09259259259259334,0.8814814814814811) node {$A$};
\draw[color=black] (-0.8318518518518517,-0.6948148148148146) node {$B$};
\draw[color=black] (1.5088888888888907,-0.7125925925925922) node {$C$};
\draw [fill=black] (0.05295851271490794,0.0864658568434467) circle (1.pt);
\draw[color=black] (-0.043703703703703065,0.09111111111111105) node {$H$};
\draw[color=black] (0.3533333333333342,-0.736296296296296) node {$M$};
\draw [fill=black] (0.34944666956847326,-0.3402699654587605) circle (1.pt);
\draw[color=black] (0.41259259259259357,-0.3392592592592591) node {$O$};
\draw [fill=black] (0.2506172839506183,-0.19802469135802492) circle (1.pt);
\draw[color=black] (0.29407407407407493,-0.12) node {$G$};
\end{tikzpicture}



\begin{tikzpicture}[line cap=round,line join=round,>=triangle 45,x=3*1.0cm,y=3*1.0cm]
\draw [line width=1.pt] (0.29056581856682023,-0.3269839728432721) circle (3*1.1246238155739503cm);
\draw [line width=1.pt] (-0.5474074074074076,-1.0770370370370375)-- (1.1177777777777793,-1.0888888888888884);
\draw [line width=1.pt] (1.1177777777777793,-1.0888888888888884)-- (0.05111111111111182,0.7718518518518515);
\draw [line width=1.pt] (-0.5474074074074076,-1.0770370370370375)-- (0.05111111111111182,0.7718518518518515);
\draw [dashed, line width=1.pt] (1.3704976124706072,-0.6408725630657572)-- (-0.23768623011828507,-0.12027458838153174);
\draw [dashed, line width=1.pt] (1.3704976124706072,-0.6408725630657572)-- (0.9869817222381007,-0.8607224364474491);
%\draw [line width=1.pt] (-0.23768623011828507,-0.12027458838153174)-- (1.367296244086496,-1.0906648210334342);
\draw [line width=1.pt] (-0.8406626189286864,-1.0749498113322598)-- (1.6213083527394943,-1.0924727363975124);
\draw [dashed, line width=1.pt] (1.3748561424656531,-0.6408725630657572)-- (1.367296244086496,-1.0906648210334342);
\draw[color=black] (0.03925925925925945,0.9362962962962963) node {$A$};
\draw[color=black] (-0.6451851851851853,-1.1348148148148152) node {$B$};
\draw[color=black] (1.2037037037037046,-1.1348148148148152) node {$C$};
\draw [fill=black] (1.3704976124706072,-0.6408725630657572) circle (1.0pt);
\draw[color=black] (1.505925925925927,-0.5748148148148151) node {$P$};
\draw [fill=black] (-0.23768623011828507,-0.12027458838153174) circle (1.0pt);
\draw[color=black] (-0.3518518518518519,-0.04148148148148166) node {$C'$};
\draw [fill=black] (0.9869817222381007,-0.8607224364474491) circle (1.0pt);
\draw[color=black] (1.0348148148148155,-0.7170370370370374) node {$B'$};
\draw [fill=black] (1.367296244086496,-1.0906648210334342) circle (1.0pt);
\draw[color=black] (1.3725925925925935,-1.1614814814814818) node {$A'$};
\end{tikzpicture}

