\setcounter{chapter}{9}

\chapter{Recursiones, Matrices y Gráficas}

Consideremos la sucesión de Fibonacci $F_0=0$, $F_1=1$, $F_{n+1}=F_{n}+F_{n-1}$, y las siguientes preguntas:

\begin{teorema}\label{Teo:Fibo}
Si $\phi_+:=\frac{1+\sqrt{5}}{2}$ y $\phi_-:=\frac{1-\sqrt{5}}{2}$, entonces 
\begin{equation}\label{eq:FibonacciGeneral}
F_n=\frac{(\phi_+)^n-(\phi_-)^n}{\sqrt{5}}
\end{equation}
\end{teorema}

\begin{itemize}
    \item P1. ¿De donde viene la fórmula \ref{eq:FibonacciGeneral}?
\end{itemize}

\begin{ejercicio}
Muestra que $\phi_+$ y $\phi_-$ son las dos raíces del polinomio $x^2-x-1$.

Muestra que $$\phi_+^2=1+\phi_+,\quad \phi_-^2=1+\phi_-,\quad \phi_+\phi_-=-1.$$
\end{ejercicio}

Supongamos que se cambian las condiciones iniciales de los números de Fibonacci:
$$\hat F_0=x,\quad  \hat F_1=y,\quad  \hat F_n=\hat F_{n-1}+\hat F_{n-2} \quad (n\geq 1).$$

\begin{itemize}
    \item P2. ¿Cómo cambian los primeros números de la recursión si cambiamos las condiciones iniciales  $(x,y)$?

    \item P3. ¿Cómo cambia la fórmula del teorema \ref{Teo:Fibo} al cambiarse las condiciones iniciales  $(x,y)$?
\end{itemize}

En esta sesión de ejercicios responderemos a las preguntas anteriores. Comenzamos con la segunda y la tercera pregunta, que tienen respuestas más o menos intuitivas.

\begin{ejercicio} Completa la siguiente tabla sobre números de Fibonacci con distintas condiciones iniciales

  \begin{center}
    \renewcommand{\arraystretch}{1.5}
    \begin{tabular}{llll}
      \hline
      $(x,y)$ & Primeros ocho términos $\hat F_0$ a $\hat F_7$ & Fórmula General \\
      \hline
      $(0,0)$ &  &  \\
      \hline
      $(1,0)$ &  &  \\
            \hline
      $(1,1)$ &  &  \\
            \hline
      $(0,-1)$ &  &  \\
            \hline
      $(3,5)$ &  &  \\
            \hline
      $(4,4)$ &  &  \\
            \hline
      $(7,8)$ &  &  \\
                  \hline
      $(2022,2202)$ &  &  \\
                  \hline
      $(\pi,2022)$ &  &  \\
                  \hline
    \end{tabular}
  \end{center}
\end{ejercicio}

A priori, la fórmula general del Teorema \ref{Teo:Fibo} nos resulta bastante misteriosa. Al involucrar raíces de cinco y la razón dorada $\phi$, ni siquiera es obvio que la fórmula arroja un número entero, en contraste con la fórmula recursiva. Si ya se conoce la fórmula, verificarla es un buen ejercicio de inducción.

\begin{ejercicio}
Demuestra el Teorema \ref{Teo:Fibo} usando inducción matemática.
\end{ejercicio}

Es un poco artificial suponer que a alguno de nosotros se nos ocurriría dicha fórmula de la nada. Sin embargo, la fórmula se deriva de una manera bastante natural y estándar si se ataca con con algunas herramientas básicas del álgebra superior (matrices y vectores de dimensión pequeña) 

\section{Multiplicación de matrices}

Una {\bf matriz} es un arreglo rectangular de números (reales o complejos), por ejemplo:
$$
\left(\begin{array}{ccc}
a_{11}&a_{12} \\
a_{21}&a_{22} 
\end{array}\right)
, \quad
\left(\begin{array}{ccc}
a_{11}&a_{12}&a_{13} \\
a_{21}&a_{22}&a_{23} \\
a_{31}&a_{32}&a_{33}  
\end{array}\right),\quad 
\left(\begin{array}{cc}
a_{11}&a_{12} \\
a_{21}&a_{22} \\
a_{31}&a_{32}  
\end{array}\right)
$$

Dos matrices cuadradas del mismo tamaño se pueden sumar y multiplicar de acuerdo a las siguientes reglas: Si

$$A=\left(\begin{array}{cc}
a_{11}&a_{12} \\
a_{21}&a_{22} 
\end{array}\right),\quad 
B=\left(\begin{array}{cc}
b_{11}&b_{12} \\
b_{21}&b_{22} 
\end{array}\right),$$
entonces $$A+B=\left(\begin{array}{cc}
a_{11}&a_{12} \\
a_{21}&a_{22} 
\end{array}\right)
+
\left(\begin{array}{ccc}
b_{11}&b_{12} \\
b_{21}&b_{22}
\end{array}\right)=
\left(\begin{array}{ccc}
a_{11}+b_{11}&a_{12}+b_{12} \\
a_{21}+b_{21}&a_{22}+b_{22}
\end{array}\right),$$
y
$$A\cdot B=\left(\begin{array}{cc}
a_{11}&a_{12} \\
a_{21}&a_{22} 
\end{array}\right)
\cdot
\left(\begin{array}{ccc}
b_{11}&b_{12} \\
b_{21}&b_{22}
\end{array}\right)=
\left(\begin{array}{ccc}
a_{11}b_{11}+a_{12}b_{21}&a_{11}b_{12}+a_{12}b_{22} \\
a_{21}b_{11}+a_{22}b_{21}&a_{21}b_{12}+a_{22}b_{22}
\end{array}\right)$$


Para matrices de $3\times 3$ y $n\times n$ la regla es similar:
$$C=\left(\begin{array}{ccc}
c_{11}&c_{12}&c_{13} \\
c_{21}&c_{22}&c_{23} \\
c_{31}&c_{32}&c_{33}  
\end{array}\right),
\quad 
D=\left(\begin{array}{ccc}
d_{11}&d_{12}&d_{13} \\
d_{21}&d_{22}&d_{23} \\
d_{31}&d_{32}&d_{33}  
\end{array}\right),$$
entonces 
$$C+D=\left(\begin{array}{ccc}
c_{11}&c_{12}&c_{13} \\
c_{21}&c_{22}&c_{23} \\
c_{31}&c_{32}&c_{33}  
\end{array}\right)
+
\left(\begin{array}{ccc}
d_{11}&d_{12}&d_{13} \\
d_{21}&d_{22}&d_{23} \\
d_{31}&d_{32}&d_{33}  
\end{array}\right)= 
\left(\begin{array}{ccc}
c_{11}+d_{11}&c_{12}+d_{12}&c_{13}+d_{13} \\
c_{21}+d_{21}&c_{22}+d_{22}&c_{23}+d_{23} \\
c_{31}+d_{31}&c_{32}+d_{32}&c_{33}+d_{33}  
\end{array}\right)
,$$
La entrada $(i,j)$ del producto se obtiene apareando las entradas del i-ésimo renglón de $C$, $(c_{i1},c_{i2},c_{i3})$ con las entradas de la $j$-ésima columna de $D$, $(d_{1j},d_{2j}, d_{3j})$. Por ejemplo, la entrada $(2,3)$ da $c_{21}d_{13}+c_{22}d_{23}+c_{23}d_{33}$ y similarmente 
$$C\cdot 
D=\left(\begin{array}{ccc}
c_{11}&c_{12}&c_{13} \\
c_{21}&c_{22}&c_{23} \\
c_{31}&c_{32}&c_{33}  
\end{array}\right)\cdot
\left(\begin{array}{ccc}
d_{11}&d_{12}&d_{13} \\
d_{21}&d_{22}&d_{23} \\
d_{31}&d_{32}&d_{33}  
\end{array}\right)$$
$$
=\left(\begin{array}{ccc}
c_{11}d_{11}+c_{12}d_{21}+c_{13}d_{31}  &  c_{11}d_{12}+c_{12}d_{22}+c_{13}d_{32}  &  c_{11}d_{13}+c_{12}d_{23}+c_{13}d_{33} \\
c_{21}d_{11}+c_{22}d_{21}+c_{23}d_{31}  &  c_{21}d_{12}+c_{22}d_{22}+c_{23}d_{32}  &  c_{21}d_{13}+c_{22}d_{23}+c_{23}d_{33}   \\
c_{31}d_{11}+c_{32}d_{21}+c_{33}d_{31}  &  c_{31}d_{12}+c_{32}d_{22}+c_{33}d_{32}  &  c_{31}d_{13}+c_{32}d_{23}+c_{33}d_{33}  
\end{array}\right)
$$

De manera análoga se definen las sumas y los productos de matrices para $n\geq 3$. La entrada $(i,j)$ del producto $A\cdot B$ de dos matrices involucra únicamente apareamientos de las entradas del $i$-ésimo renglón de $A$ y las entradas de la $j$-ésima columna de $B$.

Como ocurre con variables en álgebra, se conviene que dos matrices escritas una detrás de la otra se están multiplicando y normalmente se omite el puntito, entendiendo que $AB=A\cdot B$ y $A^n=A\cdot A\cdot A\cdot \cdots A$.

El elemento neutro multiplicativo en las matrices de $n\times n$ es la matriz identidad $I_n$ que consiste de unos en la diagonal y ceros en el resto de las entradas, por ejemplo: 
$$ I_2=
\left(\begin{array}{cc}
1&0  \\
0&1
\end{array}\right),
\quad 
I_3=
\left(\begin{array}{ccc}
1&0&0  \\
0&1&0  \\
0&0&1
\end{array}\right),
\quad 
I_4=
\left(\begin{array}{cccc}
1&0&0&0  \\
0&1&0&0  \\
0&0&1&0  \\
0&0&0&1
\end{array}\right)
$$

\begin{ejercicio}
Verifica que $I_n^k=I_n$ y calcula los siguientes productos de matrices

\begin{enumerate}
    \item     $$\left(\begin{array}{cc}
1&0 \\
0&1 
\end{array}\right)
\left(\begin{array}{cc}
a_{11}&a_{12} \\
a_{21}&a_{22} 
\end{array}\right),
\quad
\left(\begin{array}{cc}
a_{11}&a_{12} \\
a_{21}&a_{22} 
\end{array}\right)
\left(\begin{array}{cc}
1&0 \\
0&1 
\end{array}\right)
$$
    \item $$\left(\begin{array}{ccc}
1&0&0  \\
0&1&0  \\
0&0&1
\end{array}\right)
\left(\begin{array}{ccc}
a_{11}& a_{12}& a_{13} \\
a_{21}& a_{22}& a_{23} \\
a_{31}& a_{32}& a_{33}  
\end{array}\right),
\quad
\left(\begin{array}{ccc}
a_{11}& a_{12}& a_{13} \\
a_{21}& a_{22}& a_{23} \\
a_{31}& a_{32}& a_{33}  
\end{array}\right)
\left(\begin{array}{ccc}
1&0&0  \\
0&1&0  \\
0&0&1
\end{array}\right)
$$
\end{enumerate}
       
\end{ejercicio}

El objetivo principal de esta sesion de ejercicios es entender las potencias de la matriz de Fibonacci:
$$\mathbb F:=\left(\begin{array}{cc}
0&1 \\
1&1 
\end{array}\right)$$
\begin{ejercicio}
Muestra que las potencias de la matriz de Fibonacci estan dadas por la fórmula

$$\mathbb F^n=\left(\begin{array}{cc}
0&1 \\
1&1 
\end{array}\right)^n
=
\left(\begin{array}{cc}
F_n&F_{n+1} \\
F_{n+1}&F_{n+2} 
\end{array}\right),
\quad
n\geq 0
$$
Donde $F_n$, $n\geq 0$ son los números de Fibonacci.

Sug: Inducción.
\end{ejercicio}

Las matrices diagonales son fáciles de multiplicar.

\begin{ejercicio}
Calcula los siguientes productos de matrices

$$\left(\begin{array}{cc}
a&0 \\
0&b 
\end{array}\right)
\left(\begin{array}{cc}
c&0 \\
0&d 
\end{array}\right)
,\quad
\left(\begin{array}{cc}
\lambda_1&0 \\
0&\lambda_2 
\end{array}\right)^n$$
\end{ejercicio}

Recuerda que la suma y la multiplicación de dos números complejos $z=a+b\mathrm i$ y $w=c+d\mathrm i$ se define como $$z+w=(a+c)+(b+d)\mathrm i, \quad z w=(ac-bd)+(ad+bc)\mathrm i$$

El siguiente ejercicio muestra que el campo de números complejos está escondido dentro del anillo de matrices reales de $2\times 2$.
\begin{ejercicio}
Calcula los siguientes productos y sumas de matrices.
$$\left(\begin{array}{cc}
0&1 \\
-1&0 
\end{array}\right)
\left(\begin{array}{cc}
0&1 \\
-1&0 
\end{array}\right) 
,\quad
\left(\begin{array}{cc}
a&b \\
-b&a 
\end{array}\right)
+
\left(\begin{array}{cc}
c&d \\
-d&c 
\end{array}\right)
,\quad
\left(\begin{array}{cc}
a&b \\
-b&a 
\end{array}\right)
\left(\begin{array}{cc}
c&d \\
-d&c 
\end{array}\right)$$
\end{ejercicio}


\begin{ejercicio}
En particular, los complejos de norma $1$ del ejercicio anterior corresponden a {\bf matrices de rotación} reales. Calcula:
$$\left(\begin{array}{cc}
\cos\alpha&\sen\alpha \\
-\sen\alpha&\cos\alpha 
\end{array}\right)
\left(\begin{array}{cc}
\cos\beta&\sen\beta \\
-\sen\beta&\cos\beta 
\end{array}\right).
$$

\end{ejercicio}



\begin{ejercicio}
La suma y la multiplicación de complejos son conmutativas. Por el contrario la multiplicación de matrices no es conmutativa en general. Verifica lo anterior calculando:
$$\left(\begin{array}{cc}
1&2 \\
3&4 
\end{array}\right)
\left(\begin{array}{cc}
5&6 \\
7&8 
\end{array}\right),
\quad
\left(\begin{array}{cc}
5&6 \\
7&8 
\end{array}\right)
\left(\begin{array}{cc}
1&2 \\
3&4 
\end{array}\right)
.
$$
\end{ejercicio}

\begin{ejercicio}
¿Cuántas matrices distintas de $n\times n$ existen con exactamente $n$ unos y $n^2-n$ ceros, de tal forma que haya exactamente un uno en cada columna y en cada fila?
\end{ejercicio}

A las matrices como las del ejercicio anterior se les llama {\bf matrices de permutación} (más adelante, cuando apliquemos matrices a vectores veremos por qué se les llama así).

\begin{ejercicio}
Calcula las potencias de las siguientes matrices de permutación:

$$\left(\begin{array}{ccc}
0&1&0 \\
1&0&0 \\
0&0&1
\end{array}\right)
,\quad 
\left(\begin{array}{ccc}
0&1&0 \\
0&0&1 \\
1&0&0  
\end{array}\right)
,\quad 
\left(\begin{array}{ccccc}
0 & 1 & 0 & 0 & 0 \\
0 & 0 & 1 & 0 & 0 \\
0 & 0 & 0 & 1 & 0 \\
0 & 0 & 0 &  0 & 1 \\
1 & 0 & 0 & 0 & 0
\end{array}\right)
,\quad 
\left(\begin{array}{ccccc}
0 & 0 & 0 & 0 & 1 \\
0 & 0 & 1 & 0 & 0 \\
0 & 1 & 0 & 0 & 0 \\
1 & 0 & 0 & 0 & 0 \\
0 & 0 & 0 & 1 & 0
\end{array}\right)
$$
\end{ejercicio}

\begin{ejercicio}
Para las matrices de permutación del ejercicio anterior. ¿Puedes encontrar los inversos multiplicativos?
\end{ejercicio}


Para verificar productos de matrices de forma rápida, se pueden ingresar las matrices en GeoGebra renglón por renglón, separados por comas.

Por ejemplo, para ingresar las matrices 
$$A=\left(\begin{array}{cc}
a& b \\
c& d 
\end{array}\right),
\quad 
B=\left(\begin{array}{ccc}
a& b& c \\
d& e& f \\
g& h& i  
\end{array}\right),$$
en GeoGebra se escribiría $\{\{a,b\},\{c,d\}\}$ para la matriz $A$ y  $\{\{a,b,c\},\{d,e,f\},\{g,h,i\}\}$ para la matriz $B$.

\begin{ejercicio}
Utiliza Geogebra para calcular la veinteava potencia de la matriz de Fibonacci .
\end{ejercicio}
%geogebra:e8q8rtrr
Antes de continuar con el problema de entender mejor la matriz de Fibonacci, revisaremos otro tipo de matrices que describen la estructura de una gráfica.
\newpage

\section{Matrices de adyacencia}


\begin{definicion}
La {\bf matriz de adyacencia} de una gráfica $G=(V,A)$ es la matriz de unos y ceros que tiene ceros en la diagonal y un $1$ en las entradas $(i,j)$ y $(j,i)$ si y solo si la arista $v_iv_j\in A$. 
\end{definicion}

\begin{tikzpicture}[line cap=round,line join=round,>=triangle 45,x=.7*1.0cm,y=.7*1.0cm]
\draw [line width=1.pt] (4.16,1.18)-- (3.16,-0.78);
\draw [line width=1.pt] (4.16,1.18)-- (2.62,2.92);
\draw [line width=1.pt] (4.16,1.18)-- (-0.32,0.08);
\draw [line width=1.pt] (-1.92,3.)-- (0.54,3.82);
\draw [line width=1.pt] (0.54,3.82)-- (-0.32,0.08);
\draw [line width=1.pt] (-0.32,0.08)-- (2.62,2.92);
\draw [line width=1.pt] (0.54,3.82)-- (2.62,2.92);
\draw [fill=black] (0.54,3.82) circle (1.5pt);
\draw[color=black] (0.56,4.27) node {$v_2$};
\draw [fill=black] (-0.32,0.08) circle (1.5pt);
\draw[color=black] (-0.6,-0.13) node {$v_3$};
\draw [fill=black] (3.16,-0.78) circle (1.5pt);
\draw[color=black] (3.08,-1.27) node {$v_6$};
\draw [fill=black] (4.16,1.18) circle (1.5pt);
\draw[color=black] (4.42,1.47) node {$v_5$};
\draw [fill=black] (2.62,2.92) circle (1.5pt);
\draw[color=black] (2.98,3.21) node {$v_4$};
\draw [fill=black] (-1.92,3.) circle (1.5pt);
\draw[color=black] (-2.3,3.17) node {$v_1$};
\draw[color=black] (.5,-3.5) node 
{$A_{G_1}=\left(\begin{array}{cccccc}
0 & 1 & 0 & 0 & 0 & 0 \\
1 & 0 & 1 & 1 & 0 & 0\\
0 & 1 & 0 & 1 & 1 & 0\\
0 & 1 & 1 & 0 & 1 & 0\\
0 & 0 & 1 & 1 & 0 & 1\\
0 & 0 & 0 & 0 & 1 & 0
\end{array}\right)
$};
\end{tikzpicture}
\begin{tikzpicture}[line cap=round,line join=round,>=triangle 45,x=1.0cm,y=1.0cm]
\draw [line width=1.pt] (-1.5,2.52)-- (1.88,0.3);
\draw [line width=1.pt] (1.88,0.3)-- (-0.24,3.66);
\draw [line width=1.pt] (-0.24,3.66)-- (-0.8,0.22);
\draw [line width=1.pt] (-0.8,0.22)-- (2.24,2.7);
\draw [line width=1.pt] (2.24,2.7)-- (-1.5,2.52);
\draw [fill=black] (-1.5,2.52) circle (1.5pt);
\draw[color=black] (-1.82,2.69) node {$v_1$};
\draw [fill=black] (-0.24,3.66) circle (1.5pt);
\draw[color=black] (-0.2,4.11) node {$v_2$};
\draw [fill=black] (2.24,2.7) circle (1.5pt);
\draw[color=black] (2.52,2.87) node {$v_3$};
\draw [fill=black] (1.88,0.3) circle (1.5pt);
\draw[color=black] (2.16,0.15) node {$v_4$};
\draw [fill=black] (-0.8,0.22) circle (1.5pt);
\draw[color=black] (-0.96,-0.01) node {$v_5$};
\draw[color=black] (.5,-1.5) node {$A_{G_2}=\left(\begin{array}{ccccc}
0 & 0 & 1 & 1 & 0 \\
0 & 0 & 0 & 1 & 1 \\
1 & 0 & 0 & 0 & 1 \\
1 & 1 & 0 & 0 & 0 \\
0 & 1 & 1 & 0 & 0
\end{array}\right)
$};
\end{tikzpicture}
\begin{tikzpicture}[line cap=round,line join=round,>=triangle 45,x=1.0cm,y=1.0cm]
\draw [line width=1.pt] (0.,0.)-- (3.22,1.14);
\draw [line width=1.pt] (0.02,1.58)-- (3.16,2.52);
\draw [line width=1.pt] (0.02,1.58)-- (3.28,-0.34);
\draw [line width=1.pt] (0.,0.)-- (3.16,2.52);
\draw [fill=black] (0.,0.) circle (1.5pt);
\draw[color=black] (-0.3,0.03) node {$v_1$};
\draw [fill=black] (0.02,1.58) circle (1.5pt);
\draw[color=black] (-0.28,1.67) node {$v_2$};
\draw [fill=black] (3.16,2.52) circle (1.5pt);
\draw[color=black] (3.44,2.63) node {$v_3$};
\draw [fill=black] (3.22,1.14) circle (1.5pt);
\draw[color=black] (3.48,1.23) node {$v_4$};
\draw [fill=black] (3.28,-0.34) circle (1.5pt);
\draw[color=black] (3.54,-0.27) node {$v_5$};
\draw[color=black] (1.5,-2.1) node {$A_{G_3}=\left(\begin{array}{ccccc}
0 & 0 & 1 & 1 & 0 \\
0 & 0 & 1 & 0 & 1 \\
1 & 1 & 0 & 0 & 0 \\
1 & 0 & 0 & 0 & 0 \\
0 & 1 & 0 & 0 & 0
\end{array}\right)
$};
\end{tikzpicture}

\begin{ejercicio} 
Utiliza GeoGebra para calcular las primeras cuatro potencias de las matrices de adyacencia anteriores.
\end{ejercicio}

\begin{ejercicio} 
Demuestra que las entradas de la diagonal de la matriz $(A_G)^2$ son los grados de los vértices.
\end{ejercicio}

\begin{ejercicio}
Demuestra que la entrada $(i,j)$ de $A_G^k$ cuenta el número de caminos de tamaño $k$ en $G$ que van desde el vértice $v_i$ al vértice $v_j$.

En particular las entradas de la diagonal $(i,i)$ cuentan el número de ciclos de tamaño $k$ desde $v_i$ 
\end{ejercicio}

\begin{tikzpicture}[line cap=round,line join=round,>=triangle 45,x=.7*1.0cm,y=.7*1.0cm]
\draw [line width=1.pt] (4.16,1.18)-- (3.16,-0.78);
\draw [line width=1.pt] (4.16,1.18)-- (2.62,2.92);
\draw [line width=1.pt] (4.16,1.18)-- (-0.32,0.08);
\draw [line width=1.pt] (-1.92,3.)-- (0.54,3.82);
\draw [line width=1.pt] (0.54,3.82)-- (-0.32,0.08);
\draw [line width=1.pt] (-0.32,0.08)-- (2.62,2.92);
\draw [line width=1.pt] (0.54,3.82)-- (2.62,2.92);
\draw [fill=black] (0.54,3.82) circle (1.5pt);
\draw[color=black] (0.56,4.27) node {$v_2$};
\draw [fill=black] (-0.32,0.08) circle (1.5pt);
\draw[color=black] (-0.6,-0.13) node {$v_3$};
\draw [fill=black] (3.16,-0.78) circle (1.5pt);
\draw[color=black] (3.08,-1.27) node {$v_6$};
\draw [fill=black] (4.16,1.18) circle (1.5pt);
\draw[color=black] (4.42,1.47) node {$v_5$};
\draw [fill=black] (2.62,2.92) circle (1.5pt);
\draw[color=black] (2.98,3.21) node {$v_4$};
\draw [fill=black] (-1.92,3.) circle (1.5pt);
\draw[color=black] (-2.3,3.17) node {$v_1$};
\draw[color=black] (.5,-3.5) node 
{$(A_{G_1})^2=\left(\begin{array}{cccccc}
1 & 0 & 1 & 1 & 0 & 0 \\
0 & 3 & 1 & 1 & 2 & 0\\
1 & 1 & 3 & 2 & 1 & 1\\
1 & 1 & 2 & 3 & 1 & 1\\
0 & 2 & 1 & 1 & 3 & 0\\
0 & 0 & 1 & 1 & 0 & 1
\end{array}\right)
$};
\end{tikzpicture}
\begin{tikzpicture}[line cap=round,line join=round,>=triangle 45,x=1.0cm,y=1.0cm]
\draw [line width=1.pt] (-1.5,2.52)-- (1.88,0.3);
\draw [line width=1.pt] (1.88,0.3)-- (-0.24,3.66);
\draw [line width=1.pt] (-0.24,3.66)-- (-0.8,0.22);
\draw [line width=1.pt] (-0.8,0.22)-- (2.24,2.7);
\draw [line width=1.pt] (2.24,2.7)-- (-1.5,2.52);
\draw [fill=black] (-1.5,2.52) circle (1.5pt);
\draw[color=black] (-1.82,2.69) node {$v_1$};
\draw [fill=black] (-0.24,3.66) circle (1.5pt);
\draw[color=black] (-0.2,4.11) node {$v_2$};
\draw [fill=black] (2.24,2.7) circle (1.5pt);
\draw[color=black] (2.52,2.87) node {$v_3$};
\draw [fill=black] (1.88,0.3) circle (1.5pt);
\draw[color=black] (2.16,0.15) node {$v_4$};
\draw [fill=black] (-0.8,0.22) circle (1.5pt);
\draw[color=black] (-0.96,-0.01) node {$v_5$};
\draw[color=black] (.5,-1.5) node {$(A_{G_2})^2=\left(\begin{array}{ccccc}
2 & 1 & 0 & 0 & 1 \\
1 & 2 & 1 & 0 & 0 \\
0 & 1 & 2 & 1 & 0 \\
0 & 0 & 1 & 2 & 1 \\
1 & 0 & 0 & 1 & 2
\end{array}\right)
$};
\end{tikzpicture}
\begin{tikzpicture}[line cap=round,line join=round,>=triangle 45,x=1.0cm,y=1.0cm]
\draw [line width=1.pt] (0.,0.)-- (3.22,1.14);
\draw [line width=1.pt] (0.02,1.58)-- (3.16,2.52);
\draw [line width=1.pt] (0.02,1.58)-- (3.28,-0.34);
\draw [line width=1.pt] (0.,0.)-- (3.16,2.52);
\draw [fill=black] (0.,0.) circle (1.5pt);
\draw[color=black] (-0.3,0.03) node {$v_1$};
\draw [fill=black] (0.02,1.58) circle (1.5pt);
\draw[color=black] (-0.28,1.67) node {$v_2$};
\draw [fill=black] (3.16,2.52) circle (1.5pt);
\draw[color=black] (3.44,2.63) node {$v_3$};
\draw [fill=black] (3.22,1.14) circle (1.5pt);
\draw[color=black] (3.48,1.23) node {$v_4$};
\draw [fill=black] (3.28,-0.34) circle (1.5pt);
\draw[color=black] (3.54,-0.27) node {$v_5$};
\draw[color=black] (1.5,-2.1) node {$(A_{G_3})^2=\left(\begin{array}{ccccc}
2 & 1 & 0 & 0 & 0 \\
1 & 2 & 0 & 0 & 0 \\
0 & 0 & 2 & 1 & 1 \\
0 & 0 & 1 & 1 & 0 \\
0 & 0 & 1 & 0 & 1
\end{array}\right)
$};
\end{tikzpicture}

%%%cubos

\begin{tikzpicture}[line cap=round,line join=round,>=triangle 45,x=.7*1.0cm,y=.7*1.0cm]
\draw [line width=1.pt] (4.16,1.18)-- (3.16,-0.78);
\draw [line width=1.pt] (4.16,1.18)-- (2.62,2.92);
\draw [line width=1.pt] (4.16,1.18)-- (-0.32,0.08);
\draw [line width=1.pt] (-1.92,3.)-- (0.54,3.82);
\draw [line width=1.pt] (0.54,3.82)-- (-0.32,0.08);
\draw [line width=1.pt] (-0.32,0.08)-- (2.62,2.92);
\draw [line width=1.pt] (0.54,3.82)-- (2.62,2.92);
\draw [fill=black] (0.54,3.82) circle (1.5pt);
\draw[color=black] (0.56,4.27) node {$v_2$};
\draw [fill=black] (-0.32,0.08) circle (1.5pt);
\draw[color=black] (-0.6,-0.13) node {$v_3$};
\draw [fill=black] (3.16,-0.78) circle (1.5pt);
\draw[color=black] (3.08,-1.27) node {$v_6$};
\draw [fill=black] (4.16,1.18) circle (1.5pt);
\draw[color=black] (4.42,1.47) node {$v_5$};
\draw [fill=black] (2.62,2.92) circle (1.5pt);
\draw[color=black] (2.98,3.21) node {$v_4$};
\draw [fill=black] (-1.92,3.) circle (1.5pt);
\draw[color=black] (-2.3,3.17) node {$v_1$};
\draw[color=black] (.5,-3.5) node 
{$(A_{G_1})^3=\left(\begin{array}{cccccc}
0 & 3 & 1 & 1 & 2 & 0 \\
3 & 2 & 6 & 6 & 2 & 2\\
1 & 6 & 4 & 5 & 6 & 1\\
1 & 6 & 5 & 4 & 6 & 1\\
2 & 2 & 6 & 6 & 2 & 3\\
0 & 2 & 1 & 1 & 3 & 0
\end{array}\right)
$};
\end{tikzpicture}
\begin{tikzpicture}[line cap=round,line join=round,>=triangle 45,x=1.0cm,y=1.0cm]
\draw [line width=1.pt] (-1.5,2.52)-- (1.88,0.3);
\draw [line width=1.pt] (1.88,0.3)-- (-0.24,3.66);
\draw [line width=1.pt] (-0.24,3.66)-- (-0.8,0.22);
\draw [line width=1.pt] (-0.8,0.22)-- (2.24,2.7);
\draw [line width=1.pt] (2.24,2.7)-- (-1.5,2.52);
\draw [fill=black] (-1.5,2.52) circle (1.5pt);
\draw[color=black] (-1.82,2.69) node {$v_1$};
\draw [fill=black] (-0.24,3.66) circle (1.5pt);
\draw[color=black] (-0.2,4.11) node {$v_2$};
\draw [fill=black] (2.24,2.7) circle (1.5pt);
\draw[color=black] (2.52,2.87) node {$v_3$};
\draw [fill=black] (1.88,0.3) circle (1.5pt);
\draw[color=black] (2.16,0.15) node {$v_4$};
\draw [fill=black] (-0.8,0.22) circle (1.5pt);
\draw[color=black] (-0.96,-0.01) node {$v_5$};
\draw[color=black] (.5,-1.5) node {$(A_{G_2})^3=\left(\begin{array}{ccccc}
0 & 1 & 3 & 3 & 1 \\
1 & 0 & 1 & 3 & 3 \\
3 & 1 & 0 & 1 & 3 \\
3 & 3 & 1 & 0 & 1 \\
1 & 3 & 3 & 1 & 0
\end{array}\right)
$};
\end{tikzpicture}
\begin{tikzpicture}[line cap=round,line join=round,>=triangle 45,x=1.0cm,y=1.0cm]
\draw [line width=1.pt] (0.,0.)-- (3.22,1.14);
\draw [line width=1.pt] (0.02,1.58)-- (3.16,2.52);
\draw [line width=1.pt] (0.02,1.58)-- (3.28,-0.34);
\draw [line width=1.pt] (0.,0.)-- (3.16,2.52);
\draw [fill=black] (0.,0.) circle (1.5pt);
\draw[color=black] (-0.3,0.03) node {$v_1$};
\draw [fill=black] (0.02,1.58) circle (1.5pt);
\draw[color=black] (-0.28,1.67) node {$v_2$};
\draw [fill=black] (3.16,2.52) circle (1.5pt);
\draw[color=black] (3.44,2.63) node {$v_3$};
\draw [fill=black] (3.22,1.14) circle (1.5pt);
\draw[color=black] (3.48,1.23) node {$v_4$};
\draw [fill=black] (3.28,-0.34) circle (1.5pt);
\draw[color=black] (3.54,-0.27) node {$v_5$};
\draw[color=black] (1.5,-2.1) node {$(A_{G_3})^3=\left(\begin{array}{ccccc}
0 & 0 & 3 & 2 & 1 \\
0 & 0 & 3 & 1 & 2 \\
3 & 3 & 0 & 0 & 0 \\
2 & 1 & 0 & 0 & 0 \\
1 & 2 & 0 & 0 & 0
\end{array}\right)
$};
\end{tikzpicture}

Sug: Inducción.

Una gráfica se llama {\bf bipartita} si se puede elegir una partición del conjunto de vértices en dos conjuntos (disjuntos) $V=V_1\cup V_2$, donde no hay aristas entre vértices de $V_1$ ni aristas entre vértices de $V_2$. Las únicas posibles aristas unen un elemento de cada conjunto de vértices.

\begin{ejercicio} 
Demuestra que una gráfica es bipartita si y solo si no tiene ciclos impares.
\end{ejercicio}

\newpage

\section{Matrices y vectores propios}

Se pueden multiplicar matrices rectangulares $A$ de $(n\times m)$ y $B$ $(p\times q)$ siempre y cuando $m=p$ (es decir, cuando el número de columnas de $A$ coincide con el número de filas de $B$). 

El resultado $AB$ es una matriz de $n\times q$ ($n$ filas y $q$ columnas).

Por ejemplo: 

$$\left(\begin{array}{ccc}
a_{11}&a_{12}&a_{13} \\
a_{21}&a_{22}&a_{23} 
\end{array}\right)
\left(\begin{array}{cccc}
b_{11}&b_{12}&b_{13}&b_{14} \\
b_{21}&b_{22}&b_{23}&b_{24} \\
b_{31}&b_{32}&b_{33}&b_{34}
\end{array}\right)$$ 
$$=
\left(\begin{array}{cccc}
a_{11}b_{11}+a_{12}b_{21}+a_{13}b_{31} & a_{11}b_{12}+a_{12}b_{22}+a_{13}b_{32} & a_{11}b_{13}+a_{12}b_{23}+a_{13}b_{33} & a_{11}b_{14}+a_{12}b_{24}+a_{13}b_{34} \\
a_{21}b_{11}+a_{22}b_{21}+a_{23}b_{31} & a_{21}b_{12}+a_{22}b_{22}+a_{23}b_{32} & a_{21}b_{13}+a_{22}b_{23}+a_{23}b_{33} & a_{21}b_{14}+a_{22}b_{24}+a_{23}b_{34}
\end{array}\right)$$
En esta sesión nos interesa principalmente el comportamiento de las potencias de una matriz, por lo que esto solamente tiene sentido con matrices cuadradas.

Sin embargo, un caso particular que sí que nos interesa es cuando $A$ es de $n\times n$ y $B$ es de $n\times 1$. En este caso decimos que $B$ es un {\bf vector}, y escribimos $$v=\left(\begin{array}{c}
    x_{1}\\
    x_{2}\\
    \vdots \\
    x_{n}
\end{array} \right)\quad \text{en lugar de } B=\left(\begin{array}{c}
    b_{11}\\
    b_{21}\\
    \vdots \\
    b_{n1}
\end{array} \right).$$

Una matriz aplicada a un vector es nuevamente un vector. La regla es la siguiente:

$$ \left(\begin{array}{cccc}
    a_{11}& a_{12} & \dots & a_{1n} \\
    a_{21}& a_{22} & \dots &  a_{2n}\\

    \vdots& \vdots &\ddots &  \\
    a_{n1}& a_{n2} & & a_{nn}
\end{array} \right)
\left(\begin{array}{c}
    x_{1}\\
    x_{2}\\
    \vdots \\
    x_{n}
\end{array} \right) 
=
\left(\begin{array}{c}
    a_{11}x_1+ a_{12}x_2 + \dots + a_{1n}x_n\\
    a_{21}x_1+ a_{22}x_2 + \dots + a_{2n}x_n\\
    \vdots \\
    a_{n1}x_1+ a_{n2}x_2 + \dots + a_{nn}x_n
\end{array} \right)
$$

Un vector se puede estirar/encoger multiplicando todas las entradas por un mismo factor, por ejemplo. 
$$
\lambda
\left(\begin{array}{c}
x \\
y
\end{array}\right)
=
\left(\begin{array}{c}
\lambda x \\
\lambda y
\end{array}\right),
\quad 
\lambda
\left(\begin{array}{c}
x \\
y \\
z
\end{array}\right)
=
\left(\begin{array}{c}
\lambda x \\
\lambda y \\
\lambda z
\end{array}\right)
$$

\begin{ejercicio}
Las matrices son {\bf transformaciones lineales}: Demuestra que $A(\alpha v+\beta w)=\alpha A(v)+\beta A(w)$. 
\end{ejercicio}

\begin{ejercicio}
Muestra que las potencias de la matriz de Fibonacci aplicadas al vector columna $(x,y)$ arrojan los números de Fibonacci con condiciones iniciales $\hat F_0=x$, $\hat F_1=y$, $\hat F_{n+1}=\hat F_n+\hat F_{n-1}$:
$$
\left(\begin{array}{cc}
0&1 \\
1&1
\end{array}\right)^n
\left(\begin{array}{c}
x \\
y
\end{array}\right)
=
\left(\begin{array}{c}
\hat F_n \\
\hat F_{n+1}
\end{array}\right)
$$
\end{ejercicio}

A veces una matriz solo estira o contrae algunos vectores. A este tipo de vectores se les llama {\bf vectores propios} o {\bf eigenvectores} (de la matriz):  $$Av=\lambda v$$. Al factor $\lambda$ (que puede ser negativo) con el que una matriz $A$ estira a uno de sus vector propios se le denomina el {\bf valor propio}  $\lambda$ o {\bf eigenvalor} (del vector propio/eigenvector $v$).

\begin{ejercicio}
Sea $A$ una matriz y $v$ un vector propio de $A$ con valor propio $\lambda$. Muestra que para cualquier número real $\alpha\neq 0$, el vector $\alpha v$ es nuevamente un vector propio con el mismo valor propio.   
\end{ejercicio}

El siguiente ejercicio es un botón de muestra de la utilidad de los vectores propios.

\begin{ejercicio}
Sea $A$ una matriz y $v$ un vector. 

Supón que $v_1$ y $v_2$ son vectores tales que $$v=\alpha v_1+\beta v_2,\quad Av_1=\lambda_1 v_1,\quad Av=\lambda_2 v_2.$$ Calcula $A^n(v)$. 
\end{ejercicio}


Nuestro objetivo es encontrar vectores propios para la matriz de Fibonacci. Pero antes revisemos qué sucede con matrices más sencillas.

\begin{ejercicio} Matrices diagonales
$$
\left(\begin{array}{cc}
\lambda_1&0 \\
0&\lambda_2
\end{array}\right)^n
\left(\begin{array}{c}
x \\
0
\end{array}\right),
\quad
\left(\begin{array}{cc}
\lambda_1&0 \\
0&\lambda_2
\end{array}\right)^n
\left(\begin{array}{c}
0 \\
y
\end{array}\right),
\quad
\left(\begin{array}{cc}
\lambda_1&0 \\
0&\lambda_2
\end{array}\right)^n
\left(\begin{array}{c}
x \\
y
\end{array}\right)
$$

$$\left(\begin{array}{ccc}
\lambda_1&0&0 \\
0&\lambda_2&0 \\
0&0&\lambda_3
\end{array}\right)
\left(\begin{array}{c}
x \\
0 \\
0
\end{array}\right),
\quad 
\left(\begin{array}{ccc}
\lambda_1&0&0 \\
0&\lambda_2&0 \\
0&0&\lambda_3
\end{array}\right)
\left(\begin{array}{c}
x \\
y \\
z
\end{array}\right)
$$

\end{ejercicio}

\begin{ejercicio}
Matrices de Permutación.
$$
\left(\begin{array}{cc}
0&1 \\
1&0 
\end{array}\right)
\left(\begin{array}{c}
x \\
y
\end{array}\right) 
,\quad
\left(\begin{array}{cc}
0&\pi \\
\pi&0 
\end{array}\right)
\left(\begin{array}{c}
1 \\
1
\end{array}\right)
,\quad
\left(\begin{array}{cc}
0&2 \\
2&0 
\end{array}\right)
\left(\begin{array}{c}
1 \\
-1
\end{array}\right) 
$$

$$\left(\begin{array}{ccc}
0&1&0 \\
1&0&0 \\
0&0&1
\end{array}\right)
\left(\begin{array}{c}
x \\
y \\
z
\end{array}\right),
\quad
\left(\begin{array}{ccc}
0&1&0 \\
1&0&0 \\
0&0&1
\end{array}\right)
\left(\begin{array}{c}
0 \\
0 \\
1
\end{array}\right),
$$

$$
\left(\begin{array}{ccc}
0&1&0 \\
0&0&1 \\
1&0&0  
\end{array}\right)
\left(\begin{array}{c}
x \\
y \\
z
\end{array}\right)
,\quad
\left(\begin{array}{ccc}
0&1&0 \\
0&0&1 \\
1&0&0  
\end{array}\right)
\left(\begin{array}{c}
1 \\
1 \\
1
\end{array}\right)
,\quad
\left(\begin{array}{ccc}
0&1&0 \\
0&0&1 \\
1&0&0  
\end{array}\right)
\left(\begin{array}{c}
\omega \\
\omega^2 \\
\omega^3
\end{array}\right),
$$
donde $\omega=-\frac{1}{2}+\frac{\sqrt{3}}{2}\mathrm{i}$ es una raíz cúbica de $1$.
$$
\left(\begin{array}{ccccc}
0 & 1 & 0 & 0 & 0 \\
0 & 0 & 1 & 0 & 0 \\
0 & 0 & 0 & 1 & 0 \\
0 & 0 & 0 &  0 & 1 \\
1 & 0 & 0 & 0 & 0
\end{array}\right)
\left(\begin{array}{c}
x_1 \\
x_2 \\
x_3 \\
x_4 \\
x_5 
\end{array}\right)
$$

$$\left(\begin{array}{ccccc}
0 & 0 & 0 & 0 & 1 \\
0 & 0 & 1 & 0 & 0 \\
0 & 1 & 0 & 0 & 0 \\
1 & 0 & 0 & 0 & 0 \\
0 & 0 & 0 & 1 & 0
\end{array}\right)
\left(\begin{array}{c}
x_1 \\
x_2 \\
x_3 \\
x_4 \\
x_5 
\end{array}\right)
$$
\end{ejercicio}

Para encontrar los distintos valores y vectores propios de una matriz real simétrica $A$ de $n\times n$ se tienen que calcular las raíces del polinomio característico de la matriz: 
$$\mathrm{det}(A-xI_n)$$

Más adelante revisaremos la definición general del determinante y el polinomio característico.
\newpage

Por lo pronto utilizaremos que en el caso de matrices simétricas de $(2\times 2)$, tanto el determinante como el polinomio característico son muy simples: $$\mathrm{det}
\left(\begin{array}{cc}
a&b \\
c&d 
\end{array}\right)=ad-bc,
\quad
\mathrm{det}
\left(\begin{array}{cc}
p-x&q \\
q&r-x 
\end{array}\right)
=
(p-x)(r-x)-q^2
=x^2-(p+r)x-q^2+pr.
$$

\begin{ejercicio}
Muestra que los valores propios de una matriz simétrica 
$$
\left(\begin{array}{cc}
p&q \\
q&r 
\end{array}\right)
$$
estan dados por la fórmula $$\frac{1}{2}((r+p)\pm \sqrt{(r-p)^2+4q^2})$$
\end{ejercicio}

\begin{ejercicio}
Calcula los valores propios de la siguiente matriz y un vector propio para cada valor propio.
$$
\left(\begin{array}{cc}
-1&4 \\
4&5 
\end{array}\right)
$$
\end{ejercicio}
\vspace{2cm}

\begin{ejercicio}
Usando el ejercicio anterior, calcula
$$
\left(\begin{array}{cc}
-1&4 \\
4&5 
\end{array}\right)^n
\left(\begin{array}{c}
7 \\
1 
\end{array}\right)
$$
\end{ejercicio}

Finalmente terminaremos con nuestro objetivo:

\begin{ejercicio}
Encuentra un vector propio con valor propio $\phi_+=\frac{1+\sqrt{5}}{2}$ y un vector propio con valor propio $\phi_-=\frac{1-\sqrt{5}}{2}$ Recuerda que $\phi_+^2=1+\phi_+$ y $\phi_-^2=1+\phi_-$ y que $\phi_+\phi_-=-1$.

$$
\left(\begin{array}{cc}
0&1 \\
1&1
\end{array}\right)^n
\left(\begin{array}{c}
x \\
y
\end{array}\right)
= \phi_+
\left(\begin{array}{c}
x \\
y
\end{array}\right)
,\quad
\left(\begin{array}{cc}
0&1 \\
1&1
\end{array}\right)^n
\left(\begin{array}{c}
x \\
y
\end{array}\right)
= \phi_-
\left(\begin{array}{c}
x \\
y
\end{array}\right)
$$
\end{ejercicio}

\begin{ejercicio}
Expresa el vector inicial $(0,1)$ como combinación lineal de los vectores propios del ejercicio anterior para mostrar la fórmula del Teorema \ref{Teo:Fibo}: 
\begin{equation}
F_n=\frac{(\phi_+)^n-(\phi_-)^n}{\sqrt{5}}
\end{equation}
\end{ejercicio}

\newpage

\section{Productos interiores y magnitudes}

La {\bf transpuesta $A^t$} y la {\bf adjunta} $A^*$ de una matriz $A$ de $(n\times m)$ son matrices de $(m\times n)$ que se definen como sigue:
$$A=
\left(\begin{array}{cccc}
    a_{11}& a_{12} & \dots & a_{1m} \\
    a_{21}& a_{22} & \dots &  a_{2m}\\
    \vdots& \vdots &\ddots &  \\
    a_{n1}& a_{n2} & & a_{nm}
\end{array} \right),
\quad
A^t:=
\left(\begin{array}{cccc}
    a_{11}& a_{21} & \dots & a_{n1} \\
    a_{12}& a_{22} & \dots &  a_{n2}\\
    \vdots& \vdots &\ddots &  \\
    a_{1m}& a_{2m} & & a_{nm}
\end{array} \right),
\quad 
A^*:=
\left(\begin{array}{cccc}
    \bar a_{11}& \bar a_{21} & \dots & \bar a_{n1} \\
    \bar a_{12}& \bar a_{22} & \dots &  \bar a_{n2}\\
    \vdots& \vdots &\ddots &  \\
    \bar a_{1n}& \bar a_{2n} & & \bar a_{nn}
\end{array} \right)
$$

Te recordamos que $\bar z=a-b\mathrm{i}$ denota el conjugado de un número complejo $z=a+b\mathrm{i}$. La matriz adjunta es una especie de generalización del conjugado para matrices.

\begin{ejercicio}
Muestra que $(A^t)^t=A=(A^*)^*$
\end{ejercicio}
\begin{ejercicio}
Muestra que $(AA^*)$ y $(A^*A)$ es una matriz real simétrica (¿de qué dimensión?). 
\end{ejercicio}

Cuando multiplicamos un complejo $z=a+b\mathrm{i}$ por su adjunto, obtenemos su norma al cuadrado: $$z\bar z=a^2+b^2,$$ que no solo es un número real, sino positivo. De manera análoga, la matriz $(AA^*)$ no solo es real simétrica, sino positiva: Una {\bf matriz positiva} es aquella todos sus valores propios son números reales positivos. Esto no lo discutiremos aquí.

Si las entradas son reales $A^*=A^t$ es simplemente la matriz transpuesta.

En particular, la matriz transpuesta/adjunta de un vector columna son {\bf vectores fila}: 
$$
v=
\left(\begin{array}{c}
    x_{1}\\
    x_{2}\\
    \vdots \\
    x_{n}
\end{array} \right)
,\quad
v^t=(x_{1}, x_{2}, \dots, x_{n})
,\quad
v^*=(\bar x_{1},\bar x_{2}, \dots, \bar x_{n})
$$

Consideremos ahora dos vectores columna de la misma dimensión:
$$
v=
\left(\begin{array}{c}
    x_{1}\\
    x_{2}\\
    \vdots \\
    x_{n}
\end{array} \right)
,\quad
w=
\left(\begin{array}{c}
    y_{1}\\
    y_{2}\\
    \vdots \\
    y_{n}
\end{array} \right)
$$
Como matrices rectangulares con dimensiones compatibles, se pueden considerar las multiplicaciones de vectores con vectores adjuntos en cualquier orden: $v^*w, w^*v, vw^*,wv^*$. Si se multiplica un vector renglón con un vector columna, se obtiene una matriz de $1\times 1$, es decir, un número (real o complejo, según sea el caso):
$$v^*w=x_{1}\bar y_{1}+x_{2}\bar y_{2}+\dots +x_{n}\bar y_{n}$$

$$w^*v=y_{1}\bar x_{1}+y_{2}\bar x_{2}+\dots +y_{n}\bar x_{n},$$

\begin{ejercicio}
Sean $z$ y $w$ números complejos, muestra que $\overline {z+w}=\bar z+ \bar w$
\end{ejercicio}

\begin{ejercicio}
Sean $z$ y $w$ números complejos, muestra que $\overline {z+w}=\bar z+ \bar w$
\end{ejercicio}

\begin{ejercicio}
Sean $z$ y $w$ números complejos, entonces $z\bar w=\overline{\bar z w}$
\end{ejercicio}

\begin{ejercicio}
Muestra que $v^*w$ es el conjugado de $w^*v$.
\end{ejercicio}

A $v^*w$ se le denomina el {\bf producto interior} de los vectores $v,w$ y a veces se le denota por $\langle v,w\rangle:=v^*w$. Para el caso $v=w$ (ya sean vectores reales o complejos), obtenemos la magnitud del vector al cuadrado:
$$\langle v,v\rangle=z_1\bar z_1+z_2\bar z_2+ \dots +z_n\bar z_n, \quad \langle v,v\rangle=x_1^2+x_2^2+ \dots+ x_n^2, $$

\begin{definicion}
Si $\langle v,w\rangle=0$ decimos que los vectores $v$ y $w$ son {\bf ortogonales}.

Si $\langle v,v\rangle=1$ decimos que el vector $v$ es {\bf unitario}.
\end{definicion}

Si multiplicamos los vectores en el orden inverso $vw^*, vw^*$, obtenemos matrices de $n\times n$. Estas son matrices de rango $1$ y aparecerán al final cuando diagonalicemos una matriz.

\section{Volúmenes y Determinantes}

Una matriz de $n\times n$ se puede pensar como un arreglo de $n$ vectores columna cada uno de dimensión $n$. Equivalentemente se puede pensar como un arreglo de $n$ vectores renglón.

\begin{definicion}
El determinante de una matriz de $2\times 2$ se define como
\vspace{2cm}

El determinante de una matriz de $3\times 3$ se define como
\vspace{3cm}
\end{definicion}

Interpretación: Volúmen del sólido determinado por los vectores de la matriz es igual al valor absoluto del determinante.
%GeoGebra: bmxfaxvq (2x2)
%GeoGebra: jjfxxtsq (3x3)

\begin{definicion}
El determinante de una matriz de $n\times n$ se define como
\vspace{4cm}
\end{definicion}

Propiedades del determinante
\vspace{4cm}

\newpage

\begin{definicion}
Polinomio característico.
\end{definicion}

El polinomio característico es la herramienta principal para encontrar los valores propios de una matriz.


\section{Lectura: Diagonalización de matrices normales}

Una matriz se llama {\bf normal}, si conmuta con su matriz adjunta:

Ejemplos de matrices normales.
\begin{enumerate}
    \item Matrices unitarias ($UU^*=I_n=U^*U$). 
    \item Entre ellas se encuentran las matrices de permutación.
    \item Matrices autoadjuntas. Son las matrices que cumplen $A=A^*$.
    \item Entre ellas las matrices simétricas reales.
\end{enumerate}

En general una matriz al azar no es normal.

Matrices normales y su descomposición singular.

\begin{teorema}[Descomposición singular de matrices normales]
Para toda matriz normal $A$, existen $D$ matriz diagonal y $U$ matriz unitaria tales que $U$ {\bf diagonaliza} simultáneamente a $A$ y a su adjunta $A^*$. 

Es decir: $A=UDU^*\quad A^*=UD^*U^*$, 
\end{teorema}

\begin{corolario}[Diagonalización de matrices reales simétricas]
Toda matriz real simétrica se puede escribir como $A=UDU^*$, donde $D$ es una matriz diagonal y $U$ es una matriz unitaria.
\end{corolario}

La diagonalización de una matriz simétrica es muy útil porque nos permite calcular potencias fácilmente: Si $A=UDU^*$ donde $U$ es unitaria y $D$ es diagonal, tenemos que
$$A^n=(UDU^*)^n=UDU^*UDU^*\dots UDU^*UDU^*=UD^nU^*.$$

\newpage

\section{Problemas}

