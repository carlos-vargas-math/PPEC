%%%%%%%%%%%%%%%%%%%%%%%%%%%%%%%%%%%%%%%%%%%%%%%%%%%%%%%%%%%%%%%%%%%
%%                                                               %%
%% This is the sample.tex file for the ejpecp document class.    %%
%% This file is for ejpecp version 1.0                           %%
%% Please be sure that you are using the lastest version:        %%
%%  http://mirror.ctan.org/macros/latex/contrib/ejpecp/          %%
%%                                                               %%
%% The ejpecp class works *only* with a pdflatex engine.         %%
%% You need the ejpecp.cls in your current directory or in any   %%
%% directory scanned for cls files by your pdflatex engine.      %%
%%                                                               %%
%% Manual inclusion of page layout commands is useless.          %%
%%                                                               %%
%% Note that any complex file will produce delayed publication!  %%
%%                                                               %%
%%%%%%%%%%%%%%%%%%%%%%%%%%%%%%%%%%%%%%%%%%%%%%%%%%%%%%%%%%%%%%%%%%%

%%%%%%%%%%%%%%%%%%%%%%%%%%%%%%%%%%%%%%%%%%%%%%%%%%%%%%%%%%%%%%%%%%%
%%                                                               %%
%% Journal selection: ECP or EJP.                                %%
%%                                                               %%
%%%%%%%%%%%%%%%%%%%%%%%%%%%%%%%%%%%%%%%%%%%%%%%%%%%%%%%%%%%%%%%%%%%

\documentclass[ECP]{ejpecp} % replace ECP by EJP if needed.  

%%%%%%%%%%%%%%%%%%%%%%%%%%%%%%%%%%%%%%%%%%%%%%%%%%%%%%%%%%%%%%%%%%%
%%                                                               %%
%% Please uncomment and adapt to your encoding if needed:        %%
%%                                                               %%
%%%%%%%%%%%%%%%%%%%%%%%%%%%%%%%%%%%%%%%%%%%%%%%%%%%%%%%%%%%%%%%%%%%

%\usepackage[T1]{fontenc}
%\usepackage[utf8]{inputenc}

%%%%%%%%%%%%%%%%%%%%%%%%%%%%%%%%%%%%%%%%%%%%%%%%%%%%%%%%%%%%%%%%%%%
%%                                                               %%
%% Please add here your own packages (be minimalistic please!):  %%
%% Please avoid using exotic packages and keep things simple.    %%
%% It is not necessary to include ams* and graphicx packages     %%
%% since they are automatically included by the ejpecp class.    %%
%%                                                               %%
%%%%%%%%%%%%%%%%%%%%%%%%%%%%%%%%%%%%%%%%%%%%%%%%%%%%%%%%%%%%%%%%%%%

%\usepackage{enumerate}  % uncomment to use this package

%%%%%%%%%%%%%%%%%%%%%%%%%%%%%%%%%%%%%%%%%%%%%%%%%%%%%%%%%%%%%%%%%%%
%%                                                               %%
%% Title (please edit and customize):                            %%
%%                                                               %%
%%%%%%%%%%%%%%%%%%%%%%%%%%%%%%%%%%%%%%%%%%%%%%%%%%%%%%%%%%%%%%%%%%%

\SHORTTITLE{Random Matrices and their Outer-blocks} 

\TITLE{Random Matrices and their Outer-blocks \thanks{Supported
    by ?}} % \thanks is optional. Insert line breaks with \\

\DEDICATORY{} % Optional

%%%%%%%%%%%%%%%%%%%%%%%%%%%%%%%%%%%%%%%%%%%%%%%%%%%%%%%%%%%%%%%%%%%
%%                                                               %%
%% Authors (please edit and customize):                          %%
%%                                                               %%
%%%%%%%%%%%%%%%%%%%%%%%%%%%%%%%%%%%%%%%%%%%%%%%%%%%%%%%%%%%%%%%%%%%

\AUTHORS{%
   Salvador Galindo\footnote{CIMAT.
    \EMAIL{salvador.galindo@cimat.mx}}
    \and 
   Carlos Vargas\footnote{
    \EMAIL{obieta@gmail.com}}
    }%AUTHORS
%% Type \and between all consecutive authors (not only before the last author).
%% Note: you may use \BEMAIL to force a line break before e-mail display.

%% Here is a compact example with two authors with same affiliation
%% \AUTHORS{%
%%  Michael~First\footnote{Some University. \EMAIL{mf,js@uni.edu}
%%  \and 
%%  John~Second\footnotemark[2]}%AUTHORS
%% Note: The \footnotemark is the footnote number that you wish to reuse. Here
%% it is [2] (we took into account the footnote generated by \thanks in title).

%%%%%%%%%%%%%%%%%%%%%%%%%%%%%%%%%%%%%%%%%%%%%%%%%%%%%%%%%%%%%%%%%%%
%%                                                               %%
%% Please edit and customize the following items:                %%
%%                                                               %%
%%%%%%%%%%%%%%%%%%%%%%%%%%%%%%%%%%%%%%%%%%%%%%%%%%%%%%%%%%%%%%%%%%%

\KEYWORDS{Non-commutative probability; free independence ;  Operator-valued free probability;  cumulants; non-crossing partitions} % Separate items with ;

\AMSSUBJ{NA} % Edit. Separate items with ;
%\AMSSUBJSECONDARY{FIXME:} % Optional, separate items with ;

\SUBMITTED{\today} % Edit.
\ACCEPTED{\today} % Edit.

%%%%%%%%%%%%%%%%%%%%%%%%%%%%%%%%%%%%%%%%%%%%%%%%%%%%%%%%%%%%%%%%%%%
%%                                                               %%
%% Please uncomment and edit if you have an arXiv ID:            %%
%%                                                               %%
%%%%%%%%%%%%%%%%%%%%%%%%%%%%%%%%%%%%%%%%%%%%%%%%%%%%%%%%%%%%%%%%%%%

%\ARXIVID{NNNN.NNNNvn} % Edit.
%\HALID{hal-NNN} % Edit.

%%%%%%%%%%%%%%%%%%%%%%%%%%%%%%%%%%%%%%%%%%%%%%%%%%%%%%%%%%%%%%%%%%%
%%                                                               %%
%% The following items will be set by the Managing Editor.       %%
%%                                                               %%
%%%%%%%%%%%%%%%%%%%%%%%%%%%%%%%%%%%%%%%%%%%%%%%%%%%%%%%%%%%%%%%%%%%

\VOLUME{0}
\YEAR{2021}
\PAPERNUM{0}
\DOI{vVOL-PID}

%%%%%%%%%%%%%%%%%%%%%%%%%%%%%%%%%%%%%%%%%%%%%%%%%%%%%%%%%%%%%%%%%%%
%%                                                               %%
%% Please edit and customize the abstract:                       %%
%%                                                               %%
%%%%%%%%%%%%%%%%%%%%%%%%%%%%%%%%%%%%%%%%%%%%%%%%%%%%%%%%%%%%%%%%%%%

\ABSTRACT{There have been recent theoretical advances in operator-valued c-free probability. 
From a combinatorial point of view, c-free probability simply requires to distinguish outer-blocks vs inner-blocks of non-crossing partitions while constructing cumulants.

We explore applications to large random matrices. We address new models and provide simpler descriptions for known random matrix models.}

%%%%%%%%%%%%%%%%%%%%%%%%%%%%%%%%%%%%%%%%%%%%%%%%%%%%%%%%%%%%%%%%%%%
%%                                                               %%
%% Please add your own macros and environments below:            %%
%%                                                               %%
%% If possible, avoid using \def and use instead \newcommand     %%
%% If possible, avoid defining your own environments, and use    %%
%% instead the environments already defined by ejpecp:           %%
%%  assumption, assumptions, claim, condition, conjecture,       %%
%%  corollary, definition, definitions, example, exercise, fact, %%
%%  facts, heuristics, hypothesis, hypotheses, lemma, notation,  %%
%%  notations, problem, proposition, remark, theorem             %%
%%                                                               %%
%%%%%%%%%%%%%%%%%%%%%%%%%%%%%%%%%%%%%%%%%%%%%%%%%%%%%%%%%%%%%%%%%%%

\newcommand{\ABS}[1]{\left(#1\right)} % example of author macro
\newcommand{\veps}{\varepsilon} % another example of author macro

%%%%%%%%%%%%%%%%%%%%%%%%%%%%%%%%%%%%%%%%%%%%%%%%%%%%%%%%%%%%%%%%%%%
%%                                                               %%
%% No macro definitions below this line please!                  %%
%%                                                               %%
%%%%%%%%%%%%%%%%%%%%%%%%%%%%%%%%%%%%%%%%%%%%%%%%%%%%%%%%%%%%%%%%%%%

\begin{document}

%%%%%%%%%%%%%%%%%%%%%%%%%%%%%%%%%%%%%%%%%%%%%%%%%%%%%%%%%%%%%%%%%%%
%%                                                               %%
%% No need for \maketitle.                                       %%
%%                                                               %%
%%%%%%%%%%%%%%%%%%%%%%%%%%%%%%%%%%%%%%%%%%%%%%%%%%%%%%%%%%%%%%%%%%%

%%%%%%%%%%%%%%%%%%%%%%%%%%%%%%%%%%%%%%%%%%%%%%%%%%%%%%%%%%%%%%%%%%%
%%                                                               %%
%% Please replace what follows by the body of your article       %%
%% (up to the bibliography):                                     %%
%%                                                               %%
%%%%%%%%%%%%%%%%%%%%%%%%%%%%%%%%%%%%%%%%%%%%%%%%%%%%%%%%%%%%%%%%%%%

\section{Introduction}

Operator-valued free probability is now consolidated as an important theory for the study of eigenvalue distributions of large random matrix models.

Some refs. \cite{Voi91} standard classical models (Wigner, Wishart-MP).

The rough idea is bla.

Operator-valued free probability \cite{Voi95} greatly enhanced the theory's applicability. While scalar-valued asymptotic free independence is a rare phenomenon that holds only for very regular random matrix models, operator-valued free independence is everywhere, and the main task remains to find the most convenient algebras of constants (usually the smallest) for which free independence holds. 

In this manuscript, we approach random matrix models by observing the behaviour of their outer-blocks, in view of a recent, operator-valued friendly version of c-free independence.

To warm-up, we compute in Section 3 the (non-asymptotic) joint distributions Ginibre matrices and Deterministic matrices, with respect to the conditional expectation $id\otimes \mathbb E$ by emphasising our analysis on the outer-blocks. This shows in particular, that scalar asymptotic freeness may occur, even while considering vector-state distributions (and not just for normalized expected traces).

Then we study a $B$-valued generalization of the formula for products of free random variables (Section 4). For its application to random matrices the original scalar-valued work relies heavily on the tracial property of the expectation. We must consider outer-blocks more carefully to come up with a useful treatment of products of B-free random variables.

Finally we obtain simpler descriptions of the random matrix models involving unitary random matrices (Section 5).

\section{Preliminaries}

Def: Operator-valued probability space

Examples: Deterministic Matrix spaces. Their vector-state valued distributions.

Classical, free, Boolean cumulants.

PB-Boolean cumulants.

\section{Outer-blocks for (non-asymptotic) joint distributions of Ginibre and deterministic matrices}

Def: Ginibre matrices.

Joint distributions of Ginibre and deterministic matrices.

\section{Products of $\mathcal B$ free independent random variables}

Scalar case \cite{AV12}

Operator-valued version.

Issue: Non-tracial conditional expectations. Discussion
-
Solution: Outer blocks and the square root trick.

\section{Unitary matrices, Boolean and free cumulants}

Unitary matrices and their realizations.



%%%%%%%%%%%%%%%%%%%%%%%%%%%%%%%%%%%%%%%%%%%%%%%%%%%%%%%%%%%%%%%%%%%
%%                                                               %%
%% Use the two commands below for producing your bibliography    %%
%% with bibtex, then comment again the commands and include the  %%
%% content of the .bbl file in this file below the commands.     %%
%%                                                               %%
%%%%%%%%%%%%%%%%%%%%%%%%%%%%%%%%%%%%%%%%%%%%%%%%%%%%%%%%%%%%%%%%%%%

%\bibliographystyle{amsplain}
%\bibliography{yourbibfilename}

% add below the content of your .bbl file produced by bibtex.

\begin{thebibliography}{99}

\bibitem{Ari13} O. Arizmendi. Convergence of the fourth moment and infinite divisibility 
\emph{Probab. Math. Statist.} {\bf 33} (2013), no. 2, 201--212.

\bibitem{Ari18} O. Arizmendi. $k$-divisible random variables and  
\emph{Adv. App. Math.} {\bf 93} (2018), 1--68.

\bibitem{AHLV15} O. Arizmendi, T. Hasebe, F. Lehner, C. Vargas. Relations between cumulants in non-commutative probability 
\emph{Adv. Math.} {\bf 282} (2015), 10, 56--92.

\bibitem{AV12} O. Arizmendi, C. Vargas. Products of free random variables and $k$-divisible partitions  
\emph{Electron. Commun. Probab.} {\bf 17} (2012), 11, 13 pp.

\bibitem{BN08} S. Belinschi and A. Nica. On a remarkable semigroup of homomorphisms with
respect to free multiplicative convolution, \emph{Indiana Univ. Math. J.} {\bf 57}, No. 4 (2008),
1679--1713.

\bibitem{BN08} S.T. Belinschi, A. Nica, $\eta$-series and a Boolean Bercovici–Pata bijection for bounded $k$-tuples, \emph{Adv. Math. } 2217(1) (2008) 1--41.

\bibitem{BS02} A. Ben-Ghorbal and M. Schuermann. Non-commutative notions of
stochastic independence, \emph{Math. Proc. Camb. Phil. Soc.} {\bf 133}, 531--561, (2002).

\bibitem{BS12} S. T. Belinschi, D. Shlyakhtenko, Free Probability of type B: Analytic interpretation and applications, \emph{American Journal of Mathematics} Vol. {\bf 134}, No. 1, pp. 193--234, 2012.

\bibitem{BS91} M. Bozejko, R. Speicher, $\psi$-simmetrized and independent white noises, \emph{Quantum probability and related topics (L. Accardi Ed.)} vol {\bf VI}, World Scientific, Singapore, 219--236, 1991.

\bibitem{BGN03} P. Biane, F. Goodman, A. Nica. Non-crossing cumulants of type B, \emph{Transactions of the American Mathematical Society} {\bf 355} (2003), 2263--2303.

\bibitem{BP99} H. Bercovici, and V. Pata, with appendix by P. Biane, Stable laws and domains of attraction in free probability theory.
\emph{Ann. of Math.} {\bf 149}, 1023--1060, (1999).

\bibitem{BLS96} M. Bozejko, M. Leinert and R. Speicher. Convolution and limit theorems for conditionally
free random variables. \emph{Pac. J. Math.} {\bf 175}, 357--388, (1996).

\bibitem{CDM16} G. Cebron, A. Dahlqvist and C. Male. Universal constructions for spaces of traffics, (2016)

\bibitem{CHS18} B. Collins, T. Hasebe, N. Sakuma. Free probability for purely discrete eigenvalues of random matrices, \emph{J. of the Math. Soc. Japan} {\bf 70} 3, 1111--1150 (2018).

\bibitem{CMSS07} B. Collins, J. Mingo, P. Sniady, R. Speicher, Second order freeness and fluctuations of random matrices: III Higher order freeness and free cumulants. \emph{Documenta Math.} {\bf 12} (2007), 1--70.

\bibitem{CH71} C. D. Cushen and R. L. Hudson. A quantum central limit theorem. \emph{J. Appl. Prob.} {\bf 8},
454--469, (1971).

\bibitem{Fra06} U. Franz. Multiplicative monotone convolutions \emph{Quantum Probability, M. Bozejko, W. Mlotkowski and J. Wysoczanski (eds.), Banach Center Publications} {\bf 73}, 153--166, (2006).

\bibitem{Fra09} U. Franz. Boolean convolution of probability measures on the unit circle. \emph{Analyse et probabilites, P. Biane, J. Faraut, and H. Ouerdiane (eds.), Séminaires et Congrès} {\bf 16}, 83--93, (2009).

\bibitem{DGSV21} N. Giri and W. von Waldenfels. An algebraic version of the central limit theorem. \emph{Z. Wahrsch. verw. Gebiete.}
{\bf 42}, 129--134, (1978).

\bibitem{GW78} N. Giri and W. von Waldenfels. An algebraic version of the central limit theorem. \emph{Z. Wahrsch. verw. Gebiete.}
{\bf 42}, 129--134, (1978).

\bibitem{GS17} Y. Gu, P. Skoufranis, Conditionally bi-free independence for pairs of faces. \emph{J. Funct. Anal.} {\bf 273} (5), 1663--1733 (2017)

\bibitem{GHS18} Y. Gu, T. Hasebe, P. Skoufranis, Bi-monotonic independence for pairs of algebras, \emph{J. Theoret. Probab.} to appear (2018).

\bibitem{HS11} T. Hasebe, H.Saigo. The monotone cumulants. \emph{Ann. Inst. H. Poincaré Prob. Statist.} Vol {\bf 47}, No 4 (2011), 1160--1170.

\bibitem{Has11} T. Hasebe. Conditionally monotone independence I: Independence, additive convolutions and related convolutions. \emph{Infin. Dimens. Anal. Quantum. Probab. Relat. Top.} {\bf 14}, no. 3, 465--516, (2011).

\bibitem{Has13} T. Hasebe. Conditionally monotone independence II: Multiplicative convolutions and infinite divisibility. \emph{Complex Anal. Op. Th.} {\bf 7}, no. 1, 115--134, (2013).

\bibitem{Hud73} R. L. Hudson. A quantum mechanical central limit theorem for anticommuting observables.
\emph{J. Appl. Prob.} {\bf 10}, 502--509, (1973).

\bibitem{Ion12} V. Ionescu. A Note on Amalgamated boolean, orthogonal and conditionally monotone or antimonotone products of operator-valued C* algebraic-probability spaces.  \emph{ REV. ROUMAINE MATH. PURES APPL.}, {\bf 57} (2012), 4, 341--37.
%checar

\bibitem{Kre72} G. Kreweras, Sur les partitions non-croisées d’un cycle, \emph{Discrete Mathematics} {\bf 1} (1972), 333--350.

\bibitem{Len07} R. Lenczewski, Decompositions of the additive free convolution, \emph{J. Funct. Anal.} {\bf 246} (2007), 330--365.

\bibitem{Mal11} C. Male. Traffic distributions and independence: permutation invariant random matrices and the three notions of independence. Preprint, (2011).

\bibitem{MP67} V. Mar\v cenko and L. Pastur. Distribution of eigenvalues for some
sets of random matrices, \emph{Math. USSR-Sbornik} {\bf 1}, 457--483, (1967).

\bibitem{MSS07} J. Mingo, P. Sniady, and R. Speicher, Second order freeness and fluctuations of random matrices: II. Unitary random matrices. \emph{Adv. in Math.} {\bf 209} (2007), 212--240.

\bibitem{MS06}
J. Mingo and R. Speicher, Second Order Freeness and Fluctuations of Random Matrices: I. Gaussian and Wishart matrices and Cyclic Fock spaces. \emph{J. Funct. Anal.}, {\bf 235}, 2006, 2, 6--270.

\bibitem{Mlo02} W. Mlotkowski, Operator-valued version of conditionally free product, \emph{Studia Mathematica} {\bf 153} no.1, 2012.
% Segundo funcional es lineal.

\bibitem{Mur01} N. Muraki. Monotonic independence, monotonic central limit theorem and monotonic law of small numbers, \emph{Infin. Dimens. Anal. Quantum. Probab. Relat. Top.} {\bf 4}, no.{\bf 1}, 39--58, (2001).

\bibitem{Mur03} N. Muraki. The five independence as natural products, \emph{Infin. Dimens. Anal. Quantum. Probab. Relat. Top.} {\bf 06}, 337, (2003). 

\bibitem{NS06} A. Nica, R. Speicher, \emph{Lectures on the combinatorics of free probability}, London Mathematical Society Lecture Note Series, vol. {\bf 335}, Cambridge University Press, Cambridge, 2006.

\bibitem{Ora02} F. Oravecz. Fermi convolution. \emph{Infin. Dimens. Anal. Quantum. Probab. Relat. Top.} {\bf 5}, no. 2, 235--242, (2002).

\bibitem{Spe94} R. Speicher. Multiplicative functions on the lattice of noncrossing partitions
and free convolution, \emph{Math. Ann.} {\bf 298}, no. 4, 611--628, (1994).

\bibitem{Sta12} Richard P. Stanley, \emph{Enumerative Combinatorics}. Volume 1, second ed., Cambridge Studies in Advanced Mathematics, vol. {\bf 49}, Cambridge University
Press, Cambridge, (2012).

\bibitem{PVW15} M. Popa, V. Vinnikov, J.-C. Wang, On the multiplication of operator-valued C-free random variables, \emph{Colloquium Mathematicum} {\bf 153}(2), 2015.
% S.transform twisted multiplicativity, etc

\bibitem{PW11} M. Popa, J.-C. Wang, On multiplicative conditionally free convolution, \emph{Trans. Amer. Math. Soc. 363 (2011)} {\bf 363} no. 12, 6309--6335 2011.
% caso escalar

\bibitem{Pop14} M. Popa. A Fock space model for addition and multiplication of c-free random variables \emph{Proc. Amer. Math. Soc.} {\bf 142} (2014).
% modelo en espacio de fock (escalar)

\bibitem{Pop08} M. Popa, Multilinear function series in conditionally free probability with amalgamation, \emph{Communications on Stochastic Analysis} Vol. {\bf 2}, No. 2 (2008) 307--322.
% positividad, caso inverso al que nosotros presentamos.

\bibitem{Spe98} R. Speicher. \emph{Combinatorial theory of the free product with amalgamation
and operator-valued free probability theory}. Memoirs of the American Math.
Society, vol. {\bf 132}, (1998).

\bibitem{SW97} R. Speicher and R. Woroudi. Boolean convolution. \emph{Fields Inst. Commun.} vol. {\bf 12}, 1997, 267--279, (1997).

\bibitem{Spe97} R. Speicher. On universal products \emph{Fields Inst. Commun.} vol. {\bf 12}, 257--266, (1997).

\bibitem{Voi85} D. Voiculescu. Symmetries of some reduced free product $C^*$-algebras,\emph{Operator
algebras and their connections with topology and ergodic theory (Busteni,
1983), Lecture Notes in Math.}, vol. {\bf 1132}, Springer, Berlin, 556--588, (1985).


\bibitem{Voi91} D. Voiculescu. Limit laws for random matrices and free products, \emph{Invent. Math.} {\bf 104}, 201--220, (1991).

\bibitem{VDN92} D. Voiculescu, K. Dykema and A. Nica. \emph{Free Random Variables}, CRM Monograph Series, Vol. {\bf 1}, AMS, (1992).

\bibitem{Voi93} D. Voiculescu, The analogues of entropy and of Fisher's information measure in free probability theory I. \emph{Comm. Math. Phys.}, Volume {\bf 155}, Number 1 (1993), 71--92.

\bibitem{Voi95} D. Voiculescu. Operations on certain non-commutative operator-valued random
variables. \emph{Recent advances in operator algebras (Orleans, 1992). Asterisque} {\bf 232}, (1995).

\bibitem{Voi14} D. Voiculescu: Free probability for pairs of faces I. \emph{Commun. Math. Phys.} {\bf 332} (3), 955--980 (2014).

\bibitem{Wal78} W. von Waldenfels. An algebraic central limit theorem in the anti-commuting case. \emph{Z.
Wahrsch. verw. Gebiete} {\bf 42}, 135--140, (1978).

\bibitem{Wig58} E. Wigner. On the distribution of the roots of certain symmetric matrices, \emph{Ann. of Math.} {\bf 67}, 325--327, (1958).

\bibitem{Woe86} W. W\"oss. Nearest neighbour random walks on free products of discrete groups
\emph{Boll. Unione Mat. Ital. B (6)}, {\bf 5} (3), 961--982, (1986).


\end{thebibliography}

%%%%%%%%%%%%%%%%%%%%%%%%%%%%%%%%%%%%%%%%%%%%%%%%%%%%%%%%%%%%%%%%%%%
%%                                                               %%
%% You may add acknowledgments (optional).                       %%
%%                                                               %%
%%%%%%%%%%%%%%%%%%%%%%%%%%%%%%%%%%%%%%%%%%%%%%%%%%%%%%%%%%%%%%%%%%%

\ACKNO{Research supported by CONACYT Grant A1-S-976.}

 
%%%%%%%%%%%%%%%%%%%%%%%%%%%%%%%%%%%%%%%%%%%%%%%%%%%%%%%%%%%%%%%%%%%
%%                                                               %%
%% You have reached the end of your document.                    %%
%%                                                               %%
%%%%%%%%%%%%%%%%%%%%%%%%%%%%%%%%%%%%%%%%%%%%%%%%%%%%%%%%%%%%%%%%%%%

\end{document}

%%%%%%%%%%%%%%%%%%%%%%%%%%%%%%%%%%%%%%%%%%%%%%%%%%%%%%%%%%%%%%%%%%%
%%                                                               %%
%% You may put below funny messages to the Managing Editor:      %%
%%                                                               %%
%%%%%%%%%%%%%%%%%%%%%%%%%%%%%%%%%%%%%%%%%%%%%%%%%%%%%%%%%%%%%%%%%%%

%% EOF
